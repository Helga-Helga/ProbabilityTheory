\addcontentsline{toc}{chapter}{Занятие 12. Числовые характеристики случайных величин II}
\chapter*{Занятие 12. Числовые характеристики случайных величин II}

\addcontentsline{toc}{section}{Контрольные вопросы и задания}
\section*{Контрольные вопросы и задания}

\subsubsection*{Запишите основные вероятностные распределения и поясните смысл их параметров.}

Основные распределения:
\begin{enumerate}
\item дискретное распределение:
\begin{enumerate}
\item равномерное дискретное распределение
$$P \left( \xi = k \right) =
\frac{1}{N},
1 \leq k \leq N,
M \xi = \frac{N+1}{2}, D \xi = \frac{N^2 - 1}{12};$$
\item биномиальное распределение: $ \xi \sim Biom \left( n, p \right) $
$$P \left\{ \xi = k \right\} =
C_n^k p^k q^{n-k},$$
где $0 \leq k \leq n, q = 1 - p, M \xi = np, D \xi = npq$;
\item геометрическое распределение $ \xi \sim Geom \left( p \right), p \in \left( 0, 1 \right) $
$$P \left\{ \xi = k \right\} =
pq^k,
k \geq 0,
M \xi = \frac{q}{p},
D \xi = \frac{q}{p^2};$$
\item пуассоновское распределение: $ \xi \sim Pois \left( \lambda \right), \lambda > 0$
$$P \left\{ \xi = k \right\} = \frac{ \lambda^k}{k!} \cdot e^{- \lambda },
k \geq 0,
M \xi = \lambda,
D \xi = \lambda;$$
\end{enumerate}
\item непрерывные распределения:
\begin{enumerate}
\item равномерное распределение: $ \xi \sim U \left( \left[ a, b \right] \right) $
$$p \left( x \right) =
\begin{cases}
\frac{1}{b-a}, \qquad x \in \left[ a, b \right], \\
0, x \notin \left[ a, b \right], \\
\end{cases}
M \xi = \frac{a+b}{2},
D \xi = \frac{ \left( b-a \right)^2}{12};$$
\item экспоненциальное распределение: $ \xi \sim Exp \left( \lambda \right), \lambda > 0$
$$p \left( x \right) =
\begin{cases}
\lambda e^{- \lambda x}, \qquad x \geq 0, \\
0, \qquad x < 0, \\
\end{cases}
M \xi = \frac{1}{ \lambda },
D \xi = \frac{1}{ \lambda^2};$$
\item распределение Коши: $ \xi \sim C \left( \Theta \right), \Theta > 0$
$$p \left( x \right) =
\frac{ \Theta }{ \pi \left( x^2 + \Theta^2 \right) };$$
\item гауссовское (нормальное) распределение:
$$ \xi \sim \mathcal{N} \left( a, \sigma^2 \right),
a \in \mathbb{R},
\sigma^2 > 0,$$
$$p \left( x \right) = \frac{1}{ \sqrt{2 \pi } \cdot \sigma } \cdot e^{- \frac{ \left( x-a \right)^2}{2 \sigma^2}},
x \in \mathbb{R},
M \xi = a,
D \xi = \sigma^2.$$
\end{enumerate}
\end{enumerate}

\subsubsection*{Сформулируйте основные свойства математического ожидания и дисперсии.}

Свойства математического ожидания:
\begin{enumerate}
\item если $ \exists M$, то $ \forall c \in \mathbb{R} \, \exists M \left( c \xi \right) $ и $ M \left( c \xi \right) = cM \xi $;
\item если существует $M \xi $ и $M \eta $, то $ \exists M \left( \xi + \eta \right) $ и $M \left( \xi + \eta \right) = M \xi + M \eta $.
Из этих двух условий следует, что математическое ожидание является линейной функцией;
\item $ \exists M \xi \iff \exists M \left| \xi \right| $, и кроме того $ \left| M \xi \right| \leq M \left| \xi \right| $;
\item если $ \xi \geq 0$, то $M \xi \geq 0$;
\item $\xi \geq \eta, \, \exists \xi, \, \exists \eta \Rightarrow M \xi \geq M \eta $;
\item если $ \eta $ и $ \xi $ --- независимые случайные величины,
для которых существует математическое ожидание, то $ \exists M \left( \xi \eta \right) = M \xi \cdot M \eta $.
\end{enumerate}

Свойства дисперсии:
\begin{enumerate}
\item $D \xi \geq 0$'
\item $ \forall \lambda \in \mathbb{R}: \qquad D \left( \lambda \xi \right) = \lambda^2 D \xi $;
\item для независимых $ \eta $ и $ \xi \, D \left( \xi + \eta \right) = D \xi + D \eta $.
\end{enumerate}

\addcontentsline{toc}{section}{Дополнительные задачи}
\section*{Дополнительные задачи}

\addcontentsline{toc}{section}{Домашнее задание}
\section*{Домашнее задание}
