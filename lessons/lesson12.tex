\addcontentsline{toc}{chapter}{Занятие 12. Числовые характеристики случайных величин II}
\chapter*{Занятие 12. Числовые характеристики случайных величин II}

\addcontentsline{toc}{section}{Контрольные вопросы и задания}
\section*{Контрольные вопросы и задания}

\subsubsection*{Запишите основные вероятностные распределения и поясните смысл их параметров.}

Основные распределения:
\begin{enumerate}
\item дискретное распределение:
\begin{enumerate}
\item равномерное дискретное распределение
$$P \left( \xi = k \right) =
\frac{1}{N},
1 \leq k \leq N,
M \xi = \frac{N+1}{2}, D \xi = \frac{N^2 - 1}{12};$$
\item биномиальное распределение: $ \xi \sim Biom \left( n, p \right) $
$$P \left\{ \xi = k \right\} =
C_n^k p^k q^{n-k},$$
где $0 \leq k \leq n, q = 1 - p, M \xi = np, D \xi = npq$;
\item геометрическое распределение $ \xi \sim Geom \left( p \right), p \in \left( 0, 1 \right) $
$$P \left\{ \xi = k \right\} =
pq^k,
k \geq 0,
M \xi = \frac{q}{p},
D \xi = \frac{q}{p^2};$$
\item пуассоновское распределение: $ \xi \sim Pois \left( \lambda \right), \lambda > 0$
$$P \left\{ \xi = k \right\} = \frac{ \lambda^k}{k!} \cdot e^{- \lambda },
k \geq 0,
M \xi = \lambda,
D \xi = \lambda;$$
\end{enumerate}
\item непрерывные распределения:
\begin{enumerate}
\item равномерное распределение: $ \xi \sim U \left( \left[ a, b \right] \right) $
$$p \left( x \right) =
\begin{cases}
\frac{1}{b-a}, \qquad x \in \left[ a, b \right], \\
0, x \notin \left[ a, b \right], \\
\end{cases}
M \xi = \frac{a+b}{2},
D \xi = \frac{ \left( b-a \right)^2}{12};$$
\item экспоненциальное распределение: $ \xi \sim Exp \left( \lambda \right), \lambda > 0$
$$p \left( x \right) =
\begin{cases}
\lambda e^{- \lambda x}, \qquad x \geq 0, \\
0, \qquad x < 0, \\
\end{cases}
M \xi = \frac{1}{ \lambda },
D \xi = \frac{1}{ \lambda^2};$$
\item распределение Коши: $ \xi \sim C \left( \Theta \right), \Theta > 0$
$$p \left( x \right) =
\frac{ \Theta }{ \pi \left( x^2 + \Theta^2 \right) };$$
\item гауссовское (нормальное) распределение:
$$ \xi \sim \mathcal{N} \left( a, \sigma^2 \right),
a \in \mathbb{R},
\sigma^2 > 0,$$
$$p \left( x \right) = \frac{1}{ \sqrt{2 \pi } \cdot \sigma } \cdot e^{- \frac{ \left( x-a \right)^2}{2 \sigma^2}},
x \in \mathbb{R},
M \xi = a,
D \xi = \sigma^2.$$
\end{enumerate}
\end{enumerate}

\subsubsection*{Сформулируйте основные свойства математического ожидания и дисперсии.}

Свойства математического ожидания:
\begin{enumerate}
\item если $ \exists M$, то $ \forall c \in \mathbb{R} \, \exists M \left( c \xi \right) $ и $ M \left( c \xi \right) = cM \xi $;
\item если существует $M \xi $ и $M \eta $, то $ \exists M \left( \xi + \eta \right) $ и $M \left( \xi + \eta \right) = M \xi + M \eta $.
Из этих двух условий следует, что математическое ожидание является линейной функцией;
\item $ \exists M \xi \iff \exists M \left| \xi \right| $, и кроме того $ \left| M \xi \right| \leq M \left| \xi \right| $;
\item если $ \xi \geq 0$, то $M \xi \geq 0$;
\item $\xi \geq \eta, \, \exists \xi, \, \exists \eta \Rightarrow M \xi \geq M \eta $;
\item если $ \eta $ и $ \xi $ --- независимые случайные величины,
для которых существует математическое ожидание, то $ \exists M \left( \xi \eta \right) = M \xi \cdot M \eta $.
\end{enumerate}

Свойства дисперсии:
\begin{enumerate}
\item $D \xi \geq 0$'
\item $ \forall \lambda \in \mathbb{R}: \qquad D \left( \lambda \xi \right) = \lambda^2 D \xi $;
\item для независимых $ \eta $ и $ \xi \, D \left( \xi + \eta \right) = D \xi + D \eta $.
\end{enumerate}

\addcontentsline{toc}{section}{Аудиторные задачи}
\section*{Аудиторные задачи}

\subsubsection*{12.4}

\textit{Задание.} Существует $n$ типов игрушек, которые можно найти в конфете <<Киндер-сюрприз>>.
Сколько в среднем нужно перебрать конфет, чтобы набрать полную коллекцию?
Как ведёт себя количество попыток при $n \to \infty $?

\textit{Решение.} Количество игрушек не ограничено.
Сколько бы их не брали, общее количество от этого не изменится.
Все игрушки одинаково равномерно перемешаны, так что каждый тип встречается с одинаковой вероятностью $1/n$.

Вероятность в формуле 
$$M \xi =
\sum \limits_{k=0}^{ \infty } kP \left( \xi = k \right) $$
будет найти сложно, поэтому пользуемся формулой
$$M \xi =
\sum \limits_{k=0}^{ \infty } P \left( \xi > k \right).$$

Обозначим через $ \xi $ наименьшее количество конфет, которые необходимо купить, чтобы получить полный набор игрушек,
то есть каждый тип из $n$ типов игрушек должен быть там представлен. 
Напишем такое событие $ \left\{ \xi > k \right\} $.
Оно означает, что среди $k$ игрушек ещё нет полного набора.

$P \left( \xi > k \right) = P$(среди $k$ игрушек нет полного набора) $= P$(среди $k$ игрушек не хватает хотя бы одного типа).
Используем формулу включений-исключений.
Введём события $A_i^k =$ {нет $i$-того типа среди $k$ игрушек}.
Через эти события запишем формулу включений-исключений
$P \left( \xi > k \right) = \\
= P \left( A_1^k \right) C_n^1 -
C_n^2 P \left( A_1^k \cap A_2^k \right) +
C_n^3 \cdot P \left( A_1^k \cap A_2^k \cap A_3^k \right) -
\dotsc + \\
+ \left( -1 \right)^n P \left( A_1^k \cap A_2^k \cap \dotsc \cap A_{n-1}^k \right) C_n^{n-1}$.
Не может быть такого, что нет ни одного типа.
Вычисляем $P \left( A_1^k \right) = P$(среди $k$ игрушек нет игрушек первого типа) =
\begin{equation*}
\begin{split}
= \sum \limits_{m_2 + m_3 + \dotsc + m_n = k, m_i \geq 0} \frac{k!}{m_2! \dotsc m_n!} \cdot \left( \frac{1}{n} \right)^{m_2} \left( \frac{1}{n} \right)^{n_3} \cdot \dotsc \cdot \left( \frac{1}{n} \right)^{m_n} = \\
= \left( \frac{1}{n} \right)^k \left( n-1 \right)^k =
\left( 1 - \frac{1}{n} \right)^k.
\end{split}
\end{equation*}

Вычисляем $P \left( A_1^k \cap A_2^k \right) = P$(среди $k$ игрушек нет первого и второго типов игрушек) =
\begin{equation*}
\begin{split}
= \sum \limits_{m_3 + m_4 + \dotsc + m_n = k, m_i \geq 0} \frac{k!}{m_3! \dotsc m_n!} \cdot
\left( \frac{1}{n} \right)^{m_3} \left( \frac{1}{n} \right)^{m_4} \cdot \dotsc \cdot \left( \frac{1}{n} \right)^{m_n} = \\
= \left( \frac{1}{n} \right)^k \left( n-2 \right)^k =
\left( 1 - \frac{2}{n} \right)^k.
\end{split}
\end{equation*}

В формуле используется мультибиномиальное распределение
$$C_k^{m_1} C_{k-m_1}^{m_2} C_{k- m_1 - m_2}^{m_3} \dotsc.$$

Будем пользоваться формулой
\begin{equation*}
\begin{split}
M \xi =
\sum \limits_{k=0}^{ \infty } P \left( \xi > k \right) =
C_n^1 \sum \limits_{k=0}^{ \infty } P \left( A_1^k \right) -
C_n^2 \sum \limits_{k=0}^{ \infty } P \left( A_1^k \cap A_2^k \right) +
\dotsc + \\
+ \left( -1 \right)^n C_n^{n-1} \sum \limits_{k=0}^{ \infty } P \left( A_1^k \cap A_2^k \cap \dotsc \cap A_{n-1}^k \right),
\end{split}
\end{equation*}
где
\begin{equation*}
\begin{split}
\sum \limits_{k=0}^{ \infty } P \left( A_1^k \right) =
\sum \limits_{k=0}^{ \infty } \left( 1 - \frac{1}{n} \right)^k =
\frac{1}{1 - \left( 1 - \frac{1}{n} \right) } =
n,
\sum \limits_{k=0}^{ \infty } P \left( A_1^k \cap A_2^k \right) =
\frac{n}{2}, \\
\sum \limits_{k=0}^{ \infty } P \left( A_1^k \cap A_2^k \cap \dotsc \cap A_{n-1}^k \right) =
\frac{n}{n-1}.
\end{split}
\end{equation*}
Получаем
$$M \xi =
nC_n^1 - \frac{n}{2} \cdot C_n^2 + \dotsc + \left( -1 \right)^n C_n^{n-1} \cdot \frac{n}{n-1}.$$

\subsubsection*{12.5}

\textit{Задание.} В магазине находится $A$ тонн товара, который быстро портится и который нужно реализовать за один день.
Спрос на этот товар --- случайная величина, которая имеет показательное распределение с параметром $1/2$ (в тоннах).
Цена одного килограмма товара составляет одну гривну.
Найдите:
\begin{enumerate}[label=\alph*)]
\item средний спрос за день;
\item среднюю выручку за день при условии, что $A = 1$;
\item среднюю выручку за день при условии, что $A = 1$, а нереализованный остаток товара утилизируется по 30 коп. за килограмм.
При каком $A$ выручка будет максимальной?
\end{enumerate}

\textit{Решение.} Пусть $ \xi $ --- это спрос на товар ( случайная величина с показательным распределением с параметром $1/2$)
$$ \xi \sim \prod \left( \frac{1}{2} \right).$$

\begin{enumerate}[label=\alph*)]
\item Нужно найти математическое ожидание $ \xi $.
В тоннах оно равно
$$M \xi =
\frac{1}{ \frac{1}{2}} =
2.$$

Это означает, что за день в среднем люди хотели, чтобы было две тонны продукта;
\item введём новую случайную величину.
Пусть $ \eta $ --- выручка за день.
Выразим через $A = 1$ и $ \xi $.
Получим $ \eta = \min \left( A, \xi \right) \cdot 1000 = \min \left( 1, \xi \right) \cdot 1000$ (грн.).

Теперь ищем
\begin{equation*}
\begin{split}
M \eta =
10^3 \cdot M \min \left( \xi, \eta \right) =
10^3 \int \limits_{- \infty }^{+ \infty } \min \left( 1, x \right) p_{ \xi } \left( x \right) dx = \\
= \int \limits_0^{+ \infty } \min \left( 1, x \right) \cdot \frac{1}{2} \cdot e^{- \frac{x}{2}} dx \cdot 10^3.
\end{split}
\end{equation*}
Разобьём на 2 интеграла
$$M \eta =
10^3 \int \limits_0^1 x \cdot \frac{1}{2} \cdot e^{- \frac{x}{2}} dx + 10^3 \int \limits_1^{+ \infty } \frac{1}{2} \cdot e^{- \frac{x}{2}} dx.$$
Вычисляем каждый интеграл отдельно.
Первый интеграл берём по частям
$$u = x, dv = e^{- \frac{x}{2}} dx, du = dx, v = \int e^{- \frac{x}{2}} dx = -2 \int e^{- \frac{x}{2}} d \left( - \frac{x}{2} \right) = -2e^{- \frac{x}{2}}.$$
Получаем
\begin{equation*}
\begin{split}
\int \limits_0^1 x \cdot e^{- \frac{x}{2}} dx =
\left. -2xe^{- \frac{x}{2}} \right|_0^1 + 2 \int \limits_0^1 e^{- \frac{x}{2}} dx =
\left. -2e^{- \frac{1}{2}} + 2 \cdot \left( -2 \right) \cdot e^{- \frac{x}{2}} \right|_0^1 = \\
= -2e^{- \frac{1}{2}} -4e^{- \frac{1}{2}} + 4 =
-6e^{- \frac{1}{2}} + 4.
\end{split}
\end{equation*}
Подставляем
\begin{equation*}
\begin{split}
M \eta =
10^3 \cdot \left( -3e^{- \frac{1}{2}} + 2 \right) + 10^3 \cdot e^{- \frac{1}{2}} =
10^3 \left( 2 - 2e^{- \frac{1}{2}} \right) = \\
= 2 \cdot 10^3 \left( 1 - e^{- \frac{1}{2}} \right) =
2000 \left( 1 - e^{- \frac{1}{2}} \right).
\end{split}
\end{equation*}
Выражение в скобках меньше чем $1/2$, поэтому $M \xi < 1000$;

\item ищем среднюю выручку за день при условии, что $A = 1$, а нереализованный остаток товара утилизируется по 30 коп. за килограмм.
Введём $ \zeta = \min \left( 1, \xi \right) \cdot 1000 + \left[ 1 - \min \left( 1, \xi \right) \right] \cdot 300$.
Перегруппируем слагаемые $ \zeta = \min \left( 1, \xi \right) \cdot 1000 + 300 - 300 \cdot \min \left( 1, \xi \right) $.
Вынесем $ \min \left( 1, \xi \right) $ за скобки $ \zeta = 700 \cdot \min \left( 1, \xi \right) + 300$.
Из предыдущего пункта
$$M \zeta =
2 \cdot 700 \left( 1 - e^{- \frac{1}{2}} \right) + 300.$$
\end{enumerate}

\subsubsection*{12.6}

\textit{Задание.} На фрезеровочный станок с конвейера приходит в среднем 5 деталей.
Время от времени станок требует налаживания, и тогда его останавливают в среднем на 15 минут.
Найдите математическое ожидание числа деталей, которые поступают на станок за время его простоя.

\textit{Решение.}
Введём случайные величины $ \xi_i$ --- количество деталей,
которые пришли с конвейера за $i$-тую минуту простоя, $ \nu $ --- это время простоя, то есть целое количество минут.

Тогда $ \eta $ --- количество деталей, которые поступили за время простоя.
Это
$$M \eta =
M \left( \sum \limits_{i=1}^{ \nu } \xi_i \right) =
M \nu \cdot M \xi_1.$$

По условию $M \xi_i = 5, \xi_i \geq 0, M \nu = 15$.
Тогда $M \eta = 15 \cdot 5 = 75$. 

\addcontentsline{toc}{section}{Домашнее задание}
\section*{Домашнее задание}
