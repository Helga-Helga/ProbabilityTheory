\addcontentsline{toc}{chapter}{Занятие 10. Числовые характеристики случайных величин I}
\chapter*{Занятие 10. Числовые характеристики случайных величин I}

\addcontentsline{toc}{section}{Контрольные вопросы и задания}
\section*{Контрольные вопросы и задания}

\subsubsection*{Приведите формулы для вычисления математического ожидания дискретной случайной величины; случайной величины, которая имеет плотность распределения; функции от случайной величины.}

Пусть $ \xi $ --- дискретная случайная величина.
Она может принимать не более чем счётное количество значений.
Принимает значения $x_1, \dotsc, x_n, \dotsc $ с такими вероятностями $p_1, \dotsc, p_n, \dotsc $.
Тогда математическое ожидание вычисляется так
$$M \xi =
\sum \limits_{i=1}^{ \infty } x_i p_i.$$
Для того, чтобы математическое ожидание существовало, необходимо, чтобы ряд сходился адболютно, то есть
$$ \sum \limits_{i=1}^{ \infty } \left| x_i \right| \cdot p_i < + \infty.$$

Пусть $ \xi $ имеет абсолютно непрерывное распределение (имеет плотность распределения $p_{ \xi } \left( x \right) $).
В таком случае математическое ожидание вычисляется ка
$$M \xi =
\int \limits_{- \infty }^{+ \infty } xp_{ \xi } \left( x \right) dx.$$

В случае, если имеем функцию от $ \xi $, математическое ожидание
$$M \varphi \left( x \right) =
\int \limits_{- \infty }^{+ \infty } \varphi \left( x \right) p_{ \xi } \left( x \right) dx.$$

\subsubsection*{Приведите определение дисперсии случайной величины и запишите формулу для её вычисления.}

Дисперсия случайной величины $ \xi $, для которой $ \exists M \xi $,
называется
$$D \xi =
M \left[ \left( \xi - M \xi \right)^2 \right] \leq + \infty.$$
Это мера отклонения случайной величины от своего среднего (математического ожидания).

Для вычисления пользуемся формулой $D \xi = M \xi^2 - \left( M \xi \right)^2.$

\subsubsection*{Сформулируйте свойства математического ожидания и дисперсии.}

Сфойства математического ожидания:
\begin{enumerate}
\item если $ \exists M$, то $ \forall c \in \mathbb{R} \, \exists M \left( c \xi \right) $ и $ M \left( c \xi \right) = cM \xi $;
\item если существует $M \xi $ и $M \eta $, то $ \exists M \left( \xi + \eta \right) $ и $M \left( \xi + \eta \right) = M \xi + M \eta $.
Из этих двух условий следует, что математическое ожидание является линейной функцией;
\item $ \exists M \xi \iff \exists M \left| \xi \right| $, и кроме того $ \left| M \xi \right| \leq M \left| \xi \right| $;
\item если $ \xi \geq 0$, то $M \xi \geq 0$;
\item $\xi \geq \eta, \, \exists \xi, \, \exists \eta \Rightarrow M \xi \geq M \eta $;
\item если $ \eta $ и $ \xi $ --- независимые случайные величины,
для которых существует математическое ожидание, то $ \exists M \left( \xi \eta \right) = M \xi \cdot M \eta $.
\end{enumerate}

Свойства дисперсии:
\begin{enumerate}
\item $D \xi \geq 0$'
\item $ \forall \lambda \in \mathbb{R}: \qquad D \left( \lambda \xi \right) = \lambda^2 D \xi $;
\item для независимых $ \eta $ и $ \xi \, D \left( \xi + \eta \right) = D \xi + D \eta $.
\end{enumerate}

\addcontentsline{toc}{section}{Аудиторные задачи}
\section*{Аудиторные задачи}

\subsubsection*{10.3}

\textit{Задание.}
Случайная величина $ \xi $ имеет дискретное распределение:
$$P \left( \xi = 0 \right) = 0.2,
P \left( \xi = 1 \right) = 0.3,
P \left( \xi = 2 \right) = 0.5.$$
Вычислите: $M \xi, D \xi, M \xi^{10}, M \left( 2 \xi + 1 \right), M \left( \xi - 1 \right)^2$.

\textit{Решение.} Есть дискретная случайная величина, у неё есть 3 значения.
Поэтому $M \xi = 0 \cdot 0.2 + 1 \cdot 0.3 + 2 \cdot 0.5 = 1.3$.

Дисперсия $D \xi = M \xi^2 - \left( M \xi \right)^2$.

Вычислим первое слагаемое $M \xi^2 = 0 \cdot 0.02 + 1 \cdot 0.3 + 4 \cdot 0.5 = 2.3$.

Подставим $D \xi = 2.3 - 1.69 = 0.61$.

Найдём математическое ожидание $M \xi^{10} = 0 \cdot 0.2 + 1 \cdot 0.3 + 1024 \cdot 0.5 = 512.3$.

Воспользуемся линейностью математического ожидания
$$M \left( 2 \xi + 1 \right) =
2 M \xi + 1 =
3.6.$$

Раскроем скобки в следующем выражении
$$M \left( \xi - 1 \right)^2 =
M \xi^2 - 2 M \xi + 1 =
2.3 - 2.6 + 1 =
0.7.$$

\addcontentsline{toc}{section}{Дополнительные задачи}
\section*{Дополнительные задачи}

\addcontentsline{toc}{section}{Домашнее задание}
\section*{Домашнее задание}
