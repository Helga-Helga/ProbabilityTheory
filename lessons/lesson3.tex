\addcontentsline{toc}{chapter}{Занятие 3. Классическое определение вероятности}
\chapter*{Занятие 3. Классическое определение вероятности}

\addcontentsline{toc}{section}{Контрольные вопросы и задания}
\section*{Контрольные вопросы и задания}

\subsubsection*{Приведите определение вероятностного эксперимента, вероятностного пространства, случайного события.}

Вероятностным экспериментом называется явление, исход которого для нас не определён, и который можно повторить любое число раз независимым образом.

Вероятностное пространство --- совокупность всех исходов вероятностного эксперимента, $\Omega$.

Случайное событие --- подмножество всех исходов вероятностного эксперимента.

\subsubsection*{Как вычислить вероятность события в случае, когда вероятностное пространство состоит из конечного количества равновероятных элементарных событий?}

В этом случае вероятность любого события A вычисляется по формуле
$$ P(A) =
\frac{ |A| }{ |\Omega| },$$
называемой классическим определением вероятности.

\subsubsection*{Запишите формулу включений и исключений.}

Если
$ B_1, B_2,  \dotsc , B_m $ --- некоторые события, то имеет место равенство
$ P\{ \bigcup \limits_{i=1}^m B_i \} =
\sum \limits_{i=1}^m P \{ B_i \} -
\sum \limits_{ 1 \leq i_1 < i_2 \leq m } \{ B_{ i_1 } \cap B_{ i_2 } \} + \dotsc + \\
+ (-1)^{k-1} \sum \limits_{ 1 \leq i_1 < i_2 < \dotsc < i_k \leq m } P \{ B_{ i_1 } \cap B_{ i_2 } \cap \dotsc \cap B_{ i_k } \} + \dotsc + \\
+ (-1)^{m-1} P \{ B_1 \cap B_2 \cap \dotsc \cap B_m \}.$

\addcontentsline{toc}{section}{Аудиторные задачи}
\section*{Аудиторные задачи}

\subsubsection*{3.3}

\textit{Задание.} Пусть A и B такие события, что
$$P \left( A \cap B \right) = \frac{1}{4},
P \left( \overline{A} \right) = \frac{1}{3},
P \left( B \right) = \frac{1}{2}. $$
Вычислите $P \left( A \cup B \right)$.

\textit{Решение.} Вычислим вероятность события A:
$$P \left( A \right) =
1 - P \left( \overline{A} \right) =
1 - \frac{1}{3} =
\frac{2}{3}.$$

Так как события независимы, то вероятность пересечения событий равна произведению пересечений, т.е
$P \left( A \cap B \right) = P \left( A \right) P \left( B \right) $.

Вычислим вероятность объединения:
$$P \left( A \cup B \right) =
P \left( A \right) + P \left( B \right) - P \left( A \cap B \right) =
\frac{2}{3} + \frac{1}{2} - \frac{1}{4} =
\frac{8+6-3}{12} =
\frac{11}{12}.$$

\subsubsection*{3.4}

\textit{Задание.} Было подброшено три монеты.
Найдите вероятности событий:
\begin{enumerate}[label=\alph*)]
\item A = {хотя бы одна из монет выпала гербом};
\item B = {выпало ровно два герба};
\item C = {выпало не меньше двух гербов}.
\end{enumerate}

\textit{Решение.} На первой монете может появиться или герб, или цифра.
Аналогичные два элементарных исхода возможны при подбрасывании остальных монет.
Таким образом, общее число возможных элементарных исходов испытания равно $2 \cdot 2 \cdot 2 = 8$.
Эти исходы равновероятны.

\begin{enumerate}[label=\alph*)]
\item Рассмотрим событие $ \overline{A} = $ \{все монеты выпали решкой\} = \{ЦЦЦ\}.
Мощность этого события равна $\left| \overline{A} \right| = 1$.
Таким образом, вероятность события $ \overline{A} $ будет
$$P \left( A \right) =
\frac{ \left| \overline{A} \right| }{ \left| \Omega \right| } =
\frac{1}{8}.$$
Тогда вероятность необходимого события
$$P \left( A \right) =
1 - P \left( \overline{A} \right) =
1 - \frac{1}{8} =
\frac{7}{8};$$
\item благоприятствующим интересующему нас событию (выпало ровно два герба) являются следующие исходы: ГГЦ, ГЦГ, ЦГГ.

Искомая вероятность равна
$$P \left( B \right) =
\frac{3}{8};$$

\item опишем событие С = \{выпало не меньше двух гербов\} = $B \cup $ \{ГГГ\}.

Искомая вероятность равна
$$P \left( C \right) =
\frac{4}{8} =
\frac{1}{2}.$$
\end{enumerate}

\subsubsection*{3.5}

\textit{Задание.} Подброшено 12 игральных кубиков.
Найдите вероятность того, что каждое из очков $1, 2, \dotsc, 6$ выпало дважды.

\textit{Решение.}
На выпавшей грани <<первого>> игрального кубика может появиться одно очко, два очка, $ \dotsc $, шесть очков.
Аналогичные шесть элементарных исходов возможны при бросании остальных кубиков.
Таким образом, общее число возможных элементарных исходов испытания равно $6 \cdot 6 \cdot \dotsc \cdot 6 = 6^{12}$.
Эти исходы в силу симметрии кубиков равновероятны.

Благоприятствующим интересующему нас событию
(каждое из очков выпадет дважды)
является $C_{12}^2 \cdot C_{10}^2 \cdot C_8^2 \cdot C_6^2 \cdot C_4^2 \cdot C_2^2$ исходов.

Искомая вероятность равна отношению числа исходов, благоприятствующих событию,
к числу всех возможных элементарных исходов:
$$P =
\frac{C_{12}^2 \cdot C_{10}^2 \cdot C_8^2 \cdot C_6^2 \cdot C_4^2 \cdot C_2^2}{6^{12}}.$$

\subsubsection*{3.6}

\textit{Задание.}
Игральный кубик изготовлено так,
что вероятность выпадения каждой грани пропорциональна количеству очков, изображённой на этой грани.
Вычислите вероятность того, что при подбрасывании такого игрального кубика выпадет чётное количество очков.

\textit{Решение.} Вероятность выпадения единицы пропорциональна единице, т.е. равна n, где n --- любое натуральное число.
Вероятность выпадения двойки равна $2 n$, тройки --- $3 n$, четвёрки --- $4 n$, пятёрки --- $5 n$ и шестёрки --- $6 n$.
Вероятность выпадения хоть какого-то числа равна единице, т.е. сумма вероятностей выпадения всех граней равна единице: $n + 2 n + 3 n + 4 n + 5 n + \\ + 6 n = 21 n = 1$.
Отсюда
$$n =
\frac{1}{21}.$$

Интересующие нас события: выпадет 2 очка, 4 очка или 6 очков.
2 очка может выпасть с вероятностью
$$ P \left( 2 \right) =
2 \cdot \frac{1}{21} =
\frac{2}{21}.$$
4 очка может выпасть с вероятностью
$$ P \left( 4 \right) =
4 \cdot \frac{1}{21} =
\frac{4}{21}.$$
6 очков может выпасть с вероятностью
$$ P \left( 6 \right) =
6 \cdot \frac{1}{21} =
\frac{6}{21}.$$

Вероятность выпадения чётного числа равна сумме вероятностей их выпадения:
$$P \left( 2, 4, 6 \right) =
\frac{2}{21} + \frac{4}{21} + \frac{6}{21} =
\frac{12}{21}.$$

\subsubsection*{3.7}

\textit{Задание.} Найдите вероятность того, что наугад выбранное число из множества $\{ 1, 2, \dotsc, 100!\}$ делится:
\begin{enumerate}[label=\alph*)]
\item на 2;
\item на 2 и на 3;
\item хотя бы на одно из чисел 2, 3, или 5.
\end{enumerate}

\textit{Решение.} Всего чисел $100!$.
\begin{enumerate}[label=\alph*)]
\item На 2 делится каждое второе число, т.е. количество чисел, которые делятся на 2, равно
$$\frac{100!}{2}.$$
Вероятность выбора чётного числа равна
$$P =
\frac{ \frac{100!}{2} }{100!} =
\frac{1}{2};$$

\item на 2 и на 3 делится каждое шестое число, т.е. их всего
$$ \frac{100!}{6}.$$
Тогда вероятность выбора числа, которое делится и на 2, и на 3, равна
$$P =
\frac{ \frac{100!}{6} }{100!} =
\frac{1}{6};$$

\item на 2 делится каждое второе число, на 3 --- каждое третье, на 5 --- каждое пятое.
Используем формулу включений-исключений:
$$P =
\frac{1}{2} + \frac{1}{3} + \frac{1}{5} - \frac{1}{6} - \frac{1}{10} - \frac{1}{15} + \frac{1}{30} =
\frac{15+10+6-5-3-2+1}{30} =
\frac{22}{30} =
\frac{11}{15}.$$
\end{enumerate}

\subsubsection*{3.8}

\textit{Задание.} В лифт девятиэтажного дома зашло пятеро человек.
Известно, что каждый из них с одинаковой вероятностью может выйти на любом этаже, начиная со второго.
Найдите вероятность того, что:
\begin{enumerate}[label=\alph*)]
\item все пятеро выйдут на 5-м этаже;
\item все пятеро выйдут на одном и том же этаже;
\item все пятеро выйдут на разных этажах;
\item двое людей выйдут на 4-м этаже, двое --- на 7-м и один человек выйдет на 9-м этаже.
\end{enumerate}

\textit{Решение.} Каждый из пяти человек может выйти на любом из восьми этажей.
Тогда всего есть $8 \cdot 8 \cdot 8 \cdot 8 \cdot 8 = 8^5$ вариантов.

\begin{enumerate}[label=\alph*)]
\item Все пять человек могут выйти на пятом этаже одним способов.
Тогда вероятность этого события равна
$$P =
\frac{1}{8^5};$$

\item все пять человек могут выйти на одном и том же этаже восемью способами.
Тогда вероятность этого события равна
$$P =
\frac{8}{8^5} =
\frac{1}{8^4};$$

\item первый человек может выйти на любом из восьми этажей, второй --- на любом из оставшихся семи,
третий --- на любом из оставшихся шести, четвёртый --- пяти, и наконец пятый --- на оставшихся четырёх этажах.
Тогда вероятность равна
$$P =
\frac{8 \cdot 7 \cdot 6 \cdot 5 \cdot 4}{8^5} =
\frac{7 \cdot 6 \cdot 5 \cdot 4}{8^4};$$

\item на четвёртом этаже могут выйти 2 человека из пяти $ \left( C_5^2 \right) $.
На седьмом этаже --- двое из оставшихся трёх человек.
Это $C_3^2$.
На девятом --- один человек из одного.
Это один способ.
Тогда вероятность равна
$$P =
\frac{C_5^2 \cdot C_3^2}{8^5}.$$
\end{enumerate}

\subsubsection*{3.9}

\textit{Задание.} Из колоды карт наугад выбирают 3 карты (без возвращения).
Найдите вероятность того, что среди этих карт:
\begin{enumerate}[label=\alph*)]
\item окажется ровно одни туз;
\item окажется хотя бы один туз;
\item окажется тройка, семёрка, туз.
\end{enumerate}

\textit{Решение.} Всего в колоде 52 карты.
Тогда есть 4 масти по 13 карт в каждой.
3 карты из 36 можно выбрать $C_{52}^3$ способами.

\begin{enumerate}[label=\alph*)]
\item В колоде есть 4 туза.
Один из них можно выбрать $C_4^1$ способами.
Остаётся выбрать две карты из $52-4=48$.
Это можно сделать $C_{48}^2$ способами.
Тогда вероятность выбрать 3 карты так, что среди них окажется ровно 1 туз, равна
$$P =
\frac{C_4^1 \cdot C_{48}^2}{C_{52}^3};$$

\item $B =$ \{среди этих карт окажется хотя бы 1 туз\}.
Рассмотрим противоположное событие: $ \overline{B} =$ \{среди трёх карт нет тузов\}.
Вероятность этого события равна
$$P \left( \overline{B} \right) =
\frac{C_{48}^3}{C_{52}^3}.$$
Тогда вероятнотсь интересующего нас события равна
$$P \left( B \right) =
1 - P \left( \overline{B} \right) =
1 - \frac{C_{48}^3}{C_{52}^3};$$

\item есть 4 варианта выбрать тройку, 4 выбрать семёрку и 4 --- туз.
Первой картой может быть тройка, семёрка или туз ($4 \cdot 3 = 12$ карт из 52).
Далее должна быть одна из восьми карт (т.е. 8 вариантов из оставшихся 51 карты).
Последней картой может быть любая из четырёх (из 50 карт).
По правилу произведения имеем
$$P =
\frac{12}{52} \cdot \frac{8}{51} \cdot \frac{4}{50}.$$
\end{enumerate}

\subsubsection*{3.10}

\textit{Задание.} В партии, что состоит из $N$ деталей, есть $M$ бракованных.
Найдите вероятность того, что среди $n \, \left( n < N \right)$ наугад выбранных деталей окажется:
\begin{enumerate}[label=\alph*)]
\item $m \, \left( m < M \right) $ бракованных;
\item не больше $m$ бракованных.
\end{enumerate}

\textit{Решение.} Пространство элементарных событий $ \Omega $ является множеством n-элементных подмножеств множества из $N$ элементов.
Значит, $| \Omega | = C_N^n$.

\begin{enumerate}[label=\alph*)]
\item Посчитаем число исходов,
благоприятствующих интересующему нас событию
(среди $n$ деталей ровно $m$ бракованных):
$m$ бракованных деталей можно взять из $M$ бракованных деталей $C_M^m$ способами;
при этом остальные $n - m$ деталей не должны быть бракованными;
взять же $n - m$ не бракованных деталей из $N - M$ не бракованных деталей можно $C_{N-M}^{n-m}$ способами.
Следовательно, число благоприятствующих исходов равно $C_M^m C_{N-M}^{n-m}$.

Искомая вероятность равна отношению числа исходов, благоприятствующих событию, к числу всех элементарных исходов:
$$P =
\frac{C_M^m C_{N-M}^{n-m}}{C_N^n};$$

\item посчитаем число исходов,
благоприятствующих интересующему нас событию
(среди $n$ деталей от нуля до $m$ бракованных):
от нуля до $m$ бракованных деталей можно взять из $M$ бракованных деталей $ \sum \limits_{i=0}^m C_M^i$ способами;
при этом остальные от $n - m$ до $n$ деталей не должны быть бракованными;
взять же от $n - m$ до $n$ не бракованных деталей из $N - M$ не бракованных деталей можно $ \sum \limits_{i=0}^m C_{N-M}^{n-i}$ способами.
Следовательно, число благоприятствующих исходов равно $ \sum \limits_{i=0}^m C_M^i C_{N-M}^{n-i}$.

Искомая вероятность равна отношению числа исходов, благоприятствующих событию, к числу всех элементарных исходов:
$$P =
\frac{ \sum \limits_{i=0}^m C_M^i C_{N-M}^{n-i}}{C_N^n}.$$
\end{enumerate}

\subsubsection*{3.11}

\textit{Задание.} В купейный вагон (9 купе по 4 места в каждом) семи пассажирам продано семь билетов.
Найдите вероятность того, что занятыми окажутся:
\begin{enumerate}[label=\alph*)]
\item ровно два купе;
\item ровно три купе.
\end{enumerate}

\textit{Решение.}
Любой из семи пассажиров может занять любое из четырёх мест в любом из девяти купе (всего мест $4 \cdot 9 = 36$).
Тогда общее число элементарных исходов --- $C_{36}^7$.

\begin{enumerate}[label=\alph*)]
\item Опишем событие $A = $ \{заняты 2 купе\}.
Нужно выполнить 4 действия.
Первое действие: выбрать купе, где будет 4 человека.
Способов сделать это $C_9^1$.
Второе действие: разместить четырёх человек на четырёх местах.
Способов сделать это --- $1$.
Третье действие: выбрать купе, где будет 3 человека.
Это можно сделать $C_8^1$ способом.
Четвёртое действие: разместить трёх человек на четырёх местах.
Это можно сделать $C_4^3$ способами.

Тогда мощность события $A$ равна $ \left| A \right| = C_9^1 \cdot 1 \cdot C_8^1 \cdot C_4^3$.

Вероятность этого события равна
$$P \left( A \right) =
\frac{ \left| A \right| }{ \left| \Omega \right| } =
\frac{C_9^1 \cdot 1 \cdot C_8^1 \cdot C_4^3}{C_{36}^7};$$

\item Рассмотрим событие $B =$ \{занято 3 купе\}.
Тогда возможны такие ситуации:
\begin{itemize}
\item в одном купе находится 4 человека, в другом --- 2, и в оставшемся --- 1;
\item в двух купе находится по 3 человека, и в оставшемся --- 1 человек;
\item в двух купе находится по 2 человека, и в оставшемся --- 3 человека.
\end{itemize}

Тогда мощность этого события равна
$ \left| B \right| =
\left( C_9^1 C_4^4 \right) \left(C_3^1 C_4^2 \right) \left( C_7^1 C_4^1 \right) + \\
+ \left( C_9^2 \left( C_4^3 \right)^2 \right) \left( C_7^1 C_4^1 \right) +
\left( C_9^2 \left( C_4^2 \right)^2 \right) \left( C_7^1 C_4^3 \right)$.

Вероятность такого события равна
$$P \left( B \right) =
\frac{ \left| B \right| }{ \left| \Omega \right| }.$$
\end{enumerate}

\subsubsection*{3.12}

\textit{Задание.} Шутник Петя написал письма $n$ адресатам, в каждый конверт вложил по одному письму, а потом наугад написал на каждом из конвертов один из $n$ адресов.
Найдите вероятность того, что:
\begin{enumerate}[label=\alph*)]
\item хотя бы одно письмо придёт по назначению;
\item ровно $m$ писем придут по назначению.
\end{enumerate}

\textit{Решение.} Разложить $n$ писем в $n$ конвертов можно $n!$ способами.

\begin{enumerate}[label=\alph*)]
\item Найдём вероятность того, что ни одно письмо не попадёт в конверт с правильным адресом, а затем вычтем её из $1$.

Из общего числа способов необходимо вычесть число тех вариантов, при которых первое письмо попадает в 1-й конверт,
все способы, при которых второе письмо попадает во 2-й конверт и т.д.

Письмо, которое будет вложено в конверт с правильным адресом, можно выбрать $n$ способами;
остальные $n - 1$ письмо можно вложить в $n - 1$ конверт $ \left( n - 1 \right)!$ способами,
поэтому общее число вариантов размещения писем по конвертам равно $n \cdot \left( n - 1 \right)! = n!$.
Вычитая это число из общего числа возможных вариантов размещения писем по конвертам, равного $n!$,
мы не оставляем ни одного варианта.
Вариант, в котором, например, первое письмо попадает в 1-й конверт, а второе письмо --- во 2-й, мы вычитаем дважды.

Чтобы найти, сколько вариантов мы вычли слишком большое число раз, заметим, что существует 
$$C_n^2 =
\frac{n!}{2! \left( n-2 \right)!} =
\frac{n \left( n-1 \right) }{2!}$$
пар писем, и если письма, образующие пару, вложены в конверты с правильными адресами, то остальные $n - 2$ письма можно распределить по конвертам
$$\frac{n \left( n-1 \right)}{2!} \cdot \left( n-2 \right)! =
\frac{n!}{2!}$$
способами.

Прибавив число способов распределения писем в конверты, при которых два письма вложены в свои конверты, мы получим всего
$$n! - n! + \frac{n!}{2!}$$
вариантов размещения писем по конвертам.

Но теперь это слишком много,
так как все варианты, при которых в свои конверты вложены три письма, не были учтены
(мы вычли число таких вариантов трижды, а затем прибавили его столько раз,
сколько пар писем можно образовать из трёх писем, т.е. три раза).
Следовательно, мы должны вычесть число способов, которыми можно вложить в конверты с правильными адресами три письма, т.е.
$$C_n^3 \cdot \left( n-3 \right)! =
\frac{n!}{\left( n-3 \right)! 3!}\cdot \left( n-3 \right)! =
\frac{n!}{3!}$$
способов.

Теперь мы вычли слишком много раз число способов,
которыми можно вложить в конверты с правильными адресами четыре письма и т.д.
Таким образом, число способов, которыми письма можно разложить по конвертам так, что ни одно письмо не окажется в конверте с правильным адресом, равно
$$n! - n! + \frac{n!}{2!} - \frac{n!}{3!} + \dotsc + \left( -1 \right)^{n+1} \left( \frac{1}{n!} \right),$$
а вероятность этого события равна этому числу, делённому на $n!$, т.е. равна числу
$$1 - 1 + \frac{1}{2!} + \dotsc + \left( -1 \right)^{n+1} \frac{1}{n!}.$$
Следовательно, вероятность того, что по крайней мере одно письмо окажется в конверте с правильным адресом, равна
$$P =
1 - \frac{1}{2!} + \frac{1}{3!} - \dotsc + \left( -1 \right)^{n+1} \frac{1}{n!};$$

\item найдём вероятность того, что ровно $m$ писем попадут в конверты с правильными адресами.

Всего вариантов расстановки есть $n!$.

Письма, которые будут вложены в конверты с правильными адресами,
можно выбрать $C_n^m$ способами;
остальные $n - m$ писем можно вложить в $n - m$ конвертов $ \left( n-m \right)!$ способами.

От этого значения нужно отнять число вариантов расстановки, при котором $ \left( m+1 \right) $ письмо попало по назначению.
Для этого выбираем $ \left( m+1 \right) $-но письмо, которое попало но назначению, остальные переставляем произвольным образом.
Получаем
$$C_n^m \cdot \left( n-m \right)! - C_n^{m+1} \left( n-m-1 \right)!.$$

К этому значению нужно прибавить количество способов, при котором $ \left( m+2 \right) $ письма попало по назначению.
Получаем
$$C_n^m \cdot \left( n-m \right)! - C_n^{m+1} \left( n-m-1 \right)! + C_n^{m+2} \cdot \left( n-m-2 \right)!.$$

И так далее
\begin{equation*}
\begin{split}
C_n^m \cdot \left( n-m \right)! - C_n^{m+1} \left( n-m-1 \right)! + C_n^{m+2} \cdot \left( n-m-2 \right)! - \dotsc = \\
= \frac{n!}{m! \left( n-m \right)!} \cdot \left( n-m \right)! -
\frac{n!}{ \left( m+1 \right)! \left( n-m-1 \right)!} \cdot \left( n-m-1 \right)! + \\
+ \frac{n!}{ \left( m+2 \right)! \left( n-m-2 \right)!} \cdot \left( n-m-2 \right)! -
\dotsc = \\
= \frac{n!}{m!} - \frac{n!}{ \left( m+1 \right)!} + \frac{n!}{ \left( m+2 \right)!} - \dotsc.
\end{split}
\end{equation*}

Следовательно, вероятность того, что ровно $m$ писем окажется в конвертах с правильными адресами, равна
$$P =
\frac{1}{m!} - \frac{1}{ \left( m+1 \right)!} + \frac{1}{ \left( m+2 \right)!} - \dotsc $$
--- поделили на общее число способов подписать $n$ писем $n$ адресами --- $n!$.

\paragraph*{Число встреч}

В задаче используется понятие <<число встреч>>.
Под ним понимается число перестановок множества $\{ 1, \dotsc, n \}$ с заданным числом неподвижных элементов.
Для $n \geq 0$ и $0 \leq k \leq n$ число встреч $D_{n, k}$ --- это число перестановок $\{ 1, \dotsc, n \}$,
содержащих ровно $k$ элементов, не изменивших положение в перестановке.

\begin{figure}[h!]
  \centering
  \includegraphics[width=.7\textwidth]{./pictures/3_12.png}
  \caption{Фрагмент таблицы числа встреч}
  \label{fig:312}
\end{figure}

Числа в первом столбце $ \left( k=0 \right) $ показывают число беспорядков (рис. \ref{fig:312}).
Так,
$D_{0, 0} = 1, D_{1, 0} = 0, D_{n+2, 0} = \left( n+1 \right) \left( D_{n+1, 0} + D_{n, 0} \right)$ для неотрицательного $n$.

Выберем $m$ фиксированных элементов из $n$, затем посчитаем число беспорядков оставшихся $n - m$ элементов.
Это будет
$$ \left( n-m \right)! \sum \limits_{k=0}^{n-m} \frac{ \left( -1 \right)^k}{k!}.$$

$$D_{n ,m} =
C_n^m D_{n-m, 0} =
\frac{n!}{m!} \sum \limits_{k=0}^{n-m} \frac{ \left( -1 \right)^k}{k!}.$$
\end{enumerate}

\subsubsection*{3.13}

\textit{Задание.} (Статистика Максвелла-Больцмана). Каждая из $n$ разных частиц наугад попадает в одну из $N$ ячеек.
\begin{enumerate}[label=\alph*)]
\item Найдите вероятность того, что в первой, второй и т.д., N-ой ячейке будет соответственно $n_1, n_2, \dotsc, n_N$ частиц;
\item найдите вероятность $p_k$ того, что данная ячейка содержит $k$ частиц;
\item *докажите, что если $n$ и $N$ стремятся к бесконечности так, что
$$ \frac{n}{N} \rightarrow \lambda,$$
то
$$p_k \rightarrow \frac{ \lambda^k}{k!}e^{- \lambda};$$
\item найдите вероятность того, что в каждой ячейке есть хотя бы одна частица;
\item найдите вероятность того, что занято ровно $r$ ячеек.
\end{enumerate}

\textit{Решение.} В случаях а), б) и д) вероятность желаемого события оценивается по формуле
$$p =
\frac{r}{m}.$$
Для каждой из $n$ разных частиц есть $N$ вариантов попасть в одну из ячеек.
Поэтому по правилу умножения имеем общее количество вариантов $m = N^n$.
Вычислим количество $r$ вариантов, что отвечают каждому из событий, указанных в условии задачи.

\begin{enumerate}[label=\alph*)]
\item Существует $C_n^{n_1}$ способов, которыми можно отобрать $n_1$ частицу в первую ячейку.
Аналогично, количество способов отобрать $n_2$ частицы во вторую ячейку среди $n - n_1$ частицы, которые остались, равно $C_{n-n_1}^{n_2}$.
Подобным же образом отбираются частицы и для других ячеек.
Поэтому событию способствуют
\begin{equation*}
\begin{split}
r =
C_n^{n_1} \cdot C_{n-n_1}^{n_2} \cdot C_{n - n_1 - n_2}^{n_3} \cdot \dotsc = \\
= \frac{n!}{n_1! \left( n-n_1 \right)!}
\cdot \frac{ \left( n-n_1 \right)!}{n_2! \left( n - n_1 - n_2 \right)!}
\cdot \frac{ \left( n - n_1 - n_2 \right)!}{n_3! \left( n - n_1 - n_2 - n_3 \right)!} \cdot \dotsc = \\
= \frac{n!}{n_1! n_2! n_3! \dotsc n_N!}
\end{split}
\end{equation*}
способов.

Тогда вероятность равна
$$p =
\frac{\frac{n!}{n_1! n_2! n_3! \dotsc n_N!}}{N^n} =
\frac{n!}{N^n \cdot n_1! n_2! n_3! \dotsc n_N!};$$

\item сначала нужно отобрать $k$ частиц в фиксированную ячейку
(это можно сделать $C_n^k$ способами),
а потом $n - k$ частиц распределить среди $N - 1$ ячейки
($ \left( N - 1\right)^{n-k}$ способов).
По правилу умножения имеем $r = C_n^k \cdot \left( N-1 \right)^{n-k}$ способов.

Тогда вероятность равна
$$p_k =
\frac{C_n^k \cdot \left( N-1 \right)^{n-k}}{N^n}ж$$

\item *распишем $p_k$:
\begin{equation*}
\begin{split}
p_k =
\frac{C_n^k \left( N-1 \right)^{n-k}}{N^n} =
\frac{n! \left( N-1 \right)^{n-k}}{k! \left( n-k \right)! \cdot N^n} = \\
= \frac{n! \left( N-1 \right)^n}{k! \left( n-k \right)! \cdot N^n \left( N-1 \right)^k} =
\frac{n!}{k! \left( n-k \right)! \left( N-1 \right)^k} \cdot \left( \frac{N-1}{N} \right)^n = \\
= \frac{ \left( n-k \right)! \left( n-k+1 \right) \dotsc n}{k! \left( n-k \right)! \left( N-1 \right)^k} \cdot \left(\frac{N-1}{N} \right)^n =
\frac{ \prod \limits_{i=0}^{k-1} \left( n-i \right) }{k! \left( N-1 \right)^k} \cdot \left( \frac{N-1}{N} \right)^n = \\
= \frac{ \prod \limits_{i=0}^{k-1} \left( \frac{n-i}{N-1} \right)}{k!} \cdot \left( 1 - \frac{1}{N} \right)^n.
\end{split}
\end{equation*}

Умножим и поделим последнюю скобку на $n$.
Получим:
\begin{equation*}
\begin{split}
p_k =
\frac{ \prod \limits_{i=0}^{k-1} \left( \frac{n-i}{N-1} \right)}{k!} \cdot \left( 1 - \frac{ \frac{n}{N} }{n} \right)^n.
\end{split}
\end{equation*}

Возьмём предел полученного выражения при 
$$ \frac{n}{N} \rightarrow \lambda.$$
Получим:
$$ \lim \limits_{ \frac{n}{N} \rightarrow \lambda } p_k =
\lim \limits_{ \frac{n}{N} \rightarrow \lambda }
\left( \frac{ \prod \limits_{i=0}^{k-1} \left( \frac{n-i}{N-1} \right)}{k!} \cdot \left( 1 - \frac{ \frac{n}{N} }{n} \right)^n \right) =
\frac{ \lambda^k}{k!} \exp \left( - \lambda \right);$$

\item вероятность соответственного события вычисляется с помощью формулы включений и исключений:
$P \left\{ \bigcup \limits_{i=1}^m B_i \right\} =
\sum \limits_{i=1}^m P \left\{ B_i \right\} - \\
- \sum \limits_{1 \leq i_1 < i_2 \leq m} P \left\{ B_{i_1} \cap B_{i_2} \right\} + \dotsc + \\
+ \left( -1 \right)^{k-1} \sum \limits_{1 \leq i_1 < i_2 < \dotsc < i_k \leq m} P \left\{ B_{i_1} \cap B_{i_2} \cap \dotsc \cap B_{i_k} \right\} + \dotsc + \\
+ \left( -1 \right)^{m-1} P \left\{ B_1 \cap B_2 \cap \dotsc \cap B_m \right\}$.
Обозначим через $B_i$ событие <<ячейка $i$ не содержит частиц>>.
Тогда $p = 1 - P \left\{ \bigcup \limits_{i=1}^N B_i \right\}$.
Для любых $i = 1, \dotsc, N, \, 1 \leq \\
\leq i_1 < i_2 < \dotsc i_j \leq N$
$$P \left\{ B_i \right\} =
\frac{ \left( N-1 \right)^m}{N^m}, \,
P \left\{ B_{i_1} \cap B_{i_2} \cap \dotsc \cap B_{i_j} \right\} =
\frac{ \left( N-j \right)^m}{N^m}, \,
j = 2, 3, \dotsc, N-1$$
($j$ ячеек являются пустыми; $m$ разных частиц можно разместить в $N - j$ ячейках $ \left( N-j \right)^m$ способами).
Количество слагаемых в соответствующей сумме равно $C_n^j$.
Согласно с формулой включений и исключений имеем окончательный ответ
$$p =
1 - \frac{1}{N^m} \cdot \left[ C_N^1 \cdot \left( N-1 \right)^m - C_N^2 \cdot \left( N-2 \right)^m + \dotsc + \left( -1 \right)^N \cdot C_N^{N-1} \cdot 1^m \right];$$

\item решение задачи разбиваем на два шага: сначала выбираем $r$ ячеек из $N$, которые будут занятыми (это можно сделать $C_N^r$ способами),
а потом размещаем $n$ разных частиц в $r$ ячеек таким образом, чтобы ни одна из них не была пустой ( подобная задача решена в пункте г) по формуле включений и исключений).
Поэтому окончательно имеем
$r =
C_N^r \times \\
\times \left[ r^n - C_r^1 \cdot \left( r-1 \right)^n + C_r^2 \cdot \left( r-2 \right)^n - \dotsc + \left( -1 \right)^{r-1} \cdot C_r^{r-1} \cdot 1^n \right]$.
\end{enumerate}

\addcontentsline{toc}{section}{Дополнительные задачи}
\section*{Дополнительные задачи}

\subsubsection*{3.14}

\textit{Задание.} Каждое из двух чисел $a, b$ наугад выбирают из множества $ \left\{ 1, 2, \dotsc, n \right\}$.
Вычислите вероятность того, что одно из них делится на другое.
Найдите предел этой вероятности при $n \rightarrow \infty$.

\textit{Решение.} Вероятность будет равна
$$P =
\frac{k}{m},$$
где $k$ --- сумма количества делителей каждого числа из множества  $ \left\{ 1, 2, \dotsc, n \right\}$, $m$ --- количество разных пар чисел.

Найдём $m$.
Это будет число комбинаций из $n$ по $2$: $C_n^2$.

Найдём количество множителей натурального числа.
Пусть
$n =
p_1^{ \alpha_1} \cdot p_2^{ \alpha_2} \cdot \\
\cdot \dotsc \cdot p_s^{ \alpha_s}$
--- каноническое разложение на простые множители натурального числа $n$.
Тогда число $ \tau \left( n \right) $ натуральных делителей числа $n$ выражается формулой
$ \tau \left( n \right) =
\left( \alpha_1 + 1 \right) \dotsc \left( \alpha_s + 1 \right)$.

Каждый натуральный делитель $d$ числа $n$
может быть записан в виде
$d = p_1^{ \delta_1} p_2^{ \delta_2} \dotsc p_s^{ \delta_s}$, где $ \delta_i$ ---
целые числа, удовлетворяющие условиям
$ \delta_i \in  \\
\in \left\{ 0, 1, \dotsc, \alpha_i \right\}$
для $i = 1, 2, \dotsc, s$.

Докажем это утверждение.
Пусть $d$ есть какой-либо натуральный делитель $n$.
Так как каждый простой делитель числа $d$
является делителем числа $n$, то ввиду
$n =
p_1^{ \alpha_1} \cdot p_2^{ \alpha_2} \cdot \dotsc \cdot p_s^{ \alpha_s}$
в разложении $d$ на простые множители могут встречаться только числа множества $ \left\{ p_1, \dotsc, p_s \right\}$.
Поэтому число $d$ представимо в виде $d = p_1^{ \delta_1} p_2^{ \delta_2} \dotsc p_s^{ \delta_s}$, где $ \delta_i$,
причём показатели $ \delta_i$ должны удовлетворять условиям
$ \delta_i \in \left\{ 0, 1, \dotsc, \alpha_i \right\}$ для $i = 1, 2, \dotsc, s$.

С другой стороны, если $d$ представимо в виде
$d =
p_1^{ \delta_1} p_2^{ \delta_2} \dotsc p_s^{ \delta_s}$
и показатели удовлетворяют условиям
$ \delta_i \in \left\{ 0, 1, \dotsc, \alpha_i \right\}$
для $i = 1, 2, \dotsc, s$, то
$n =
d \left( p_1^{ \alpha_1 - \delta_1} \dotsc p_s^{ \alpha_s - \delta_s} \right) \left( \alpha_i - \delta_i \geq 0 \right)$,
т.е. $d$ является натуральным делителем числа $n$.

Чтобы найти число всех натуральных делителей числа $n$,
достаточно посчитать число всевозможных упорядоченных наборов
$ \delta_1, \dotsc, \delta_s$, удовлетворяющих условиям
$ \delta_i \in \left\{ 0, 1, \dotsc, \alpha_i \right\}$ для $i = 1, 2, \dotsc, s$.
Ввиду условий $ \delta_i$ может принимать $ \alpha_i + 1$ значение,
выборы различных значений $ \delta_1, \dotsc, \delta_s$
не зависят один от другого
и в силу единственности разложения на простые множители разным наборам соответствуют различные делители $n$.
Следовательно, число всех натуральных делителей числа $n$ равно
$ \left( \alpha_1 + 1 \right) \dotsc \left( \alpha_s + 1 \right)$.

Можно оценить количество делителей числа.
В терминах о-малое,
функция делителей удовлетворяет неравенству для всех $ \epsilon > 0, \, d \left( n \right) = o \left( n^{\epsilon} \right)$.

Тогда имеем вероятность
$$P =
\frac{ \sum \limits_{i=1}^n i^{\epsilon}}{C_n^2} \, \forall \epsilon > 0.$$

\addcontentsline{toc}{section}{Домашнее задание}
\section*{Домашнее задание}

\subsubsection*{3.15}

\textit{Задание.} Докажите, что для произвольных событий A и B
$ P( AB ) = \\
= 1 - P\left( \overline{ A } \right) - P \left( \overline{ B } \right) + P \left( \overline{ A } \, \overline{ B } \right) $.

\textit{Решение.}
$ P( AB ) =
P(A) + P(B) - P \left( A \cup B \right) = \\
= 1 - P \left( \overline{ A }\right) + 1 - P \left( \overline{ B } \right) - \left( 1 - P \left( \overline{ A \cup B } \right) \right) =
1 - P \left( \overline{ A } \right) + 1 - P \left( \overline{ B } \right) - 1 + P \left( \overline{ A } \, \overline{ B } \right) = \\
= 1 - P \left( \overline{ A } \right) - P \left( \overline{ B } \right) + P \left( \overline{ A } \, \overline{ B } \right) $.

\subsubsection*{3.16}

\textit{Задание.} Подбрасывают 4 игральных кубика.
Найдите вероятность того, что на них выпадет одинаковое количество очков.

\textit{Решение.} На выпавшей грани <<первого>> игрального кубика может появиться одно очко, два очка,  $ \dotsc $ , шесть очков.
Аналогичные шесть элементарных исходов возможны при бросании остальных кубиков.
Таким образом, общее число возможных элементарных исходов испытания равно $ 6 \cdot 6 \cdot 6 \cdot 6 = 1296 $.
Эти исходы в силу симметрии кубиков равновероятны.

Благоприятствующие интересующему нас событию
(на всех гранях появится одинаковое количество очков)
являются следующие шесть исходов
(первым записано число очков,
выпавших на <<первом>> кубике,
вторым --- число очков, выпавших на <<втором>> кубике, и т.д.): 1) 1, 1, 1, 1, 2) 2, 2, 2, 2, 3) 3, 3, 3, 3, 4) 4, 4, 4, 4, 5) 5, 5, 5, 5, 6) 6, 6, 6, 6.

Искомая вероятность равна отношению числа исходов, благоприятствующих событию, к числу всех возможных элементарных исходов:
$$ P =
\frac{6}{6^4} = \frac{1}{6^3}.$$

\subsubsection*{3.17}

\textit{Задание.} Группа состоит из r студентов.
Найдите вероятность того, что по крайней мере 2 студента родились в одном и том же месяце (считайте, что все месяцы года являются равновероятными для рождения).

\textit{Решение.} Год имеет 12 месяцев.
День рождения каждого из студентов может приходиться на любой месяц года (12 вариантов).

Если студентов больше чем месяцев ($ r > 12 $), то хотя бы в одном месяце родилось хотя бы 2 человека:
$ P = 1 $.

Рассмотрим случай, когда студентов в группе не больше чем месяцев в году ($ r \leq 12 $).
По правилу умножения существует всего $ n = 12^r $ вариантов размещения дней рождения студентов.
Найдём количество вариантов, когда никакие два студента не имеют день рождения в тот же месяц.
Для этого нужно вычислить количество способов, которыми из 12 месяцев можно выбрать упорядоченное множество из r месяцев.
Используя формулу для размещения из 12 элементов по r, имеем $ A_{12}^r $.
Поэтому
$$ P =
1 - \frac{ A_{ 12 }^r }{ 12^r }.$$

\subsubsection*{3.18}

\textit{Задание.} Сколько людей должно быть в комнате, чтобы вероятность того, что хотя бы двое из присутствовавших родились в один и тот же день года, была большей чем 1/2?

\textit{Решение.} Пусть r --- число людей, и будем считать, что все дни рождения равновероятны.
Вычислим вероятность противоположного события A = \{все люди родились в разные дни\}.
Число способов, благоприятствующих этому событию --- это число размещений из 365 по r.
Всего же имеется $ n = 365^r $ возможностей распределения дней рождения.
То есть
$$ P(A) =
\frac{A_{365}^r}{365^r}.$$

Вероятность интересующего нас события $ \overline{A} $ тогда равна
$$ P \left( \overline{A} \right) =
1 - P \left( A \right) =
1 - \frac{A_{365}^r}{365^r}.$$

Вычислим вероятность $ P \left( \overline{A} \right) $ для различных значений r.
\begin{equation*}
\begin{split}
r = 5 : P \left( \overline{A} \right) =
1 - \frac{A_{365}^5}{365^5} =
1 - \frac{5! \cdot C_{365}^5}{365^5} = \\
= 1 - \frac{5! \cdot \frac{365!}{5! \cdot 360!} }{365^5} =
1 - \frac{365!}{360! \cdot 365^5} =
1 - \frac{360! \cdot 361 \cdot 362 \cdot 363 \cdot 364 \cdot 365}{360! \cdot 365^5} = \\
= 1 - \frac{361 \cdot 362 \cdot 363 \cdot 364}{365^4} =
1 - \frac{1.72 \cdot 10^{10}}{1.77 \cdot 10^{10}} =
1 - 0.97 =
0.03.
\end{split}
\end{equation*}

\begin{equation*}
\begin{split}
r = 10 :
P \left( \overline{A} \right) =
1 - \frac{A_{365}^{10}}{365^{10}} =
1- \frac{10! \cdot C_{365}^{10}}{365^{10}} = \\
= 1 - \frac{10! \cdot \frac{365!}{10! \cdot 355!}}{365^{10}} =
1 - \frac{365!}{355! \cdot 365^{10}} =
0.12.
\end{split}
\end{equation*}

\begin{equation*}
\begin{split}
r = 20 :
P \left( \overline{A} \right) =
1 - \frac{A_{365}^{20}}{365^{20}} =
1 - \frac{20! \cdot C_{365}^{20}}{365^{20}} =
1 - \frac{365!}{345! \cdot 365^{20}} =
0.41.
\end{split}
\end{equation*}

\begin{equation*}
\begin{split}
r = 22 :
P \left( \overline{A} \right) =
1 - \frac{A_{365}^{22}}{365^{22}} =
1 - \frac{365!}{343! \cdot 365^{22}} =
0.48.
\end{split}
\end{equation*}

\begin{equation*}
\begin{split}
r = 23 :
P \left( \overline{A} \right) =
1 - \frac{A_{365}^{23}}{365^{23}} =
1 - \frac{365!}{342! \cdot 365^{23}} =
0.51.
\end{split}
\end{equation*}

При
$ r = 23 $
вероятность по крайней мере одного совпадения равна
$ 0.51 > \\
> 0.5 $,
то есть
$ r = 23 $ ---
наименьшее число, удовлетворяющее условиям задачи.

\subsubsection*{3.19}

\textit{Задание.} Числа $ 1, 2,  \dotsc , n $ размещены в случайном порядке.
Найдите вероятность того, что числа:
\begin{enumerate}[label=\alph*)]
\item 1 и 2;
\item 1, 2 и 3 размещены рядом в указанном порядке.
\end{enumerate}

\textit{Решение.} Пространство элементарных событий $  \Omega$ является множеством всех перестановок множества из n элементов.
Значит $ |\Omega| = n! $.

\begin{enumerate}[label=\alph*)]
\item Поскольку числа 1 и 2 стоят рядом, то их можно рассматривать как одно число (обозначим его $ n + 1 $).
Количество возможных перестановок чисел $ \{ 3, 4,  \dotsc , n + 1 \} $ равно $ \left( n - 1 \right)! $.
Тогда вероятность данного события равна:
$$ P =
\frac{ \left( n - 1 \right)! }{ n! } =
\frac{1}{n};$$

\item поскольку числа 1, 2 и 3 стоят рядом, то их можно рассматривать как одно число (обозначим его $ n + 1 $).
Количество возможных перестановок чисел $ \{ 4, 5,  \dotsc , n + 1 \} $ равно $ \left( n - 2 \right)! $.
Тогда вероятность данного события равна:
$$ P = \frac{ \left( n - 2 \right)!}{ n! } =
\frac{1}{ n (n - 1 ) }.$$
\end{enumerate}

\subsubsection*{3.20}

\textit{Задание.} Экзамен состоит из 10 вопросов, на каждый из которых нужно дать ответ <<да>> или <<нет>>.
Найдите вероятность того, что студент правильно ответил хотя бы на $ 70 \% $ вопросов, выбирая ответ наугад.
Решите задачу, если тест состоит из 30 и из 50 вопросов.

\textit{Решение.} На каждый вопрос есть возможность ответить двумя способами (<<да>> или <<нет>>).

В случае, когда экзамен состоит из десяти вопросов, пространство элементарных событий содержит $ | \Omega | = 2^{10} $ элементов.
Найдём вероятность события А = \{студент правильно ответит хотя бы на 70\% вопросов из десяти\}.
Ответить правильно хотя бы на 70\% вопросов означает ответить правильно хотя бы на 7 вопросов из 10, т. е. событие А допускает правильный ответ на 7, 8, 9 или 10 вопросов.
На 7 вопросов правильно можно ответить $ C_{10}^7 $ способами.
На 8 вопросов правильно можно ответить $ C_{10}^8 $ способами.
На 9 --- $ C_{10}^9 $.
Аналогично, на 10 вопросов можно дать правильный ответ $ C_{10}^{10} $ способами.
По правилу суммы имеем $ |A| = C_{10}^7 + C_{10}^8 + C_{10}^9 + C_{10}^{10} = \sum \limits_{i=7}^{10} C_{10}^i $.
Тогда вероятность события A равна
$$ P \left( A \right) =
\frac{|A|}{| \Omega |} =
\frac{ \sum \limits_{i=7}^{10} C_{10}^i}{2^{10}}.$$

Аналогично находим вероятность правильного ответа на хотя бы 70\% вопросов из тридцати и пятидесяти.

Когда всего есть 30 вопросов (событие B), то 70\% от них --- это $ 30 \cdot 0.7 = 21$ вопрос.
Тогда
$$ P \left( B \right) =
\frac{ \sum \limits_{i=21}^{30} C_{30}^i}{2^{30}}.$$

Когда всего есть 50 вопросов (событие C), то 70\% от них --- это $ 50 \cdot 0.7 = 35$ вопросов.
Тогда
$$ P \left( C \right) =
\frac{ \sum \limits_{i=35}^{50} C_{50}^i}{2^{50}}.$$

\subsubsection*{3.21}

\textit{Задание.} Список из 4N участников турнира разбито на четыре равные группы.
Найдите вероятность того, что четыре самых сильных участника турнира окажутся в разных группах. 

\textit{Решение.} Найдём сначала количество вариантов, которыми N участников можно выбрать в первую группу.
Для этого достаточно из 4N участников выбрать N, т.е. $ n = C_{4N}^N $.
Выберем того самого сильного участника, который попадёт в первую группу (это можно сделать четырьмя способами).
В эту же группу нужно дополнительно выбрать $ N - 1 $ участника из $ 4N - 4 $ участников, что остались ($ C_{ 4N - 4 }^{ N - 1 } $ способов).
Отсюда следует, что четыре самых сильных участника попадут в разные группы с вероятностью
\begin{equation*}
\begin{split}
p =
\frac{4 C_{4N-4}^{N-1} \cdot 3 C_{3N-3}^{N-1} \cdot 2 C_{2N-2}^{N-1} \cdot C_{4N}^{N-1}}{C_{2N}^N C_{3N}^N C_{2N}^N C_{N}^N} = \\
\frac{6 N^3}{ \left(4N-3 \right) \left( 4N-2 \right) \left( 4N-1 \right) }.
\end{split}
\end{equation*}

\subsubsection*{3.22}

\textit{Задание.} При игре в покер игрок имеет на руках пять карт из колоды в 52 карты.
Найдите вероятность следующих комбинаций:
\begin{enumerate}[label=\alph*)]
\item <<royal flush>> (десятка, валет, дама, король и туз одной масти);
\item <<straight flush>> (пять карт подряд одной масти, но не <<royal flush>>);
\item <<four of a kind>> (четыре карты одного порядка);
\item <<full house>> (2 карты одинакового порядка и 3 карты одинакового порядка);
\item <<flush>> (пять карт одой масти, но не <<royal flush>> и не <<straight flush>>);
\item <<straight>> (пять карт подряд не все одной масти).
\end{enumerate}

\textit{Решение.} Выбрать 5 карт из 52 можно $ C_{52}^5 $ способами.

\begin{enumerate}[label=\alph*)]
\item Так как есть 4 масти, то нужно выбрать одну из них.
Поэтому вероятность выпадения комбинации <<royal flush>> равна
$$ p =
\frac{C_4^1}{ C_{52}^5 };$$

\item есть 9 комбинаций пяти карт подряд одной масти.
Так как мастей 4, то этих комбинаций $ 9 \cdot 4 = 36 $.
Поэтому вероятность комбинации <<straight flush>> равна
$$ p =
\frac{ C_{36}^1 }{ C_{52}^5 }.$$
После исключения комбинации <<royal flush>>, получим
$$ p =
\frac{ C_{36}^1 - C_4^1 }{ C_{52}^5 };$$

\item есть 4 масти, в каждой из которых $ 52 : 4 = 13 $ карт.
Есть 13 способов выбрать 4 карты одного порядка.
Так же нужно дополнительно выбрать одну любую карту из оставшихся $ 52 - 4 = 48 $ карт (это 48 способов).
Отсюда имеем, что вероятность комбинации <<four of a kind>> равна
$$ p =
\frac{ C_{13}^1 \cdot C_{48}^1 }{ C_{52}^5 };$$

\item есть карты тринадцати порядков четырёх мастей.
Две карты одного порядка могут иметь любой из тринадцати порядков и любые две из четырёх мастей (это $ C_{13}^1 \cdot C_4^2 $).
Осталось выбрать 3 карты одного порядка. Имеем теперь 12 порядков карт.
Поэтому есть $ C_{12}^1 \cdot C_4^3 $ способов их выбрать.
Итого имеем, что вероятность комбинации <<full house>> равна
$$ p =
\frac{ C_{13}^1 \cdot C_4^2 \cdot C_{12}^1 \cdot C_4^3 }{ C_{52}^5 };$$

\item пять карт одной масти можно выбрать $ 4 \cdot C_{13}^5 $ способами.
Отнимем из этих комбинаций <<royal flush>> и <<straight flush>>.
Получим $ 4 \cdot C_{13}^5 - \left( С_4^1 \right)^5 - C_{36}^1 $.
Поэтому вероятность комбинации <<flush>> равна
$$ p =
\frac{ 4 \cdot C_{13}^5 - \left( С_4^1 \right)^5 - C_{36}^1 }{ C_{52}^5 };$$

\item имея только одну масть есть 9 комбинаций пяти карт подряд.
Каждая из этих пяти карт может быть любой масти.
Главное, чтобы хотя бы одна карта имела масть, отличающуюся от остальных четырёх карт.
Имеем $ 9 \cdot 4^5 $ комбинаций.
Теперь отнимем от них комбинации, где все карты имеют одну масть (их 9).
Поэтому вероятность <<straight>> равна
$$ p =
\frac{ C_9^1 \cdot 4^5 }{ C_{52}^5 }.$$
\end{enumerate}

\subsubsection*{3.23}

\textit{Задание.} Компьютерный центр имеет три процессора, на которые поступило n заданий.
Каждое задание выполняется на наугад выбранном процессоре.
Найдите вероятность того, что ровно один процессор останется незадействованным.

\textit{Решение.} Используем классическое определение вероятности:
$$ P \left( A \right) =
\frac{ \left| A \right| }{ \left| \Omega \right| },$$
где $ \left| A \right| $ --- число исходов, благоприятствующих осуществлению события, а $ \left| \Omega \right| $ --- число всех равновероятных элементарных исходов.

Посчитаем $ \left| \Omega \right| $.
$ \Omega = \left\{ \left( x_1, x_2, \dotsc, x_n \right), \, x_i = \overline{1,3} \right\} $, где 1 --- выполнение задания на первом процессоре, 2 --- на втором,
3 --- на третьем.

Значит, $ \left| \Omega \right| = 3^n $ ---
число различных способов распределить n заданий по трём процессорам,
причём каждый процессор может получить любое количество заданий.

Теперь посчитаем $ \left| A \right| $.
Из трёх возможных процессоров выбираем 1 нерабочий.
Тогда 2 процессора будут решать $n$ заданий: $ \left| A_1 \right| = C_3^1 \cdot 2^n$.
Но тут учтена возможность, когда работает только 1 процессор.
Её необходимо вычесть: $ \left| A_2 \right| = C_3^1 \cdot C_2^1 \cdot 1^n$, где $C_2^1$ --- выбор второго нерабочего процессора.
Имеем, что $ \left| A \right| = \left| A_1 \right| - \left| A_2 \right| = C_3^1 \cdot 2^n - C_3^1 \cdot C_2^1 \cdot 1^n$.

Искомая вероятность
$$ P \left( A \right) =
\frac{C_3^1 \cdot 2^n - C_3^1 \cdot C_2^1 \cdot 1^n}{3^n}.$$

\subsubsection*{3.24}

\textit{Задание.} (Статистика Бозе-Эйнштейна).
Каждая из n одинаковых частиц наугад попадает в одну из N ячеек.
\begin{enumerate}[label=\alph*)]
\item Найдите вероятность того, что в первой, второй и т.д., N-ой ячейке будет соответственно $ n_1, n_2, \dotsc , n_N $ частиц;
\item докажите, что вероятность $ q_k $ того, что в данной ячейке будет k частиц, равна
$$ q_k =
\frac{ C_{ N+n-k-2 }^{ n-k } }{ C_{ N+n-1 }^n };$$
\item докажите, что при $ N > 2 : q_0 > q_1 > q_2 > \dotsc $;
\item докажите, что если n и N стремятся к бесконечности так, что
$$ \frac{ n }{ N } \rightarrow \lambda,$$
то
$$ q_k \rightarrow \frac{ \lambda^k }{ \left( 1 + \lambda \right)^{ k+1 } };$$
\item найдите вероятность того, что ровно r ячеек будут пустыми.
\end{enumerate}

\textit{Решение.}
\begin{enumerate}[label=\alph*)]
\item Разместим частицы в ряд.
Поставим между ними $ N - 1 $ перегородку.
Обозначим частицу символом <<0>>, а перегородку символом <<1>>.
Тогда распределение частиц однозначно характеризуется последовательностью из n нулей и $ N - 1 $ единицы.
Например, последовательность
$ 1000101100001 \dotsc $
означает, что в первую ячейку не попало ни одной частицы, во вторую --- 3, в третью попала одна частица, в четвёртую не попало ни одной частицы и т.д.
Количество способов, которыми можно распределить n частиц, совпадает с количеством разных последовательностей указанного вида.
Последовательность определяется однозначно, если выбрать $ N - 1 $ место из $ n + N - 1$, где будут размещены единицы.
Количество таких комбинаций равно $ | \Omega | = C_{n+N-1}^{N-1} $.

Событию A = \{ в первой, второй и т.д., N-ой ячейке окажется соответственно $ n_1, n_2, \dotsc , n_N $ частиц\}
способствует одно элементарное событие, а именно
$ \omega_0 = \left( n_1, n_2, \dotsc , n_N \right) $.

Таким образом,
$$ P(A) =
\frac{1}{C_{n+N-1}^{N-1}};$$

\item вероятность желаемого события оценивается по формуле
$$ q_k =
\frac{r}{| \Omega |}.$$

Аналогично предыдущему пункту $  | \Omega | = C_{n+N-1}^{N-1}  $.

Сначала надо отобрать k частиц в фиксированную ячейку (это можно сделать одним способом), а потом
$ n - k $
частиц распределить среди
$ N - 1 $ ячейки.

Используем метод перегородок.
Разместим все $ n - k $ частиц в ряд.
Чтобы отделить частицы, которые попадут в разные ячейки, поставим между ними $ N - 2 $ перегородки.
Обозначим частицу символом <<0>>, а перегородку символом <<1>>.
Тогда распределение частиц по ячейкам однозначно характеризуется последовательностью из $ n - k $ нулей и $ N - 2$ единиц.
Количество способов, которыми можно распределить
$ n - k $
частиц, совпадает с количеством разных последовательностей указанного вида.
Последовательность определяется однозначно, если выбрать $ N - 2 $ места из $ n - k + N - 2 $, где будут размещены единицы.
Количество таких комбинаций равно $ r = C_{n-k+N-2}^{N-2} $.

Тогда
$$ q_k =
\frac{C_{n-k+N-2}^{N-2}}{C_{n+N-1}^{N-1}};$$

\item найдём вероятность $ q_0 $ того, что в данной ячейке будет 0 частиц.
Вероятность желаемого события оценивается по формуле
$$ q_0 =
\frac{r_0}{| \Omega |}.$$

Аналогично пункту а) $  | \Omega | = C_{n+N-1}^{N-1}  $.

Нужно n одинаковых частиц распределить среди $ N - 1 $ ячейки.

Используем метод перегородок.
Разместим все n частиц в ряд.
Чтобы отделить частицы, которые попадут в разные ячейки, поставим между ними $ N - 2 $ перегородки.
Обозначим частицу символом <<0>>, а перегородку символом <<1>>.
Тогда распределение частиц по ячейкам характеризуется последовательностью из n нулей и $ N - 2 $ единиц.
Количество способов, которыми можно распределить n частиц, совпадает с количеством разных последовательностей указанного вида.
Последовательность определяется однозначно, если выбрать $ N - 2 $ места из $ n + N - 2 $, где будут размещены единицы.
Количество таких комбинаций равно $ C_{n+N-2}^{N-2} $.

Тогда
$$ q_0 =
\frac{C_{n+N-2}^{N-2}}{C_{n+N-1}^{N-1}}.$$

Проверим формулу из пункта б).
Подставим $ k = 0 $.
Получим:
$$ q_0 =
\frac{C_{n+N-2}^{N-2}}{C_{n+N-1}^{N-1}}.$$
Видим, что получили такую же формулу, значит для дальнейшего решения можно использовать её, подставляя вместо k нужное значение.

Упростим:
\begin{equation*}
\begin{split}
q_0 =
\frac{ \frac{ \left( n+N-2 \right)! }{ \left( N-2 \right)! \left( n+N-2-N+2 \right)!  } }{ \frac{ \left( n+N-1\right)! }{\left( N-1 \right)! \left( n+N-1-N+1 \right)!  } } =
\frac{ \left( n+N-2\right)! \left( N-1 \right)!n!}{ \left( N-2 \right)!n! \left( n+N-1 \right)! } = \\
= \frac{ \left( n+N-2 \right)! \left( N-2 \right)! \left( N-1 \right) }{ \left( N-2 \right)! \left( n+N-2 \right)! \left( n+N-1 \right) } =
\frac{N-1}{n+N-1}.
\end{split}
\end{equation*}

Упростим общую формулу:
\begin{equation*}
\begin{split}
q_k =
\frac{ \frac{ \left( n-k+N-2 \right)! }{ \left( N-2 \right)! \left( n-k+N-2-N+2 \right)! } }{ \frac{ \left( n+N-1 \right)!}{ \left( N-1 \right)! \left(n+N-1-N+1 \right)!} } =
\frac{ \left( n-k+N-2 \right)! \left( N-1 \right)!n!}{ \left( N-2 \right)! \left( n-k \right)! \left( n+N-1 \right)!} = \\
= \frac{ \left( n-k+N-2 \right)! \left( N-1 \right) n!}{ \left( n-k \right)! \left( n+N-1 \right)!}.
\end{split}
\end{equation*}

Найдём вероятность $ q_1 $ того, что в данной ячейке будет одна частица:
\begin{equation*}
\begin{split}
q_1 =
\frac{ \left( n-1+N-2 \right)! \left( N-1 \right) n!}{ \left( n-1 \right)! \left( n+N-1 \right)!} = \\
= \frac{ \left( n+N-3 \right)! \left( N-1 \right)! \left( n-1 \right)!n }{ \left( n-1 \right)! \left( n+N-3 \right)! \left( n+N-2 \right) \left( n+N-1 \right) } =
\frac{ \left( N-1 \right) n}{ \left( n+N-2 \right) \left( n+N-1 \right).}
\end{split}
\end{equation*}

Сравним $ q_0 $ и $ q_1 $.
Приведём к общему знаменателю.
Для этого $ q_0 $ умножим на $ \left( n+N-2 \right) $.
Получим:
$$ q_0 =
\frac{ \left( N-1 \right) \left( n+N-2 \right) }{ \left( n+N-1 \right) \left( n+N-2 \right) }.$$

Видим, что $ q_0 $ и $ q_1 $ отличаются только выражением в числителе.
Теперь сравним n и $ \left( n+N-2 \right)$.
По условию задачи $ N > 2 $, поэтому $ N - 2 > 0 $.
Это значит, что $ n + N - 2  > n $, т.е. $ q_0 > q_1 $.

Докажем, что условие выполняется для k-го члена.
Найдём вероятность $ q_{k+1} $ того, что в данной ячейке будет $ k + 1 $ частица:
$$ q_{k+1} =
\frac{ \left( n-k-1+N-2 \right)! \left( N-1 \right) n!}{ \left( n-k-1 \right)! \left( n+N-1 \right)!} =
\frac{ \left( n-k+N-3 \right)! \left( N-1 \right) n!}{ \left( n-k-1 \right)! \left( n+N-1 \right)!}.$$

Сравним полученное выражение с $ q_k $.
В выражении для $ q_{k+1} $ и числитель, и знаменатель меньше: $ n - k + N -3 < n - k + N - 2 $, а также $ n - k - 1 < n - k $.
Отсюда следует, что $ q_k > q_{k+1}$.

Итого, доказали, что условие верно для первого члена $ \left( q_0 > q_1 \right) $.
Также из верности условия для k-го члена вытекает его верность для $ k + 1 $-го.
Отсюда по индукции Пеано следует, что условие верно для любого натурального числа;

\item распишем $ p_k $ (вероятность того, что в данную ячейку попадут k частиц):

\begin{equation*}
\begin{split}
p_k =
\frac{C_{N+n-k-2}^{n-k}}{C_{N+n-1}^n} =
\frac{ \left( N+n-k-2 \right)!n! \left( N-1 \right)!}{ \left( n-k \right)! \left( N-2 \right)! \left( N+n-1 \right)!} = \\
= \frac{ \left( N+n-k-2 \right)! \left( n-k \right)! \left( N-2 \right)! \left( N-1 \right) \prod \limits_{i=0}^{k-1} \left( n-i \right) }{ \left( n-k \right)! \left( N-2 \right)! \left( N+n-k-2 \right)! \prod \limits_{j=1}^{k+1} \left( N+n-j \right) } =
\frac{ \left( N-1 \right) \prod \limits_{i=0}^{k-1} \left( n-i \right) }{ \prod \limits_{j=1}^{k+1} \left( N+n-j \right) }.
\end{split}
\end{equation*}

Поделим числитель и знаменатель на N:
\begin{equation*}
\begin{split}
p_k =
\frac{ \left( 1- \frac{1}{N} \right) \prod \limits_{i=0}^{k-1} \left( \frac{n}{N} - \frac{i}{N} \right) }{ \prod \limits_{j=1}^{k+1} \left( 1+ \frac{n}{N} - \frac{j}{N} \right) } = \\
= \frac{ \frac{n}{N} \left( \frac{n}{N} - \frac{1}{N} \right) \dotsc \left( \frac{n}{N} - \frac{k-1}{N} \right) }{ \left( 1+ \frac{n}{N} - \frac{1}{N} \right) \left( 1+ \frac{n}{N} - \frac{2}{N} \right) \dotsc \left( 1+ \frac{n}{N} - \frac{k+1}{N} \right) } =
\frac{ \lambda^k}{ \left( 1+ \lambda \right)^{k+1}};
\end{split}
\end{equation*}

\item вероятность желаемого события оценивается по формуле
$$ p =
\frac{r}{| \Omega |}.$$

Как и в пункте а) имеем общее количество вариантов $ | \Omega | = C_{n+N-1}^{N-1} $.
Вычислим количество вариантов r, которые отвечают событию, указанному в условии задачи.

Решение задачи разбиваем на два шага:
сначала выбираем $ N - r $ ячеек из N, которые будут занятыми (это можно сделать $ C_{N}^{N-r} $ способами),
а затем размещаем n одинаковых частиц в $ N - r $ ячеек таким образом, чтобы ни одна их них не была пустой.
Поскольку все частицы одинаковы, то в каждую ячейку можно заранее поместить по одной частице.
В этом случае задача сводится к распределению $ n - \left( N - r \right) = n - N + r $ частиц на $ N - r $ ячеек.

Используем метод перегородок.
Разместим все частицы в ряд.
Чтобы отделить частицы, которые будут находиться в разных ячейках, поставим между ними $ N - r - 1 $ перегородку.
Обозначим частицу символом <<1>>, а перегородку символом <<0>>.

Тогда распределение частиц по ячейкам характеризуется последовательностью из $ n - N + r $ нулей и $ N - r - 1 $ единицы.
Количество способов, которыми можно распределить $ n - N + r $ частиц, совпадает с количеством разных комбинаций указанного вида.
Последовательность определяется однозначно, если выбрать $ N - r - 1 $ мест из
$ \left( n - N + r \right) + \left( N - r - 1 \right) = \\
= n - N + r + N - r - 1 = m - 1 $,
где будут стоять единицы.
Количество таких комбинаций равно $ C_{n-1}^{N - r - 1} $.

Поэтому окончательно имеем $ r = C_{N}^{N-r} \cdot C_{n-1}^{N - r - 1} $.

Тогда вероятность указанного события равна
$$ p =
\frac{C_{N}^{N-r} \cdot C_{n-1}^{N - r - 1}}{C_{n+N-1}^{N-1}}.$$
\end{enumerate}
