\addcontentsline{toc}{chapter}{Занятие 18. Условное математическое ожидание}
\chapter*{Занятие 18. Условное математическое ожидание}

\addcontentsline{toc}{section}{Контрольные вопросы и задания}
\section*{Контрольные вопросы и задания}

\subsubsection*{Приведите определение условного математического ожидания.}

Условным математическим ожиданием случайной величины $ \xi $ относительно $ \sigma $-алгебры
$ \mathbb{F'}$ называется случайная величина $ \eta $ такая, что:
\begin{enumerate}
  \item $ \eta $ измерима относительно $ \mathbb{F'}$;
  \item для любой случайной величины $ \zeta $, измеримой относительно $ \mathbb{F'}$ и ограниченной,
  $M \xi \zeta = M \eta \zeta $.
\end{enumerate}

\subsubsection*{Запишите формулы для вычисления условного математического ожидания.}

$ \sigma $-алгебра $ \mathbb{F'}$ --- конечная.
Есть набор $H_1, \dotsc, H_n$ такой, что
$$ \bigcup \limits_{k = 1}^n H_k = \Omega, \,
  H_k \cap H_j = \emptyset, \,
  k \neq j, \,
  P \left( H_k \right) > 0.$$

Уловное математическое ожидание случайной величины $ \xi $ относительно
$$ \mathbb{F'} =
  \left\{ \bigcup \limits_{k = 1}^n H_k^{ \varepsilon_k} \right\},$$
где
$$ \varepsilon_k =
  \begin{cases}
    0, \\
    1,
  \end{cases}
  H_k^0 = \emptyset, \,
  H_k^1 = H_k,$$
имеет вид
$$ \eta =
  \sum \limits_{k = 1}^n
    \frac{M \xi \cdot \mathbbm{1}_{H_k}}{P \left( H_k \right) } \cdot \mathbbm{1}_{H_k}.$$

\subsubsection*{Сформулируйте свойства условного математического ожидания.}

\begin{enumerate}
  \item $ \mathbb{F'} = \left\{ \emptyset, \Omega \right\} $.
  Из этого следует, что $M \left( \xi \; \middle| \; \mathbb{F'} \right) = M \xi $;
  \item $ \mathbb{F'} \subset \mathbb{F"} \subset \mathbb{F}$.
  Тогда
  $M \left( M \left( \xi \; \middle| \; \mathbb{F"} \right) \; \middle| \; \mathbb{F'} \right) =
    M \left( \xi \; \middle| \; \mathbb{F'} \right) $;
  \item $M \left( M \left( \xi \; \middle| \; \mathbb{F'} \right) \right) = M \xi $;
  \item $ \zeta $ измерима относительно $ \mathbb{F'}$.
  Тогда
  $M \left( \xi \zeta \; \middle| \; \mathbb{F'} \right) =
    \zeta M \left( \xi \; \middle| \; \mathbb{F'} \right) $;
  \item $ \xi $ не зависит от $ \mathbb{F'}$, следовательно,
  $M \left( \xi \; \middle| \; \mathbb{F'} \right) =
    M \xi $;
  \item $M \left( \alpha_1 \xi_1 + \alpha_2 \xi_2 \; \middle| \; \mathbb{F'} \right) =
    \alpha_1 M \left( \xi_1 \; \middle| \; \mathbb{F'} \right) +
    \alpha_2 M \left( \xi_2 \; \middle| \; \mathbb{F'} \right) $;
  \item если $ \xi \geq 0$, то $M \left( \xi \; \middle| \; \mathbb{F'} \right) \geq 0$;
  \item $ \left| M \left( \xi \; \middle| \; \mathbb{F'} \right) \right| \leq
    M \left( \left| \xi \right| \; \middle| \; \mathbb{F'} \right) $,
  так как модуль --- выпуклая вниз функция;
  \item если $f$ --- выпуклая вниз функция,
  то $f \left( M \left( \xi \; \middle| \; \mathbb{F'} \right) \right) \leq
    M \left( f \left( \xi \right) \; \middle| \; \mathbb{F'} \right) $;
  \item предельные теоремы.
  Если $ \xi_n \geq 0, \, \xi_n \nearrow \xi $ (возрастает) при $n \to \infty $,
  то
  $M \left( \xi_n \; \middle| \; \mathbb{F'} \right) \nearrow
    M \left( \xi \; \middle| \; \mathbb{F'} \right) $
  почти наверное, $n \to \infty $;
  \item если $ \left| \xi_n \right| \leq \eta, \, \exists M \eta $ и
  $ \xi_n \overset{P} \xi, \, n \to \infty $,
  то
  $M \left( \xi_n \; \middle| \; \mathbb{F'} \right) \overset{P}{ \to }
    M \left( \xi \; \middle| \; \mathbb{F'} \right), \,
    n \to \infty.$
\end{enumerate}

\addcontentsline{toc}{section}{Аудиторные задачи}
\section*{Аудиторные задачи}

\subsubsection*{18.3}

\textit{Задание.}
На вероятностном пространстве $ \left( \Omega, \mathbb{F}, P \right) $,
где $ \Omega = \left[ 0, 1 \right], \mathbb{F}$ --- мера Лебега,
задана случайная величина $ \xi = \omega $.

Пусть $ \sigma $-алгебра $ \mathbb{B}$ порождена множествами
$$ \left[ 0, \frac{1}{3} \right), \,
  \left\{ \frac{1}{3} \right\}, \,
  \left( \frac{1}{3}, \frac{1}{2} \right).$$
Вычислите условное математическое ожидание $M \left[ \xi \; \middle| \; \mathbb{B} \right] $.

\textit{Решение.} Мера Лебега в данном случае --- это длина.

Есть 3 атома
$$ \sigma \left( \beta \right) =
  \left\{
    \left[ 0, \frac{1}{3} \right), \,
    \left[ \frac{1}{3}, \frac{1}{2} \right), \,
    \left[ \frac{1}{2}, 1 \right]
  \right\}.$$

Условное математическое ожидание ---
это случайная величина
$$ \eta =
  M \left[ \xi \; \middle| \; \mathbb{B} \right].$$
Будет 3 слагаемых
\begin{equation*}
  \begin{split}
    \eta = \\
    = \frac{M \left( \omega \cdot \mathbbm{1} \left\{ \omega \in \left[ 0, \frac{1}{3} \right] \right\} \right) }{ \frac{1}{3}} \cdot
    \mathbbm{1} \left\{ \omega \in \left[ 0, \frac{1}{3} \right] \right\} + \\
    + \frac{M \left( \omega \cdot \mathbbm{1} \left\{ \omega \in \left[ \frac{1}{3}, \frac{1}{2} \right) \right\} \right) }{ \frac{1}{6}} \cdot
    \mathbbm{1} \left\{ \omega \in \left[ \frac{1}{3}, \frac{1}{2} \right) \right\} + \\
    + \frac{M \left( \omega \cdot \mathbbm{1} \left\{ \omega \in \left[ \frac{1}{2}, 1 \right] \right\} \right) }{ \frac{1}{2}} \cdot
    \mathbbm{1} \left\{ \omega \in \left[ \frac{1}{2}, 1 \right] \right\}.
  \end{split}
\end{equation*}
Числители --- интегралы в соответствующих пределах
\begin{equation*}
  \begin{split}
    \eta = \\
    = 3 \int \limits_0^{ \frac{1}{3}} xdx \cdot
    \mathbbm{1} \left\{ \omega \in \left[ 0, \frac{1}{3} \right) \right\} +
    6 \int \limits_{ \frac{1}{3}}^{ \frac{1}{2}} xdx \cdot
    \mathbbm{1} \left\{ \omega \in \left[ \frac{1}{3}, \frac{1}{2} \right) \right\} + \\
    + 2 \int \limits_{ \frac{1}{2}}^1 xdx \cdot
    \mathbbm{1} \left\{ \omega \in \left[ \frac{1}{2}, 1 \right] \right\} = \\
    = \frac{1}{6} \cdot \mathbbm{1} \left\{ \omega \in \left[ 0, \frac{1}{3} \right) \right\} +
    \frac{5}{12} \cdot
    \mathbbm{1} \left\{ \omega \in \left[ \frac{1}{3}, \frac{1}{2} \right) \right\} +
    \frac{3}{4} \cdot \mathbbm{1} \left\{ \omega \in \left[ \frac{1}{2}, 1 \right] \right\}.
  \end{split}
\end{equation*}

На каждом атоме случайная величина принимает своё значение.

\subsubsection*{18.4}

\textit{Задание.} Пусть $ \xi $ и $ \eta $ --- независимые случайные величины.
Вычислите $D \left( M \left[ \xi \eta \; \middle| \; \xi \right] \right) $,
если $ \xi $ равномерно распределена на отрезке $ \left[ 0, 1 \right] $,
а $ \eta $ имеет нормальное распределение с параметрами $1, \sigma^2$.

\textit{Решение.} Упростим случайную величину
$M \left( \xi \eta \; \middle| \; \xi \right) =
  M \left[ \xi \eta \; \middle| \; \sigma \left( \xi \right) \right] $.
Случайная величина $ \xi $ измерима относительно своей же $ \sigma $-алгебры
$ \sigma \left( \xi \right) $.
Используем свойство условного математического ожидания, которое говорит,
что в данном случае можем вынести $ \xi $ за знак условного математического ожидания
$M \left[ \xi \eta \; \middle| \; \sigma \left( \xi \right) \right] =
  \xi M \left[ \eta \; \middle| \; \sigma \left( \xi \right) \right].$
Случайные величины $ \xi $ и $ \eta $ независимы, поэтому $ \eta $ не зависит от $ \sigma $-алгебры,
порождённой $ \xi $.
По свойствам условного математического ожидания
$ \xi M \left[ \eta \; \middle| \; \sigma \left( \xi \right) \right] =
  \xi M \eta =
  \xi a$,
где
$$a = const.$$

Ищем дисперсию
$D \left( M \left[ \xi \eta \; \middle| \; \xi \right] \right) =
  D \left( a \xi \right) =
  a^2 D \xi.$
Случайная величина имеет равномерное распределение
$$a^2 D \xi =
  \frac{a^2}{12}.$$

\subsubsection*{18.5}

\textit{Задание.} Пусть $ \xi $ имеет стандартное нормальное распределение.
Вычислите условное математическое ожидание:
\begin{enumerate}[label=\alph*)]
  \item $M \left( \xi \; \middle| \; \left| \xi \right| \right) $;
  \item $M \left( \xi^2 \; \middle| \; \left| \xi \right| \right) $.
\end{enumerate}

\textit{Решение.}
\begin{enumerate}[label=\alph*)]
  \item $M \left( \xi \; \middle| \; \left| \xi \right| \right) =
    \eta $.

  Не можем $ \xi $ хорошей функцией выразить через $ \left| \xi \right| $.

  Случайная величина $ \eta $ должна удовлетворять свойствам:
  \begin{enumerate}
    \item $ \eta $ --- измеримая относительно $ \sigma $-алгебры, порождённо $ \left| \xi \right| $,
    то есть $ \sigma \left( \left| \xi \right| \right) $;
    \item для любого множества
    $A \in \sigma \left( \left| \xi \right| \right): \,
    M \left( \xi \cdot \mathbbm{1}_A \right) = M \left( \eta \cdot \mathbbm{1}_A \right) $.
  \end{enumerate}

  Вычисляем математическое ожидание
  $$M \left( \xi \cdot \mathbbm{1}_A \right) =
    \int \limits_{- \infty }^{+ \infty }
      x \cdot \mathbbm{1}_A \left( x \right) p \left( x \right) dx,$$
  где $p \left( x \right) $ --- плотность нормального распределения, $A$ ---
  симметричная относительно нуля область (множество).
  Это некоторая функция от $ \left| x \right| $.
  Учитывая это, получаем
  $$ \int \limits_{- \infty }^{+ \infty }
      x \cdot \mathbbm{1}_A \left( x \right) p \left( x \right) dx =
    \int \limits_{- \infty }^{+ \infty }
      x \cdot \varphi \left( \left| x \right| \right) p \left( x \right) dx.$$
  Функция $ \varphi \left( \left| x \right| \right) $ --- чётная, $p \left( x \right) $ --- чётная,
  $x$ --- нечётная.
  Интегрируем по всей оси
  $$ \int \limits_{- \infty }^{+ \infty }
      x \cdot \varphi \left( \left| x \right| \right) p \left( x \right) dx =
    0,$$
  следовательно, $ \eta = 0$;

  \item ищем
  $M \left( \xi^2 \; \middle| \; \left| \xi \right| \right) =
    M \left[ \xi^2 \; \middle| \; \sigma \left( \left| \xi \right| \right) \right] $.
  Перепишем случайную величину $ \xi^2 = \left| \xi \right|^2$, откуда следует, что $ \xi^2$ ---
  измеримая относительно $ \sigma $-алгебры $ \sigma \left( \left| \xi \right| \right) $.
  Согласно со свойствами
  $M \left[ \xi^2 \; \middle| \; \sigma \left( \left| \xi \right| \right) \right] =
    \xi^2.$
\end{enumerate}

\subsubsection*{18.6}

\textit{Задание.}
Пусть независимые случайные величины $ \xi $ и $ \eta $ имеют распределение Пуассона с
параметром $ \lambda $ каждая.
Вычислите:
\begin{enumerate}[label=\alph*)]
  \item условное распределение $P \left( \xi = k \; \middle| \; \xi + \eta = n \right) $;
  \item условное математическое ожидание $M \left[ \xi^2 \; \middle| \; \xi + \eta \right] $.
\end{enumerate}

\textit{Решение.}
$$M \left[ \xi^2 \; \middle| \; \xi + \eta \right] =
  M \left( \xi^2 \; \middle| \; \xi + \eta = t \right).$$

\begin{enumerate}[label=\alph*)]
  \item По определению условной вероятности
  $$P \left( \xi = k \; \middle| \; \xi + \eta = t \right) =
    \frac{P \left( \xi = k, \, \xi + \eta = t \right) }{P \left( \xi + \eta = t \right) }.$$
  Сумма пуассоновских случайных величин имеет распределение Пуассона с параметром,
  который равен сумме параметров распределений случайных величин, входящих в сумму
  $$ \frac{P \left( \xi = k, \, \xi + \eta = t \right) }{P \left( \xi + \eta = t \right) } =
    \frac{P \left( \xi = k \right) P \left( \eta = t - k \right) }{P \left( \zeta = t \right) } =
    \frac{ \frac{ \lambda^k e^{- \lambda }}{k!} \cdot \frac{ \lambda^{t - k} e^{- \lambda }}{ \left( t - k \right)!}}{ \left( 2 \lambda \right)^y e^{-2 \lambda }} \cdot
    t!.$$
  Сокращаем
  $$ \frac{ \frac{ \lambda^k e^{- \lambda }}{k!} \cdot \frac{ \lambda^{t - k} e^{- \lambda }}{ \left( t - k \right)!}}{ \left( 2 \lambda \right)^y e^{-2 \lambda }} \cdot
    t! =
    C_t^k \cdot \frac{1}{2^t} =
    C_t^k \cdot 2^{-t},$$
  где $ \zeta = \xi + \eta \sim Pois \left( 2 \lambda \right) $;
  \item условное математическое ожидание равно
  $$M \left( \xi^2 \; \middle| \; \xi + \eta = t \right) =
    \sum \limits_{k = 0}^t k^2 \cdot \frac{t! \cdot 2^{-t}}{k! \cdot \left( t - k \right)!} =
    2^{-t} \sum \limits_{k = 1}^t \frac{kt!}{ \left( k - 1 \right)! \left( t - k \right)!}.$$
  Разбиваем на две суммы, $k$ представляем как $\left( k - 1 + 1 \right) $.
  Получем
  \begin{equation*}
    \begin{split}
      2^{-t} \sum \limits_{k = 1}^t \frac{kt!}{ \left( k - 1 \right)! \left( t - k \right)!} = \\
      = 2^{-t} \cdot
      \sum \limits_{k = 1}^t
        \frac{ \left( t - 2 \right)!}{ \left( k - 2 \right)! \left( t - k \right)!} \cdot
      \left( t - 1 \right) +
      2^{-t} \cdot
      \sum \limits_{k = 2}^t
        \frac{ \left( t - 1 \right)!}{ \left( k - 1 \right)! \left( t - k \right)!} = \\
      = 2^{-t} \left[ t \left( t - 1 \right) \cdot 2^{t - 2} + t \cdot 2^{t - 1} \right] =
      \frac{t \left( t - 1 \right) }{4} + \frac{t}{2} =
      \frac{t^2 + t}{4}.
    \end{split}
  \end{equation*}
  Отсюда следует, что
  $$M \left( \xi^2 \; \middle| \; \xi + \eta \right) =
    \frac{1}{4} \left[ \left( \xi + \eta \right)^2 + \xi + \eta \right].$$
\end{enumerate}

\addcontentsline{toc}{section}{Домашнее задание}
\section*{Домашнее задание}

\subsubsection*{18.15}

\textit{Задание.}
На вероятностном пространстве $ \left( \omega, \mathcal{F}, P \right) $,
где $ \Omega = \left[ 0, 1 \right], \, \mathcal{F}$ ---
$ \sigma $-алгебра борелевских подмножетсв $ \Omega, \, P$ --- мера Лебега,
задана случайная величина $ \xi = \sin \pi \omega $.
Пусть $ \sigma $-алгебра $ \mathcal{B}$ порождена множествами
$$ \left[0, \frac{1}{3} \right), \,
  \left\{ \frac{1}{3} \right\}, \,
  \left( \frac{1}{3}, \frac{1}{2} \right).$$

Вычислите условное математическое ожидание $M \left[ \xi \; \middle| \; \mathcal{B} \right] $.

\textit{Решение.} Мера Лебега в данном случае --- это длина.

Есть 3 атома
$$ \sigma \left( \mathcal{B} \right) =
  \left\{
    \left[ 0, \frac{1}{3} \right), \,
    \left[ \frac{1}{3}, \frac{1}{2} \right), \,
    \left[ \frac{1}{2}, 1 \right]
  \right\}.$$

Условное математическое ожидание ---
это случайная величина
$$ \eta =
  M \left( \xi \; \middle| \; \mathcal{B} \right).$$

Будет 3 слагаемых,
умноженные на соответствующие
индикоторы и поделенные на вероятность попадания в соответствующий отрезок.

Первое слагаемое имеет вид
$$ \int \limits_0^1
  \sin \left( \pi \omega \right) \cdot
  \mathbbm{1} \left\{ \omega \in \left[ 0, \frac{1}{3} \right) \right\} d \omega =
  \int \limits_0^{ \frac{1}{3}} \sin \left( \pi \omega \right) d \omega =
  \frac{1}{ \pi } \cdot
  \int \limits_0^{ \frac{1}{3}} \sin \left( \pi \omega \right) d \left( \pi \omega \right).$$
Берём интеграл
$$ \frac{1}{ \pi } \cdot
  \int \limits_0^{ \frac{1}{3}} \sin \left( \pi \omega \right) d \left( \pi \omega \right)
  \left. - \frac{1}{ \pi } \cdot \cos \left( \pi \omega \right) \right|_0^{ \frac{1}{3}} =
  - \frac{1}{ \pi } \cdot \cos \left( \frac{ \pi }{3} \right) + \frac{1}{ \pi } \cdot \cos 0=
  - \frac{1}{ \pi } \cdot \frac{1}{2} + \frac{1}{ \pi } =
  \frac{1}{2 \pi }.$$

Второе слагаемое имеет вид
$$ \int \limits_0^1
  \sin \left( \pi \omega \right) \cdot
  \mathbbm{1} \left\{ \omega \in \left[ \frac{1}{3}, \frac{1}{2} \right) \right\} d \omega =
  \int \limits_{ \frac{1}{3}}^{ \frac{1}{2}} \sin \left( \pi \omega \right) d \omega =
  \left.
    - \frac{1}{ \pi } \cdot \cos \left( \pi \omega \right)
  \right|_{ \frac{1}{3}}^{ \frac{1}{2}}.$$
Подставляем пределы интегрирования
$$ \left.
    - \frac{1}{ \pi } \cdot \cos \left( \pi \omega \right)
  \right|_{ \frac{1}{3}}^{ \frac{1}{2}} =
  - \frac{1}{ \pi } \cdot \cos \frac{ \pi }{2} + \frac{1}{ \pi } \cdot \cos \frac{ \pi }{3} =
  - \frac{1}{ \pi } \cdot 0 + \frac{1}{ \pi } \cdot \frac{1}{2} =
  \frac{1}{2 \pi }.$$

Аналогично получаем третье слагаемое
$$ \int \limits_0^1
    \sin \left( \pi \omega \right) \cdot
    \mathbbm{1} \left\{ \omega \in \left[ \frac{1}{2}, 1 \right] \right\} d \omega =
  \int \limits_{ \frac{1}{2}}^1 \sin \left( \pi \omega \right) d \omega =
  \left. - \frac{1}{ \pi } \cdot \cos \left( \pi \omega \right) \right|_{ \frac{1}{2}}^1.$$
Подставляем пределы интегрирования
$$ \left. - \frac{1}{ \pi } \cdot \cos \left( \pi \omega \right) \right|_{ \frac{1}{2}}^1 =
  - \frac{1}{ \pi } \cdot \cos \pi + \frac{1}{ \pi } \cdot \cos \frac{ \pi }{2} =
  - \frac{1}{ \pi } \cdot \left( -1 \right) + \frac{1}{ \pi } \cdot 0 =
  \frac{1}{ \pi }.$$

Получаем условное математическое ожидание
$$ \eta =
  \frac{3}{2 \pi } \cdot
  \mathbbm{1} \left\{ \omega \in \left[ 0, \frac{1}{3} \right) \right\} +
  \frac{6}{2 \pi } \cdot
  \mathbbm{1} \left\{ \omega \in \left[ \frac{1}{3}, \frac{1}{2} \right) \right\} +
  \frac{2}{ \pi } \cdot
  \mathbbm{1} \left\{ \omega \in \left[ \frac{1}{2}, 1 \right) \right\}.$$

Случайная величина на каждом атоме принимает своё значение.

\subsubsection*{18.16}

\textit{Задание.} Пусть $ \xi $ и $ \eta $ --- независимые случайные величины.
Вычислите $D \left( M \left[ \xi + \eta \; \middle| \; \xi \right] \right) $,
если $ \xi $ и $ \eta $ имеют показательное распределение с параметрами $ \lambda $
и $ \mu $ соответственно.

\textit{Решение.}
Воспользуемся линейностью условного математического ожидания
$M \left( \xi + \eta \; \middle| \; \xi \right) =
  M \left( \xi \; \middle| \; \xi \right) + M \left( \eta \; \middle| \; \xi \right) $.

Поскольку случайная величина $ \eta $ не зависит от $ \xi $, то
$$M \left( \eta \; \middle| \; \xi \right) =
  M \eta =
  \frac{1}{ \mu }.$$

Случайная величина $ \xi $ измерима относительно $ \sigma $-алгебры $ \sigma \left( \xi \right) $,
порождённой случайной величиной $ \xi $, а поэтому
$$M \left( \xi + \eta \; \middle| \; \xi \right) =
  \xi + \frac{1}{ \mu }.$$

Ищем дисперсию
$$D \left[ M \left( \xi + \eta \; \middle| \; \xi \right) \right] =
  D \left( \xi + \frac{1}{ \mu } \right) =
  D \xi + D \frac{1}{ \mu }.$$
Случайная величина $ \xi $ имеет показательное распределение
$$D \xi + D \frac{1}{ \mu } =
  \frac{1}{ \lambda^2} + 0 =
  \frac{1}{ \lambda^2}.$$

\subsubsection*{18.17}

\textit{Задание.} Пусть $ \xi $ имеет геометрическое распределение с параметром $p$.
Вычислите:
\begin{enumerate}[label=\alph*)]
  \item $M \left( \xi \; \middle| \; \xi^3 \right) $;
  \item $M \left( \xi \; \middle| \; \max \left( \xi, \eta \right) \right) $.
\end{enumerate}

\textit{Решение.}
\begin{enumerate}[label=\alph*)]
  \item Ищем
  $M \left( \xi \; \middle| \; \xi^3 \right) =
    M \left[ \xi \; \middle| \; \sigma \left( \xi^3 \right) \right] $.
  Перепишем случайную величину $ \xi = \sqrt[3]{ \xi^3}$, откуда следует, что $ \xi $ ---
  измеримая относительно $ \sigma $-алгебры $ \sigma \left( \xi^3 \right) $.
  Согласно со свойствами $M \left[ \xi \; \middle| \; \sigma \left( \xi^3 \right) \right] = \xi $;
  \item $M \left( \xi \; \middle| \; \max \left( \xi, \eta \right) \right) $.
\end{enumerate}
