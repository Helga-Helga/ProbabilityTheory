\addcontentsline{toc}{chapter}{Занятие 18. Условное математическое ожидание}
\chapter*{Занятие 18. Условное математическое ожидание}

\addcontentsline{toc}{section}{Контрольные вопросы и задания}
\section*{Контрольные вопросы и задания}

\subsubsection*{Приведите определение условного математического ожидания.}

Условным математическим ожиданием случайной величины $ \xi $ относительно $ \sigma $-алгебры
$ \mathbb{F'}$ называется случайная величина $ \eta $ такая, что:
\begin{enumerate}
  \item $ \eta $ измерима относительно $ \mathbb{F'}$;
  \item для любой случайной величины $ \zeta $, измеримой относительно $ \mathbb{F'}$ и ограниченной,
  $M \xi \zeta = M \eta \zeta $.
\end{enumerate}

\subsubsection*{Запишите формулы для вычисления условного математического ожидания.}

$ \sigma $-алгебра $ \mathbb{F'}$ --- конечная.
Есть набор $H_1, \dotsc, H_n$ такой, что
$$ \bigcup \limits_{k = 1}^n H_k = \Omega, \,
  H_k \cap H_j = \emptyset, \,
  k \neq j, \,
  P \left( H_k \right) > 0.$$

Уловное математическое ожидание случайной величины $ \xi $ относительно
$$ \mathbb{F'} =
  \left\{ \bigcup \limits_{k = 1}^n H_k^{ \varepsilon_k} \right\},$$
где
$$ \varepsilon_k =
  \begin{cases}
    0, \\
    1,
  \end{cases}
  H_k^0 = \emptyset, \,
  H_k^1 = H_k,$$
имеет вид
$$ \eta =
  \sum \limits_{k = 1}^n
    \frac{M \xi \cdot \mathbbm{1}_{H_k}}{P \left( H_k \right) } \cdot \mathbbm{1}_{H_k}.$$

\subsubsection*{Сформулируйте свойства условного математического ожидания.}

\begin{enumerate}
  \item $ \mathbb{F'} = \left\{ \emptyset, \Omega \right\} $.
  Из этого следует, что $M \left( \xi \; \middle| \; \mathbb{F'} \right) = M \xi $;
  \item $ \mathbb{F'} \subset \mathbb{F"} \subset \mathbb{F}$.
  Тогда
  $M \left( M \left( \xi \; \middle| \; \mathbb{F"} \right) \; \middle| \; \mathbb{F'} \right) =
    M \left( \xi \; \middle| \; \mathbb{F'} \right) $;
  \item $M \left( M \left( \xi \; \middle| \; \mathbb{F'} \right) \right) = M \xi $;
  \item $ \zeta $ измерима относительно $ \mathbb{F'}$.
  Тогда
  $M \left( \xi \zeta \; \middle| \; \mathbb{F'} \right) =
    \zeta M \left( \xi \; \middle| \; \mathbb{F'} \right) $;
  \item $ \xi $ не зависит от $ \mathbb{F'}$, следовательно,
  $M \left( \xi \; \middle| \; \mathbb{F'} \right) =
    M \xi $;
  \item $M \left( \alpha_1 \xi_1 + \alpha_2 \xi_2 \; \middle| \; \mathbb{F'} \right) =
    \alpha_1 M \left( \xi_1 \; \middle| \; \mathbb{F'} \right) +
    \alpha_2 M \left( \xi_2 \; \middle| \; \mathbb{F'} \right) $;
  \item если $ \xi \geq 0$, то $M \left( \xi \; \middle| \; \mathbb{F'} \right) \geq 0$;
  \item $ \left| M \left( \xi \; \middle| \; \mathbb{F'} \right) \right| \leq
    M \left( \left| \xi \right| \; \middle| \; \mathbb{F'} \right) $,
  так как модуль --- выпуклая вниз функция;
  \item если $f$ --- выпуклая вниз функция,
  то $f \left( M \left( \xi \; \middle| \; \mathbb{F'} \right) \right) \leq
    M \left( f \left( \xi \right) \; \middle| \; \mathbb{F'} \right) $;
  \item предельные теоремы.
  Если $ \xi_n \geq 0, \, \xi_n \nearrow \xi $ (возрастает) при $n \to \infty $,
  то
  $M \left( \xi_n \; \middle| \; \mathbb{F'} \right) \nearrow
    M \left( \xi \; \middle| \; \mathbb{F'} \right) $
  почти наверное, $n \to \infty $;
  \item если $ \left| \xi_n \right| \leq \eta, \, \exists M \eta $ и
  $ \xi_n \overset{P} \xi, \, n \to \infty $,
  то
  $M \left( \xi_n \; \middle| \; \mathbb{F'} \right) \overset{P}{ \to }
    M \left( \xi \; \middle| \; \mathbb{F'} \right), \,
    n \to \infty.$
\end{enumerate}

\addcontentsline{toc}{section}{Аудиторные задачи}
\section*{Аудиторные задачи}


\addcontentsline{toc}{section}{Домашнее задание}
\section*{Домашнее задание}
