\addcontentsline{toc}{chapter}{Занятие 18. Условное математическое ожидание}
\chapter*{Занятие 18. Условное математическое ожидание}

\addcontentsline{toc}{section}{Контрольные вопросы и задания}
\section*{Контрольные вопросы и задания}

\subsubsection*{Приведите определение условного математического ожидания.}

Условным математическим ожиданием случайной величины $ \xi $ относительно $ \sigma $-алгебры
$ \mathbb{F'}$ называется случайная величина $ \eta $ такая, что:
\begin{enumerate}
  \item $ \eta $ измерима относительно $ \mathbb{F'}$;
  \item для любой случайной величины $ \zeta $, измеримой относительно $ \mathbb{F'}$ и ограниченной,
  $M \xi \zeta = M \eta \zeta $.
\end{enumerate}

\subsubsection*{Запишите формулы для вычисления условного математического ожидания.}

$ \sigma $-алгебра $ \mathbb{F'}$ --- конечная.
Есть набор $H_1, \dotsc, H_n$ такой, что
$$ \bigcup \limits_{k = 1}^n H_k = \Omega, \,
  H_k \cap H_j = \emptyset, \,
  k \neq j, \,
  P \left( H_k \right) > 0.$$

Уловное математическое ожидание случайной величины $ \xi $ относительно
$$ \mathbb{F'} =
  \left\{ \bigcup \limits_{k = 1}^n H_k^{ \varepsilon_k} \right\},$$
где
$$ \varepsilon_k =
  \begin{cases}
    0, \\
    1,
  \end{cases}
  H_k^0 = \emptyset, \,
  H_k^1 = H_k,$$
имеет вид
$$ \eta =
  \sum \limits_{k = 1}^n
    \frac{M \xi \cdot \mathbbm{1}_{H_k}}{P \left( H_k \right) } \cdot \mathbbm{1}_{H_k}.$$

\subsubsection*{Сформулируйте свойства условного математического ожидания.}

\begin{enumerate}
  \item $ \mathbb{F'} = \left\{ \emptyset, \Omega \right\} $.
  Из этого следует, что $M \left( \xi \; \middle| \; \mathbb{F'} \right) = M \xi $;
  \item $ \mathbb{F'} \subset \mathbb{F"} \subset \mathbb{F}$.
  Тогда
  $M \left( M \left( \xi \; \middle| \; \mathbb{F"} \right) \; \middle| \; \mathbb{F'} \right) =
    M \left( \xi \; \middle| \; \mathbb{F'} \right) $;
  \item $M \left( M \left( \xi \; \middle| \; \mathbb{F'} \right) \right) = M \xi $;
  \item $ \zeta $ измерима относительно $ \mathbb{F'}$.
  Тогда
  $M \left( \xi \zeta \; \middle| \; \mathbb{F'} \right) =
    \zeta M \left( \xi \; \middle| \; \mathbb{F'} \right) $;
  \item $ \xi $ не зависит от $ \mathbb{F'}$, следовательно,
  $M \left( \xi \; \middle| \; \mathbb{F'} \right) =
    M \xi $;
  \item $M \left( \alpha_1 \xi_1 + \alpha_2 \xi_2 \; \middle| \; \mathbb{F'} \right) =
    \alpha_1 M \left( \xi_1 \; \middle| \; \mathbb{F'} \right) +
    \alpha_2 M \left( \xi_2 \; \middle| \; \mathbb{F'} \right) $;
  \item если $ \xi \geq 0$, то $M \left( \xi \; \middle| \; \mathbb{F'} \right) \geq 0$;
  \item $ \left| M \left( \xi \; \middle| \; \mathbb{F'} \right) \right| \leq
    M \left( \left| \xi \right| \; \middle| \; \mathbb{F'} \right) $,
  так как модуль --- выпуклая вниз функция;
  \item если $f$ --- выпуклая вниз функция,
  то $f \left( M \left( \xi \; \middle| \; \mathbb{F'} \right) \right) \leq
    M \left( f \left( \xi \right) \; \middle| \; \mathbb{F'} \right) $;
  \item предельные теоремы.
  Если $ \xi_n \geq 0, \, \xi_n \nearrow \xi $ (возрастает) при $n \to \infty $,
  то
  $M \left( \xi_n \; \middle| \; \mathbb{F'} \right) \nearrow
    M \left( \xi \; \middle| \; \mathbb{F'} \right) $
  почти наверное, $n \to \infty $;
  \item если $ \left| \xi_n \right| \leq \eta, \, \exists M \eta $ и
  $ \xi_n \overset{P} \xi, \, n \to \infty $,
  то
  $M \left( \xi_n \; \middle| \; \mathbb{F'} \right) \overset{P}{ \to }
    M \left( \xi \; \middle| \; \mathbb{F'} \right), \,
    n \to \infty.$
\end{enumerate}

\addcontentsline{toc}{section}{Аудиторные задачи}
\section*{Аудиторные задачи}

\subsubsection*{18.3}

\textit{Задание.}
На вероятностном пространстве $ \left( \Omega, \mathbb{F}, P \right) $,
где $ \Omega = \left[ 0, 1 \right], \mathbb{F}$ --- мера Лебега,
задана случайная величина $ \xi = \omega $.

Пусть $ \sigma $-алгебра $ \mathbb{B}$ порождена множествами
$$ \left[ 0, \frac{1}{3} \right), \,
  \left\{ \frac{1}{3} \right\}, \,
  \left( \frac{1}{3}, \frac{1}{2} \right).$$
Вычислите условное математическое ожидание $M \left[ \xi \; \middle| \; \mathbb{B} \right] $.

\textit{Решение.} Мера Лебега в данном случае --- это длина.

Есть 3 атома
$$ \sigma \left( \beta \right) =
  \left\{
    \left[ 0, \frac{1}{3} \right), \,
    \left[ \frac{1}{3}, \frac{1}{2} \right), \,
    \left[ \frac{1}{2}, 1 \right]
  \right\}.$$

Условное математическое ожидание ---
это случайная величина
$$ \eta =
  M \left[ \xi \; \middle| \; \mathbb{B} \right].$$
Будет 3 слагаемых
\begin{equation*}
  \begin{split}
    \eta = \\
    = \frac{M \left( \omega \cdot \mathbbm{1} \left\{ \omega \in \left[ 0, \frac{1}{3} \right] \right\} \right) }{ \frac{1}{3}} \cdot
    \mathbbm{1} \left\{ \omega \in \left[ 0, \frac{1}{3} \right] \right\} + \\
    + \frac{M \left( \omega \cdot \mathbbm{1} \left\{ \omega \in \left[ \frac{1}{3}, \frac{1}{2} \right) \right\} \right) }{ \frac{1}{6}} \cdot
    \mathbbm{1} \left\{ \omega \in \left[ \frac{1}{3}, \frac{1}{2} \right) \right\} + \\
    + \frac{M \left( \omega \cdot \mathbbm{1} \left\{ \omega \in \left[ \frac{1}{2}, 1 \right] \right\} \right) }{ \frac{1}{2}} \cdot
    \mathbbm{1} \left\{ \omega \in \left[ \frac{1}{2}, 1 \right] \right\}.
  \end{split}
\end{equation*}
Числители --- интегралы в соответствующих пределах
\begin{equation*}
  \begin{split}
    \eta = \\
    = 3 \int \limits_0^{ \frac{1}{3}} xdx \cdot
    \mathbbm{1} \left\{ \omega \in \left[ 0, \frac{1}{3} \right) \right\} +
    6 \int \limits_{ \frac{1}{3}}^{ \frac{1}{2}} xdx \cdot
    \mathbbm{1} \left\{ \omega \in \left[ \frac{1}{3}, \frac{1}{2} \right) \right\} + \\
    + 2 \int \limits_{ \frac{1}{2}}^1 xdx \cdot
    \mathbbm{1} \left\{ \omega \in \left[ \frac{1}{2}, 1 \right] \right\} = \\
    = \frac{1}{6} \cdot \mathbbm{1} \left\{ \omega \in \left[ 0, \frac{1}{3} \right) \right\} +
    \frac{5}{12} \cdot
    \mathbbm{1} \left\{ \omega \in \left[ \frac{1}{3}, \frac{1}{2} \right) \right\} +
    \frac{3}{4} \cdot \mathbbm{1} \left\{ \omega \in \left[ \frac{1}{2}, 1 \right] \right\}.
  \end{split}
\end{equation*}

На каждом атоме случайная величина принимает своё значение.

\subsubsection*{18.4}

\textit{Задание.} Пусть $ \xi $ и $ \eta $ --- независимые случайные величины.
Вычислите $D \left( M \left[ \xi \eta \; \middle| \; \xi \right] \right) $,
если $ \xi $ равномерно распределена на отрезке $ \left[ 0, 1 \right] $,
а $ \eta $ имеет нормальное распределение с параметрами $1, \sigma^2$.

\textit{Решение.} Упростим случайную величину
$M \left( \xi \eta \; \middle| \; \xi \right) =
  M \left[ \xi \eta \; \middle| \; \sigma \left( \xi \right) \right] $.
Случайная величина $ \xi $ измерима относительно своей же $ \sigma $-алгебры
$ \sigma \left( \xi \right) $.
Используем свойство условного математического ожидания, которое говорит,
что в данном случае можем вынести $ \xi $ за знак условного математического ожидания
$M \left[ \xi \eta \; \middle| \; \sigma \left( \xi \right) \right] =
  \xi M \left[ \eta \; \middle| \; \sigma \left( \xi \right) \right].$
Случайные величины $ \xi $ и $ \eta $ независимы, поэтому $ \eta $ не зависит от $ \sigma $-алгебры,
порождённой $ \xi $.
По свойствам условного математического ожидания
$ \xi M \left[ \eta \; \middle| \; \sigma \left( \xi \right) \right] =
  \xi M \eta =
  \xi a$,
где
$$a = const.$$

Ищем дисперсию
$D \left( M \left[ \xi \eta \; \middle| \; \xi \right] \right) =
  D \left( a \xi \right) =
  a^2 D \xi.$
Случайная величина имеет равномерное распределение
$$a^2 D \xi =
  \frac{a^2}{12}.$$

\subsubsection*{18.5}

\textit{Задание.} Пусть $ \xi $ имеет стандартное нормальное распределение.
Вычислите условное математическое ожидание:
\begin{enumerate}[label=\alph*)]
  \item $M \left( \xi \; \middle| \; \left| \xi \right| \right) $;
  \item $M \left( \xi^2 \; \middle| \; \left| \xi \right| \right) $.
\end{enumerate}

\textit{Решение.}
\begin{enumerate}[label=\alph*)]
  \item $M \left( \xi \; \middle| \; \left| \xi \right| \right) =
    \eta $.

  Не можем $ \xi $ хорошей функцией выразить через $ \left| \xi \right| $.

  Случайная величина $ \eta $ должна удовлетворять свойствам:
  \begin{enumerate}
    \item $ \eta $ --- измеримая относительно $ \sigma $-алгебры, порождённо $ \left| \xi \right| $,
    то есть $ \sigma \left( \left| \xi \right| \right) $;
    \item для любого множества
    $A \in \sigma \left( \left| \xi \right| \right): \,
    M \left( \xi \cdot \mathbbm{1}_A \right) = M \left( \eta \cdot \mathbbm{1}_A \right) $.
  \end{enumerate}

  Вычисляем математическое ожидание
  $$M \left( \xi \cdot \mathbbm{1}_A \right) =
    \int \limits_{- \infty }^{+ \infty }
      x \cdot \mathbbm{1}_A \left( x \right) p \left( x \right) dx,$$
  где $p \left( x \right) $ --- плотность нормального распределения, $A$ ---
  симметричная относительно нуля область (множество).
  Это некоторая функция от $ \left| x \right| $.
  Учитывая это, получаем
  $$ \int \limits_{- \infty }^{+ \infty }
      x \cdot \mathbbm{1}_A \left( x \right) p \left( x \right) dx =
    \int \limits_{- \infty }^{+ \infty }
      x \cdot \varphi \left( \left| x \right| \right) p \left( x \right) dx.$$
  Функция $ \varphi \left( \left| x \right| \right) $ --- чётная, $p \left( x \right) $ --- чётная,
  $x$ --- нечётная.
  Интегрируем по всей оси
  $$ \int \limits_{- \infty }^{+ \infty }
      x \cdot \varphi \left( \left| x \right| \right) p \left( x \right) dx =
    0,$$
  следовательно, $ \eta = 0$;

  \item ищем
  $M \left( \xi^2 \; \middle| \; \left| \xi \right| \right) =
    M \left[ \xi^2 \; \middle| \; \sigma \left( \left| \xi \right| \right) \right] $.
  Перепишем случайную величину $ \xi^2 = \left| \xi \right|^2$, откуда следует, что $ \xi^2$ ---
  измеримая относительно $ \sigma $-алгебры $ \sigma \left( \left| \xi \right| \right) $.
  Согласно со свойствами
  $M \left[ \xi^2 \; \middle| \; \sigma \left( \left| \xi \right| \right) \right] =
    \xi^2.$
\end{enumerate}

\subsubsection*{18.6}

\textit{Задание.}
Пусть независимые случайные величины $ \xi $ и $ \eta $ имеют распределение Пуассона с
параметром $ \lambda $ каждая.
Вычислите:
\begin{enumerate}[label=\alph*)]
  \item условное распределение $P \left( \xi = k \; \middle| \; \xi + \eta = n \right) $;
  \item условное математическое ожидание $M \left[ \xi^2 \; \middle| \; \xi + \eta \right] $.
\end{enumerate}

\textit{Решение.}
$$M \left[ \xi^2 \; \middle| \; \xi + \eta \right] =
  M \left( \xi^2 \; \middle| \; \xi + \eta = t \right).$$

\begin{enumerate}[label=\alph*)]
  \item По определению условной вероятности
  $$P \left( \xi = k \; \middle| \; \xi + \eta = t \right) =
    \frac{P \left( \xi = k, \, \xi + \eta = t \right) }{P \left( \xi + \eta = t \right) }.$$
  Сумма пуассоновских случайных величин имеет распределение Пуассона с параметром,
  который равен сумме параметров распределений случайных величин, входящих в сумму
  $$ \frac{P \left( \xi = k, \, \xi + \eta = t \right) }{P \left( \xi + \eta = t \right) } =
    \frac{P \left( \xi = k \right) P \left( \eta = t - k \right) }{P \left( \zeta = t \right) } =
    \frac{ \frac{ \lambda^k e^{- \lambda }}{k!} \cdot \frac{ \lambda^{t - k} e^{- \lambda }}{ \left( t - k \right)!}}{ \left( 2 \lambda \right)^y e^{-2 \lambda }} \cdot
    t!.$$
  Сокращаем
  $$ \frac{ \frac{ \lambda^k e^{- \lambda }}{k!} \cdot \frac{ \lambda^{t - k} e^{- \lambda }}{ \left( t - k \right)!}}{ \left( 2 \lambda \right)^y e^{-2 \lambda }} \cdot
    t! =
    C_t^k \cdot \frac{1}{2^t} =
    C_t^k \cdot 2^{-t},$$
  где $ \zeta = \xi + \eta \sim Pois \left( 2 \lambda \right) $;
  \item условное математическое ожидание равно
  $$M \left( \xi^2 \; \middle| \; \xi + \eta = t \right) =
    \sum \limits_{k = 0}^t k^2 \cdot \frac{t! \cdot 2^{-t}}{k! \cdot \left( t - k \right)!} =
    2^{-t} \sum \limits_{k = 1}^t \frac{kt!}{ \left( k - 1 \right)! \left( t - k \right)!}.$$
  Разбиваем на две суммы, $k$ представляем как $\left( k - 1 + 1 \right) $.
  Получем
  \begin{equation*}
    \begin{split}
      2^{-t} \sum \limits_{k = 1}^t \frac{kt!}{ \left( k - 1 \right)! \left( t - k \right)!} = \\
      = 2^{-t} \cdot
      \sum \limits_{k = 1}^t
        \frac{ \left( t - 2 \right)!}{ \left( k - 2 \right)! \left( t - k \right)!} \cdot
      \left( t - 1 \right) +
      2^{-t} \cdot
      \sum \limits_{k = 2}^t
        \frac{ \left( t - 1 \right)!}{ \left( k - 1 \right)! \left( t - k \right)!} = \\
      = 2^{-t} \left[ t \left( t - 1 \right) \cdot 2^{t - 2} + t \cdot 2^{t - 1} \right] =
      \frac{t \left( t - 1 \right) }{4} + \frac{t}{2} =
      \frac{t^2 + t}{4}.
    \end{split}
  \end{equation*}
  Отсюда следует, что
  $$M \left( \xi^2 \; \middle| \; \xi + \eta \right) =
    \frac{1}{4} \left[ \left( \xi + \eta \right)^2 + \xi + \eta \right].$$
\end{enumerate}

\subsubsection*{18.8}

\textit{Задание.}
Пусть $ \xi $ и $ \eta $ --- независимые случайные величины,
равномерно распределённые на отрезке $ \left[ 0, 1 \right] $.
Вычислите:
\begin{enumerate}[label=\alph*)]
  \item $M \left[ \xi^2 - \eta^2 \; \middle| \; \xi + \eta \right] $;
  \item $M \left[ \xi \; \middle| \; \xi + 2 \eta \right] $.
\end{enumerate}

\textit{Решение.}
\begin{enumerate}[label=\alph*)]
  \item Распишем разность квадратов
  $$M \left[ \xi^2 - \eta^2 \; \middle| \; \xi + \eta \right] =
    M \left[
      \left( \xi - \eta \right) \left( \xi + \eta \right) \; \middle| \; \xi + \eta
    \right].$$
  Случайная величина $ \left( \xi + \eta \right) $ измерима относительно
  $ \sigma \left( \xi + \eta \right) $,
  потому её можно вынести за знак условного математического ожидания:
  $M \left[
      \left( \xi - \eta \right) \left( \xi + \eta \right) \; \middle| \; \xi + \eta
    \right] =
    \left( \xi + \eta \right) \left[ M \left( \xi \; \middle| \; \xi + \eta \right) -
    M \left( \eta \; \middle| \; \xi + \eta \right) \right] $.

  Найти распределение суммы можем.
  Чтобы вычислить первое математическое ожидание,
  необходимо найти распределение вектора $ \left( \xi, \xi + \eta \right) $.

  Чтобы найти второе математическое ожидание,
  нужно найти распределение $ \left( \eta, \eta + \xi \right) $.
  Случайные величины $ \xi $ и $ \eta $ --- независимы и одинаково распределённые.
  Из этого следует, что распределение векторов одинаковое,
  потому что $ \xi $ и $ \eta $ одинаково входят в функцию
  $$ \xi + \eta,$$
  то есть
  $ \left( \xi, \xi + \eta \right) \sim \left( d \right) \,
    \left( \eta, \eta + \xi \right) $.
  Отсюда следует, что эти математические ожидание совпадают:
  $M \left( \xi \; \middle| \; \xi + \eta \right) =
    M \left( \eta \; \middle| \; \xi + \eta \right) $.
  Тогда
  $$ \left( \xi + \eta \right) \left[ M \left( \xi \; \middle| \; \xi + \eta \right) -
    M \left( \eta \; \middle| \; \xi + \eta \right) \right] =
    \left( \xi + \eta \right) \cdot 0 =
    0;$$
  \item по определению
  $M \left[ \xi \; \middle| \; \xi + 2 \eta \right] =
    f \left( \xi + 2 \eta \right) $.

  Как и в предыдущем пункте
  $$f \left( y \right) =
    M \left( \xi \; \middle| \; \xi + 2 \eta = y \right) =
    \frac{ \int \limits_{ \mathbb{R}} xp_{ \left( \xi, \xi + 2 \eta \right) } \left( x, y \right) dx}{ \int \limits_{ \mathbb{R}} p_{ \left( \xi, \xi + 2 \eta \right) } \left( x, y \right) dx}.$$

  Нужно найти плотность такого вектора.
  Начнём с функции распределения и дифференцируем, находя плотность.

  По определению
  $F_{ \left( \xi, \xi + 2 \eta \right) } \left( x, y \right) =
    P \left\{ \xi \leq x, \xi + 2 \eta \leq y \right\} $.
  Это математическое ожидание индикатора
  $$P \left\{ \xi \leq x, \xi + 2 \eta \leq y \right\} =
    M \mathbbm{1} \left\{ \xi \leq x, \xi + 2 \eta \leq y \right\}.$$
  Есть две случайные величины, следовательно, есть 2 интеграла
  $$M \mathbbm{1} \left\{ \xi \leq x, \xi + 2 \eta \leq y \right\} =
    \iint \limits_{ \mathbb{R}^2}
      \mathbbm{1} \left\{ z \leq x, z + 2t \leq y \right\} \cdot
      p_{ \left( \xi, \eta \right) } \left( z, t \right)
    dzdt.$$
  По условию $ \xi $ и $ \eta $ --- независимые, следовательно,
  совместная плотность равна произведению плотностей
  \begin{equation*}
    \begin{split}
      \iint \limits_{ \mathbb{R}^2}
        \mathbbm{1} \left\{ z \leq x, z + 2t \leq y \right\} \cdot
        p_{ \left( \xi, \eta \right) } \left( z, t \right)
      dzdt = \\
      = \iint \limits_{ \mathbb{R}^2}
        \mathbbm{1} \left\{ x \leq x, z + 2t \leq y \right\} \cdot
        \mathbbm{1} \left\{ z \in \left[ 0, 1 \right] \right\} \cdot
        \mathbbm{1} \left\{ t \in \left[ 0, 1 \right] \right\}
      dzdt = \\
      = \int \limits_0^x
        \int \limits_0^{ \frac{y - z}{2}} p_{ \xi } \left( z \right) p_{ \eta } \left( t \right) dt
      dz.
    \end{split}
  \end{equation*}

  Найдём плотность.
  Теперь нужно продифференцировать по $x$ и по $y$, то есть
  $$p_{ \left( \xi, \xi + 2 \eta \right) } \left( x, y \right) =
    \frac{ \partial^2 F_{ \left( \xi, \xi + 2 \eta \right) } \left( x, y \right) }{ \partial x \partial y}.$$
  Дифференцируем по $x$ как интеграл по сменным верхним пределом
  $$ \frac{ \partial^2 F_{ \left( \xi, \xi + 2 \eta \right) } \left( x, y \right) }{ \partial x \partial y} =
    \frac{ \partial }{ \partial y} \left[
      \int \limits_0^{ \frac{y - x}{2}} p_{ \xi } \left( x \right) p_{ \eta } \left( t \right) dt
    \right].$$
  Дифференцируем по $y$ как интеграл со сменным верхним пределом
  $$ \frac{ \partial }{ \partial y} \left[
      \int \limits_0^{ \frac{y - x}{2}} p_{ \xi } \left( x \right) p_{ \eta } \left( t \right) dt
    \right] =
    \frac{1}{2} \cdot p_{ \xi } \left( x \right) p_{ \eta } \left( \frac{y - x}{2} \right).$$
  Подставляем значения плотностей
  $$ \frac{1}{2} \cdot p_{ \xi } \left( x \right) p_{ \eta } \left( \frac{y - x}{2} \right) =
    \frac{1}{2} \cdot
    \mathbbm{1} \left\{ x \in \left[ 0, 1 \right] \right\} \cdot
    \mathbbm{1} \left\{ \frac{y - x}{2} \in \left[ 0, 1 \right] \right\}.$$
  Из последнего индикатора следует, что $y - 2 \leq x \leq y$.

  Нашли совместную плотность.
  Сначала найдём знаменатель, то есть проигтегрируем, чтобы найти одномерную плотность.
  Нужно интегрировать по пересечению $ \left[ 0, 1 \right] \cap \left[ y - 2, y \right] $, то есть
  $$ \int \limits_{ \mathbb{R}} p_{ \left( \xi, \xi + 2 \eta \right) } \left( x, y \right) dx =
    \int \limits_{ \left[ 0, 1 \right] \cap \left[ y - 2, y \right] } \frac{1}{2} dx =
    \begin{cases}
      \frac{y}{2}, \qquad y \in \left[ 0, 1 \right], \\
      \frac{1}{2}, \qquad y \in \left[ 1, 2 \right], \\
      \frac{3 - y}{2}, \qquad y \in \left[ 2, 3 \right].
    \end{cases}$$

  Находим числитель
  $$ \int \limits_{ \mathbb{R}} xp_{ \left( \xi, \xi + 2 \eta \right) } \left( x, y \right) dx =
    \int \limits_{ \left[ 0, 1 \right] \cap \left[ y - 2, y \right] } \frac{x}{2} dx.$$
  Вычисляем интеграл
  $$ \int \limits_{ \left[ 0, 1 \right] \cap \left[ y - 2, y \right] } \frac{x}{2} dx =
    \begin{cases}
      \frac{y^2}{4}, \qquad y \in \left[ 0, 1 \right], \\
      \frac{1}{4}, \qquad y \in \left[ 1, 2 \right], \\
      \frac{1 - \left( y - 2 \right)^2}{4} =
      \frac{ \left( 3 - y \right) \left( y - 1 \right) }{4}, \qquad y \in \left[ 2, 3 \right].
    \end{cases}$$

  Будем делить на каждом интервале на своё значение.
  Таким образом,
  $$f \left( y \right) =
    \begin{cases}
      \frac{y}{2}, \qquad y \in \left[ 0, 1 \right], \\
      \frac{1}{2}, \qquad y \in \left[ 1, 2 \right], \\
      \frac{y - 1}{2}, \qquad y \in \left[ 2, 3 \right].
    \end{cases}$$

  Нужно написать ответ
  $$f \left( \xi + 2 \eta \right) =
    \begin{cases}
      \frac{ \xi + 2 \eta }{2}, \qquad \xi + 2 \eta \in \left[ 0, 1 \right], \\
      \frac{1}{2}, \qquad \xi + 2 \eta \in \left[ 1, 2 \right], \\
      \frac{ \xi + 2 \eta - 1}{2}, \qquad \xi + 2 \eta \in \left[ 2, 3 \right].
    \end{cases}$$
\end{enumerate}

\subsubsection*{18.9}

\textit{Задание.}
Пусть случайная величина $ \xi $ имеет показательное распределение с параметром 1, а $t > 0$.
Вычислите:
\begin{enumerate}[label=\alph*)]
  \item $M \left[ \xi \; \middle| \; \min \left( \xi, t \right) \right] $;
  \item $M \left[ \xi \; \middle| \; \max \left( \xi, t \right) \right] $.
\end{enumerate}

\textit{Решение.}
\begin{enumerate}[label=\alph*)]
  \item $t$ --- фиксированное число.
  Решим по определению
  $$M \left[ \xi \; \middle| \; \min \left( \xi, t \right) \right] =
    \eta =
    f \left( \min \left( \xi, t \right) \right),$$
  где функция $f$ --- борелевская, такая, что:
  $$M \left[ \xi \varphi \left( \min \left( \xi, t \right) \right) \right] =
    M \left[ f \left( \min \left( \xi, t \right) \right) \cdot
    \varphi \left( \min \left( \xi, t \right) \right) \right] $$
  для произвольной ограниченной борелевой $ \varphi $.

  В явном виде вычислим оба математических ожидания и их приравняем
  $$M \left[ \xi \varphi \left( \min \left( \xi, t \right) \right) \right] =
    \int \limits_0^{+ \infty } x \varphi \left( \min \left( x, t \right) \right) e^{-x} dx.$$
  Разбиваем интеграл на 2
  $$ \int \limits_0^{+ \infty } x \varphi \left( \min \left( x, t \right) \right) e^{-x} dx =
    \int \limits_0^t x \varphi \left( x \right) e^{-x} dx +
    \varphi \left( t \right) \int \limits_t^{+ \infty } xe^{-x} dx.$$
  Второй интеграл вычисляем в явном виде
  \begin{equation*}
    \begin{split}
      \int \limits_0^t x \varphi \left( x \right) e^{-x} dx +
      \varphi \left( t \right) \int \limits_t^{+ \infty } xe^{-x} dx = \\
      = \int \limits_0^t x \varphi \left( x \right) e^{-x} dx +
      \varphi \left( t \right) \left( \left. te^{-t} - e^{-x} \right|_t^{+ \infty } \right) = \\
      = \int \limits_0^t x \varphi \left( x \right) e^{-x} dx +
      \varphi \left( t \right) e^{-t} \left( t + 1 \right).
    \end{split}
  \end{equation*}

  В явном виде вычисляем правую часть
  $$M \left[
    f \left( \min \left( \xi, t \right) \right) \varphi \left( \min \left( \xi, t \right) \right)
  \right] =
  \int \limits_0^{+ \infty }
    f \left( \min \left( x, t \right) \right) \varphi \left( \min \left( x, t \right) \right) e^{-x}
  dx.$$
  Разбиваем на 2
  \begin{equation*}
    \begin{split}
      \int \limits_0^{+ \infty }
        f \left( \min \left( x, t \right) \right) \cdot
        \varphi \left( \min \left( x, t \right) \right) e^{-x}
      dx = \\
      = \int \limits_0^t f \left( x \right) \varphi \left( x \right) e^{-x} dx +
      f \left( t \right) \varphi \left( t \right) \int \limits_t^{+ \infty } e^{-x} dx = \\
      = \int \limits_0^t f \left( x \right) \varphi \left( x \right) e^{-x} dx +
      f \left( t \right) \varphi \left( t \right) e^{-t}.
    \end{split}
  \end{equation*}

  Нужно подобрать $f \left( x \right) $ так, чтобы математические ожидания совпадали:
  $f \left( x \right) =
    x \cdot \mathbbm{1} \left\{ x < t \right\} +
    \left( t + 1 \right) \cdot \mathbbm{1} \left\{ x \geq t \right\} $.

  Вместо $x$ нужно подставить $ \min \left( \xi, t \right) $.
  Это даст возможность упростить выражение
  $$f \left( \min \left( \xi, t \right) \right) =
    \min \left( \xi, t \right) \cdot \mathbbm{1} \left\{ \min \left( \xi, t \right) < t \right\} +
    \left( t + 1 \right) \cdot \mathbbm{1} \left\{ \min \left( \xi, t \right) \geq t \right\}.$$
  Упрощаем неравенства
  \begin{equation*}
    \begin{split}
      \min \left( \xi, t \right) \cdot \mathbbm{1} \left\{ \min \left( \xi, t \right) < t \right\} +
      \left( t + 1 \right) \cdot \mathbbm{1} \left\{ \min \left( \xi, t \right) \geq t \right\} = \\
      = \xi \cdot \mathbbm{1} \left\{ \xi < t \right\} +
      \left( t + 1 \right) \cdot \mathbbm{1} \left\{ \xi \geq t \right\};
    \end{split}
  \end{equation*}
  \item ищем
  $M \left[ \xi \; \middle| \; \max \left( \xi, t \right) \right] =
    \eta =
    f \left( \max \left( \xi, t \right) \right) $.

  Находим $f$ из условия
  $$M \left[ \xi \varphi \left( \max \left( \xi, t \right) \right) \right] =
    M \left[
      f \left( \max \left( \xi, t \right) \right) \cdot
      \varphi \left( \max \left( \xi, t \right) \right)
    \right],$$
  где $ \varphi $ --- произвольная ограниченная борелевская функция.
  Сначала найдём левую часть
  $$M \left[ \xi \varphi \left( \max \left( \xi, t \right) \right) \right] =
    \int \limits_0^{+ \infty } x \varphi \left( \max \left( x, t \right) \right) e^{-x} dx.$$
  Разбиваем на 2 интеграла согласно функции максимума
  $$ \int \limits_0^{+ \infty } x \varphi \left( \max \left( x, t \right) \right) e^{-x} dx =
    \int \limits_0^t x \varphi \left( t \right) e^{-x} dx +
    \int \limits_t^{+ \infty } x \varphi \left( x \right) e^{-x} dx.$$
  Берём первый интеграл, второй оставляем без изменений
  $$ \int \limits_0^t x \varphi \left( t \right) e^{-x} dx +
    \int \limits_t^{+ \infty } x \varphi \left( x \right) e^{-x} dx =
    \left[ 1 - e^{-t} \left( t + 1 \right) \right] \varphi \left( t \right) +
    \int \limits_t^{+ \infty } x \varphi \left( x \right) e^{-x} dx.$$

  Теперь вычисляем правую часть
  $$M \left[
      f \left( \max \left( \xi, t \right) \right) \varphi \left( \max \left( \xi, t \right) \right)
    \right] =
    \int \limits_0^{+ \infty }
      f \left( \max \left( x, t \right) \right) \cdot
      \varphi \left( \max \left( x, t \right) \right) e^{-x}
    dx.$$
  Разбиваем на 2 интеграла согласно функции максимума
  \begin{equation*}
    \begin{split}
      \int \limits_0^{+ \infty }
        f \left( \max \left( x, t \right) \right) \cdot
        \varphi \left( \max \left( x, t \right) \right) e^{-x}
      dx = \\
      = \int \limits_0^t f \left( t \right) \varphi \left( t \right) e^{-x} dx +
      \int \limits_t^{+ \infty } f \left( x \right) \varphi \left( x \right) e^{-x} dx.
    \end{split}
  \end{equation*}
  Берём первый интеграл
  \begin{equation*}
    \begin{split}
      \int \limits_0^t f \left( t \right) \varphi \left( t \right) e^{-x} dx +
      \int \limits_t^{+ \infty } f \left( x \right) \varphi \left( x \right) e^{-x} dx = \\
      = f \left( t \right) \varphi \left( t \right) \left( 1 - e^{-t} \right) +
      \int \limits_t^{+ \infty } f \left( x \right) \varphi \left( x \right) e^{-x} dx.
    \end{split}
  \end{equation*}

  Подбираем функцию $f$ так, чтобы выполнялось равенство
  $$f \left( x \right) =
    \frac{1 - e^{-t} \left( t + 1 \right) }{1 - e^{-t}} \cdot \mathbbm{1} \left\{ x < t \right\} +
    x \cdot \mathbbm{1} \left\{ x \geq t \right\}.$$
  Записываем, когда вместо $x$ ставим $ \max \left( \xi, t \right) $.
  Получаем
  \begin{equation*}
    \begin{split}
      f \left( \max \left( \xi, t \right) \right) = \\
      = \frac{1 - e^{-t} \left( t + 1 \right) }{1 - e^{-t}} \cdot
      \mathbbm{1} \left\{ \max \left( \xi, t \right) < t \right\} +
      \max \left( \xi, t \right) \cdot
      \mathbbm{1} \left\{ \max \left( \xi, t \right) \geq t \right\} = \\
      = \frac{1 - e^{-t} \left( t + 1 \right) }{1 - e^{-t}} \cdot
      \mathbbm{1} \left\{ \xi < t \right\} +
      \xi \cdot \mathbbm{1} \left\{ \xi \geq t \right\}.
    \end{split}
  \end{equation*}
\end{enumerate}

\subsubsection*{18.11}

\textit{Задание.}
Вычислите $M \left[ \xi \; \middle| \; \eta \right] $,
если совместная плотность распределения случайного вектора $ \left( \xi, \eta \right) $ равна:
$$p \left( x, y \right) =
  \begin{cases}
    2ye^{-x} + 2e^{-2x}, \qquad x \geq 0, \, 0 \leq y \leq 1, \\
    0, \qquad otherwise.
  \end{cases}$$

\textit{Решение.} $M \left( \xi \; \middle| \; \eta \right) = f \left( \eta \right) $.

По определению
$$f \left( y \right) =
  M \left( \xi \; \middle| \; \eta = y \right) =
  \int \limits_{ \mathbb{R}} xp_{ \left. \xi \right| \eta } \left( \left. x \right| y \right) dx.$$
Распишем условную плотность
$$ \int \limits_{ \mathbb{R}} xp_{ \left. \xi \right| \eta } \left( \left. x \right| y \right) dx =
  \frac{ \int \limits_{ \mathbb{R}} xp_{ \left( \xi, \eta \right) } \left( x, y \right) dy}{ \int \limits_{ \mathbb{R}} p_{ \left( \xi, \eta \right) } \left( x, y \right) dx} =
  \frac{ \int \limits_{ \mathbb{R}} xp_{ \left( \xi, \eta \right) } \left( x, y \right) dy}{p_{ \eta } \left( t \right) }.$$
Подставим значения из условия
$$ \frac{ \int \limits_{ \mathbb{R}} xp_{ \left( \xi, \eta \right) } \left( x, y \right) dy}{p_{ \eta } \left( t \right) } =
  \frac{ \int \limits_0^{+ \infty } x \left( 2ye^{-x} + 2e^{-2x} \right) dx}{ \int \limits_0^{+ \infty } \left( 2ye^{-x} + 2e^{-2x} \right) dx} =
  \frac{2y + \frac{1}{2}}{2y + 1}.$$

Таким образом,
$$M \left( \xi \; \middle| \; \eta \right) =
  f \left( \eta \right) =
  \frac{2 \eta + \frac{1}{2}}{2 \eta + 1}.$$

\addcontentsline{toc}{section}{Домашнее задание}
\section*{Домашнее задание}

\subsubsection*{18.15}

\textit{Задание.}
На вероятностном пространстве $ \left( \omega, \mathcal{F}, P \right) $,
где $ \Omega = \left[ 0, 1 \right], \, \mathcal{F}$ ---
$ \sigma $-алгебра борелевских подмножетсв $ \Omega, \, P$ --- мера Лебега,
задана случайная величина $ \xi = \sin \pi \omega $.
Пусть $ \sigma $-алгебра $ \mathcal{B}$ порождена множествами
$$ \left[0, \frac{1}{3} \right), \,
  \left\{ \frac{1}{3} \right\}, \,
  \left( \frac{1}{3}, \frac{1}{2} \right).$$

Вычислите условное математическое ожидание $M \left[ \xi \; \middle| \; \mathcal{B} \right] $.

\textit{Решение.} Мера Лебега в данном случае --- это длина.

Есть 3 атома
$$ \sigma \left( \mathcal{B} \right) =
  \left\{
    \left[ 0, \frac{1}{3} \right), \,
    \left[ \frac{1}{3}, \frac{1}{2} \right), \,
    \left[ \frac{1}{2}, 1 \right]
  \right\}.$$

Условное математическое ожидание ---
это случайная величина
$$ \eta =
  M \left( \xi \; \middle| \; \mathcal{B} \right).$$

Будет 3 слагаемых,
умноженные на соответствующие
индикоторы и поделенные на вероятность попадания в соответствующий отрезок.

Первое слагаемое имеет вид
$$ \int \limits_0^1
  \sin \left( \pi \omega \right) \cdot
  \mathbbm{1} \left\{ \omega \in \left[ 0, \frac{1}{3} \right) \right\} d \omega =
  \int \limits_0^{ \frac{1}{3}} \sin \left( \pi \omega \right) d \omega =
  \frac{1}{ \pi } \cdot
  \int \limits_0^{ \frac{1}{3}} \sin \left( \pi \omega \right) d \left( \pi \omega \right).$$
Берём интеграл
$$ \frac{1}{ \pi } \cdot
  \int \limits_0^{ \frac{1}{3}} \sin \left( \pi \omega \right) d \left( \pi \omega \right)
  \left. - \frac{1}{ \pi } \cdot \cos \left( \pi \omega \right) \right|_0^{ \frac{1}{3}} =
  - \frac{1}{ \pi } \cdot \cos \left( \frac{ \pi }{3} \right) + \frac{1}{ \pi } \cdot \cos 0=
  - \frac{1}{ \pi } \cdot \frac{1}{2} + \frac{1}{ \pi } =
  \frac{1}{2 \pi }.$$

Второе слагаемое имеет вид
$$ \int \limits_0^1
  \sin \left( \pi \omega \right) \cdot
  \mathbbm{1} \left\{ \omega \in \left[ \frac{1}{3}, \frac{1}{2} \right) \right\} d \omega =
  \int \limits_{ \frac{1}{3}}^{ \frac{1}{2}} \sin \left( \pi \omega \right) d \omega =
  \left.
    - \frac{1}{ \pi } \cdot \cos \left( \pi \omega \right)
  \right|_{ \frac{1}{3}}^{ \frac{1}{2}}.$$
Подставляем пределы интегрирования
$$ \left.
    - \frac{1}{ \pi } \cdot \cos \left( \pi \omega \right)
  \right|_{ \frac{1}{3}}^{ \frac{1}{2}} =
  - \frac{1}{ \pi } \cdot \cos \frac{ \pi }{2} + \frac{1}{ \pi } \cdot \cos \frac{ \pi }{3} =
  - \frac{1}{ \pi } \cdot 0 + \frac{1}{ \pi } \cdot \frac{1}{2} =
  \frac{1}{2 \pi }.$$

Аналогично получаем третье слагаемое
$$ \int \limits_0^1
    \sin \left( \pi \omega \right) \cdot
    \mathbbm{1} \left\{ \omega \in \left[ \frac{1}{2}, 1 \right] \right\} d \omega =
  \int \limits_{ \frac{1}{2}}^1 \sin \left( \pi \omega \right) d \omega =
  \left. - \frac{1}{ \pi } \cdot \cos \left( \pi \omega \right) \right|_{ \frac{1}{2}}^1.$$
Подставляем пределы интегрирования
$$ \left. - \frac{1}{ \pi } \cdot \cos \left( \pi \omega \right) \right|_{ \frac{1}{2}}^1 =
  - \frac{1}{ \pi } \cdot \cos \pi + \frac{1}{ \pi } \cdot \cos \frac{ \pi }{2} =
  - \frac{1}{ \pi } \cdot \left( -1 \right) + \frac{1}{ \pi } \cdot 0 =
  \frac{1}{ \pi }.$$

Получаем условное математическое ожидание
$$ \eta =
  \frac{3}{2 \pi } \cdot
  \mathbbm{1} \left\{ \omega \in \left[ 0, \frac{1}{3} \right) \right\} +
  \frac{6}{2 \pi } \cdot
  \mathbbm{1} \left\{ \omega \in \left[ \frac{1}{3}, \frac{1}{2} \right) \right\} +
  \frac{2}{ \pi } \cdot
  \mathbbm{1} \left\{ \omega \in \left[ \frac{1}{2}, 1 \right) \right\}.$$

Случайная величина на каждом атоме принимает своё значение.

\subsubsection*{18.16}

\textit{Задание.} Пусть $ \xi $ и $ \eta $ --- независимые случайные величины.
Вычислите $D \left( M \left[ \xi + \eta \; \middle| \; \xi \right] \right) $,
если $ \xi $ и $ \eta $ имеют показательное распределение с параметрами $ \lambda $
и $ \mu $ соответственно.

\textit{Решение.}
Воспользуемся линейностью условного математического ожидания
$M \left( \xi + \eta \; \middle| \; \xi \right) =
  M \left( \xi \; \middle| \; \xi \right) + M \left( \eta \; \middle| \; \xi \right) $.

Поскольку случайная величина $ \eta $ не зависит от $ \xi $, то
$$M \left( \eta \; \middle| \; \xi \right) =
  M \eta =
  \frac{1}{ \mu }.$$

Случайная величина $ \xi $ измерима относительно $ \sigma $-алгебры $ \sigma \left( \xi \right) $,
порождённой случайной величиной $ \xi $, а поэтому
$$M \left( \xi + \eta \; \middle| \; \xi \right) =
  \xi + \frac{1}{ \mu }.$$

Ищем дисперсию
$$D \left[ M \left( \xi + \eta \; \middle| \; \xi \right) \right] =
  D \left( \xi + \frac{1}{ \mu } \right) =
  D \xi + D \frac{1}{ \mu }.$$
Случайная величина $ \xi $ имеет показательное распределение
$$D \xi + D \frac{1}{ \mu } =
  \frac{1}{ \lambda^2} + 0 =
  \frac{1}{ \lambda^2}.$$

\subsubsection*{18.17}

\textit{Задание.} Пусть $ \xi $ имеет геометрическое распределение с параметром $p$.
Вычислите:
\begin{enumerate}[label=\alph*)]
  \item $M \left( \xi \; \middle| \; \xi^3 \right) $;
  \item $M \left( \xi \; \middle| \; \max \left( \xi, n \right) \right) $.
\end{enumerate}

\textit{Решение.}
\begin{enumerate}[label=\alph*)]
  \item Ищем
  $M \left( \xi \; \middle| \; \xi^3 \right) =
    M \left[ \xi \; \middle| \; \sigma \left( \xi^3 \right) \right] $.
  Перепишем случайную величину $ \xi = \sqrt[3]{ \xi^3}$, откуда следует, что $ \xi $ ---
  измеримая относительно $ \sigma $-алгебры $ \sigma \left( \xi^3 \right) $.
  Согласно со свойствами $M \left[ \xi \; \middle| \; \sigma \left( \xi^3 \right) \right] = \xi $;
  \item ищем
    $M \left( \xi \; \middle| \; \max \left( \xi, \eta \right) \right) =
      \eta =
      f \left( \max \left( \xi, n \right) \right) $.

    Находим $f$ из условия
    $$M \left[ \xi \varphi \left( \max \left( \xi, n \right) \right) \right] =
      M \left[
        f \left( \max \left( \xi, n \right) \right) \cdot
        \varphi \left( \max \left( \xi, n \right) \right)
      \right],$$
    где $ \varphi $ --- произвольная ограниченная борелевская функция.
    Сначала находим левую часть
    $$M \left[ \xi \varphi \left( \max \left( \xi, n \right) \right) \right] =
      \sum \limits_{x = 0}^{ \infty }
        x \varphi \left( \max \left( x, n \right) \right) p \left( 1 - p \right)^x.$$
    Разделим на две суммы согласно функции максимума
    $$ \sum \limits_{x = 0}^{ \infty } x \varphi \left( \max \left( x, n \right) \right) pq^x =
      \sum \limits_{x = 0}^n x \varphi \left( n \right) p \left( 1 - p \right)^x +
      \sum \limits_{x = n}^{ \infty } x \varphi \left( x \right) p \left( 1 - p \right)^x.$$
    Первое слагаемое в первой сумме нулевое, можно суммировать от единицы
    \begin{equation*}
      \begin{split}
        \sum \limits_{x = 0}^n x \varphi \left( n \right) p \left( 1 - p \right)^x +
        \sum \limits_{x = n}^{ \infty } x \varphi \left( x \right) p \left( 1 - p \right)^x = \\
        = \varphi \left( n \right) p \left( 1 - p \right) \cdot
        \sum \limits_{x = 1}^n x \left( 1 - p \right)^{x - 1} +
        p \sum \limits_{x = n}^{ \infty } x \varphi \left( x \right) \left( 1 - p \right)^x = \\
        = \varphi \left( n \right) p \left( 1 - p \right) \cdot
        \sum \limits_{x = 1}^n \left[ \left( 1 - p \right)^x \right]' +
        p \sum \limits_{x = n}^{ \infty } x \varphi \left( x \right) \left( 1 - p \right)^x = \\
        = \varphi \left( n \right) p \left( 1 - p \right) \cdot
        \left[ \sum \limits_{x = 1}^n \left( 1 - p \right)^x \right]' +
        p \sum \limits_{x = n}^{ \infty } x \varphi \left( x \right) \left( 1 - p \right)^x.
      \end{split}
    \end{equation*}
    В первом слагаемом имеем геометрическую прогрессию, сумма которой вычисляется по формуле
    $$S_n =
      \frac{b_1 \left( 1 - q^n \right) }{1 - q},$$
    где первый член прогрессии $b_1 = 1 - p$, знаменатель
    $$q =
      \frac{b_2}{b_1} =
      \frac{ \left( 1 - p \right)^2}{1 - p} =
      1 - p.$$
    Получаем
    \begin{equation*}
      \begin{split}
        \varphi \left( n \right) p \left( 1 - p \right) \cdot
        \left[ \sum \limits_{x = 1}^n  \left( 1 - p \right)^x \right]' +
        p \sum \limits_{x = n}^{ \infty } x \varphi \left( x \right)  \left( 1 - p \right)^x = \\
        = \varphi \left( n \right) p \left( 1 - p \right) \cdot
        \left[ \frac{ \left( 1 - p \right) \left[ 1 - \left( 1 - p \right)^n \right] }{p} \right]' +
        p \sum \limits_{x = n}^{ \infty } x \varphi \left( x \right) \left( 1 - p \right)^x = \\
        = \varphi \left( n \right) p \left( 1 - p \right) \times \\
        \times \frac{ \left[ n \left( 1 - p \right)^{n - 1} - 1 + \left( 1 - p \right)^n - pn \left( 1 - p \right)^{n - 1} \right] p - 1 + \left( 1 - p \right)^n + p}{p^2} - \\
        - \frac{p \left( 1 - p \right)^n}{p^2} +
        p \sum \limits_{x = n}^{ \infty } x \varphi \left( x \right) \left( 1 - p \right)^x = \\
        = \varphi \left( n \right) \left( 1 - p \right) \times \\
        \times \frac{np \left( 1 - p \right)^{n - 1} - p + p \left( 1 - p \right)^n - p^2 n \left( 1 - p \right)^{n - 1} - 1 + \left( 1 - p \right)^n + p}{p} - \\
        - \frac{p \left( 1 - p \right)^n}{p} +
        p \sum \limits_{x = n}^{ \infty } x \varphi \left( x \right) \left( 1 - p \right)^x = \\
        = \varphi \left( n \right) \left( 1 - p \right) \cdot
        \frac{np \left( 1 - p \right)^{n - 1} \left( 1 - p \right) - 1 + \left( 1 - p \right)^n}{p} + \\
        + p \sum \limits_{x = n}^{ \infty } x \varphi \left( x \right) \left( 1 - p \right)^x = \\
        = \varphi \left( n \right) \left( 1 - p \right) \cdot
        \frac{np \left( 1 - p \right)^n - 1 + \left( 1 - p \right)^n}{p} +
        p \sum \limits_{x = n}^{ \infty } x \varphi \left( x \right) \left( 1 - p \right)^x = \\
        = \varphi \left( n \right) \left( 1 - p \right) \cdot
        \frac{ \left( 1 - p \right)^n \left( np + 1 \right) - 1}{p} +
        p \sum \limits_{x = n}^{ \infty } x \varphi \left( x \right) \left( 1 - p \right)^x.
      \end{split}
    \end{equation*}

  Теперь вычисляем правую часть
  \begin{equation*}
    \begin{split}
      M \left[
        f \left( \max \left( \xi, n \right) \right) \varphi \left( \max \left( \xi, n \right) \right)
      \right] = \\
      = \sum \limits_{x = 0}^{ \infty }
        f \left( \max \left( x, n \right) \right) \cdot
        \varphi \left( \max \left( x, n \right) \right) p \left( 1 - p \right)^x.
    \end{split}
  \end{equation*}
  Разбиваем на две суммы согласно максимуму
  \begin{equation*}
    \begin{split}
      \sum \limits_{x = 0}^{ \infty }
        f \left( \max \left( x, n \right) \right) \cdot
        \varphi \left( \max \left( x, n \right) \right) p \left( 1 - p \right)^x = \\
      = \sum \limits_{x = 0}^n f \left( n \right) \varphi \left( n \right) p \left( 1 - p \right)^x +
      \sum \limits_{x = n}^{ \infty }
        f \left( x \right) \varphi \left( x \right) p \left( 1 - p \right)^x = \\
      = f \left( n \right) \varphi \left( n \right) p \sum \limits_{x = 0}^n \left( 1 - p \right)^x +
      p \cdot
      \sum \limits_{x = n}^{ \infty }
        f \left( x \right) \varphi \left( x \right) \left( 1 - p \right)^x = \\
      = f \left( n \right) \varphi \left( n \right) p \cdot
      \frac{1 - \left( 1 - p \right)^{n + 1}}{1 - \left( 1 - p \right) } +
      p \cdot
      \sum \limits_{x = n}^{ \infty }
        f \left( x \right) \varphi \left( x \right) \left( 1 - p \right)^x = \\
      = f \left( n \right) \varphi \left( n \right) \left[ 1 - \left( 1 - p \right)^{n + 1} \right] +
      p \cdot
      \sum \limits_{x = n}^{ \infty }
        f \left( x \right) \varphi \left( x \right) \left( 1 - p \right)^x.
    \end{split}
  \end{equation*}

  Подбираем функцию $f$ так, чтобы выполнялось равенство
  $$f \left( x \right) =
    \frac{ \left( 1 - p \right) \cdot \frac{ \left( 1 - p \right)^n \left( np + 1 \right) - 1}{p}}{1 - \left( 1 - p \right)^{n + 1}} \cdot
    \mathbbm{1} \left\{ x < n \right\} +
    x \cdot \mathbbm{1} \left\{ x \geq n \right\}.$$

  Записываем, когда вместо $x$ стоит $ \max \left( \xi, n \right) $.
  Получаем
  \begin{equation*}
    \begin{split}
      f \left( \max \left( \xi, n \right) \right) = \\
      = \frac{ \left( 1 - p \right) \cdot \frac{ \left( 1 - p \right)^n \left( np + 1 \right) - 1}{p}}{1 - \left( 1 - p \right)^{n + 1}} \cdot
      \mathbbm{1} \left\{ \max \left( \xi, n \right) < n \right\} + \\
      + \max \left( \xi, n \right) \cdot
      \mathbbm{1} \left\{ \max \left( \xi, n \right) \geq n \right\} = \\
      = \frac{ \left( 1 - p \right) \cdot \frac{ \left( 1 - p \right)^n \left( np + 1 \right) - 1}{p}}{1 - \left( 1 - p \right)^{n + 1}} \cdot
      \mathbbm{1} \left\{ \xi < n \right\} +
      \xi \cdot \mathbbm{1} \left\{ \xi \geq n \right\}.
    \end{split}
  \end{equation*}
\end{enumerate}

\subsubsection*{18.18}

\textit{Задание.}
Пусть $X_1, \dotsc, X_n$ ---
независимые случайные величины показательно распределены с параметром $ \lambda $ каждая.
Вычислите
$$M \left( \sum \limits_{k = 1}^n X_k^2 \; \middle| \; X_1 \right).$$

\textit{Решение.} Воспользуемся линейностью условного математического ожидания:
$$M \left( \sum \limits_{k = 1}^n X_k^2 \; \middle| \; X_1 \right) =
  \sum \limits_{i = 1}^n M \left( X_k^2 \; \middle| X_1 \right).$$

Поскольку случайная величина $X_k^2$ не зависит от $X_1$ при произвольном $k \neq 1$, то
$$M \left( X_k^2 \; \middle| \; X_1 \right) =
  MX_k^2 =
  \frac{2}{ \lambda^2}$$
при $k \neq 1$.
Случайная величина $X_1^2$ измерима относительно $ \sigma $-алгебры $ \sigma \left( X_1 \right) $,
порождённой случайной величиной $X_1$,
а поэтому
$$M \left( X_1^2 \; \middle| \; X_1 \right) =
  X_1^2.$$
Таким образом:
$$M \left( \sum \limits_{k = 1}^n X_k^2 \; \middle| \; X_1 \right) =
  X_1^2 + \frac{2 \left( n - 1 \right) }{ \lambda^2}.$$

\subsubsection*{18.19}

\textit{Задание.}
Пусть случайная величина $ \xi $ имеет равномерное распределение на отрезке $ \left[ 0, 2 \right] $.
Вычислите:
\begin{enumerate}[label=\alph*)]
  \item $M \left[ \xi \; \middle| \; \min \left( \xi, t \right) \right] $;
  \item $M \left[ \xi \; \middle| \; \max \left( \xi, t \right) \right] $.
\end{enumerate}

\textit{Решение.}
\begin{enumerate}[label=\alph*)]
  \item $t$ --- фиксированное значение.
  Решим по определению
  $$M \left[ \xi \; \middle| \min \left( \xi, t \right) \right] =
    \eta =
    f \left( \min \left( \xi, t \right) \right),$$
  где $f$ --- борелевская, такая,
  что
  $$M \left[ \xi \varphi \left( \min \left( \xi, t \right) \right) \right] =
    M \left[
      f \left( \min \left( \xi, t \right) \right) \cdot
      \varphi \left( \min \left( \xi, t \right) \right)
    \right] $$
  для произвольной ограниченной борелевской $ \varphi $.

  В явном виде вычислим оба математических ожидания и их приравняем
  $$M \left[ \xi \varphi \left( \min \left( \xi, t \right) \right) \right] =
    \int \limits_0^2 \frac{1}{2} \cdot x \varphi \left( \min \left( x, t \right) \right) dx.$$
  Разделим на 2 интеграла
  $$ \int \limits_0^2 \frac{1}{2} \cdot x \varphi \left( \min \left( x, t \right) \right) dx =
    \int \limits_0^t \frac{1}{2} \cdot x  \varphi \left( x \right) dx +
    \int \limits_t^2 \frac{1}{2} \cdot x \varphi \left( x \right) dx.$$
  Второй интеграл вычисляем в явном виде
  $$ \int \limits_0^t \frac{1}{2} \cdot x  \varphi \left( x \right) dx +
    \int \limits_t^2 \frac{1}{2} \cdot x \varphi \left( x \right) dx =
    \frac{1}{2} \int \limits_0^t x \varphi \left( x \right) dx +
    \left. \frac{1}{2} \cdot \varphi \left( x \right) \cdot \frac{x^2}{2} \right|_t^2.$$
  Подставляем пределы интегрирования
  $$ \frac{1}{2} \int \limits_0^t x \varphi \left( x \right) dx +
    \left. \frac{1}{2} \cdot \varphi \left( x \right) \cdot \frac{x^2}{2} \right|_t^2 =
    \frac{1}{2} \int \limits_0^t x \varphi \left( x \right) dx +
    \frac{1}{4} \cdot \varphi \left( x \right) \left( 4 - t^2 \right).$$
  Разобъём второе слагаемое на множители,
  чтобы потом было проще получить функцию $f \left( x \right) $.
  Получим
  \begin{equation*}
    \begin{split}
      \frac{1}{2} \int \limits_0^t x \varphi \left( x \right) dx +
      \frac{1}{4} \cdot \varphi \left( x \right) \left( 4 - t^2 \right) = \\
      = \frac{1}{2} \int \limits_0^t x \varphi \left( x \right) dx +
      \frac{1}{2} \cdot \frac{1}{2} \cdot
      \varphi \left( t \right) \left( 2 - t \right) \left( 2 + t \right).
    \end{split}
  \end{equation*}

  В явном виде вычисляем правую часть
  $$ M \left[
      f \left( \min \left( \xi, t \right) \right) \cdot
      \varphi \left( \min \left( \xi, t \right) \right)
    \right] =
    \int \limits_0^2
      f \left( \min \left( x, t \right) \right) \cdot
      \varphi \left( \min \left( x, t \right) \right) \cdot \frac{1}{2}
    dx.$$
  Разделим на 2 интеграла
  \begin{equation*}
    \begin{split}
      \int \limits_0^2
        f \left( \min \left( x, t \right) \right) \cdot
        \varphi \left( \min \left( x, t \right) \right) \cdot \frac{1}{2}
      dx = \\
      = \frac{1}{2} \int \limits_0^t f \left( x \right) \varphi \left( x \right) dx +
      \frac{1}{2} \int \limits_t^2 \varphi \left( t \right) f \left( t \right) dx = \\
      = \frac{1}{2} \int \limits_0^t f \left( x \right) \varphi \left( x \right) dx +
      \frac{1}{2} \cdot \left. f \left( t \right) \varphi \left( t \right) x \right|_t^2 = \\
      = \frac{1}{2} \int \limits_0^t f \left( x \right) \varphi \left( x \right) dx +
      \frac{1}{2} \cdot f \left( t \right) \varphi \left( t \right) \left( 2 - t \right).
    \end{split}
  \end{equation*}

  Нужно подобрать $f \left( x \right) $ так, чтобы математические ожидания совпадали
  $$f \left( x \right) =
    x \cdot \mathbbm{1} \left\{ x < t \right\} +
    \frac{1}{2} \cdot \left( 2 + t \right) \cdot \mathbbm{1} \left\{ x \geq t \right\}.$$
  Упростим второе слагаемое
  $$x \cdot \mathbbm{1} \left\{ x < t \right\} +
    \frac{1}{2} \cdot \left( 2 + t \right) \cdot \mathbbm{1} \left\{ x \geq t \right\} =
    x \cdot \mathbbm{1} \left\{ x < t \right\} +
    \left( 1 + \frac{t}{2} \right) \cdot \mathbbm{1} \left\{ x \geq t \right\}.$$

  Вместо $x$ нужно подставить $ \min \left( \xi, t \right) $.
  Это даст возможность упростить выражение
  $$f \left( \min \left( \xi, t \right) \right) =
    \min \left( \xi, t \right) \cdot \mathbbm{1} \left\{ \min \left( \xi, t \right) < t \right\} +
    \left( 1 + \frac{t}{2} \right) \cdot
    \mathbbm{1} \left\{ \min \left( \xi, t \right) \geq t \right\}.$$
  Упростим выражение
  \begin{equation*}
    \begin{split}
      \min \left( \xi, t \right) \cdot \mathbbm{1} \left\{ \min \left( \xi, t \right) < t \right\} +
      \left( 1 + \frac{t}{2} \right) \cdot
      \mathbbm{1} \left\{ \min \left( \xi, t \right) \geq t \right\} = \\
      = \xi \cdot \mathbbm{1} \left\{ \xi < t \right\} +
      \left( 1 + \frac{t}{2} \right) \cdot \mathbbm{1} \left\{ \xi \geq t \right\};
    \end{split}
  \end{equation*}
  \item ищем
  $M \left[ \xi \; \middle| \; \max \left( \xi, t \right) \right] =
    \eta =
    f \left( \max \left( \xi, t \right) \right) $.

  Находим $f$ из условия
  $$M \left[ \xi \varphi \left( \max \left( \xi, t \right) \right) \right] =
    M \left[
      f \left( \max \left( \xi, t \right) \right) \cdot
      \varphi \left( \max \left( \xi, t \right) \right)
    \right],$$
  где $ \varphi $ --- произвольная ограниченная борелевская функция.
  Сначала найдём левую часть
  $$M \left[ \xi \varphi \left( \max \left( \xi, t \right) \right) \right] =
    \int \limits_0^2 x \varphi \left( \max \left( x, t \right) \right) \cdot \frac{1}{2} dx.$$
  Разобъём на 2 интеграла согласно функции максимума
  $$ \int \limits_0^2 x \varphi \left( \max \left( x, t \right) \right) \cdot \frac{1}{2} dx =
    \frac{1}{2} \int \limits_0^t x \varphi \left( t \right) dx +
    \frac{1}{2} \int \limits_t^2 x \varphi \left( x \right) dx.$$
  Берём первый интеграл, второй оставляем без изменений
  $$ \frac{1}{2} \int \limits_0^t x \varphi \left( t \right) dx +
    \frac{1}{2} \int \limits_t^2 x \varphi \left( x \right) dx =
    \frac{1}{2} \cdot \left. \varphi \left( t \right) \cdot \frac{x^2}{2} \right|_0^t +
    \frac{1}{2} \int \limits_t^2 x \varphi \left( x \right) dx.$$
  Подставляем пределы интегрирования
  $$ \frac{1}{2} \cdot \left. \varphi \left( t \right) \cdot \frac{x^2}{2} \right|_0^t +
    \frac{1}{2} \int \limits_t^2 x \varphi \left( x \right) dx =
    \frac{1}{4} \cdot \varphi \left( t \right) t^2 +
    \frac{1}{2} \int \limits_t^2 x \varphi \left( x \right) dx.$$

  Теперь вычислим правую часть
  $$M \left[
      f \left( \max \left( \xi, t \right) \right) \cdot
      \varphi \left( \max \left( \xi, t \right) \right)
    \right] =
    \int \limits_0^2
      f \left( \max \left( x, t \right) \right) \cdot
      \varphi \left( \max \left( x, t \right) \right) \cdot \frac{1}{2}
    dx.$$
  Разделим на 2 интеграла согласно функции максимума
  $$ \int \limits_0^2
      f \left( \max \left( x, t \right) \right) \cdot
      \varphi \left( \max \left( x, t \right) \right) \cdot \frac{1}{2}
    dx =
    \frac{1}{2} \int \limits_0^t f \left( t \right) \varphi \left( t \right) dx +
    \frac{1}{2} \int \limits_t^2 f \left( x \right) \varphi \left( x \right) dx.$$
  Берём первый интеграл
  \begin{equation*}
    \begin{split}
      \frac{1}{2} \int \limits_0^t f \left( t \right) \varphi \left( t \right) dx +
      \frac{1}{2} \int \limits_t^2 f \left( x \right) \varphi \left( x \right) dx = \\
      = \frac{1}{2} \cdot \left. f \left( t \right) \varphi \left( t \right) x \right|_0^t +
      \frac{1}{2} \int \limits_t^2 f \left( x \right) \varphi \left( x \right) dx = \\
      = \frac{1}{2} \cdot f \left( t \right) \varphi \left( t \right) t +
      \frac{1}{2} \int \limits_t^2 f \left( x \right) \varphi \left( x \right) dx.
    \end{split}
  \end{equation*}

  Подбираем функцию $f$ так, чтобы выполнялось равенство
  $$f \left( x \right) =
    \frac{t}{2} \cdot \mathbbm{1} \left\{ x < t \right\} +
    x \cdot \mathbbm{1} \left\{ x \geq t \right\}.$$

  Запишем, когда вместо $x$ стоит $ \max \left( \xi, t \right) $.
  Получим
  $$f \left( \max \left( \xi, t \right) \right) =
    \frac{t}{2} \cdot \mathbbm{1} \left\{ \max \left( \xi, t \right) < t \right\} +
    \max \left( \xi, t \right) \cdot
    \mathbbm{1} \left\{ \max \left( \xi, t \right) \geq t \right\}.$$
  Упростим выражение
  \begin{equation*}
    \begin{split}
      \frac{t}{2} \cdot \mathbbm{1} \left\{ \max \left( \xi, t \right) < t \right\} +
      \max \left( \xi, t \right) \cdot
      \mathbbm{1} \left\{ \max \left( \xi, t \right) \geq t \right\} = \\
      = \frac{t}{2} \cdot \mathbbm{1} \left\{ \xi < t \right\} +
      \xi \cdot \mathbbm{1} \left\{ \xi \geq t \right\}.
    \end{split}
  \end{equation*}
\end{enumerate}

\subsubsection{18.20}

\textit{Задание.}
Пусть $ \xi $ и $ \eta $ --- независимые случайные величины,
равномерно распределённые на отрезке $ \left[ 0, 1 \right] $.
Вычислите:
\begin{enumerate}[label=\alph*)]
  \item $M \left( \xi \; \middle| \; \xi + \eta \right) $;
  \item $M \left( \xi - \eta \; \middle| \; \xi + \eta \right) $.
\end{enumerate}

\textit{Решение.}
\begin{enumerate}[label=\alph*)]
  \item $M \left( \xi \; \middle| \; \xi + \eta \right) =
    M \left( \xi + \eta - \eta \; \middle| \; \xi + \eta \right) =
    M \left( \xi + \eta \; \middle| \; \xi + \eta \right) -
    M \left( \eta \; \middle| \; \xi + \eta \right) $.
  Случайная величина $ \left( \xi + \eta \right) $ измерима относительно
  $ \sigma \left( \xi + \eta \right) $,
  потому её можно вынести за знак условного математического ожидания:
  $M \left( \xi + \eta \; \middle| \; \xi + \eta \right) -
    M \left( \eta \; \middle| \; \xi + \eta \right) =
    \xi + \eta - M \left( \eta \; \middle| \; \xi + \eta \right) $.

  Чтобы вычислить условное математическое ожидание,
  нужно найти распределение $ \left( \eta, \xi + \eta \right) $.
  Случайные величины $ \xi $ и $ \eta $ --- независимы и одинаково распределённые.
  Из этого следует, что распределения векторов одинаковые,
  потому что $ \xi $ и $ \eta $ одинаково входят в функцию $ \left( \xi + \eta \right) $,
  то есть $ \left( \xi, \xi + \eta \right) \sim \left( d \right) \left( \eta , \xi + \eta \right) $.
  Отсюда следует, что эти математические ожидания совпадают:
  $M \left( \xi \; \middle| \; \xi + \eta \right) =
    M \left( \eta \; \middle| \; \xi + \eta \right) $.
  Тогда $2M \left( \xi \; \middle| \xi + \eta \right) = \xi + \eta $ и
  $$M \left( \xi \; \middle| \; \xi + \eta \right) =
    \frac{ \xi + \eta }{2};$$
  \item из доказанного в предыдущем пункте
  $$M \left( \xi - \eta \; \middle| \; \xi + \eta \right) =
    M \left( \xi \; \middle| \; \xi + \eta \right) -
    M \left( \eta \; \middle| \; \xi + \eta \right) =
    0.$$
\end{enumerate}
