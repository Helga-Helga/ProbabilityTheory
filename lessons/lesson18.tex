\addcontentsline{toc}{chapter}{Занятие 18. Условное математическое ожидание}
\chapter*{Занятие 18. Условное математическое ожидание}

\addcontentsline{toc}{section}{Контрольные вопросы и задания}
\section*{Контрольные вопросы и задания}

\subsubsection*{Приведите определение условного математического ожидания.}

Условным математическим ожиданием случайной величины $ \xi $ относительно $ \sigma $-алгебры
$ \mathbb{F'}$ называется случайная величина $ \eta $ такая, что:
\begin{enumerate}
  \item $ \eta $ измерима относительно $ \mathbb{F'}$;
  \item для любой случайной величины $ \zeta $, измеримой относительно $ \mathbb{F'}$ и ограниченной,
  $M \xi \zeta = M \eta \zeta $.
\end{enumerate}

\subsubsection*{Запишите формулы для вычисления условного математического ожидания.}

$ \sigma $-алгебра $ \mathbb{F'}$ --- конечная.
Есть набор $H_1, \dotsc, H_n$ такой, что
$$ \bigcup \limits_{k = 1}^n H_k = \Omega, \,
  H_k \cap H_j = \emptyset, \,
  k \neq j, \,
  P \left( H_k \right) > 0.$$

Уловное математическое ожидание случайной величины $ \xi $ относительно
$$ \mathbb{F'} =
  \left\{ \bigcup \limits_{k = 1}^n H_k^{ \varepsilon_k} \right\},$$
где
$$ \varepsilon_k =
  \begin{cases}
    0, \\
    1,
  \end{cases}
  H_k^0 = \emptyset, \,
  H_k^1 = H_k,$$
имеет вид
$$ \eta =
  \sum \limits_{k = 1}^n
    \frac{M \xi \cdot \mathbbm{1}_{H_k}}{P \left( H_k \right) } \cdot \mathbbm{1}_{H_k}.$$

\subsubsection*{Сформулируйте свойства условного математического ожидания.}

\begin{enumerate}
  \item $ \mathbb{F'} = \left\{ \emptyset, \Omega \right\} $.
  Из этого следует, что $M \left( \xi \; \middle| \; \mathbb{F'} \right) = M \xi $;
  \item $ \mathbb{F'} \subset \mathbb{F"} \subset \mathbb{F}$.
  Тогда
  $M \left( M \left( \xi \; \middle| \; \mathbb{F"} \right) \; \middle| \; \mathbb{F'} \right) =
    M \left( \xi \; \middle| \; \mathbb{F'} \right) $;
  \item $M \left( M \left( \xi \; \middle| \; \mathbb{F'} \right) \right) = M \xi $;
  \item $ \zeta $ измерима относительно $ \mathbb{F'}$.
  Тогда
  $M \left( \xi \zeta \; \middle| \; \mathbb{F'} \right) =
    \zeta M \left( \xi \; \middle| \; \mathbb{F'} \right) $;
  \item $ \xi $ не зависит от $ \mathbb{F'}$, следовательно,
  $M \left( \xi \; \middle| \; \mathbb{F'} \right) =
    M \xi $;
  \item $M \left( \alpha_1 \xi_1 + \alpha_2 \xi_2 \; \middle| \; \mathbb{F'} \right) =
    \alpha_1 M \left( \xi_1 \; \middle| \; \mathbb{F'} \right) +
    \alpha_2 M \left( \xi_2 \; \middle| \; \mathbb{F'} \right) $;
  \item если $ \xi \geq 0$, то $M \left( \xi \; \middle| \; \mathbb{F'} \right) \geq 0$;
  \item $ \left| M \left( \xi \; \middle| \; \mathbb{F'} \right) \right| \leq
    M \left( \left| \xi \right| \; \middle| \; \mathbb{F'} \right) $,
  так как модуль --- выпуклая вниз функция;
  \item если $f$ --- выпуклая вниз функция,
  то $f \left( M \left( \xi \; \middle| \; \mathbb{F'} \right) \right) \leq
    M \left( f \left( \xi \right) \; \middle| \; \mathbb{F'} \right) $;
  \item предельные теоремы.
  Если $ \xi_n \geq 0, \, \xi_n \nearrow \xi $ (возрастает) при $n \to \infty $,
  то
  $M \left( \xi_n \; \middle| \; \mathbb{F'} \right) \nearrow
    M \left( \xi \; \middle| \; \mathbb{F'} \right) $
  почти наверное, $n \to \infty $;
  \item если $ \left| \xi_n \right| \leq \eta, \, \exists M \eta $ и
  $ \xi_n \overset{P} \xi, \, n \to \infty $,
  то
  $M \left( \xi_n \; \middle| \; \mathbb{F'} \right) \overset{P}{ \to }
    M \left( \xi \; \middle| \; \mathbb{F'} \right), \,
    n \to \infty.$
\end{enumerate}

\addcontentsline{toc}{section}{Аудиторные задачи}
\section*{Аудиторные задачи}

\subsubsection*{18.3}

\textit{Задание.}
На вероятностном пространстве $ \left( \Omega, \mathbb{F}, P \right) $,
где $ \Omega = \left[ 0, 1 \right], \mathbb{F}$ --- мера Лебега,
задана случайная величина $ \xi = \omega $.

Пусть $ \sigma $-алгебра $ \mathbb{B}$ порождена множествами
$$ \left[ 0, \frac{1}{3} \right), \,
  \left\{ \frac{1}{3} \right\}, \,
  \left( \frac{1}{3}, \frac{1}{2} \right).$$
Вычислите условное математическое ожидание $M \left[ \xi \; \middle| \; \mathbb{B} \right] $.

\textit{Решение.} Мера Лебега в данном случае --- это длина.

Есть 3 атома
$$ \sigma \left( \beta \right) =
  \left\{
    \left[ 0, \frac{1}{3} \right), \,
    \left[ \frac{1}{3}, \frac{1}{2} \right), \,
    \left[ \frac{1}{2}, 1 \right]
  \right\}.$$

Условное математическое ожидание ---
это случайная величина
$$ \eta =
  M \left[ \xi \; \middle| \; \mathbb{B} \right].$$
Будет 3 слагаемых
\begin{equation*}
  \begin{split}
    \eta = \\
    = \frac{M \left( \omega \cdot \mathbbm{1} \left\{ \omega \in \left[ 0, \frac{1}{3} \right] \right\} \right) }{ \frac{1}{3}} \cdot
    \mathbbm{1} \left\{ \omega \in \left[ 0, \frac{1}{3} \right] \right\} + \\
    + \frac{M \left( \omega \cdot \mathbbm{1} \left\{ \omega \in \left[ \frac{1}{3}, \frac{1}{2} \right) \right\} \right) }{ \frac{1}{6}} \cdot
    \mathbbm{1} \left\{ \omega \in \left[ \frac{1}{3}, \frac{1}{2} \right) \right\} + \\
    + \frac{M \left( \omega \cdot \mathbbm{1} \left\{ \omega \in \left[ \frac{1}{2}, 1 \right] \right\} \right) }{ \frac{1}{2}} \cdot
    \mathbbm{1} \left\{ \omega \in \left[ \frac{1}{2}, 1 \right] \right\}.
  \end{split}
\end{equation*}
Числители --- интегралы в соответствующих пределах
\begin{equation*}
  \begin{split}
    \eta = \\
    = 3 \int \limits_0^{ \frac{1}{3}} xdx \cdot
    \mathbbm{1} \left\{ \omega \in \left[ 0, \frac{1}{3} \right) \right\} +
    6 \int \limits_{ \frac{1}{3}}^{ \frac{1}{2}} xdx \cdot
    \mathbbm{1} \left\{ \omega \in \left[ \frac{1}{3}, \frac{1}{2} \right) \right\} + \\
    + 2 \int \limits_{ \frac{1}{2}}^1 xdx \cdot
    \mathbbm{1} \left\{ \omega \in \left[ \frac{1}{2}, 1 \right] \right\} = \\
    = \frac{1}{6} \cdot \mathbbm{1} \left\{ \omega \in \left[ 0, \frac{1}{3} \right) \right\} +
    \frac{5}{12} \cdot
    \mathbbm{1} \left\{ \omega \in \left[ \frac{1}{3}, \frac{1}{2} \right) \right\} +
    \frac{3}{4} \cdot \mathbbm{1} \left\{ \omega \in \left[ \frac{1}{2}, 1 \right] \right\}.
  \end{split}
\end{equation*}

На каждом атоме случайная величина принимает своё значение.

\subsubsection*{18.4}

\textit{Задание.} Пусть $ \xi $ и $ \eta $ --- независимые случайные величины.
Вычислите $D \left( M \left[ \xi \eta \; \middle| \; \xi \right] \right) $,
если $ \xi $ равномерно распределена на отрезке $ \left[ 0, 1 \right] $,
а $ \eta $ имеет нормальное распределение с параметрами $1, \sigma^2$.

\textit{Решение.} Упростим случайную величину
$M \left( \xi \eta \; \middle| \; \xi \right) =
  M \left[ \xi \eta \; \middle| \; \sigma \left( \xi \right) \right] $.
Случайная величина $ \xi $ измерима относительно своей же $ \sigma $-алгебры
$ \sigma \left( \xi \right) $.
Используем свойство условного математического ожидания, которое говорит,
что в данном случае можем вынести $ \xi $ за знак условного математического ожидания
$M \left[ \xi \eta \; \middle| \; \sigma \left( \xi \right) \right] =
  \xi M \left[ \eta \; \middle| \; \sigma \left( \xi \right) \right].$
Случайные величины $ \xi $ и $ \eta $ независимы, поэтому $ \eta $ не зависит от $ \sigma $-алгебры,
порождённой $ \xi $.
По свойствам условного математического ожидания
$ \xi M \left[ \eta \; \middle| \; \sigma \left( \xi \right) \right] =
  \xi M \eta =
  \xi a$,
где
$$a = const.$$

Ищем дисперсию
$D \left( M \left[ \xi \eta \; \middle| \; \xi \right] \right) =
  D \left( a \xi \right) =
  a^2 D \xi.$
Случайная величина имеет равномерное распределение
$$a^2 D \xi =
  \frac{a^2}{12}.$$

\addcontentsline{toc}{section}{Домашнее задание}
\section*{Домашнее задание}
