\addcontentsline{toc}{chapter}{Занятие 5. Условные вероятности. Независимость}
\chapter*{Занятие 5. Условные вероятности. Независимость}

\addcontentsline{toc}{section}{Контрольные вопросы и задания}
\section*{Контрольные вопросы и задания}

\subsubsection*{Какие события называются независимыми?}

$A$ и $B$ --- независимы, если $P \left( A \cap B \right) = P \left( A \right) P \left( B \right) $.

\subsubsection*{Какие события называются несовместимыми?}

В теории вероятностей несколько событий называются несовместными, или несовместимыми, если никакие из них не могут появиться одновременно в результате однократного проведения эксперимента (опыта).

\subsubsection*{Запишите формулу для вычисления условной вероятности.}

Условная вероятность события $A$ при условии, что событие $B$ произошло --- это выражение
$$P \left( A/B \right) =
\frac{P \left( A \cap B \right) }{P \left( B \right) }.$$

\addcontentsline{toc}{section}{Домашнее задание}
\section*{Домашнее задание}

\subsubsection*{5.15}

\textit{Задание.} Из колоды игральных карт наугад вынута одна карта.
Найдите вероятность того, что:
\begin{enumerate}[label=\alph*)]
\item эта карта является красной масти, при условии, что она является красной;
\item порядок карты является выше чем 10, если известно что она красной масти;
\item эта карта является тузом, если известно, что она является красной.
\end{enumerate}

\textit{Решение.} 
\begin{enumerate}[label=\alph*)]
\item нужно найти вероятность события $A$ при условии выполнения события $B$.
Опишем оба события.
Событие $А =$ \{вынята карта красной масти\}.
Красных карт половина от всего количества.
Событие $B =$ \{карта является красной\}.
Видим, что события $A$ и $B$ идентичны, поэтому $P \left( A \cap B \right) = P \left( A \right) = P \left( B \right)$.
Тогда условная вероятность события равна
$$P \left( A/B \right) =
\frac{P \left( A \cap B \right)}{P \left( B \right) } =
\frac{ \frac{1}{2} }{ \frac{1}{2} } =
1;$$
\item нужно найти вероятность события $A$ при условии выполнения события $B$.
Опишем оба события.
Событие $А =$ \{порядок выбранной карты больше десяти\}.
Такими картами является валет, дама, король и туз любой масти.
Событие $B =$ \{карта является красной\}.
Его вероятность равна
$$P \left( B \right) =
\frac{1}{2}.$$
Пересечение событий $A$ и $B$ даёт такие карты как валет, дама, король и туз красных мастей.
Вероятность пересечения равна
$$P \left( A \cap B \right) =
\frac{2 \cdot 4}{52} =
\frac{4}{26} =
\frac{2}{13}.$$
Тогда условная вероятность события равна
$$P \left( A/B \right) =
\frac{P \left( A \cap B \right)}{P \left( B \right) } =
\frac{ \frac{2}{13} }{ \frac{1}{2} } =
\frac{4}{13};$$
\item нужно найти вероятность события $A$ при условии выполнения события $B$.
Опишем оба события.
Событие $А =$ \{карта является тузом\}.
Такими картами являются 4 туза.
Событие $B =$ \{карта является красной\}.
Его вероятность равна
$$P \left( B \right) =
\frac{1}{2}.$$
Пересечение событий $A$ и $B$ даёт два красных туза.
Вероятность пересечения равна
$$P \left( A \cap B \right) =
\frac{2}{52} =
\frac{1}{26}.$$
Тогда условная вероятность события равна
$$P \left( A/B \right) =
\frac{P \left( A \cap B \right)}{P \left( B \right) } =
\frac{ \frac{1}{26} }{ \frac{1}{2} } =
\frac{2}{26} =
\frac{1}{13}.$$
\end{enumerate}

\subsubsection*{5.16}

\textit{Задание.} Игральный кубик подбросили дважды.
Найдите вероятность того, что сумма очков является больше 7, если известно, что:

\begin{enumerate}[label=\alph*)]
\item при первом подбрасывании быпало одно очко;
\item при первом подбрасывании выпало меньше, чем 5 очков.
\end{enumerate}

\textit{Решение.}
Вероятностное пространство эксперимента,
которое состоит в подбрасывании игрального кубика дважды,
опишем множеством
$ \Omega =
\left\{ \left( x, y \right), \, x = \overline{1, 6}, \, y = \overline{1, 6} \right\} $,
где через $x$ обозначим количество очков,
которые выпали при первом подбрасывании игрального кубика, а через $y$ --- количество очков, которые выпали при втором его подбрасывании.
Нам нужно вычислить вероятность $P \left( \left. x+y>7 \right| x=1 \right) $ и $P \left( \left. x+y>7 \right| x<5 \right) $.
По определению вероятности:
$$P \left( \left. x+y>7 \right| x=1 \right) =
\frac{P \left( x+y>7, \, x=1 \right) }{P \left( x=1 \right) }.$$
Поскольку
$ \left\{ \left( x, y \right) \in \Omega: \, x + y > 7, \, x = 1 \right\} =
\varnothing $,
а $ \left\{ \left( x, y \right) \in \Omega: \, x = 1 \right\} =  \\ = \left\{ \left( 1, y \right), \, y = \overline{1, 6} \right\} $, то
$$P \left( \left. x+y>7 \right| x=1 \right) =
\frac{ \varnothing }{ \frac{6}{36} } =
\varnothing.$$
Далее
$$P \left( \left. x+y>7 \right| x<5 \right) =
\frac{P \left( \left. x+y>7 \right| x<5 \right) }{P \left( x<5 \right) }.$$
Понятно, что
$$P \left( x<5 \right) =
\frac{4}{6} =
\frac{2}{3}.$$
А для вероятности в числителе имеем:
\begin{equation*}
\begin{split}
P \left( \left. x+y>7 \right| x<5 \right) =
\sum \limits_{k=1}^4 P \left(x=k, \, k+y>7 \right) = \\
= \sum \limits_{k=1}^4 P \left( x=k \right) P \left( y>7-k \right) =
P \left( x=1 \right) P \left( y>6 \right) +
P \left( x=2 \right) P \left( y>5 \right) + \\
+ P \left( x=3 \right) \left( y>4 \right) +
P \left( x=4 \right) P \left( y>3 \right) =
\frac{1}{6} \left(\varnothing +\frac{1}{6} + \frac{2}{6} + \frac{3}{6} \right) =
\frac{1}{6} \cdot \frac{6}{6} = \\
= \frac{6}{36} =
\frac{1}{6}.
\end{split}
\end{equation*}

Тут мы воспользовались тем, что результаты первого и второго подбрасываний являются независимыми событиями.
Таким образом:
$$P \left( \left. x+y>7 \right| x<5 \right) =
\frac{P \left( x+y>5, \, x<5 \right) }{P \left( x<5 \right) } =
\frac{ \frac{1}{6} }{ \frac{2}{3} } =
\frac{1}{4}.$$

\subsubsection*{5.17}

\textit{Задание.} Дважды подброшена монета.
Рассмотрим следующие события:
$A =$ \{при первом подбрасывании выпала решка\}, $B =$ \{при втором подбрасывании выпала решка\}, $C =$ \{результат обеих подбрасываний одинаковый\}.
Покажите, что события $A, \, B, \, C$ попарно независимы, но не независыми в совокупности.

\textit{Решение.} События $A_i$ и $A_j$ называются попарно независимыми, если для $ \forall i \neq j \rightarrow A_i $ и $A_j$ --- независимы.

Пространством элементарных исходов является множество векторов
$ \Omega = \\
= \left\{ \left( i, j \right), \, i, j \in \left\{ 0, 1 \right\} \right\}$,
где $0$ означает, что выпал герб, $1$ --- выпала решка, $i$ --- результат первого подбрасывания, $j$ --- результат второго подбрасывания.
Пространство элементарных событий содержит $ \left| \Omega \right| = 2^2 = 4$ элемента.

Опишем данные события.
Событие $A = \left\{ \left( 1, 0 \right), \, \left( 1, 1 \right) \right\} $.
Вероятность выпадения при первом подбрасывании решки равна
$$P \left( A \right) =
\frac{2}{4} =
\frac{1}{2}.$$

Опишем событие $B = \left\{ \left( 0, 1 \right), \, \left( 1, 1 \right) \right\} $.
Вероятность выпадения решки при втором подбрасывании равна
$$P \left( B \right) =
\frac{2}{4} =
\frac{1}{2}.$$

Опишем событие $C = \left\{ \left( 0, 0 \right), \, \left( 1, 1 \right) \right\} $.
Вероятность того, что результат двух подбрасываний будет одинаковым, равна
$$P \left( C \right) =
\frac{2}{4} =
\frac{1}{2}.$$

Найдём вероятности пересечения событий $A$ и $B$:
$$P \left( A \cap B \right) =
P \left( \left\{ \left( 1, 1 \right) \right\} \right) =
\frac{1}{4}.$$
Найдём произведение вероятностей этих событий:
$$P \left( A \right) P \left( B \right) =
\frac{1}{2} \cdot \frac{1}{2} =
\frac{1}{4}.$$
Получаем, что $P \left( A \cap B \right) = P \left( A \right) P \left( B \right) $, значит эти события независимы.

Найдём вероятности пересечения событий $A$ и $C$:
$$P \left( A \cap C \right) =
P \left( \left\{ \left( 1, 1 \right) \right\} \right) =
\frac{1}{4}.$$
Найдём произведение вероятностей этих событий:
$$P \left( A \right) P \left( C \right) =
\frac{1}{2} \cdot \frac{1}{2} =
\frac{1}{4}.$$
Получаем, что $P \left( A \cap C \right) = P \left( A \right) P \left( C \right) $, значит эти события независимы.

Найдём вероятности пересечения событий $B$ и $C$:
$$P \left( B \cap C \right) =
P \left( \left\{ \left( 1, 1 \right) \right\} \right) =
\frac{1}{4}.$$
Найдём произведение вероятностей этих событий:
$$P \left( B \right) P \left( C \right) =
\frac{1}{2} \cdot \frac{1}{2} =
\frac{1}{4}.$$
Получаем, что $P \left( B \cap C \right) = P \left( B \right) P \left( C \right) $, значит эти события независимы.

События $A, \, B, \, C$ называются независимыми в совокупности, если имеет место равенство:
$P \left( A \cap B \cap C \right) =
P \left( A \right) P \left( B \right) P \left( C \right) $.

Проверим это равенство для данной задачи:
$$P \left( A \cap B \cap C \right) =
P \left( \left\{ \left( 1, 1 \right) \right\} \right) =
\frac{1}{4}.$$
Найдём произведение вероятностей всех трёх событий:
$$P \left( A \right) P \left( B \right) P \left( C \right) =
\frac{1}{2} \cdot \frac{1}{2} \cdot \frac{1} {2} =
\frac{1}{8}.$$
Получили, что равенство не выполняется, поэтому события не независимы.

\subsubsection*{5.18}

\textit{Задание.} В урне $7$ белых и $3$ чёрных шарика.
Из неё наугад вынимают три шарика.
Известно, что среди них есть чёрный шарик.
Найдите вероятность того, что два других шарика белые.

\textit{Решение.}
Пространство элементарных событий имеет вид
$ \Omega = \\ = \left\{ \left( i, j, k \right): i, k, k \in \right.$ \{белый, чёрный\}, где $i, j, k$ --- цвета вытянутых шариков.
Выпишем все его элементы: $ \Omega = $ \{(чёрный, чёрный, чёрный), (чёрный, белый, белый), (белый, чёрный, белый), (белый, белый, чёрный),
(чёрный, чёрный, белый), (чёрный, белый, белый), (белый, чёрный, чёрный), (белый, белый, белый)\}.
Видим, что $ \left| \Omega \right| = 2^3 = 8$.

Событие $B =$ \{среди выбранных шаров есть чёрный шар\} = \{(чёрный, чёрный, чёрный),
(чёрный, белый, белый), (белый, чёрный, белый), (белый, белый, чёрный),
(чёрный, чёрный, белый), (чёрный, белый, чёрный), (белый, чёрный, чёрный)\}.
Вероятность этого события равна
$$P \left( B \right) =
\frac{ \left| B \right| }{ \left| \Omega \right| } =
\frac{7}{8}.$$

Опишем событие $A =$ \{два шара белые\} = \{(чёрный, белый, белый)\}.

Найдём вероятность пересечения событий $A$ и $B: \, P \left( A \cap B \right) =$ \{(чёрный, белый, белый)\}.

Тогда условная вероятность события $A$ при условии, что $B$ произошло равна
$$P \left( A/B \right) =
\frac{P \left( A \cap B \right) }{P \left( B \right) } =
\frac{ \frac{1}{8} }{ \frac{7}{8} } =
\frac{1}{7}.$$
