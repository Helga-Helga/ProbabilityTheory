\addcontentsline{toc}{chapter}{Занятие 6. Формула полной вероятности. Формула Байеса}
\chapter*{Занятие 6. Формула полной вероятности. Формула Байеса}

\addcontentsline{toc}{section}{Контрольные вопросы и задания}
\section*{Контрольные вопросы и задания}

\subsubsection*{Приведите определение полной группы гипотез.}

Набор случайных событий $H_1, \dotsc, H_n$ называется полным набором гипотез, если выполнены следующие условия:
\begin{enumerate}
\item $P \left( H_i \right) > 0, \, i = \overline{1, n} $;
\item $H_i \cap H_j = \varnothing, \, i \neq j$ (они несовместны);
\item $ \bigcup \limits_{i=1}^n H_i = \Omega $.
\end{enumerate}

\subsubsection*{Запишите формулу для вычисления условной вероятности, формулу полной вероятности, формулу Байеса.}

Условная вероятность события $A$ при условии, что событие $B$ произошло --- это выражение
$$P \left( \left. A \right| B \right) =
\frac{P \left( A \cap B \right) }{P \left( B \right) }.$$

Формула полной вероятности: $P \left( A \right) = \sum \limits_{i=1}^n P \left( \left. A \right| H_i \right) P \left( H_i \right) $.

Формула Байеса:
$$P \left( \left. H_i \right| A \right) =
\frac{P \left( \left. A \right| H_i \right) P \left( H_i \right) }{ \sum \limits_{k=1}^n P \left( \left. A \right| H_k \right) P \left( H_k \right) }.$$

\addcontentsline{toc}{section}{Домашнее задание}
\section*{Домашнее задание}

\subsubsection*{6.13}

\textit{Задание.} В первой урне содержатся 1 белый и 9 чёрных шаров; во второй урне --- 1 чёрный и 5 белых шаров.
Из каждой урны наугад вынуто по одному шару, а остальные пересыпали в третью урну.
Найдите вероятность того, что шар, вынятый наугад из третьей урны, окажется белым.

\textit{Решение.}
Введём полную группу событий:
$H_1 =$ \{в третью урну переложили 10 чёрных и 4 белых шара\} = \{из обеих урн вынули по одному белому шару\};
$H_2 =$ \{в третью урну переложили 9 чёрных и 5 белых шаров\} = \{из одной урны достали 1 белый шар, и из второй --- 1 чёрный шар\};
$H_3 =$ \{в третью урну переложили 8 чёрных и 6 белых шаров\} = \{из обеих урн вынули по одному чёрному шару\}.
Определим вероятности этих событий:
\begin{equation*}
\begin{split}
P \left\{ H_1 \right\} =
\frac{1}{10} \cdot \frac{5}{6} =
\frac{1}{12}, \,
P \left\{ H_2 \right\} =
\frac{1}{10} \cdot \frac{1}{6} + \frac{9}{10} \cdot \frac{5}{6} =
\frac{1}{60} + \frac{3}{4} =
\frac{1+15}{60} =
\frac{16}{60} = \\
= \frac{1}{5}, \,
P \left\{ H_3 \right\} =
\frac{9}{10} \cdot \frac{1}{6} =
\frac{3}{10 \cdot 2} =
\frac{3}{20}.
\end{split}
\end{equation*}
Введём событие $A =$ \{из третьей урны вытянули белый шар\}.
В то же время
$$P \left\{ \left. A \right| H_1 \right\} =
\frac{4}{14} =
\frac{2}{12}, \,
P \left\{ \left. A \right| H_2 \right\} =
\frac{5}{14}, \,
P \left\{ \left. A \right| H_3 \right\} =
\frac{6}{14} =
\frac{3}{7}.$$
Согласно с формулой полной вероятности $P \left\{ A \right\} = \sum \limits_{i=1}^\infty P \left\{ H_i \right\} \cdot P \left\{ \left. A \right| H_i \right\} $ имеем
\begin{equation*}
\begin{split}
P \left\{ A \right\} =
\frac{1}{12} \cdot \frac{2}{12} + \frac{1}{5} \cdot \frac{5}{14} + \frac{3}{20} + \frac{3}{7} =
\frac{1}{72} + \frac{1}{14} + \frac{9}{140} =
\frac{1960+10080+9072}{141120} = \\
= \frac{21112}{141120}=
\frac{377}{2520}.
\end{split}
\end{equation*}

\subsubsection*{6.14}

\textit{Задание.} Два стрелка стреляют по мишени.
Один из них попадает в 5 случаях, а другой --- в 8 случаях из 10.
Перед выстрелом они подбрасывают правильную монету для определения очередности.
Сторонний наблюдатель знает условия стрельбы, но не видит, кто в данный момент стреляет.
Он видит, что стрелок попал по мишени.
Найдите вероятность того, что стрелял первый стрелок.

\textit{Решение.} Первый стрелок попадает по мишени с вероятностью
$$P \left( \left. A \right| H_1 \right) =
\frac{5}{10} =
\frac{1}{2},$$
второй стрелок --- с вероятностью
$$P \left( \left. A \right| H_2 \right) =
\frac{8}{10} =
\frac{4}{5}.$$
Априорные вероятности этих гипотез одинаковы:
$$P \left( H_1 \right) =
P \left( H_2 \right) =
\frac{1}{2}.$$
Рассмотрим событие $A =$ \{стрелок попал в цель\}.
Поэтому вероятность стрелять первому стрелку
\begin{equation*}
\begin{split}
P \left( \left. A \right| H_1 \right) =
\frac{P \left( \left. A \right| H_1 \right) P \left( H_1 \right) }{P \left( \left. A \right| H_1 \right) P \left( H_1 \right) +
P \left( \left. A \right| H_2 \right) P \left( H_2 \right) } =
\frac{ \frac{1}{2} \cdot \frac{1}{2} }{ \frac{1}{2} \cdot \frac{1}{2} + \frac{4}{5} \cdot \frac{1}{2} } =
\frac{ \frac{1}{4} }{ \frac{1}{4} + \frac{2}{5} } = \\
= \frac{ \frac{1}{4} }{ \frac{5+8}{20} } =
\frac{20}{4 \cdot 13} =
\frac{5}{13}.
\end{split}
\end{equation*}

\subsubsection*{6.15}

\textit{Задание.} Среди $N$ экзаменационных билетов есть $n$ <<счастливых>>.
Студенты подходят за билетами по одному.
У кого больше шансов вынуть счастливый билет: у того, кто подошёл первым, или у того, что подошёл вторым?
Найдите вероятность того, что студент, который подошёл первым, вынул счастливый билет, если известно, что студент, который подошёл вторым, вынул счастливый билет.

\textit{Решение.} Если студент сдаёт экзамен первым, то вероятность вынуть счастливый билет равна $n/N$.
Рассмотрим ситуацию, когда он сдаёт вторым.
Введём гипотезы: $H_1$ --- первый забрал <<счастливый>> билет, $H_2$ --- первый не забрал <<счастливый>> билет.
Событие $A =$ \{студент сдал экзамен\}.
\begin{equation*}
\begin{split}
P \left( \left. A \right| H_1 \right) =
\frac{n-1}{N-1}, \,
P \left( \left. A \right| H_2 \right) =
\frac{n}{N-1}, \,
P \left( H_1 \right) =
\frac{n}{N}, \,
P \left( H_2 \right) =
1 - \frac{n}{N} = \\
= \frac{N-n}{N}.
\end{split}
\end{equation*}
Используем формулу полной вероятности:
\begin{equation*}
\begin{split}
P \left( A \right) =
\frac{n-1}{N-1} \cdot \frac{n}{N} + \frac{n}{N-1} \cdot \frac{N-n}{N} =
\frac{n}{N \left( N-1 \right) } \cdot \left( n-1+N-n \right) = \\
= \frac{n \left( N-1 \right) }{N \left( N-1 \right) } =
\frac{n}{N}.
\end{split}
\end{equation*}
Вывод: без разницы, когда сдавать.

Вероятность того, что студент, который пошёл первым, вынул счастливый билет, если известно, что студент, который пошёл вторым, вынул счастливый билет:
\begin{equation*}
\begin{split}
P \left( \left. H_1 \right| A \right) =
\frac{P \left( \left. A \right| H_1 \right) P \left( H_1 \right) }{P \left( \left. A \right| H_1 \right) P \left( H_1 \right) +
P \left( \left. A \right| H_2 \right) P \left( H_2 \right) } = \\
= \frac{ \frac{n-1}{N-1} \cdot \frac{n}{N} }{ \frac{n-1}{N-1} \cdot \frac{n}{N} + \frac{n}{N-1} \cdot \frac{N-n}{N} } =
\frac{ \frac{n-1}{N-1} \cdot \frac{n}{N} }{ \frac{n}{N \left( N-1 \right) } \cdot \left( n-1+N-n \right) } =
\frac{n-1}{N-1}.
\end{split}
\end{equation*}
