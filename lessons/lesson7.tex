\addcontentsline{toc}{chapter}{Занятие 7. Алгебры и $ \sigma $-алгебры. Мера}
\chapter*{Занятие 7. Алгебры и $ \sigma $-алгебры. Мера}

\addcontentsline{toc}{section}{Контрольные вопросы и задания}
\section*{Контрольные вопросы и задания}

\subsubsection*{Приведите определение алгебры событий,
$ \sigma $-алгебры событий, конечно-аддитивной меры, счётно-аддитивной меры, вероятностной меры, монотонного класса.}

Совокупность $\mathcal{A}$ подмножеств называется алгеброй, если выполнены свойства:
\begin{enumerate}
\item $ \emptyset \in \mathcal{A}$,
\item $A \in \mathcal{A} \Rightarrow \overline{A} \in \mathcal{A}$,
\item $A, B \in \mathcal{A} \Rightarrow A \cup B \in \mathcal{A}$.
\end{enumerate}

Совокупность множеств называется $ \sigma $-алгеброй, если выполняются следующие условия:
\begin{enumerate}
\item $ \Omega \in \mathcal{A}$,
\item $A \in \mathcal{A} \Rightarrow \overline{A} \in \mathcal{A}$,
\item $A_1, \dotsc, A_n \in \mathcal{A} \Rightarrow \bigcup \limits_{i=1}^n A_i \in \mathcal{A}$.
\end{enumerate}

Функция $ \mu: \mathcal{F} \rightarrow \left[ 0, \infty \right] $ называется конечно-аддитивной мерой (иногда объёмом), если она удовлетворяет следующим аксиомам:
\begin{enumerate}
\item $ \mu \left( \emptyset \right) = 0$,
\item если
$ \left\{ E_n \right\}_{n=1}^N \subset \mathcal{F}$
--- конечное семейство попарно непересекающихся множеств из
$ \mathcal{F}$, то есть $E_i \cap E_j = \emptyset, \, i \neq j$,
то $ \mu \left( \bigcup \limits_{n=1}^N E_n \right) = \sum \limits_{n=1}^N \mu \left( E_n \right)$.
\end{enumerate}

Функция $ \mu: \mathcal{F} \rightarrow \left[ 0, \infty \right] $ называется счётно-аддитивной (или $ \sigma $-аддитивной) мерой, если она удовлетворяет следующим аксиомам:
\begin{enumerate}
\item $ \mu \left( \emptyset \right) = 0$,
\item ($ \sigma $-аддитивность) Если
$ \left\{ E_n \right\}_{n=1}^{ \infty } \subset \mathcal{F}$
--- конечное семейство попарно непересекающихся множеств из
$ \mathcal{F}$, то есть $E_i \cap E_j = \emptyset, \, i \neq j$,
то $ \mu \left( \bigcup \limits_{n=1}^{ \infty } E_n \right) = \sum \limits_{n=1}^{ \infty } \mu \left( E_n \right)$.
\end{enumerate}

Вероятностная мера (вероятность) --- это функция $P: \, \mathcal{F} \rightarrow \left[ 0, 1 \right], \mathcal{F} - \sigma $-алгебра:
\begin{enumerate}
\item $P \left( \Omega \right) = 1$,
\item если
$A_n \in \mathcal{F}, \,
A_n \cap A_m = \emptyset, \,
n \neq m$, то $P \left( \bigcup \limits_{n=1}^{ \infty } A_n \right) = \sum \limits_{n=1}^{ \infty  P \left( A_n \right) }$.
\end{enumerate}

Совокупность $ \mu $ подмножеств множества $ \Omega $ называется монотонным классом,
если для любой монотонной последовательности $ \left\{ A_n \right\}_{n=1}^{ \infty } $
(т.е. такой,
что $A_1 \supset A_2 \supset A_3 \supset \dotsc $ или $A_1 \subset A_2 \subset \dotsc $) её предел $ \lim \limits_{n \rightarrow \infty } A_n$ тоже лежит в $ \mu $.

\subsubsection*{Сформулируйте теорему про монотонный класс.}

Пусть $ \mathcal{A}$ --- алгебра подмножеств множества $ \Omega $.
Тогда монотонный класс, порождаемый $ \mathcal{A}$ совпадает с алгеброй, порождаемой $ \mathcal{A}: \, m \left( \mathcal{A} \right) = \sigma \left( \mathcal{A} \right) $.

\addcontentsline{toc}{section}{Аудиторные задачи}
\section*{Аудиторные задачи}

\subsubsection*{7.3}

\textit{Задание.} Опишите $ \sigma $-алгебру подмножеств отрезка $ \left[ 0, 1 \right] $, порождённую множествами:
\begin{enumerate}[label=\alph*)]
\item $ \left[ 0, 1/2 \right] $;
\item $ \left[ 0, 2/3 \right] $ и $ \left[ 1/3, 1 \right] $;
\item $ \left[ 0, 1/2 \right] $ и $ \left[ 1/2,  1 \right] $.
\end{enumerate}

\textit{Решение.} Множество $ \Omega = \left[ 0, 1 \right] $.

\begin{enumerate}[label=\alph*)]
\item $A = \left[ 0, 1/2 \right] $.
Опишем $ \sigma $-алгебру подмножеств:
$$ \mathcal{A}_A =
\left\{ \emptyset, \Omega, A, \overline{A} \right\} =
\left\{ \emptyset, \left[ 0, 1 \right], \left[ 0, \frac{1}{2} \right], \left( \frac{1}{2}, 1 \right] \right\};$$
\item $B = \left[ 0, 2/3 \right], \, C = \left[ 1/3, 1 \right] $.
Введём непересекающиеся множества, которые в объединении дают всё множество:
$$D_1 =
\left[ 0, \frac{1}{3} \right], \,
D_2 =
\left( \frac{1}{3}, \frac{1}{2} \right],
D_3 =
\left( \frac{2}{3}, 1 \right].$$
Тогда
$$ \mathcal{A}_D =
\left\{ \emptyset, \Omega, \bigcup \limits_{i=1}^k D_i, k = \overline{1,3} \right\};$$
\item $A = \left[ 0, 1/2 \right], \, B = \left[ 1/2,  1 \right] $.
Введём непересекающиеся множества, которые в объединении дают всё множество:
$$D_1 =
\left[ 0, \frac{1}{2} \right), \,
D_2 \left\{ \frac{1}{2} \right\}, \,
D_3 = \left( \frac{1}{2}, 1 \right].$$
Тогда
$$ \mathcal{A}_D =
\left\{ \emptyset, \Omega, \bigcup \limits_{i=1}^k D_i, k = \overline{1,3} \right\}.$$
\end{enumerate}

\subsubsection*{7.4}

\textit{Задание.} Пусть $ \mathcal{B} - \sigma $-алгебра, порождённая интервалами $ \left( a, b \right] $.
Докажите, что множества $ \left( a, b \right), \left[ a, b \right], \left\{ a \right\} $ тоже принадлежат $ \mathcal{B} $.

\textit{Решение.} $ \mathcal{B} $ замкнутая относительно конечного объединения или пересечения множеств, которые в неё входят.
$ \mathcal{B} - \sigma $-алгебра.
Рассмотрим интервалы
$$ \left( a, b - \frac{1}{n} \right].$$
Возьмём бесконечное пересечение
$$ \bigcup \limits_{n=1}^{ \infty } \left( a, b - \frac{1}{n} \right].$$
Тогда приблизимся к точке $b$, но сама она входить не будет:
$$ \left( a, b \right) =
\bigcup \limits_{n=1}^{ \infty } \left( a, b - \frac{1}{n} \right].$$
Для любого $n$ отрезок
$$ \left( a, b - \frac{1}{n} \right] $$
по условию принадлежит $ \sigma $-алгебре $ \mathcal{B}$.
Тогда его счётное объединение тоже принадлежит $ \sigma $-алгебре:
$$ \bigcup \limits_{n=1}^{ \infty } \left( a, b - \frac{1}{n} \right] \in \mathcal{B} $$
по определению.
Тогда $ \left( a, b \right) \in \mathcal{B} $.

Отрезок, где оба конца входят,
$$ \left[ a, b \right] =
\bigcap \limits_{n=1}^{ \infty } \left( a - \frac{1}{n}, b \right].$$
В каждый этот интервал точка $a$ входит.
Тогда в их пересечение точка $a$ входит:
$$ \forall n \,
\left( a - \frac{1}{n}, b \right] \in \mathcal{B} \Rightarrow
\bigcap \limits_{n=1}^{ \infty } \left( a - \frac{1}{n}, b \right] \in \mathcal{B} \Rightarrow
\left[ a, b \right] \in \mathcal{B}.$$
Множество, состоящее из одной точки:
$$ \left\{ a \right\} =
\left[ a, a \right] =
\left( a - \frac{1}{n}, a \right].$$
Для любого $n$ отрезок такого вида принадлежит $ \mathcal{B} $, тогда их счётное пересечение принадлежит $ \mathcal{B} $, тогда $ \left[ a, a \right] \in \mathcal{B} $.
А это значит, что $ \left\{ a \right\} \in \mathcal{B} $.

\subsubsection*{7.5}

\textit{Задание.} Пусть $ \Omega = \mathbb{R}^2, \, \mathcal{B} $ --- борелевская $ \sigma $-алгебра в $ \mathbb{R}^2$.
Докажите, что:
\begin{enumerate}[label=\alph*)]
\item $ \left\{ x \in \mathbb{R}^2: \left| \left| x \right| \right| < 1 \right\} \in \mathcal{B} $;
\item $ \left\{ x \in \mathbb{R}^2: \left| \left| x \right| \right| \leq 1 \right\} \in \mathcal{B} $;
\item $ \left\{ \left( 1, 1 \right) \right\} \in \mathcal{B} $;
\item $ \mathbb{Q} \times \mathbb{Q} \in \mathcal{B} $;
\item $ \left[ 0, 1 \right]^2 \in \mathcal{B} $;
\item $ \left( 0, 1 \right) \times \left[ 2, 3 \right) \in \mathcal{B} $;
\item $ \left\{ \left( x_1, x_2 \right) \in \mathbb{R}^2: sin x_1 + cos \left( x_1 + x_2^3 \right) > 1/7 \right\} \in \mathcal{B} $.
\end{enumerate}

\textit{Решение.} Есть борелевская $ \sigma $-алгебра в $ \mathbb{R}^2$.

\begin{enumerate}[label=\alph*)]
\item $ \left\{ x \in \mathbb{R}^2: \left| \left| x \right| \right| < 1 \right\} \in \mathcal{B} $ ---
внутренняя часть круга с центром в точке 0 и радиусом 1, открытое множество, а значит принадлежит $ \sigma $-алгебре, как открытое множество;
\item это замкнутое множество.
И как дополнение к открытому множеству принадлежит $ \sigma $-алгебре;
\item $ \left\{ \left( 1, 1 \right) \right\} $ --- замкнутое множество.
Поэтому принадлежит $ \sigma $-алгебре.
Или $ \left( 1, 1 \right) = \left\{ 1 \right\} \times \left\{ 1 \right\} \in \mathcal{B} \left( \mathbb{R}^2 \right) $;
\item $ \mathbb{Q} $ --- множество рациональных чисел --- счётное множество.
Всякие счётные объединения и пересечения не выводят за границы $ \sigma $-алгебры.
$ \left( r_i, r_j \right) \in \mathcal{B} $ как замкнутое множество (точка).
$ \forall i, j \, \bigcup \limits_{i=1}^{ \infty } \bigcup \limits_{j=1}^{ \infty } \left( r_i, r_j \right) \in \mathcal{B} $.
\item $ \left[ 0, 1 \right]^2$ --- замкнутое множество в $ \mathbb{R}^2 \Rightarrow \left[ 0, 1 \right]^2 \in \\
\in \mathcal{B} \left( \mathbb{R}^2 \right) $.
Второй способ: $ \left[ 0, 1 \right]^2 = \left[ 0, 1 \right] \times \left[ 0, 1 \right] $.
Знаем, что $ \left[ 0, 1 \right] \in \mathcal{B} \left( \mathbb{R} \right) $.
Тогда
$ \left[ 0, 1 \right] \times \left[ 0, 1 \right] \in \mathcal{B} \left( \mathbb{R}^2 \right) \Rightarrow
\left[ 0, 1 \right]^2 \in \mathcal{B} \left( \mathbb{R}^2 \right) $;
\item покажем, что $ \left[ 2, 3 \right) \in \mathcal{B} \left( \mathbb{R} \right) $.
Будем считать, что $ \sigma $-алгебра порождается интервалами $ \left( a, b \right) \in \mathcal{B} \left( \mathbb{R} \right) $.
Можем записать данный интервал в виде
$$ \left( 2, 3 \right) =
\bigcap \limits_{n=1}^{ \infty } \left( 2 - \frac{1}{n}, 3 \right).$$
В каждый этот интервал точка 2 входит.
Тогда точка 2 входит в пересечение, потому что отступили влево.
Тогда $ \left[ 2, 3 \right) \in \mathcal{B} \left( \mathbb{R} \right) $.
По определению $ \left( 0, 1 \right) \in \mathcal{B} \left( \mathbb{R} \right) $ (это открытое множество).
Тогда $ \left( 0, 1 \right) \times \left[ 2, 3 \right) \in \mathcal{B} $;
\item функции $sin x_1 $ и $cos \left( x_1 + x_2^3 \right) $ ---
ограниченные: $-1 \leq sin x_1 \leq 1, \, -1 \leq \\
\leq cos \left( x_1 + x_2^3 \right) \leq 1$.
А интервалы по определению $ \mathcal{B} \left( \mathbb{R} \right) $ ей принадлежат.
Тогда $ \left\{ \left( x_1, x_2 \right) \in \mathbb{R}^2: sin x_1 + cos \left( x_1 + x_2^3 \right) > 1/7 \right\} \in \mathcal{B} $.
\end{enumerate}

\subsubsection*{7.6}

\textit{Задание.} Пусть $ \mu $ --- конечно-аддитивная мера на алгебре $ \mathcal{B} $ и пусть множества $A, A_1, A_2, \dotsc $ принадлежат $ \mathcal{B} $.
Докажите, что если:
\begin{enumerate}[label=\alph*)]
\item $A = \bigcup \limits_{n=1}^k A_n$ и $A_n$ попарно не пересекаются, то $ \mu \left( A \right) = \sum \limits_{n=1}^k \mu \left( A_n \right) $;
\item $A = \bigcup \limits_{n=1}^{ \infty } A_n$ и $A_n$ попарно не пересекаются, то $ \mu \left( A \right) \geq \sum \limits_{n=1}^{ \infty } \mu \left( A_n \right) $.
\end{enumerate}

\textit{Решение.}
\begin{enumerate}[label=\alph*)]
\item $A = \bigcup \limits_{n=1}^k A_n, \, A_i \cap A_j = \emptyset, \, i \neq j$.
Тогда $ \mu \left( A \right) = \sum \limits_{n=1}^k \mu \left( A_n \right) $ --- следует из определения конечно-аддитивной меры;
\item $A = \bigcup \limits_{n=1}^{ \infty } A_n, \, A_i \cap A_j = \emptyset, \, i \neq j$.
Каждое конечное объединение принадлежит множеству $A$.
Для произвольного $k \geq 1$ рассмотрим конечное объединение: $ \bigcup \limits_{n=1}^k A_n \subset \bigcup \limits_{n=1}^{ \infty } A_n = A$.
Из монотонности можем утверждать, что $ \mu \left( \bigcup \limits_{n=1}^k A_n \right) \leq \mu \left( A \right) $.
Мера конечно-аддитивна, и множества не пересекаются.
Из определения конечно-аддитивной меры
$ \mu \left( \bigcup \limits_{n=1}^k A_n \right) \leq
\sum \limits_{n=1}^k \mu \left( A_n \right) \leq
\mu \left( A \right) \,
\forall k$.
Переходя к границе при $ k \rightarrow \infty $ получаем $ \sum \limits_{n=1}^{ \infty } \mu \left( A_n \right) \leq \mu \left( A \right) $ ---
для $A_i \cap A_j = \emptyset, \, i \neq j$.
\end{enumerate}

\subsubsection*{7.7}

\textit{Задание.} Пусть $ \mu $ --- неотрицательная аддитивная функция, которая задана на алгебре $ \mathcal{B} $.
Докажите, что $ \mu $ является $ \sigma $-аддитивной функцией тогда и только тогда,
когда для произвольных множеств $A, A_1, A_2, \dotsc $,
которые принадлежат $ \mathcal{B} $, из того,
что $A \subset \bigcup \limits_{n=1}^{ \infty } A_n$ следует, что $ \mu \left( A \right) \leq \sum \limits_{n=1}^{ \infty } \mu \left( A_n \right) $.

\textit{Решение.}$ \mu $ --- некоторая неотрицательная аддитивная функция на алгебре $ \mathcal{B} $.
Доказать, что $ \mu - \sigma $-аддитивная мета тогда и только тогда, если из того,
что $A \subset \bigcup \limits_{n=1}^{ \infty }$ следует, что $ \mu \left( A \right) \leq \sum \limits_{n=1}^{ \infty } \mu \left( A_n \right) $.

Пусть $ \mu - \sigma $-аддитивная.
Это означает,
что имеем $ \mu \left( \bigcup \limits_{n=1}^{ \infty } A_n \right) = \sum \limits_{n=1}^{ \infty } \mu \left( A_n \right) $ для множеств,
которые не пересекаются $ \left( A_i \cap A_j = \emptyset, \, i \neq j \right) $.

Если $A \subset \bigcup \limits_{n=1}^{ \infty } A_n$, то по монотонности $ \mu \left( A \right) \leq \mu \left( \bigcup \limits_{n=1}^{ \infty } A_n \right) $.

Объединение множеств, которые не пересекаются обозначим как $ \bigcup \limits_{n=1}^{ \infty } A_m = \\
= \bigsqcup \limits_{n=1}^{ \infty } B_n$,
где $B_1 = A_1, \, B_2 = A_2 \cap \overline{A_1}, \, B_3 = A_3 \cap \overline{A_2} \cap \overline{A_1}, \dotsc $.

Откуда следует,
что
$ \mu \left( \bigcup \limits_{n=1}^{ \infty } A_n \right) =
\mu \left( \bigsqcup \limits_{n=1}^{ \infty } B_n \right) =
\sum \limits_{n=1}^{ \infty } \mu \left( B_n \right) \leq
\sum \limits_{n=1}^{ \infty } \mu \left( A_n \right) $.

Имеем, что $B_2 = A_2 \cap C \Rightarrow B_2 \subseteq A_2 \Rightarrow \mu \left( B_2 \right) \leq \mu \left( A_2 \right) $.
Поэтому $ \mu \left( A \right) \leq \\
\leq \sum \limits_{n=1}^{ \infty } \mu \left( A_n \right) $.
В одну сторону доказали.
Знаем,
что
$A \subset \bigcup \limits_{n=1}^{ \infty } A_n \Rightarrow
\mu \left( A \right) \leq \\
\leq \mu \left( \bigcup \limits_{n=1}^{ \infty } A_n \right) \leq
\sum \limits_{n=1}^{ \infty } \mu \left( A_n \right) $
выполняется для любых множеств $A, A_1, A_2, \dotsc $.
По предыдущей задаче (задача 7.6) $ \mu \left( A \right) \geq \sum \limits_{n=1}^k \mu \left( A_n \right) $ для множеств, которые не пересекаются.
Для множеств, которые не пересекаются возможно только равенство:
$ \mu \left( A \right) = \sum \limits_{n=1}^{ \infty } \mu \left( A_n \right) $ для $A_i \cap A_j = \emptyset, \, i \neq j$.

\subsubsection*{7.8}

\textit{Задание.} Пусть $ \Omega $ --- множество рациональных точек на $ \left[ 0, 1 \right] $, а $ \mathcal{F} $ ---
алгебра множеств,
каждое из которых является бесконечной суммой множеств $A$ вида
$ \left\{ r: \, a < r < b \right\}, \,
\left\{ r: \, a < r \leq b \right\}, \,
\left\{ r: \, a \leq r < b \right\} $,
которые не пересекаются, и пусть $P \left( A \right) = b - a$.
Докажите, что $P$ является конечно-аддитивной, но не счётно-аддитивной функцией множеств.

\textit{Решение.} $ \Omega = \left\{ \left[ 0, 1 \right] \cap \mathbb{Q} \right\} $ --- множество рациональных точек.
$ \mathcal{F} $ --- алгебра множеств, порождённая интервалами любого вида (открытыми, полуоткрытыми, замкнутыми).
Хотим показать, что $P = b - a$ --- конечно-аддитивная, но не счётно-аддитивная.

$ \mathbb{Q} $ --- счётное множество, $ \mathbb{Q} = \bigcup \limits_{i=1}^{ \infty } \left\{ r_i \right\} $.
Длина отрезка $ \left[ 0, 1 \right] - P \left( \Omega \right) = 1$.
Если бы $P$ была счётно-аддитивной мерой, то должно бы быть,
что
$P \left( \Omega \right) = \\
= P \left( \bigcup \limits_{i=1}^{ \infty } \left\{ r_i \right\} \right) =
\sum \limits_{i=1}^{ \infty } P \left( \left\{ r_i \right\} \right) $.
Каждое $P \left( \left\{ r_i \right\} \right) = 0$, так как $ \left\{ r_i \right\} $ --- замкнутый интервал с концами $r_i, r_i$.
Поэтому $P \left( \Omega \right) = 0$ --- противоречие.
Поэтому $P$ не является счётно-аддитивной мерой.

\subsubsection*{7.9}

\textit{Задание.}
На алгебре всех подмножеств множества рациональных чисел отрезка $ \left[ 0, 1 \right] $ введите меру так,
чтобы мера каждого рационального числа была положительной, а мера множества всех рациональных чисел из $ \left[ 0, 1 \right] $ была бы равна единице.

\textit{Решение.}
$ \Omega = \left\{ \mathbb{Q} \cap \left[ 0, 1 \right] \right\} $ --- счётное множество, счётный набор рациональных точек:
$ \Omega =
\left\{ r_1, r_2, \dotsc, r_n, \dotsc \right\} $.

Хотим,
чтобы
$P \left( \Omega \right) =
1 =
\sum \limits_{n=1}^{ \infty } P \left( \left\{ r_n \right\} \right) =
P \left( \left\{ r_1 \right\} \right) + P \left( \left\{ r_2 \right\} \right) + \dotsc $.

Каждое $P \left( \left\{ r_n \right\} \right) $ больше нуля и меньше единицы.

Просуммируем:
$$\sum \limits_{k=1}^{ \infty } \left( \frac{1}{2} \right)^k =
\frac{ \frac{1}{2} }{1 - \frac{1}{2} } =
1.$$
Поэтому все вероятности не нулевые и в сумме дают единицу.

\subsubsection*{7.10}

\textit{Задание.} Может ли число всех событий некоторого вероятностного пространства равняться 129; 130; 128?

\textit{Решение.} Число всех событий некоторого вероятностного пространства может равняться числу, которое является степенью двойки.
Числа 129 и 130 не являются степенями двойки, поэтому не могут быть числом всех событий некоторого вероятностного пространства.
А число $128 = 2^7$, поэтому может быть числом всех событий некоторого вероятностного пространства.

\addcontentsline{toc}{section}{Дополнительные задачи}
\section*{Дополнительные задачи}

\subsubsection*{7.11}

\textit{Задание.} Пусть $ \left( \Omega, \mathcal{F}, \mathbb{P} \right) $ --- произвольное вероятностное пространство.
Докажите,
что множество значений функции
$ \mathbb{P} \left( A \right), \, A \in \mathcal{F} $ является закрытым подмножеством отрезка $ \left[ 0, 1 \right] $.

\textit{Решение.}
Для любого события $A \in \mathcal{F} $ выполняется равенство $ \mathbb{P} \left( A \right) \geq 0$ ---
по определению вероятностной меры на $ \left( \Omega, \mathcal{F} \right) $,
а вероятность достоверного события равна единице: $ \mathbb{P} \left( \Omega \right) = 1$.
Так как $A \subseteq \Omega $, то $P \left( A \right) \leq P \left( \Omega \right) = 1$.
То есть $ \mathbb{P} \left( A \right) \in \left[ 0, 1 \right] $.

Пусть $y_1, y_2, \dotsc $ --- точки $Y$.
Пусть $ \Gamma $ ---
множество всех последовательностей $ \gamma = \left\{ \epsilon_1, \epsilon_2, \dotsc \right\} $, где $ \epsilon_i = 0$ или 1.
В прямом произведении топологических пространств $ \Gamma $ --- это компактное топологическое пространство,
и каждая из функций $ \epsilon_i = \epsilon_i \left( \gamma \right) $ --- это непрерывная функция.
Из ограниченности $ \mathbb{P} \left(Y \right) $ и признака Вейерштрасса следует,
что функция $ \phi \left( \gamma \right) $ определена рядом
$$ \phi \left( \gamma \right) =
\sum \limits_{i=1}^{ \infty } \epsilon_i \mathbb{P} \left( y_i \right) $$
--- это тоже непрерывная функция на $ \Gamma $.
Так как непрерывный образ компактного пространства это компакт,
то он закрыт и так как образ $ \phi \left( \Gamma \right) $ ---
это множество всех значений $ \mathbb{P} $ на подмножествах $Y$, доказательство закончено. 

\addcontentsline{toc}{section}{Домашнее задание}
\section*{Домашнее задание}

\subsubsection*{7.12}

\textit{Задание.} Опишите $ \sigma $-алгебру подмножеств отрезка $ \left[ 0, 1 \right] $, порождённую множествами:
\begin{enumerate}[label=\alph*)]
\item $ \left[ 1/3, 1/2 \right] $;
\item множеством рациональных точек отрезка $ \left[ 0, 1 \right] $;
\item $ \left\{ 0 \right\} $ и $ \left\{ 1 \right\} $.
\end{enumerate}

\textit{Решение.} Множество $ \Omega = \left[ 0, 1 \right] $.

\begin{enumerate}[label=\alph*)]
\item $A = \left[ 1/3, 1/2 \right] $.
Опишем $ \sigma $-алгебру подмножеств:
$$ \mathcal{A}_A =
\left\{ \emptyset, \Omega, A, \overline{A} \right\} =
\left\{ \emptyset, \left[ 0, 1 \right],
\left[ \frac{1}{3}, \frac{1}{2} \right], \left[ 0, \frac{1}{3} \right) \cup \left( \frac{1}{2}, 1 \right] \right\};$$
\item пусть $B = \mathbb{Q} \cap \left[ 0, 1 \right] $ --- множество рациональных точек отрезка $ \left[ 0, 1 \right] $.
Тогда
$ \mathcal{A}_{ \mathbb{Q} } =
\left\{ \emptyset, \Omega, B, \overline{B} \right\} =
\left\{ \emptyset, \left[ 0, 1 \right], \mathbb{Q} \cap \left[ 0, 1 \right], \left[ 0, 1 \right] \setminus \mathbb{Q} \right\};$
\item $C = \left\{ 0 \right\}, E = \left\{ 1 \right\} $.
Введём непересекающиеся множества,
которые в объединении дают весь отрезок: $D_1 = \left\{ 0 \right\}, \, D_2 = \left( 0, 1 \right), \, D_3 = \left\{ 1 \right\} $.
Тогда
$ \mathcal{A}_D =
\left\{ \emptyset, \Omega, \bigcup \limits_{i=1}^k D_i, k = \overline{1, 3} \right\} = \\
= \left\{ \emptyset, \left[0, 1 \right], \left\{ 0 \right\}, \left( 0, 1 \right),
\left\{ 1 \right\}, \left[0, 1 \right), \left(0, 1 \right], \left\{ 0 \right\} \cup \left\{ 1 \right\} \right\} $.
\end{enumerate}

\subsubsection*{7.13}

\textit{Задание.} Пусть $ \Omega = \mathbb{R}, \, \mathcal{B} $ --- борелевская $ \sigma $-алгебра на $ \mathbb{R} $.
Докажите, что $ \mathbb{R} \setminus \mathbb{Q} \in \mathcal{B} $.

\textit{Решение.} Минимальная $ \sigma $-алгебра,
содержащая множество всех интервалов на вещественной прямой,
называется борелевской $ \sigma $-алгеброй в $ \mathbb{R} $ и обозначается $ \mathcal{B} \left( \mathbb{R} \right) $.

Борелевская $ \sigma $-алгебра содержит все закрытые интервалы на $ \mathbb{R} $.
Любое счётное объединение закрытых интервалов на $ \mathbb{R} $ принадлежит $ \mathcal{B} \left( \mathbb{R} \right) $.
Тогда можем взять все рациональные числа $q$ из $ \mathbb{Q} $, которых счётное количество.
Тогда берём множество закрытых интервалов вида $ \left[ q, q \right], q \in \mathbb{Q} $.
Объединение всех этих множеств,
которые принадлежат $ \mathcal{B} \left( \mathbb{R} \right) $ по определению даст множество,
принадлежащее $ \mathcal{B} \left( \mathbb{R} \right) $, т.е. множество $ \mathbb{Q} \in \mathcal{B} \left( \mathbb{R} \right) $.
Тогда его дополнение $ \overline{ \mathbb{Q} } = \mathbb{R} \setminus \mathbb{Q} \in \mathcal{B} \left( \mathbb{R} \right) $.

\subsubsection*{7.14}

\textit{Задание.} Пусть $ \Omega = \mathbb{R}, \, \mathcal{B} $ --- борелевская $ \sigma $-алгебра на $ \mathbb{R}^2$.
Докажите, что:
\begin{enumerate}[label=\alph*)]
\item $ \left( \mathbb{R} \setminus \mathbb{Q} \right) \times \mathbb{Q} \in \mathcal{B} \left( \mathbb{R}^2 \right) $;
\item $ \left\{ \left( x_1, x_2 \right) \in \mathbb{R}^2:
max \left( sin \left( x_1 x_2 \right), arctg \left( x_2 - x_1 \right) \right) > 0.1 \right\} \in \mathcal{B} \left( \mathbb{R}^2 \right) $.
\end{enumerate}

\textit{Решение.}
\begin{enumerate}[label=\alph*)]
\item $ \mathbb{R} \setminus \mathbb{Q} \in \mathcal{B} \left( \mathbb{R} \right), \,
\mathbb{Q} \in \mathcal{B} \left( \mathbb{R} \right) \Rightarrow
\left( \mathbb{R} \setminus \mathbb{Q} \right) \times \mathbb{Q} \in \mathcal{B} \left( \mathbb{R}^2 \right) $;
\item функции $arctg \phi $ и $sin \phi $ --- ограниченные.
В случае максимума синуса функция будет лежать в пределах от $0.1$ до единицы.
В случае максимума арктангенса --- от $0.1$ до $ \pi/2$ не включая $ \pi/2$.
А интервалы по определению $ \mathcal{B} \left( \mathbb{R} \right) $ ей принадлежат.
Тогда $ \\
\left\{ \left( x_1, x_2 \right) \in \mathbb{R}^2:
max \left( sin \left( x_1 x_2 \right), arctg \left( x_2 - x_1 \right) \right) > 0.1 \right\} \in \mathcal{B} \left( \mathbb{R}^2 \right) $.
\end{enumerate}

\subsubsection*{7.15}

\textit{Задание.} Является ли алгеброй ($ \sigma $-алгеброй) совокупность множеств, которые состоят из:
\begin{enumerate}[label=\alph*)]
\item всех подмножеств $ \mathbb{R} $;
\item всех счётных множеств и их дополнений;
\item всех множеств вида $A \cap B$, где $A$ --- произвольное открытое, а $B$ --- произвольное замкнутое множество;
\item всех множеств вида $A \cup B$, где $A$ --- произвольное открытое, а $B$ --- произвольное замкнутое множество.
\end{enumerate}

\textit{Решение.} Множество $ \mathcal{A} $, элементами которого являются множества $ \Omega $ (не обязательно все), называется алгеброй (алгеброй событий), если оно удовлетворяет следующим условиям:
\begin{enumerate}
\item $ \Omega \in \mathcal{A} $ (алгебра событий содержит достоверное событие);
\item если $A \in \mathcal{A} $, то $ \overline{A} \in \mathcal{A} $ (вместе с любым событием алгебра содержит противоположное событие);
\item если $A \in \mathcal{A} $ и $B \in \mathcal{B} $, то $A \cup B \in \mathcal{A} $
(вместе с любыми двумя событиями алгебра содержит их объединение).
\end{enumerate}

Множество $ \mathcal{F} $,
элементами которого являются подмножества $ \Omega $ (не обязательно все),
называется $ \sigma $-алгеброй ($ \sigma $-алгеброй событий), если выполнены следующие условия:
\begin{enumerate}
\item $ \Omega \in \mathcal{F} $ ($ \sigma $-алгебра событий содержит достоверное событие);
\item если $A \in \mathcal{F} $,
то $ \overline{A} \in \mathcal{F} $ (вместе с любым событием $ \sigma $-алгебра содержит противоположное событие);
\item если $A_1, A_2, \dotsc \in \mathcal{F} $, то $A_1 \cup A_2 \cup \dotsc \in \mathcal{F} $
(вместе с любым счётным набором событий $ \sigma $-алгебра содержит их объединение).
\end{enumerate}

\begin{enumerate}[label=\alph*)]
\item Совокупность множеств, которая состоит из всех подмножеств $ \mathbb{R} $ является $ \sigma $-алгеброй, так как содержит всё множество $ \mathbb{R} $, все подмножества $ \mathbb{R} $ и их дополнения, счётные объединения подмножеств;
\item $ \mathcal{A} =$ {$ \left. A \subset \Omega \right| $ A --- счётно} --- $ \sigma $-алгебра, если $ \Omega $ --- счётно.

В данном случае $ \Omega $ --- совокупность всех счётных множеств и их дополнений.
Среди них будут не только счётные множества.
Поэтому такая совокупность множеств не будет алгеброй;
\item \item совокупность всех открытых множеств образует борелевскую $ \sigma $-алгебру, что в свою очередь является $ \sigma $-алгеброй.
Аналогично с замкнутыми множествами.

Если рассматривать вещественную прямую,
то это любого вида интервалы (открытые ---
$ \left[ a, b \right] $, полуоткрытые--- $\left[ a, b \right), \, \left( a, b \right] $, замкнутые --- $ \left[ a, b \right] $.

Тогда что пересечение, что объединение даст любые интервалы (открытые, полуоткрытые, замкнутые),
которые принадлежат заданной совокупности множеств.

Пересечение и объединение открытого и замкнутого множеств даст открытое или замкнутое множество, которое принадлежит этой совокупности.

Получаем $ \sigma $-алгебру.
\end{enumerate}

\subsubsection*{7.16}

\textit{Задание.} Опишите $ \sigma $-алгебру, порождённую:
\begin{enumerate}[label=\alph*)]
\item событиями нулевой вероятности;
\item событиями вероятности единица.
\end{enumerate}

\textit{Решение.} $ \mathcal{A} = \left\{ \left. A \right| P \left( A \right) = 0 \right\} \cup \left\{ \left. A \right| P \left( A \right) = 1 \right\} $.
Вероятность дополнения к событию нулевой вероятности --- событие вероятности $1 - 0$, которое равно 1.
Множество элементарных исходов имеет вероятность 1, потому оно уже включено.
Пустое множество имеет вероятность 0, поэтому оно тоже включено.
Объединение событий нулевой вероятности с событием вероятности 1 согласно правилу $ \sigma $-аддитивности даст событие вероятности 1, т.е. это уже учли.

Объединение событий нулевой вероятности с непустым пересечением можно представить следующим образом.
Пусть $B$ --- это пересечение всех объединяемых множеств.
Тогда объединение можно записать в виде $A_1 \setminus B \cup \\
\cup A_2 \setminus B \cup \dotsc \cup B$.
Имеем объединение непересекающихся случайных событий.
Если вероятностная мера $P$ монотонная,
т.е. если $A \subset B$, то $P \left( A \right) \leq P \left( B \right) $, тогда поскольку пересечение принадлежит каждому множеству, то его вероятность нулевая.
По свойству сигма-аддитивности результат равен нулю.

Объединение событий вероятности 1 --- это будет дополнение к пересечению множеств вероятности 0 согласно правилу де Моргана.
Каждое множество имеет вероятность 0.
И если выбранная вероятностная мера обладает свойством монотонности, то пересечение имеет вероятность 0, а дополнение к нему --- вероятность 1.

Если вероятностная мера не обладает свойством монотонности, то возможно лишь одно решение в общем случае:
$ \left\{ \emptyset, \Omega \right\} $ при условии, что нет других событий с вероятностью 0 и 1.

\subsubsection*{7.17}

\textit{Задание.} Пусть $ \left\{ A_n \right\}_{n \geq 1}$ --- некоторая последовательность событий.
Докажите, что событие $ \varliminf A_n$ принадлежит $ \sigma $-алгебре, порождённой этой последовательность.

\textit{Решение.} Нижний предел последовательности запишем в виде $ \varliminf A_n = \\
= \bigcup \limits_{n=1}^{ \infty } \bigcap \limits_{m=n}^{ \infty } A_m$.

Пусть $ \mathcal{A}$ --- $ \sigma $-алгебра, порождённая последовательностью событий $ \\
\left\{ A_n \right\}_{n \geq 1}$.
По условию каждое $A_n$ принадлежит $ \sigma $-алгебре.
По определению $ \sigma $-алгебры любое счётное пересечение и объединение $A_m$ принадлежит $ \sigma $-алгебре.
Тогда и нижний предел принадлежит $ \sigma $-алгебре.

\subsubsection*{7.18}

\textit{Задание.}
Докажите, что если $ \mathcal{A}_1, \mathcal{A}_2, \dotsc $ ---
неубывающая последовательность $ \sigma $-алгебр, то их объединение $ \mathcal{A} = \bigcup \limits_{n=1}^{ \infty } \mathcal{A}_n$ является алгеброй.

\textit{Решение.} Последовательность является неубывающей.
Это значит, что $ \mathcal{A}_1 \subseteq \mathcal{A}_2 \subseteq \dotsc $.
То есть объединение элементов этой последовательности даст наибольшую $ \sigma $-алгебру из последовательности, то есть такую $ \sigma $-алгебру $ \mathcal{A}_n$, у которой $n$ будет наибольшим.

Получаем,
что
$ \mathcal{A} =
\bigcup \limits_{n=1}^{ \infty } \mathcal{A}_n =
\lim \limits_{n \rightarrow \infty } \mathcal{A}_n =
\mathcal{A}_{ \infty}$.
А это и есть $ \sigma $-алгебра.

Если элемент $B$ принадлежит какой-то из $ \sigma $-алгебр из последовательности,
то он принадлежит и объединению: $B \in \mathcal{A}_i \Rightarrow B \in \mathcal{A} \forall i > 0$.

Если элемент $B$ принадлежит какой-то из $ \sigma $-алгебр из последовательности,
то и $ \overline{B} $ ей принадлежит, а значит принадлежит и объединению:
$B \in \mathcal{A}_i \Rightarrow
\overline{B} \in \mathcal{A}_i \Rightarrow
\overline{B} \in \mathcal{A} $.

Если счётное количество событий принадлежит хоть одной $ \sigma $-алгебре из заданной последовательности,
то ей принадлежит и их счётное объединение, которое принадлежит и объединению данных $ \sigma $-алгебр:
$A_1, A_2, \dotsc \in \mathcal{A}_i \Rightarrow
A_1 \cup A_2 \cup \dotsc \in \mathcal{A}_i \Rightarrow
A_1 \cup A_2 \cup \dotsc \in \mathcal{A} $.

Получаем, что $ \mathcal{A} $ --- алгебра.

\subsubsection*{7.19}

\textit{Задание.} На отрезке $ \left[ 0, 1 \right] $ наугад выбрана точка.
Пусть событие $A_n$ означает, что точка выбрана в отрезке
$$ \left[ 0, \frac{n}{n+1} \right),$$
а событие $B_n$ --- в том, что точка выбрана в интервале $ \left( 0, 1/n \right) $.
Что означают события $ \bigcup \limits_{n=1}^{ \infty } A_n, \, \bigcap \limits_{n=1}^{ \infty } B_n$?

\textit{Решение.} Распишем объединение отрезков:
\begin{equation*}
\begin{split}
\bigcup \limits_{n=1}^{ \infty } A_n =
\bigcup \limits_{n=1}^{ \infty } \left[ 0, \frac{n}{n+1} \right) =
\bigcup \limits_{n=1}^{ \infty } \left[ 0, \frac{n+1-1}{n+1} \right) =
\bigcup \limits_{n=1}^{ \infty } \left[ 0, 1 - \frac{1}{n+1} \right) = \\
= \lim \limits_{n \rightarrow \infty } \left[ 0, 1 - \frac{1}{n+1} \right) =
\left[ 0, 1 \right).
\end{split}
\end{equation*}
Точку выбираем из интервала $ \left[ 0, 1 \right) $.

Распишем пересечение отрезков:
$$ \bigcap \limits_{n=1}^{ \infty } B_n =
\bigcap \limits_{n=1}^{ \infty } \left( 0, \frac{1}{n} \right) =
\lim \limits_{n \rightarrow \infty } \left( 0, \frac{1}{n} \right) =
\left( 0, 0 \right) =
\emptyset.$$
Точку выбираем из пустого множества.

\subsubsection*{7.19}

\textit{Задание.} Пусть $ \Omega $ --- некоторое счётное множество, а $ \mathcal{F} $ --- совокупность всех его подмножеств.
Для $A \in \mathcal{F} $ положим $ \mu \left( A \right) = 0$, если $A$ конечно, и $ \mu \left( A \right) = \infty $, если $A$ бесконечно.
Докажите, что функция множеств $ \mu $ является конечно-аддитивной, но не счётно-аддитивной.

\textit{Решение.} Счётное объединение конечных множеств даёт счётное множество.
Его мера по условию равна бесконечности: .
При этом сумма мер равна нулю, так как мера конечного множества равна нулю: $ \sum \limits_{n=1}^{ \infty } \mu \left( A_n \right) = 0$.
Получили, что не выполняется свойство счётной аддитивности:
$ \mu \left( \bigcup \limits_{n=1}^{ \infty } A_n \right) = \\
= \infty \neq
0 =
\sum \limits_{n=1}^{ \infty } \mu \left( A_n \right) $.

\subsubsection*{7.21}

\textit{Задание.} Докажите, что:
\begin{enumerate}[label=\alph*)]
\item
$P \left( \varliminf \limits_{n \rightarrow \infty } A_n \right) =
\lim \limits_{n \rightarrow \infty } P \left( \bigcap \limits_{k=n}^{ \infty } A_k \right) $;
\item
$P \left( \varliminf \limits_{n \rightarrow \infty } A_n \right) \leq
\varliminf \limits_{n \rightarrow \infty } P \left( A_n \right) \leq
\varlimsup \limits_{n \rightarrow \infty } P \left( A_n \right) \leq
P \left( \varlimsup \limits_{n \rightarrow \infty } A_n \right) $.
\end{enumerate}

\textit{Решение.}
\begin{enumerate}[label=\alph*)]
\item По определению нижнего предела: 
$P \left( \varliminf \limits_{n \rightarrow \infty } A_n \right) =
P \left( \bigcup \limits_{n=1}^{ \infty } \bigcap \limits_{k=n}^{ \infty } A_k \right) $.

Получаем неубывающую последовательность пересечений событий:
$ \\
\bigcap \limits_{k=n}^{ \infty } A_k \subseteq \bigcap \limits_{k=n+1}^{ \infty } A_k$.
Их объединение будет стремиться к последнему элементу.
Поэтому
$$P \left( \varliminf \limits_{n \rightarrow \infty } A_n \right) =
P \left( \bigcup \limits_{n=1}^{ \infty } \bigcap \limits_{k=n}^{ \infty } A_k \right) =
\lim \limits_{n \rightarrow \infty } P \left( \bigcap \limits_{k=n}^{ \infty } A_k \right);$$
\item пересечение множеств входит в каждое множество
$$ \bigcap \limits_{k=n}^{ \infty } A_k \subseteq
A_t, \qquad
\forall n \in \mathbb{N}, \,
\forall t \geq n.$$
По свойству монотонности получаем
$$P \left( \bigcap \limits_{k=n}^{ \infty } A_k \right) \leq
P \left( A_t \right), \qquad
\forall n \in \mathbb{N}, \,
\forall t \geq n.$$
Для любого $t$, которое не меньше фиксированного $n$, данное неравенство справедливо.
Нижний предел вероятностей --- это предел одной из подпоследовательностей этих вероятностей.
Поскольку это предел, то $t$ будет стремиться к бесконечности, что больше любого заранее заданного $n$.
Это значит, что неравенство сохраняется и для нижнего предела, то есть
$$P \left( \bigcap \limits_{k=n}^{ \infty } A_k \right) \leq
\varliminf \limits_{t \rightarrow \infty } P \left( A_t \right), \qquad
\forall n \in \mathbb{N}.$$
В левой части неравенства можем увеличивать $n$ и сделать сколь угодно большим.
Иными словами
$$ \lim \limits_{n \rightarrow \infty } P \left( \bigcap \limits_{k=n}^{ \infty } A_k \right) \leq
\varliminf \limits_{t \rightarrow \infty } P \left( A_t \right).$$
По доказанному из первого пункта
$$P \left( \varliminf \limits_{n \rightarrow \infty } A_n \right) \leq
\varliminf \limits_{n \rightarrow \infty } P \left( A_n \right).$$
По свойствам нижнего и верхнего пределов
$$\varliminf \limits_{n \rightarrow \infty } P \left( A_n \right) \leq
\varlimsup \limits_{n \rightarrow \infty } P \left( A_n \right).$$
Распишем вероятность верхнего предела
$$P \left( \varlimsup \limits_{n \rightarrow \infty } A_n \right) =
P \left( \bigcap \limits_{n=1}^{ \infty } \bigcup \limits_{k=n}^{ \infty } A_k \right).$$
Верхний предел последовательности случайных событий ---
это пересечение монотонно невозрастающей последовательности объединений случайных событий.
Значит вероятность верхнего предела последовательности сходится к пределу вероятности объединений
$$P \left( \bigcap \limits_{n=1}^{ \infty } \bigcup \limits_{k=n}^{ \infty } A_k \right) =
\lim \limits_{n \rightarrow \infty } P \left( \bigcup \limits_{k=n}^{ \infty } A_k \right).$$
Любое множество последовательности входит в объединение множеств из этой последовательности
$$A_t \subseteq \bigcup \limits_{k=n}^{ \infty } A_k, \qquad
\forall n \in \mathbb{N}, \,
\forall t \geq n.$$
По свойству монотонности
$$P \left( A_t \right) \leq
P \left( \bigcup \limits_{k=n}^{ \infty } A_k \right), \qquad
\forall n \in \mathbb{N}, \,
\forall t \geq n.$$
Неравенство справедливо для любого $t$, которое не меньше фиксированного натурального $n$.
Поэтому можем перейти к пределу.
Он может не существовать, поэтому используем верхний предел
$$ \varlimsup \limits_{t \rightarrow \infty } P \left( A_t \right) \leq
P \left( \bigcup \limits_{k=n}^{ \infty } A_k \right), \qquad
\forall n \in \mathbb{N}.$$
В правой части неравенства может увеличивать $n$ и сделать его сколь угодно большим.
Иными словами
$$ \varlimsup \limits_{t \rightarrow \infty } P\left( A_t \right) \leq
\lim \limits_{n \rightarrow \infty }P \left( \bigcup \limits_{k=n}^{ \infty } A_k \right).$$
Получаем
$$ \varlimsup \limits_{n \rightarrow \infty } P \left( A_n \right) \leq
P \left( \varlimsup \limits_{n \rightarrow \infty } A_n \right).$$
\end{enumerate}
