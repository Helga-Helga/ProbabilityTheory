\addcontentsline{toc}{chapter}{Занятие 7. Алгебры и $ \sigma $-алгебры. Мера}
\chapter*{Занятие 7. Алгебры и $ \sigma $-алгебры. Мера}

\addcontentsline{toc}{section}{Контрольные вопросы и задания}
\section*{Контрольные вопросы и задания}

\subsubsection*{Приведите определение алгебры событий,
$ \sigma $-алгебры событий, конечно-аддитивной меры, счётно-аддитивной меры, вероятностной меры, монотонного класса.}

Совокупность $\mathcal{A}$ подмножеств называется алгеброй, если выполнены свойства:
\begin{enumerate}
\item $ \emptyset \in \mathcal{A}$,
\item $A \in \mathcal{A} \Rightarrow \overline{A} \in \mathcal{A}$,
\item $A, B \in \mathcal{A} \Rightarrow A \cup B \in \mathcal{A}$.
\end{enumerate}

Совокупность множеств называется $ \sigma $-алгеброй, если выполняются следующие условия:
\begin{enumerate}
\item $ \Omega \in \mathcal{A}$,
\item $A \in \mathcal{A} \Rightarrow \overline{A} \in \mathcal{A}$,
\item $A_1, \dotsc, A_n \in \mathcal{A} \Rightarrow \bigcup \limits_{i=1}^n A_i \in \mathcal{A}$.
\end{enumerate}

Функция $ \mu: \mathcal{F} \rightarrow \left[ 0, \infty \right] $ называется конечно-аддитивной мерой (иногда объёмом), если она удовлетворяет следующим аксиомам:
\begin{enumerate}
\item $ \mu \left( \emptyset \right) = 0$,
\item если
$ \left\{ E_n \right\}_{n=1}^N \subset \mathcal{F}$
--- конечное семейство попарно непересекающихся множеств из
$ \mathcal{F}$, то есть $E_i \cap E_j = \emptyset, \, i \neq j$,
то $ \mu \left( \bigcup \limits_{n=1}^N E_n \right) = \sum \limits_{n=1}^N \mu \left( E_n \right)$.
\end{enumerate}

Функция $ \mu: \mathcal{F} \rightarrow \left[ 0, \infty \right] $ называется счётно-аддитивной (или $ \sigma $-аддитивной) мерой, если она удовлетворяет следующим аксиомам:
\begin{enumerate}
\item $ \mu \left( \emptyset \right) = 0$,
\item ($ \sigma $-аддитивность) Если
$ \left\{ E_n \right\}_{n=1}^{ \infty } \subset \mathcal{F}$
--- конечное семейство попарно непересекающихся множеств из
$ \mathcal{F}$, то есть $E_i \cap E_j = \emptyset, \, i \neq j$,
то $ \mu \left( \bigcup \limits_{n=1}^{ \infty } E_n \right) = \sum \limits_{n=1}^{ \infty } \mu \left( E_n \right)$.
\end{enumerate}

Вероятностная мера (вероятность) --- это функция $P: \, \mathcal{F} \rightarrow \left[ 0, 1 \right], \mathcal{F} - \sigma $-алгебра:
\begin{enumerate}
\item $P \left( \Omega \right) = 1$,
\item если
$A_n \in \mathcal{F}, \,
A_n \cap A_m = \emptyset, \,
n \neq m$, то $P \left( \bigcup \limits_{n=1}^{ \infty } A_n \right) = \sum \limits_{n=1}^{ \infty  P \left( A_n \right) }$.
\end{enumerate}

Совокупность $ \mu $ подмножеств множества $ \Omega $ называется монотонным классом,
если для любой монотонной последовательности $ \left\{ A_n \right\}_{n=1}^{ \infty } $
(т.е. такой,
что $A_1 \supset A_2 \supset A_3 \supset \dotsc $ или $A_1 \subset A_2 \subset \dotsc $) её предел $ \lim \limits_{n \rightarrow \infty } A_n$ тоже лежит в $ \mu $.

\subsubsection*{Сформулируйте теорему про монотонный класс.}

Пусть $ \mathcal{A}$ --- алгебра подмножеств множества $ \Omega $.
Тогда монотонный класс, порождаемый $ \mathcal{A}$ совпадает с алгеброй, порождаемой $ \mathcal{A}: \, m \left( \mathcal{A} \right) = \sigma \left( \mathcal{A} \right) $.

\addcontentsline{toc}{section}{Аудиторные задачи}
\section*{Аудиторные задачи}

\subsubsection*{7.3}

\textit{Задание.} Опишите $ \sigma $-алгебру подмножеств отрезка $ \left[ 0, 1 \right] $, порождённую множествами:
\begin{enumerate}[label=\alph*)]
\item $ \left[ 0, 1/2 \right] $;
\item $ \left[ 0, 2/3 \right] $ и $ \left[ 1/3, 1 \right] $;
\item $ \left[ 0, 1/2 \right] $ и $ \left[ 1/2,  1 \right] $.
\end{enumerate}

\textit{Решение.} Множество $ \Omega = \left[ 0, 1 \right] $.

\begin{enumerate}[label=\alph*)]
\item $A = \left[ 0, 1/2 \right] $.
Опишем $ \sigma $-алгебру подмножеств:
$$ \mathcal{A}_A =
\left\{ \emptyset, \Omega, A, \overline{A} \right\} =
\left\{ \emptyset, \left[ 0, 1 \right], \left[ 0, \frac{1}{2} \right], \left( \frac{1}{2}, 1 \right] \right\};$$
\item $B = \left[ 0, 2/3 \right], \, C = \left[ 1/3, 1 \right] $.
Введём непересекающиеся множества, которые в объединении дают всё множество:
$$D_1 =
\left[ 0, \frac{1}{3} \right], \,
D_2 =
\left( \frac{1}{3}, \frac{1}{2} \right],
D_3 =
\left( \frac{2}{3}, 1 \right].$$
Тогда
$$ \mathcal{A}_D =
\left\{ \emptyset, \Omega, \bigcup \limits_{i=1}^k D_i, k = \overline{1,3} \right\};$$
\item $A = \left[ 0, 1/2 \right], \, B = \left[ 1/2,  1 \right] $.
Введём непересекающиеся множества, которые в объединении дают всё множество:
$$D_1 =
\left[ 0, \frac{1}{2} \right), \,
D_2 \left\{ \frac{1}{2} \right\}, \,
D_3 = \left( \frac{1}{2}, 1 \right].$$
Тогда
$$ \mathcal{A}_D =
\left\{ \emptyset, \Omega, \bigcup \limits_{i=1}^k D_i, k = \overline{1,3} \right\}.$$
\end{enumerate}

\subsubsection*{7.4}

\textit{Задание.} Пусть $ \mathcal{B} - \sigma $-алгебра, порождённая интервалами $ \left( a, b \right] $.
Докажите, что множества $ \left( a, b \right), \left[ a, b \right], \left\{ a \right\} $ тоже принадлежат $ \mathcal{B} $.

\textit{Решение.} $ \mathcal{B} $ замкнутая относительно конечного объединения или пересечения множеств, которые в неё входят.
$ \mathcal{B} - \sigma $-алгебра.
Рассмотрим интервалы
$$ \left( a, b - \frac{1}{n} \right].$$
Возьмём бесконечное пересечение
$$ \bigcup \limits_{n=1}^{ \infty } \left( a, b - \frac{1}{n} \right].$$
Тогда приблизимся к точке $b$, но сама она входить не будет:
$$ \left( a, b \right) =
\bigcup \limits_{n=1}^{ \infty } \left( a, b - \frac{1}{n} \right].$$
Для любого $n$ отрезок
$$ \left( a, b - \frac{1}{n} \right] $$
по условию принадлежит $ \sigma $-алгебре $ \mathcal{B}$.
Тогда его счётное объединение тоже принадлежит $ \sigma $-алгебре:
$$ \bigcup \limits_{n=1}^{ \infty } \left( a, b - \frac{1}{n} \right] \in \mathcal{B} $$
по определению.
Тогда $ \left( a, b \right) \in \mathcal{B} $.

Отрезок, где оба конца входят,
$$ \left[ a, b \right] =
\bigcap \limits_{n=1}^{ \infty } \left( a - \frac{1}{n}, b \right].$$
В каждый этот интервал точка $a$ входит.
Тогда в их пересечение точка $a$ входит:
$$ \forall n \,
\left( a - \frac{1}{n}, b \right] \in \mathcal{B} \Rightarrow
\bigcap \limits_{n=1}^{ \infty } \left( a - \frac{1}{n}, b \right] \in \mathcal{B} \Rightarrow
\left[ a, b \right] \in \mathcal{B}.$$
Множество, состоящее из одной точки:
$$ \left\{ a \right\} =
\left[ a, a \right] =
\left( a - \frac{1}{n}, a \right].$$
Для любого $n$ отрезок такого вида принадлежит $ \mathcal{B} $, тогда их счётное пересечение принадлежит $ \mathcal{B} $, тогда $ \left[ a, a \right] \in \mathcal{B} $.
А это значит, что $ \left\{ a \right\} \in \mathcal{B} $.

\subsubsection*{7.5}

\textit{Задание.} Пусть $ \Omega = \mathbb{R}^2, \, \mathcal{B} $ --- борелевская $ \sigma $-алгебра в $ \mathbb{R}^2$.
Докажите, что:
\begin{enumerate}[label=\alph*)]
\item $ \left\{ x \in \mathbb{R}^2: \left| \left| x \right| \right| < 1 \right\} \in \mathcal{B} $;
\item $ \left\{ x \in \mathbb{R}^2: \left| \left| x \right| \right| \leq 1 \right\} \in \mathcal{B} $;
\item $ \left\{ \left( 1, 1 \right) \right\} \in \mathcal{B} $;
\item $ \mathbb{Q} \times \mathbb{Q} \in \mathcal{B} $;
\item $ \left[ 0, 1 \right]^2 \in \mathcal{B} $;
\item $ \left( 0, 1 \right) \times \left[ 2, 3 \right) \in \mathcal{B} $;
\item $ \left\{ \left( x_1, x_2 \right) \in \mathbb{R}^2: sin x_1 + cos \left( x_1 + x_2^3 \right) > 1/7 \right\} \in \mathcal{B} $.
\end{enumerate}

\textit{Решение.} Есть борелевская $ \sigma $-алгебра в $ \mathbb{R}^2$.

\begin{enumerate}[label=\alph*)]
\item $ \left\{ x \in \mathbb{R}^2: \left| \left| x \right| \right| < 1 \right\} \in \mathcal{B} $ ---
внутренняя часть круга с центром в точке 0 и радиусом 1, открытое множество, а значит принадлежит $ \sigma $-алгебре, как открытое множество;
\item это замкнутое множество.
И как дополнение к открытому множеству принадлежит $ \sigma $-алгебре;
\item $ \left\{ \left( 1, 1 \right) \right\} $ --- замкнутое множество.
Поэтому принадлежит $ \sigma $-алгебре.
Или $ \left( 1, 1 \right) = \left\{ 1 \right\} \times \left\{ 1 \right\} \in \mathcal{B} \left( \mathbb{R}^2 \right) $;
\item $ \mathbb{Q} $ --- множество рациональных чисел --- счётное множество.
Всякие счётные объединения и пересечения не выводят за границы $ \sigma $-алгебры.
$ \left( r_i, r_j \right) \in \mathcal{B} $ как замкнутое множество (точка).
$ \forall i, j \, \bigcup \limits_{i=1}^{ \infty } \bigcup \limits_{j=1}^{ \infty } \left( r_i, r_j \right) \in \mathcal{B} $.
\item $ \left[ 0, 1 \right]^2$ --- замкнутое множество в $ \mathbb{R}^2 \Rightarrow \left[ 0, 1 \right]^2 \in \\
\in \mathcal{B} \left( \mathbb{R}^2 \right) $.
Второй способ: $ \left[ 0, 1 \right]^2 = \left[ 0, 1 \right] \times \left[ 0, 1 \right] $.
Знаем, что $ \left[ 0, 1 \right] \in \mathcal{B} \left( \mathbb{R} \right) $.
Тогда
$ \left[ 0, 1 \right] \times \left[ 0, 1 \right] \in \mathcal{B} \left( \mathbb{R}^2 \right) \Rightarrow
\left[ 0, 1 \right]^2 \in \mathcal{B} \left( \mathbb{R}^2 \right) $;
\item покажем, что $ \left[ 2, 3 \right) \in \mathcal{B} \left( \mathbb{R} \right) $.
Будем считать, что $ \sigma $-алгебра порождается интервалами $ \left( a, b \right) \in \mathcal{B} \left( \mathbb{R} \right) $.
Можем записать данный интервал в виде
$$ \left( 2, 3 \right) =
\bigcap \limits_{n=1}^{ \infty } \left( 2 - \frac{1}{n}, 3 \right).$$
В каждый этот интервал точка 2 входит.
Тогда точка 2 входит в пересечение, потому что отступили влево.
Тогда $ \left[ 2, 3 \right) \in \mathcal{B} \left( \mathbb{R} \right) $.
По определению $ \left( 0, 1 \right) \in \mathcal{B} \left( \mathbb{R} \right) $ (это открытое множество).
Тогда $ \left( 0, 1 \right) \times \left[ 2, 3 \right) \in \mathcal{B} $;
\item функции $sin x_1 $ и $cos \left( x_1 + x_2^3 \right) $ ---
ограниченные: $-1 \leq sin x_1 \leq 1, \, -1 \leq \\
\leq cos \left( x_1 + x_2^3 \right) \leq 1$.
А интервалы по определению $ \mathcal{B} \left( \mathbb{R} \right) $ ей принадлежат.
Тогда $ \left\{ \left( x_1, x_2 \right) \in \mathbb{R}^2: sin x_1 + cos \left( x_1 + x_2^3 \right) > 1/7 \right\} \in \mathcal{B} $.
\end{enumerate}

\subsubsection*{7.6}

\textit{Задание.} Пусть $ \mu $ --- конечно-аддитивная мера на алгебре $ \mathcal{B} $ и пусть множества $A, A_1, A_2, \dotsc $ принадлежат $ \mathcal{B} $.
Докажите, что если:
\begin{enumerate}[label=\alph*)]
\item $A = \bigcup \limits_{n=1}^k A_n$ и $A_n$ попарно не пересекаются, то $ \mu \left( A \right) = \sum \limits_{n=1}^k \mu \left( A_n \right) $;
\item $A = \bigcup \limits_{n=1}^{ \infty } A_n$ и $A_n$ попарно не пересекаются, то $ \mu \left( A \right) \geq \sum \limits_{n=1}^{ \infty } \mu \left( A_n \right) $.
\end{enumerate}

\textit{Решение.}
\begin{enumerate}[label=\alph*)]
\item $A = \bigcup \limits_{n=1}^k A_n, \, A_i \cap A_j = \emptyset, \, i \neq j$.
Тогда $ \mu \left( A \right) = \sum \limits_{n=1}^k \mu \left( A_n \right) $ --- следует из определения конечно-аддитивной меры;
\item $A = \bigcup \limits_{n=1}^{ \infty } A_n, \, A_i \cap A_j = \emptyset, \, i \neq j$.
Каждое конечное объединение принадлежит множеству $A$.
Для произвольного $k \geq 1$ рассмотрим конечное объединение: $ \bigcup \limits_{n=1}^k A_n \subset \bigcup \limits_{n=1}^{ \infty } A_n = A$.
Из монотонности можем утверждать, что $ \mu \left( \bigcup \limits_{n=1}^k A_n \right) \leq \mu \left( A \right) $.
Мера конечно-аддитивна, и множества не пересекаются.
Из определения конечно-аддитивной меры
$ \mu \left( \bigcup \limits_{n=1}^k A_n \right) \leq
\sum \limits_{n=1}^k \mu \left( A_n \right) \leq
\mu \left( A \right) \,
\forall k$.
Переходя к границе при $ k \rightarrow \infty $ получаем $ \sum \limits_{n=1}^{ \infty } \mu \left( A_n \right) \leq \mu \left( A \right) $ ---
для $A_i \cap A_j = \emptyset, \, i \neq j$.
\end{enumerate}

\subsubsection*{7.7}

\textit{Задание.} Пусть $ \mu $ --- неотрицательная аддитивная функция, которая задана на алгебре $ \mathcal{B} $.
Докажите, что $ \mu $ является $ \sigma $-аддитивной функцией тогда и только тогда,
когда для произвольных множеств $A, A_1, A_2, \dotsc $,
которые принадлежат $ \mathcal{B} $, из того,
что $A \subset \bigcup \limits_{n=1}^{ \infty } A_n$ следует, что $ \mu \left( A \right) \leq \sum \limits_{n=1}^{ \infty } \mu \left( A_n \right) $.

\textit{Решение.}$ \mu $ --- некоторая неотрицательная аддитивная функция на алгебре $ \mathcal{B} $.
Доказать, что $ \mu - \sigma $-аддитивная мета тогда и только тогда, если из того,
что $A \subset \bigcup \limits_{n=1}^{ \infty }$ следует, что $ \mu \left( A \right) \leq \sum \limits_{n=1}^{ \infty } \mu \left( A_n \right) $.

Пусть $ \mu - \sigma $-аддитивная.
Это означает,
что имеем $ \mu \left( \bigcup \limits_{n=1}^{ \infty } A_n \right) = \sum \limits_{n=1}^{ \infty } \mu \left( A_n \right) $ для множеств,
которые не пересекаются $ \left( A_i \cap A_j = \emptyset, \, i \neq j \right) $.

Если $A \subset \bigcup \limits_{n=1}^{ \infty } A_n$, то по монотонности $ \mu \left( A \right) \leq \mu \left( \bigcup \limits_{n=1}^{ \infty } A_n \right) $.

Объединение множеств, которые не пересекаются обозначим как $ \bigcup \limits_{n=1}^{ \infty } A_m = \\
= \bigsqcup \limits_{n=1}^{ \infty } B_n$,
где $B_1 = A_1, \, B_2 = A_2 \cap \overline{A_1}, \, B_3 = A_3 \cap \overline{A_2} \cap \overline{A_1}, \dotsc $.

Откуда следует,
что
$ \mu \left( \bigcup \limits_{n=1}^{ \infty } A_n \right) =
\mu \left( \bigsqcup \limits_{n=1}^{ \infty } B_n \right) =
\sum \limits_{n=1}^{ \infty } \mu \left( B_n \right) \leq
\sum \limits_{n=1}^{ \infty } \mu \left( A_n \right) $.

Имеем, что $B_2 = A_2 \cap C \Rightarrow B_2 \subseteq A_2 \Rightarrow \mu \left( B_2 \right) \leq \mu \left( A_2 \right) $.
Поэтому $ \mu \left( A \right) \leq \\
\leq \sum \limits_{n=1}^{ \infty } \mu \left( A_n \right) $.
В одну сторону доказали.
Знаем,
что
$A \subset \bigcup \limits_{n=1}^{ \infty } A_n \Rightarrow
\mu \left( A \right) \leq \\
\leq \mu \left( \bigcup \limits_{n=1}^{ \infty } A_n \right) \leq
\sum \limits_{n=1}^{ \infty } \mu \left( A_n \right) $
выполняется для любых множеств $A, A_1, A_2, \dotsc $.
По предыдущей задаче (задача 7.6) $ \mu \left( A \right) \geq \sum \limits_{n=1}^k \mu \left( A_n \right) $ для множеств, которые не пересекаются.
Для множеств, которые не пересекаются возможно только равенство:
$ \mu \left( A \right) = \sum \limits_{n=1}^{ \infty } \mu \left( A_n \right) $ для $A_i \cap A_j = \emptyset, \, i \neq j$.

\subsubsection*{7.8}

\textit{Задание.} Пусть $ \Omega $ --- множество рациональных точек на $ \left[ 0, 1 \right] $, а $ \mathcal{F} $ ---
алгебра множеств,
каждое из которых является бесконечной суммой множеств $A$ вида
$ \left\{ r: \, a < r < b \right\}, \,
\left\{ r: \, a < r \leq b \right\}, \,
\left\{ r: \, a \leq r < b \right\} $,
которые не пересекаются, и пусть $P \left( A \right) = b - a$.
Докажите, что $P$ является конечно-аддитивной, но не счётно-аддитивной функцией множеств.

\textit{Решение.} $ \Omega = \left\{ \left[ 0, 1 \right] \cap \mathbb{Q} \right\} $ --- множество рациональных точек.
$ \mathcal{F} $ --- алгебра множеств, порождённая интервалами любого вида (открытыми, полуоткрытыми, замкнутыми).
Хотим показать, что $P = b - a$ --- конечно-аддитивная, но не счётно-аддитивная.

$ \mathbb{Q} $ --- счётное множество, $ \mathbb{Q} = \bigcup \limits_{i=1}^{ \infty } \left\{ r_i \right\} $.
Длина отрезка $ \left[ 0, 1 \right] - P \left( \Omega \right) = 1$.
Если бы $P$ была счётно-аддитивной мерой, то должно бы быть,
что
$P \left( \Omega \right) = \\
= P \left( \bigcup \limits_{i=1}^{ \infty } \left\{ r_i \right\} \right) =
\sum \limits_{i=1}^{ \infty } P \left( \left\{ r_i \right\} \right) $.
Каждое $P \left( \left\{ r_i \right\} \right) = 0$, так как $ \left\{ r_i \right\} $ --- замкнутый интервал с концами $r_i, r_i$.
Поэтому $P \left( \Omega \right) = 0$ --- противоречие.
Поэтому $P$ не является счётно-аддитивной мерой.

\addcontentsline{toc}{section}{Домашнее задание}
\section*{Домашнее задание}

\subsubsection*{7.12}

\textit{Задание.} Опишите $ \sigma $-алгебру подмножеств отрезка $ \left[ 0, 1 \right] $, порождённую множествами:
\begin{enumerate}[label=\alph*)]
\item $ \left[ 1/3, 1/2 \right] $;
\item множеством рациональных точек отрезка $ \left[ 0, 1 \right] $;
\item $ \left\{ 0 \right\} $ и $ \left\{ 1 \right\} $.
\end{enumerate}

\textit{Решение.} Множество $ \Omega = \left[ 0, 1 \right] $.

\begin{enumerate}[label=\alph*)]
\item $A = \left[ 1/3, 1/2 \right] $.
Опишем $ \sigma $-алгебру подмножеств:
$$ \mathcal{A}_A =
\left\{ \emptyset, \Omega, A, \overline{A} \right\} =
\left\{ \emptyset, \left[ 0, 1 \right],
\left[ \frac{1}{3}, \frac{1}{2} \right], \left[ 0, \frac{1}{3} \right) \cup \left( \frac{1}{2}, 1 \right] \right\};$$
\item пусть $B = \mathbb{Q} \cap \left[ 0, 1 \right] $ --- множество рациональных точек отрезка $ \left[ 0, 1 \right] $.
Тогда
$ \mathcal{A}_{ \mathbb{Q} } =
\left\{ \emptyset, \Omega, B, \overline{B} \right\} =
\left\{ \emptyset, \left[ 0, 1 \right], \mathbb{Q} \cap \left[ 0, 1 \right], \left[ 0, 1 \right] \setminus \mathbb{Q} \right\};$
\item $C = \left\{ 0 \right\}, E = \left\{ 1 \right\} $.
Введём непересекающиеся множества,
которые в объединении дают весь отрезок: $D_1 = \left\{ 0 \right\}, \, D_2 = \left( 0, 1 \right), \, D_3 = \left\{ 1 \right\} $.
Тогда
$ \mathcal{A}_D =
\left\{ \emptyset, \Omega, \bigcup \limits_{i=1}^k D_i, k = \overline{1, 3} \right\} = \\
= \left\{ \emptyset, \left[0, 1 \right], \left\{ 0 \right\}, \left( 0, 1 \right),
\left\{ 1 \right\}, \left[0, 1 \right), \left(0, 1 \right], \left\{ 0 \right\} \cup \left\{ 1 \right\} \right\} $.
\end{enumerate}

\subsubsection*{7.13}

\textit{Задание.} Пусть $ \Omega = \mathbb{R}, \, \mathcal{B} $ --- борелевская $ \sigma $-алгебра на $ \mathbb{R} $.
Докажите, что $ \mathbb{R} \setminus \mathbb{Q} \in \mathcal{B} $.

\textit{Решение.} Минимальная $ \sigma $-алгебра,
содержащая множество всех интервалов на вещественной прямой,
называется борелевской $ \sigma $-алгеброй в $ \mathbb{R} $ и обозначается $ \mathcal{B} \left( \mathbb{R} \right) $.

Борелевская $ \sigma $-алгебра содержит все закрытые интервалы на $ \mathbb{R} $.
Любое счётное объединение закрытых интервалов на $ \mathbb{R} $ принадлежит $ \mathcal{B} \left( \mathbb{R} \right) $.
Тогда можем взять все рациональные числа $q$ из $ \mathbb{Q} $, которых счётное количество.
Тогда берём множество закрытых интервалов вида $ \left[ q, q \right], q \in \mathbb{Q} $.
Объединение всех этих множеств,
которые принадлежат $ \mathcal{B} \left( \mathbb{R} \right) $ по определению даст множество,
принадлежащее $ \mathcal{B} \left( \mathbb{R} \right) $, т.е. множество $ \mathbb{Q} \in \mathcal{B} \left( \mathbb{R} \right) $.
Тогда его дополнение $ \overline{ \mathbb{Q} } = \mathbb{R} \setminus \mathbb{Q} \in \mathcal{B} \left( \mathbb{R} \right) $.

\subsubsection*{7.14}

\textit{Задание.} Пусть $ \Omega = \mathbb{R}, \, \mathcal{B} $ --- борелевская $ \sigma $-алгебра на $ \mathbb{R}^2$.
Докажите, что:
\begin{enumerate}[label=\alph*)]
\item $ \left( \mathbb{R} \setminus \mathbb{Q} \right) \times \mathbb{Q} \in \mathcal{B} \left( \mathbb{R}^2 \right) $;
\item $ \left\{ \left( x_1, x_2 \right) \in \mathbb{R}^2:
max \left( sin \left( x_1 x_2 \right), arctg \left( x_2 - x_1 \right) \right) > 0.1 \right\} \in \mathcal{B} \left( \mathbb{R}^2 \right) $.
\end{enumerate}

\textit{Решение.}
\begin{enumerate}[label=\alph*)]
\item $ \mathbb{R} \setminus \mathbb{Q} \in \mathcal{B} \left( \mathbb{R} \right), \,
\mathbb{Q} \in \mathcal{B} \left( \mathbb{R} \right) \Rightarrow
\left( \mathbb{R} \setminus \mathbb{Q} \right) \times \mathbb{Q} \in \mathcal{B} \left( \mathbb{R}^2 \right) $;
\item функции $arctg \phi $ и $sin \phi $ --- ограниченные.
В случае максимума синуса функция будет лежать в пределах от $0.1$ до единицы.
В случае максимума арктангенса --- от $0.1$ до $ \pi/2$ не включая $ \pi/2$.
А интервалы по определению $ \mathcal{B} \left( \mathbb{R} \right) $ ей принадлежат.
Тогда $ \\
\left\{ \left( x_1, x_2 \right) \in \mathbb{R}^2:
max \left( sin \left( x_1 x_2 \right), arctg \left( x_2 - x_1 \right) \right) > 0.1 \right\} \in \mathcal{B} \left( \mathbb{R}^2 \right) $.
\end{enumerate}

\subsubsection*{7.16}

\textit{Задание.} Опишите $ \sigma $-алгебру, порождённую:
\begin{enumerate}[label=\alph*)]
\item событиями нулевой вероятности;
\item событиями вероятности единица.
\end{enumerate}

\textit{Решение.} $ \mathcal{A} = \left\{ \left. A \right| P \left( A \right) = 0 \right\} \cup \left\{ \left. A \right| P \left( A \right) = 1 \right\} $.
Вероятность дополнения к событию нулевой вероятности --- событие вероятности $1 - 0$, которое равно 1.
Множество элементарных исходов имеет вероятность 1, потому оно уже включено.
Пустое множество имеет вероятность 0, поэтому оно тоже включено.
Объединение событий нулевой вероятности с событием вероятности 1 согласно правилу $ \sigma $-аддитивности даст событие вероятности 1, т.е. это уже учли.

Объединение событий нулевой вероятности с непустым пересечением можно представить следующим образом.
Пусть $B$ --- это пересечение всех объединяемых множеств.
Тогда объединение можно записать в виде $A_1 \setminus B \cup \\
\cup A_2 \setminus B \cup \dotsc \cup B$.
Имеем объединение непересекающихся случайных событий.
Если вероятностная мера $P$ монотонная,
т.е. если $A \subset B$, то $P \left( A \right) \leq P \left( B \right) $, тогда поскольку пересечение принадлежит каждому множеству, то его вероятность нулевая.
По свойству сигма-аддитивности результат равен нулю.

Объединение событий вероятности 1 --- это будет дополнение к пересечению множеств вероятности 0 согласно правилу де Моргана.
Каждое множество имеет вероятность 0.
И если выбранная вероятностная мера обладает свойством монотонности, то пересечение имеет вероятность 0, а дополнение к нему --- вероятность 1.

Если вероятностная мера не обладает свойством монотонности, то возможно лишь одно решение в общем случае:
$ \left\{ \emptyset, \Omega \right\} $ при условии, что нет других событий с вероятностью 0 и 1.

\subsubsection*{7.17}

\textit{Задание.} Пусть $ \left\{ A_n \right\}_{n \geq 1}$ --- некоторая последовательность событий.
Докажите, что событие $ \varliminf A_n$ принадлежит $ \sigma $-алгебре, порождённой этой последовательность.

\textit{Решение.} Нижний предел последовательности запишем в виде $ \varliminf A_n = \\
= \bigcup \limits_{n=1}^{ \infty } \bigcap \limits_{m=n}^{ \infty } A_m$.

Пусть $ \mathcal{A}$ --- $ \sigma $-алгебра, порождённая последовательностью событий $ \\
\left\{ A_n \right\}_{n \geq 1}$.
По условию каждое $A_n$ принадлежит $ \sigma $-алгебре.
По определению $ \sigma $-алгебры любое счётное пересечение и объединение $A_m$ принадлежит $ \sigma $-алгебре.
Тогда и нижний предел принадлежит $ \sigma $-алгебре.
