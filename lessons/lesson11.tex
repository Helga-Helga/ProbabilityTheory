\addcontentsline{toc}{chapter}{Занятие 11. Совместное распределение случайных величин}
\chapter*{Занятие 11. Совместное распределение случайных величин}

\addcontentsline{toc}{section}{Контрольные вопросы и задания}
\section*{Контрольные вопросы и задания}

\subsubsection*{Запишите основные вероятностные распределения и поясните смысл их параметров.}

Основные распределения:
\begin{enumerate}
\item дискретное распределение:
\begin{enumerate}
\item равномерное дискретное распределение
$$P \left( \xi = k \right) =
\frac{1}{N},
1 \leq k \leq N,
M \xi = \frac{N+1}{2}, D \xi = \frac{N^2 - 1}{12};$$
\item биномиальное распределение: $ \xi \sim Biom \left( n, p \right) $
$$P \left\{ \xi = k \right\} =
C_n^k p^k q^{n-k},$$
где $0 \leq k \leq n, q = 1 - p, M \xi = np, D \xi = npq$;
\item геометрическое распределение $ \xi \sim Geom \left( p \right), p \in \left( 0, 1 \right) $
$$P \left\{ \xi = k \right\} =
pq^k,
k \geq 0,
M \xi = \frac{q}{p},
D \xi = \frac{q}{p^2};$$
\item пуассоновское распределение: $ \xi \sim Pois \left( \lambda \right), \lambda > 0$
$$P \left\{ \xi = k \right\} = \frac{ \lambda^k}{k!} \cdot e^{- \lambda },
k \geq 0,
M \xi = \lambda,
D \xi = \lambda;$$
\end{enumerate}
\item непрерывные распределения:
\begin{enumerate}
\item равномерное распределение: $ \xi \sim U \left( \left[ a, b \right] \right) $
$$p \left( x \right) =
\begin{cases}
\frac{1}{b-a}, \qquad x \in \left[ a, b \right], \\
0, x \notin \left[ a, b \right], \\
\end{cases}
M \xi = \frac{a+b}{2},
D \xi = \frac{ \left( b-a \right)^2}{12};$$
\item экспоненциальное распределение: $ \xi \sim Exp \left( \lambda \right), \lambda > 0$
$$p \left( x \right) =
\begin{cases}
\lambda e^{- \lambda x}, \qquad x \geq 0, \\
0, \qquad x < 0, \\
\end{cases}
M \xi = \frac{1}{ \lambda },
D \xi = \frac{1}{ \lambda^2};$$
\item распределение Коши: $ \xi \sim C \left( \Theta \right), \Theta > 0$
$$p \left( x \right) =
\frac{ \Theta }{ \pi \left( x^2 + \Theta^2 \right) };$$
\item гауссовское (нормальное) распределение:
$$ \xi \sim \mathcal{N} \left( a, \sigma^2 \right),
a \in \mathbb{R},
\sigma^2 > 0,$$
$$p \left( x \right) = \frac{1}{ \sqrt{2 \pi } \cdot \sigma } \cdot e^{- \frac{ \left( x-a \right)^2}{2 \sigma^2}},
x \in \mathbb{R},
M \xi = a,
D \xi = \sigma^2.$$
\end{enumerate}
\end{enumerate}

\subsubsection*{Что называется случайным вектором?}

Случайный вектор --- это набор упорядоченных величин
$$ \vec{ \xi } =
\left( \xi_1, \xi_2, \dotsc, \xi_n \right).$$

\subsubsection*{Приведите критерий независимости компонент случайного вектора.}

$ \xi_1, \dotsc, \xi_n$ --- независимы тогда и только тогда, когда
$$p_{ \vec{ \xi }} \left( x_1, \dotsc, x_n \right) =
\prod \limits_{i=1}^n p_{ \xi_i} \left( x \right).$$

\subsubsection*{Как вычислить плотность суммы независимых случайных величин?}

Для независимых случайных величин
$$p_{ \xi + \eta } \left( x \right) =
\int \limits_{ \mathbb{R}} p_{ \xi } \left( x-y \right) p_{ \eta } \left(t y \right) dy.$$

\addcontentsline{toc}{section}{Аудиторные задачи}
\section*{Аудиторные задачи}

\subsubsection*{11.3}

\textit{Задание.} Пусть $ \xi, \eta $ --- независимые случайные величины, причём $ \xi $ имеет показательное распределение с параметром $ \lambda $, а $ \nu $ равномерно распределена на отрезке $ \left[ 0, h \right] $.
Найдите плотностьраспределения случайной величины $ \xi $. 

\textit{Решение.} Случайная величина $ \xi $ имеет плотность распределения
$$p_{ \xi } \left( x \right) =
\begin{cases}
\lambda e^{- \lambda x}, \qquad x \geq 0, \\
0, \qquad x < 0.
\end{cases}$$

Случайная величина $ \eta $ имеет плотность распределения
$$p_{ \eta } \left( x \right) =
\begin{cases}
0, \qquad x \notin \left[ 0, h \right], \\
\frac{1}{h}, \qquad x \in \left[ 0, h \right].
\end{cases}$$

Случайные величины $ \xi, \eta $ --- независимы.
Записываем плотность распределения суммы
$$p_{ \xi + \eta } \left( x \right) =
\int \limits_{ \mathbb{R}} p_{ \xi } \left( x - y \right) p_{ \eta } \left( y \right) dy.$$
Будем использовать индикатор
$$p_{ \xi + \eta } \left( x \right) =
\int \limits_{ \mathbb{R}} \lambda e^{- \lambda \left( x-y \right) } \cdot
\mathbbm{1} \left( x-y \geq 0 \right) \cdot
\frac{1}{h} \cdot
\mathbbm{1} \left( y \in \left[ 0, h \right] \right) dy.$$
Эти индикаторы будут задавать, где именно будем интегрировать эти плотности.
Должны интегрировать по пересечению $ \left[ 0, h \right] \cap \left( - \infty, x \right)$, где $h$ --- фиксированное число, а $x$ --- параметр
$$p_{ \xi + \eta } \left( x \right) =
\begin{cases}
0, \qquad x < 0, \\
\int \limits_0^x \lambda e^{- \lambda \left( x-y \right) } dy \cdot \frac{1}{h}, \qquad x \in \left[ 0, h \right], \\
\int \limits_0^h \lambda e^{- \lambda \left( x-y \right) } dy \cdot \frac{1}{h}, \qquad x > h.
\end{cases}$$

Отдельно вычисляем каждый интеграл.
Второй случай
$$ \frac{ \lambda }{h} \int \limits_0^x e^{- \lambda x} \cdot e^{ \lambda y} dy =
\left. \frac{1}{h} \cdot e^{- \lambda x} \cdot e^{ \lambda y} \right|_0^x =
\frac{1}{h} \cdot e^{- \lambda x} \cdot e^{ \lambda x} - \frac{1}{h} \cdot e^{- \lambda x} \cdot e^0.$$
Выражение $e^{- \lambda x} \cdot e^{ \lambda x}$ даёт единицу
$$\frac{ \lambda }{h} \int \limits_0^x e^{- \lambda x} \cdot e^{ \lambda y} dy =
\frac{1}{h} - \frac{1}{h} \cdot e^{- \lambda x} =
\frac{1}{h} \left( 1 - e^{- \lambda x} \right) =
\frac{1-e^{- \lambda x}}{h}.$$

Вычисляем третий случай --- такой же итеграл, только в пределах от нуля до $h$.
Имеем
$$ \int \limits_0^h \lambda e^{- \lambda \left( x-y \right) } dy \cdot \frac{1}{h} =
\left. \frac{1}{h} \cdot e^{- \lambda x} \cdot e^{ \lambda y} \right|_0^h =
\frac{e^{- \lambda x}}{h} \left( e^{ \lambda h} - 1 \right),$$
где $x$ --- это параметр.
Следовательно
$$p_{ \xi + \eta } \left( x \right) =
\begin{cases}
0, \qquad x < 0, \\
\frac{1-e^{- \lambda x}}{h}, \qquad x \in \left[ 0, h \right], \\
\frac{e^{- \lambda x}}{h} \left( e^{ \lambda h} - 1 \right), \qquad x > h.
\end{cases}$$

\subsubsection*{11.4}

\textit{Задание.} Пусть $ \xi, \eta, \zeta $ --- независимые случайные величины.
Известно,
что $ \xi $ и $ \eta $ имеют функции распределения $F$ и $G$ соответственно,
а $ \zeta $ принимат значения 0 или 1 с вероятностями $p$ и $q$ соответственно $ \left( p + q = 1 \right) $.
Найдите функцию распределения случайной величины $ \zeta \xi + \left( 1 - \zeta \right) \eta $.

\textit{Решение.} Начинаем с определения функции распределения
$$P \left( \xi \leq x \right) = F_{ \xi } \left( x \right) = F,
P \left( \eta \leq x \right) = G.$$

По условию $P \left( \zeta = 0 \right) = p = 1 - P \left( \zeta = 1 \right) = 1 - q$.

Откуда $P \left( \zeta = 1 \right) = q$.

Ищем функцию распределения по определению
$$F_{ \zeta \xi + \left( 1 - \zeta \right) \eta } \left( x \right) =
P \left(  \zeta \xi + \left( 1 - \zeta \right) \eta \leq x \right).$$
Применяем формулу полной вероятности.
Если $ \zeta = 0$, то случайная величина --- это $ \eta $, если $ \zeta = 1$, то это $ \xi $.
Получаем
$$F_{ \zeta \xi + \left( 1 - \zeta \right) \eta } \left( x \right) =
P \left( \zeta = 0 \right) P \left( \eta \leq x \right) + P \left( \zeta = 1 \right) P \left( \eta \leq x \right) =
pG + \left( 1 - p \right) F.$$
В итоге получаем $F_{ \zeta \xi + \left( 1 - \zeta \right) \eta } \left( x \right) = pG + qF.$

\subsubsection*{11.5}

\textit{Задание.} Пусть $ \xi $ --- случайная величина с функцией распределения $F$.
Найдите функцию распределения случайного вектора $ \left( \xi, \left| \xi \right| \right) $.

\textit{Решение.} Есть случайный вектор, который имеет две компоненты.
Поэтому функция распределения будет иметь 2 аргумента.

По определению $F_{ \left( \xi, \left| \xi \right| \right) } \left( x, y \right) = P \left( \xi \leq x, \left| \xi \right| \leq y \right).$
Раскрываем модуль $F_{ \left( \xi, \left| \xi \right| \right) } \left( x, y \right) = P \left( \xi \leq x, -y \leq \xi \leq y \right) $.
Всё касается одной и той же случайной величины
$$F_{ \left( \xi, \left| \xi \right| \right) } \left( x, y \right) =
\begin{cases}
0, \qquad x < - y, \\
P \left( \xi \in \left[ -y, x \right] \right), \qquad x \in \left[ -y, y \right], \\
P \left( \xi \in \left[ -y, y \right] \right), \qquad x > y
\end{cases}
\cdot \mathbbm{1} \left( y > 0 \right).$$
Выражаем это через функцию распределения $ \xi $.
Получаем
$$F_{ \left( \xi, \left| \xi \right| \right) } \left( x, y \right) =
\begin{cases}
0, \qquad x < - y, \\
F_{ \xi } \left( x \right) - F_{ \xi } \left( -y \right), \qquad x \in \left[ -y, y \right], \\
F_{ \xi } \left( y \right) - F_{ \xi } \left( -y \right), \qquad x > y
\end{cases}
\cdot \mathbbm{1} \left( y > 0 \right).$$

\subsubsection*{11.6}

\textit{Задание.}
Случайный вектор $ \left( \xi_1, \xi_2 \right) $ имеет плотность распределения
$$p \left( x \right) =
\left( x+ay^2 \right) \cdot \mathbbm{1}_{ \left[ 0, 1 \right]^2}.$$
Найдите:
\begin{enumerate}[label=\alph*)]
\item параметр $a$;
\item $M \xi_1, M \xi_1 \xi_2, P \left( \xi_1 < \xi_2 \right) $;
\item плотности $p_{ \xi_1} \left( x \right), p_{ \xi_2} \left( x \right) $;
\item плотность распределения случайной величины $ \max \left( \xi_1, \xi_2 \right) $.
\end{enumerate}
Являются ли $ \xi_1, \xi_2$ независимыми?

\textit{Решение.}
\begin{enumerate}[label=\alph*)]
\item Находим неизвестный параметр $a$, который входит в плотность из условия нормировки.

Плотность определена, когда $0 \leq x \leq 1, 0 \leq y \leq 1$.

Условие нормировки выполняется для всякой плотности
$$1 =
\int \limits_{- \infty }^{+ \infty } \int \limits_{- \infty }^{+ \infty } p \left( x, y \right) dxdy =
\int \limits_0^1 \int \limits_0^1 \left( x+ay^2 \right) dxdy.$$
Подставляем пределы интегрирования
$$1 =
\left. \int \limits_0^1 \left( \frac{x^2}{2} + ay^2 x \right) \right|_0^1 dy =
\int \limits_0^1 \left( \frac{1}{2} + ay^2 \right) dy =
\frac{1}{2} + \frac{a}{3}.$$
Отсюда следует, что
$$a =
\frac{2}{3};$$
\item ищем $M \xi_1$.
Математическое ожидание $ \phi \left( \xi_1, \xi_2 \right) $ вычислили бы так
$$M \phi \left( \xi_1, \xi_2 \right) =
\iint_{ \mathbb{R}^2} \phi \left( x, y \right) p \left( x, y \right) dxdy,$$
где $p$ --- это их совместная вероятность, то есть вероятность вектора.

В данном случае $ \phi \left( \xi_1, \xi_2 \right) = \xi_1$.

Вычисляем математическое ожидание
$$M \xi_1 =
\iint_{ \mathbb{R}^2} x \cdot p \left( x, y \right) dxdy =
\int \limits_0^1 \int \limits_0^1 x \left( x + ay^2 \right) dxdy.$$
Раскрываем скобки
$$M \xi_1 =
\int \limits_0^1 \int \limits_0^1 \left( x^2 + ay^2 x \right) dxdy =
\left. \int \limits_0^1 \left( \frac{x^3}{3} + \frac{ay^2 x^2}{2} \right) \right|_0^1 dy.$$
Подставляем пределы интегрирования
$$M \xi_1 =
\int \limits_0^1 \left( \frac{1}{3} + \frac{ay^2}{2} \right) dy =
\left. \left( \frac{y}{3} + \frac{ay^3}{6} \right) \right|_0^1 =
\frac{1}{3} + \frac{a}{6}.$$
Подставляем
$$a =
\frac{2}{3}.$$
Получаем
$$M \xi_1 =
\frac{1}{3} + \frac{3}{2 \cdot 6} =
\frac{1}{3} + \frac{1}{4}.$$
Приводим к общему знаменателю, для чего первое слагаемое умножаем и делим на 4, а второе --- на 3
$$M \xi_1 =
\frac{7}{12}.$$

Найдём следующее математическое ожидание.

В данном случае $ \phi \left( \xi_1, \xi_2 \right) = \xi_1 \xi_2$.

Это будет
$$M \xi_1 \xi_2 =
\iint_{ \mathbb{R}^2} xyp \left( x, y \right) dxdy.$$
Подставляем вид $p \left( x, y \right) $.
Получаем
$$M \xi_1 \xi_2 =
\int \limits_0^1 \int \limits_0^1 xy \left( x + \frac{3}{2} \cdot y^2 \right) dxdy =
\int \limits_0^1 \int \limits_0^1 \left( x^2 y + \frac{3xy^3}{2} \right) dxdy.$$
Берём внутненний интеграл
$$M \xi_1 \xi_2 =
\int \limits_0^1 \left( \frac{y}{3} + \frac{3y^3}{4} \right) dy =
\left. \left( \frac{y^2}{6} + \frac{3y^4}{16} \right) \right|_0^1 =
\frac{1}{6} + \frac{3}{16} =
\frac{17}{48}.$$

Любую вероятность можно записать как математическое ожидание индикатора
$$P \left( \xi_1 < \xi_2 \right) =
M \mathbbm{1} \left( \xi_1 < \xi_2 \right).$$
В данный момент $\mathbbm{1} \left( \xi_1 < \xi_2 \right) = \phi \left( \xi_1, \xi_2 \right) $.
Поэтому
$$P \left( \xi_1 < \xi_2 \right) =
\iint_{ \mathbb{R}^2} \mathbbm{1} \left( x < y \right) p \left( x, y \right) dxdy =
\iint_{x < y} p \left( x, y \right) dxdy.$$
Подставляем плотность распределения
$$P \left( \xi_1 < \xi_2 \right) =
\int \limits_0^1 \int \limits_0^y \left( x + \frac{3y^2}{2} \right) dxdy =
\left. \int \limits_0^1 \left( \frac{x^2}{2} + \frac{2y^2 x}{2} \right) \right|_0^y.$$
Подставляем пределы интегрирования
$$P \left( \xi_1 < \xi_2 \right) =
\int \limits_0^1 \left( \frac{y^2}{2} + \frac{3y^3}{2} \right) dxdy =
\left. \left( \frac{y^3}{6} + \frac{3y^4}{8} \right) \right|_0^1 =
\frac{1}{6} + \frac{3}{8}.$$
Приводим к общему знаменателю.
Для этого первую дробь умножаем и делим на 4, а вторую --- на 3
$$P \left( \xi_1 < \xi_2 \right) =
\frac{4+9}{24} =
\frac{13}{24};$$
\item находим плотность каждой из случайных величин.
Находим как плотности компонент вектора
$$p_{ \xi_1} \left( x \right) =
\int \limits_{ \mathbb{R}} p \left( x, y \right) dy =
\int \limits_0^1 \left( x + \frac{3y^2}{2} \right) dy \cdot \mathbbm{1} \left( x \in \left[ 0, 1 \right] \right).$$
Вычисляем интеграл
$$p_{ \xi_1} \left( x \right) =
\left. \left( xy + \frac{3y^3}{6} \right) \right|_0^1 \cdot \mathbbm{1} \left( x \in \left[ 0, 1 \right] \right) =
\left. \left( xy + \frac{y^3}{2} \right) \right|_0^1 \cdot \mathbbm{1} \left( x \in \left[ 0, 1 \right] \right).$$
Подставляем пределы интегрирования
$$p_{ \xi_1} \left( x \right) =
\left( x + \frac{1}{2} \right) \cdot \mathbbm{1} \left( x \int \left[ 0, 1 \right] \right).$$

Теперь ищем плотность $ \xi_2$ аналогичным образом
$$p_{ \xi_2} \left( y \right) =
\int \limits_{ \mathbb{R}} p \left( x, y \right) dx =
\int \limits_0^1 \left( x + \frac{3y^2}{2} \right) dx \cdot \mathbbm{1} \left( y \in \left[ 0, 1 \right] \right).$$
Вычисляем интеграл
$$p_{ \xi_2} \left( y \right) =
\left. \left( \frac{x^2}{2} + \frac{3y^2 x}{2} \right) \right|_0^1 \cdot \mathbbm{1} \left( y \in \left[ 0, 1 \right] \right) =
\left( \frac{1}{2} + \frac{2y^2}{2} \right) \cdot \mathbbm{1} \left( y \in \left[ 0, 1 \right] \right).$$

Поскольку совместная плотность $p_{ \vec{ \xi }} \left( x, y \right) \neq p_{ \xi_1} \left( x \right) \cdot p_{ \xi_2} \left( y \right) $,
то $ \xi_1$ и $ \xi_2$ не могут быть независимыми;
\item $F_{ \max \left( \xi_1, \xi_2 \right) } \left( x \right) $ --- ищем.

$$P \left( \max \left( \xi_1, \xi_2 \right) \leq x \right) =
P \left( \xi_1 \leq x, \xi_2 \leq x \right) =
M \mathbbm{1} \left( \xi_1 \leq x, \xi_2 \leq x \right).$$
Вычисляем математическое ожидание по определению
$$P \left( \max \left( \xi_1, \xi_2 \right) \leq x \right) =
\iint_{ \mathbb{R}^2} \mathbbm{1} \left( u \leq x, v \leq x \right) \cdot p \left( u, v \right) dudv.$$
При этом $x \in \left[ 0, 1 \right] $.
Получаем
$$P \left( \max \left( \xi_1, \xi_2 \right) \leq x \right) =
\int \limits_0^x \int \limits_0^x \left( u + \frac{3v^2}{2} \right) dudv =
\left. \int \limits_0^x \left( \frac{u^2}{2} + \frac{3u^2 \cdot u}{2} \right) \right|_0^x dv.$$
Подставляем пределы интегрирования
$$P \left( \max \left( \xi_1, \xi_2 \right) \leq x \right) =
\int \limits_0^x \left( \frac{x^2}{2} + \frac{3v^2 x}{2} \right) dv =
\frac{x^3}{2} + \frac{x^4}{2}, x \in \left[ 0, 1 \right].$$

Записываем функцию распределения
$$F_{ \max \left( \xi_1, \xi_2 \right) } \left( x \right) =
\begin{cases}
0, \qquad x < 0, \\
\frac{x^3}{2} + \frac{x^4}{2}, \qquad x \in \left[ 0, 1 \right], \\
1, \qquad x > 1.
\end{cases}$$

Ищем плотность распределения
$$p_{ \max \left( \xi_1, \xi_2 \right) } \left( x \right) =
F_{ \max \left( \xi_1, \xi_2 \right) }' \left( x \right) =
\left( \frac{3}{2} \cdot x^2 + 2x^3 \right) \cdot \mathbbm{1} \left( x \in \left[ 0, 1 \right] \right).$$
\end{enumerate}

\subsubsection*{11.7}

\textit{Задание.} Пусть $ \xi_1, \xi_2$ --- независимые случайные величины со стандартным нормальным распределением каждая.
Найдите плотность оаспределения случайной величины $ \sqrt{ \xi_1^2 + \xi_2^2}$.

\textit{Решение.} Выпишем плотность $ \xi_1, \xi_2$ каждую из условия, $ \sigma = 1, a = 0$.
Имеем
$$p_{ \xi_1} \left( x \right) =
p_{ \xi_2} \left( x \right) =
\frac{1}{ \sqrt{2 \pi }} \cdot e^{- \frac{x^2}{2}}.$$

Тогда совместная плотность вектора (из независимости)
$$p_{ \left( \xi_1, \xi_2 \right) } \left( x, y \right) =
p_{ \xi_1} \left( x \right) \cdot p_{ \xi_2} \left( x \right) =
\frac{1}{2 \pi } \cdot e^{- \frac{x^2 + y^2}{2}}.$$

Функция распределения этой случайной величины по определению
$$F_{ \sqrt{ \xi_1^2 + \xi_2^2}} \left( t \right) =
P \left( \sqrt{ \xi_1^2 + \xi_2^2} \leq t \right).$$
Это целесообразно рассматривать при $t \geq 0$.
Записываем вероятность как индикатор этого события $F_{ \sqrt{ \xi_1^2 + \xi_2^2}} \left( t \right) = M \mathbbm{1} \left( \sqrt{ \xi_1^2 + \xi_2^2} \leq t \right) $.
Есть две случайные величины, значит будет двойной интеграл
$$F_{ \sqrt{ \xi_1^2 + \xi_2^2}} \left( t \right) =
\iint_{ \mathbb{R}^2} \mathbbm{1} \left( \sqrt{x^2 + y^2} \leq t \right) p_{ \left( \xi_1, \xi_2 \right) } \left( x, y \right) dxdy.$$
Перепишем плотность
$$F_{ \sqrt{ \xi_1^2 + \xi_2^2}} \left( t \right) =
\iint_{ \mathbb{R}^2} \mathbbm{1} \left( \sqrt{x^2 + y^2} \leq t \right) \cdot \frac{1}{2 \pi } \cdot e^{- \frac{x^2 + y^2}{2}} \cdot dxdy.$$
Перейдём в полярную систему координат
$x = \rho \cos \phi,
y = \rho \sin \phi,
dxdy = \\
= \rho d \rho d \phi,
\rho: 0 \rightarrow t,
\phi: 0 \rightarrow 2 \pi,
x^2 + y^2 = \rho^2,
\mathbbm{1} \left( \sqrt{x^2 + y^2} \leq t \right) = \mathbbm{1} \left( \rho \leq t \right) $.
Получаем
$$F_{ \sqrt{ \xi_1^2 + \xi_2^2}} \left( t \right) =
\int \limits_0^{2 \pi } \int \limits_0^t \frac{1}{2 \pi } \cdot e^{- \frac{ \rho^2}{2}} \rho d \rho d \phi.$$
По $d \phi $ проинтегрируем сразу
$$F_{ \sqrt{ \xi_1^2 + \xi_2^2}} \left( t \right) =
- \int \limits_0^t d \left( e^{- \frac{ \rho^2}{2}} \right) =
\left. - e^{- \frac{ \rho^2}{2}} \right|_0^t =
1 - e^{- \frac{t^2}{2}}$$
--- это при $t > 0$.

Нужно продифференцировать по $t$.
Получаем
$$p_{ \sqrt{ \xi_1^2 + \xi_2^2}} \left( t \right) =
- e^{- \frac{t^2}{2}} \cdot \left( - \frac{2t}{2} \right) \cdot \mathbbm{1} \left( t > 0 \right) =
te^{- \frac{t^2}{2}} \cdot \mathbbm{1} \left( t > 0 \right).$$

\subsubsection*{11.8}

\textit{Задание.} Пусть $ \xi_1, \dotsc, \xi_5$ --- независимые случайные величины.
Случайные величины $ \xi_1, \dotsc, \xi_4$ равномерно распределены на отрезке $ \left[ 0, 2 \right] $,
а $ \xi_5$ имеет плотность распределения $p \left( x \right) = 2x \mathbbm{1}_{ \left[ 0, 1 \right] }$.
Вычислите вероятность $P \left( \xi_1 \leq \xi_2 \leq \dotsc \leq \xi_5 \leq x \right), x \in \mathbb{R}$.

\textit{Решение.}
\begin{enumerate}
\item При $x \leq 0$ искомая вероятность равна нулю;
\item при $x \geq 2$ искомая вероятность равна единице;
\item при $0 < x < 2$ искомая вероятность равна
\begin{equation*}
\begin{split}
\int \limits_0^2 \int \limits_0^2 \int \limits_0^2 \int \limits_0^2 \int \limits_0^2 \mathbbm{1} \left\{ z_1 \leq z_2 \leq \dotsc \leq z_5 \leq x \right\} \cdot
\left( \frac{1}{2} \right)^4 \cdot 2 \cdot z_5 \cdot \mathbbm{1}_{ \left[ 0, 1 \right] } \left( z_5 \right) \times \\
\times dz_1 \dotsc dz_5 = \\
= \frac{1}{8} \int \limits_0^2 \int \limits_0^2 \int \limits_0^2 \int \limits_0^2 \int \limits_0^1
\mathbbm{1} \left\{ z_1 \leq z_2 \leq \dotsc \leq z_5 \leq x \right\} z_5 dz_1 \dotsc dz_5 = \\
= \frac{1}{8} \int \limits_0^1 \int \limits_0^1 \int \limits_0^1 \int \limits_0^1 \int \limits_0^1
\mathbbm{1} \left\{ z_1 \leq z_2 \leq \dotsc \leq z_5 \leq x \right\} z_5 dz_1 \dotsc dz_5.
\end{split}
\end{equation*}

Это получается при $x > 1$.

При $x \leq 1$ в верхний предел интегралов будет $x$.
\end{enumerate}

\subsubsection*{11.9}

\textit{Задание.} Пусть $ \xi $ и $ \eta $ --- независимые случайные величины, каждая из которых распределена по показательному распределению с параметром $ \alpha $.
Найдите функцию и плотность распределения случайной величины $ \xi/\eta $.

\textit{Решение.}
Плотности распределения случайных величин
$$p_{ \eta } \left( x \right) =
p_{ \eta } \left( x \right) =
\alpha e^{- \alpha x} \cdot \mathbbm{1} \left( x \geq 0 \right).$$

Искомая функция распределения
$$F_{ \frac{ \xi }{ \eta }} \left( x \right) =
P \left( \frac{ \xi }{ \eta } \leq x \right) =
P \left( \xi \leq \eta x \right).$$

Умеем считать плотность для суммы
$$ \frac{ \xi }{ \eta } \rightarrow \tilde{ \xi } + \tilde{ \eta }.$$

Есть случайная величина
$$ \zeta =
\frac{ \xi }{ \eta }.$$
Можно взять логарифм того и другого и получить
$$ \ln \zeta =
\ln \xi - \ln \eta =
\ln \xi + \left( - \ln \eta \right) =
\xi' + \eta'.$$

Так как $ \xi $ и $ \eta $ --- независимы, то $ \ln \xi, - \ln \eta $ --- тоже независимые.
Нужно найти функцию распределения этих случайных величин
$$F_{ \xi'} \left( x \right) =
P \left( \xi' \leq x \right) =
P \left( \xi \leq e^x \right) =
F_{ \xi } \left( e^x \right).$$
Плотность распределения $p_{ \xi'} \left( x \right) = \left( F_{ \xi } \left( e^x \right) \right)' = e^x p_{ \xi } \left( e^x \right) $.
Подставим плотность распределения из условия $p_{ \xi'} \left( x \right) = \mathbbm{1} \left( e^x \geq 0 \right) \alpha e^x \cdot e^{- \alpha e^x}$.
Индикатор равен единице, поэтому $p_{ \xi'} \left( x \right) = \alpha e^x \cdot e^{- \alpha e^x}$.
Хотим распределить функцию распределения $ \eta'$.
Получаем
$F_{ \eta'} \left( x \right) =
P \left( \eta' \leq x \right) =
P \left( - \ln \eta \leq x \right) = \\
= P \left( \ln \eta \geq -x \right) =
P \left( \eta \geq -e^x \right) =
1 - P \left( \eta \geq e^{-x} \right).$
Случайная величина имеет равномерное распределение $F_{ \eta'} \left( x \right) = 1 - F_{ \xi } \left( e^{-x} \right) $.

Плотность распределения
$$p_{ \eta'} \left( x \right) =
\left( 1 - F_{ \eta } \left( -e^{-x} \right) \right)' =
e^x p_{ \xi } \left( e^{-x} \right) =
\alpha e^{-x} \cdot e^{- \alpha e^{-x}}.$$

Получаем плотность распределения суммы
$$p_{ \xi' + \eta'} \left( x \right) =
\int \limits_{- \infty }^{+ \infty } e^y e^{- \alpha e^y} e^{-x+y} e^{- \alpha e^{-x+y}} dy =
\alpha^2 \int \limits_{ \mathbb{R}} e^{2y-x} \cdot e^{- \alpha \left( e^y + e^{-x+y} \right)} dy.$$
Нужно сделать замену переменных.
Нужно взять
$$y = \ln z,
dy = \frac{1}{z}dz.$$
Отсюда следует, что $z = e^y, y = + \infty \Rightarrow z = + \infty, y = - \infty \Rightarrow z = 0$.
Получаем
$$p_{ \xi' + \eta'} \left( x \right) =
\alpha^2 \int \limits_0^{+ \infty } e^{-x} \cdot x^2 e^{- \alpha x \left( 1 + e^{-x} \right) } dz \cdot \frac{1}{z}.$$
Сокращаем $z$, а $e^{-x}$ выносим за знак интеграла
$$p_{ \xi' + \eta'} \left( x \right) =
\alpha^2 e^{-x} \int \limits_0^{+ \infty } ze^{- \alpha z \left( 1 + e^{-x} \right) } dz.$$
Берём по частям
$$u = z,
du = dz,
dv = e^{- \alpha x \left( 1 + e^{-x} \right) } dz,
v = - \frac{1}{ \alpha \left( 1 + e^{-x} \right) } \cdot e^{- \alpha z \left( 1 + e^{-x} \right) }.$$
Получаем
\begin{equation*}
\begin{split}
p_{ \xi' + \eta'} \left( x \right) = \\
= \alpha^2 e^{-x} \left( \left. - \frac{z}{ \alpha \left( 1 + e^{-x} \right) } \cdot e^{- \alpha + \left( 1 + e^{-x} \right) } \right|_0^{+ \infty } +
\int \limits_0^{+ \infty } \frac{1}{ \alpha \left( 1 + e^{-x} \right) } \cdot e^{- \alpha z \left( 1 + e^{-x} \right) } dx \right) = \\
= \alpha^2 e^{-x} \left( \left. - \frac{z}{ \alpha \left( 1 + e^{-x} \right) } \cdot e^{- \alpha z \left( 1 + e^{-x} \right) } \right|_0^{+ \infty } -
\left. \frac{1}{ \alpha^2 \left( 1 + e^{-x} \right)^2 } \cdot e^{- \alpha z \left( 1 + e^{-x} \right)} \right|_0^{+ \infty } \right).
\end{split}
\end{equation*}
Первое слагаемое превращается в 0
$$p_{ \xi' + \eta'} \left( x \right) =
\alpha^2 e^{-x} \cdot \frac{1}{ \alpha^2 \left( 1 + e^{-x} \right) ^2}.$$
Сокращаем $ \alpha^2$ и получаем
$$p_{ \xi' + \eta'} \left( x \right) = \frac{e^{-x}}{ \left( 1 + e^{-x} \right)^2} =
p_{ \ln \left( \frac{ \xi }{ \eta } \right) } \left( x \right) =
p_{ \ln \zeta } \left( x \right).$$

В результате получаем
$$p_{ \frac{ \xi }{ \eta }} \left( x \right) =
\frac{x}{ \left( 1 + x \right)^2} =
p_{ \zeta } \left( x \right).$$

\subsubsection*{11.10}

\textit{Задание.}
Пусть $X_1, \dotsc, X_n$ --- независимые случайные величины, каждая из которых имеет геометрическое распределение с параметром $p$.
Вычислите вероятность $P \left( X_1 = k, X_{ \left( 1 \right) } = l \right) $,
где $X_{ \left( 1 \right) } = \min \left\{ X_1, \dotsc, X_n \right\} $.

\textit{Решение.} Геометрическое распределение имеет вид
$$P \left( X_i = k \right) = q^{k-1} p,
k > 0,
i = \overline{1,n},
q = 1 - p.$$

Если $k < l$, то искомая вероятность нулевая, потому что минимум не может быть больше чем одно из значений.
Поэтому нужно рассматривать случай, когда $k \geq l$.
Сначала будем искать вероятность $P \left( X_1 = k, X_{ \left( 1 \right) } \geq l \right) $.
Тогда
$P \left( X_1 = k, X_{ \left( 1 \right) } = l \right) =
P \left( X_1 = k, X_{ \left( 1 \right) } \geq l \right) - P \left( X_1 = k, X_{ \left( 1 \right) } \geq l + 1 \right) $.

Ищем вероятность
$$P \left( X_1 = k, X_{ \left( 1 \right) } \geq l \right) =
P \left\{ X_1 = k, X_1 \geq l, X_2 \geq l, \dotsc, X_n \geq l \right\}.$$
Если $l \geq l$, то если $X_1 = k$, то автоматически $X_1 \geq l$.
Получаем
$$P \left( X_1 = k, X_{ \left( 1 \right) } \geq l \right) =
P \left( X_1 = k, X_2 \leq l, \dotsc, X_n \leq l \right\}.$$
Пользуемся независимостью
\begin{equation*}
\begin{split}
P \left( X_1 = k, X_{ \left( 1 \right) } \geq l \right) =
P \left\{ X_1 = k \right\} \cdot P \left\{ X_2 \geq l \right\} \cdot \dotsc P \left\{ X_n \geq l \right\} = \\
= p \left( 1-p \right)^{k-1} \cdot \left( P \left\{ X_2 \geq l \right\} \right)^{n-1} =
p \left( 1-p \right)^{k-1} \cdot \left( \sum \limits_{i=1}^{+ \infty } P \left\{ X_2 = i \right\} \right)^{n-1}.
\end{split}
\end{equation*}
Подставим значение вероятности
$$P \left( X_1 = k, X_{ \left( 1 \right) } \geq l \right) =
p \left( 1-p \right)^{k-1} \cdot \left( \sum \limits_{i=l}^{+ \infty } p \left( 1-p \right)^{i-1} \right)^{n-1}.$$
Пользуемся формулой для суммы геометрической прогрессии
$$P \left( X_1 = k, X_{ \left( 1 \right) } \geq l \right) =
pp^{n-1} \left( 1-p \right)^{k-1} \cdot \left( \frac{ \left( 1-p \right)^{l-1}}{1 - \left( 1-p \right) } \right)^{n-1}.$$
Раскроем скобки в знаменателе
$$P \left( X_1 = k, X_{ \left( 1 \right) } \geq l \right) =
p^n \left( 1-p \right)^{k-1} \cdot \left( \frac{ \left( 1-p \right)^{l-1}}{p} \right)^{n-1}.$$
Сократим на $p^{n-1}$ и получим
$$P \left( X_1 = k, X_{ \left( 1 \right) } \geq l \right) =
p \left( 1-p \right)^{k-1} \cdot \left( 1-p \right)^{\left( l-1 \right) \left( n-1 \right)} =
p \left( 1-p \right)^{l \left( n-1 \right) + n}$$
при условии, что $k \geq l$.
А иначе --- 0.

Воспользуемся формулой 
$P \left( X_1 = k, X_{ \left( 1 \right) } = l \right) =
P \left( X_1 = k, X_{ \left( 1 \right) } \geq l \right) - P \left( X_1 = k, X_{ \left( 1 \right) } \geq l + 1 \right) $.
Возникает 2 случая.
Если $k < l$, то обе вероятности равны нулю.
В случае, когда $k = l$, вторая вероятность равна нулю и ответ: $p \left( 1-p \right)^{n \left( n-1 \right) }$.
При $k \geq l + 1$ нужно найти разность
$P \left( X_1 = k, X_{ \left( 1 \right) } = l \right) = \\
= P \left( X_1 = k, X_{ \left( 1 \right) } \geq l \right) - P \left( X_1 = k, X_{ \left( 1 \right) } \geq l + 1 \right) =
p \left( 1-p \right)^{k-1 + \left( n-1 \right) \left( l-1 \right) } - \\
- p \left( 1-p \right)^{ \left( k-1 \right) \left( n-1 \right) l} =
p \left( 1-p \right)^{ \left( k-1 \right) + \left( n-1 \right) \left( l-1 \right) } \left( 1- \left( 1-p \right)^{n-1} \right)$.

\subsubsection*{11.11}

\textit{Задание.} Пусть $ \xi_1, \xi_2$ --- показательно распределённые с параметром $ \lambda $ независимые случайные величины.
Вычислите вероятность
$$P \left\{ \xi_1 < x, \xi_1 + \xi_2 < y \right\},
x, y \in \left[ 0, + \infty \right).$$

\textit{Решение.} 
\begin{equation*}
\begin{split}
P \left\{ \xi_1 < x, \xi_1 + \xi_2 < y \right\} =
M \mathbbm{1} \left( \xi_1 < x, \xi_1 + \xi_2 < y \right\} = \\
= \int \limits_{- \infty }^{+ \infty } \mathbbm{1} \left\{ z_1 < x, z_1 + z_2 < y \right\} p_{ \xi_1} \left( z_1 \right) p_{ \xi_2} \left( z_2 \right) dz_1 dz_2.
\end{split}
\end{equation*}
Случайные величины $ \xi_1, \xi_2$ --- независимые, поэтому плотность их совместного распределения равна произведению их плотностей
$$P \left\{ \xi_1 < x, \xi_1 + \xi_2 < y \right\} =
\int \limits_0^x \int \limits_0^{+ \infty } \mathbbm{1} \left\{ z_1 < x, z_1 + z_2 < y \right\} \lambda^2 e^{- \lambda \left( z_1 + z_2 \right) } dz_1 dz_2.$$
Нужно рассматривать 2 случая:
\begin{enumerate}
\item при $x < y$ получаем
$$P \left\{ \xi_1 < x, \xi_1 + \xi_2 < y \right\} =
\int \limits_0^x \int \limits_0^{y - z_1} \lambda^2 e^{- \lambda \left( z_1 + z_2 \right) } dz_1 dz_2.$$
Разбиваем на две экспоненты
$$P \left\{ \xi_1 < x, \xi_1 + \xi_2 < y \right\} =
\int \limits_0^x \int \limits_0^{y-z_1} \lambda^2 e^{- \lambda z_1} e^{- \lambda z_2} dz_1 dz_2.$$
Берём внутренний интеграл
$$P \left\{ \xi_1 < x, \xi_1 + \xi_2 < y \right\} =
\int \limits_0^x \left. \lambda^2 e^{- \lambda z_1} \cdot \frac{1}{- \lambda } \cdot e^{- \lambda z_2} \right|_0^{y-z_1} dz_1.$$
Сокращаем $ \lambda $ и получаем
$$P \left\{ \xi_1 < x, \xi_1 + \xi_2 < y \right\} =
- \lambda \int \limits_0^x e^{- \lambda z_1} \left( e^{- \lambda \left( y-z_1 \right) } - 1 \right) dz_1.$$
Раскрываем скобки
$$P \left\{ \xi_1 < x, \xi_1 + \xi_2 < y \right\} =
- \lambda \int \limits_0^x \left( e^{- \lambda z_1 - \lambda y + \lambda z_1} - e^{- \lambda z_1} \right) dz_1.$$
Слагаемые $- \lambda z_1$ и $ \lambda z_1$ уничтожаются
$$P \left\{ \xi_1 < x, \xi_1 + \xi_2 < y \right\} =
- \lambda \int \limits_0^x \left( e^{- \lambda y} - e^{- \lambda z_1} \right) dz_1.$$
Берём интеграл
$$P \left\{ \xi_1 < x, \xi_1 + \xi_2 < y \right\} =
\left. - \lambda e^{- \lambda y} \cdot z_1 \right|_0^x - \left. \lambda \cdot \frac{1}{ \lambda } \cdot e^{- \lambda z_1} \right|_0^x.$$
Подставляем пределы интегрирования
$$P \left\{ \xi_1 < x, \xi_1 + \xi_2 < y \right\} =
- \lambda e^{- \lambda y} x - e^{- \lambda x} + 1;$$
\item при $x > y$ получаем
\begin{equation*}
\begin{split}
P \left\{ \xi_1 < x, \xi_1 + \xi_2 < y \right\} =
P \left\{ \xi_1 < y, \xi_1 + \xi_2 < y \right\} = \\
= P \left\{ \xi_1 \leq \min \left( x, y \right), \xi_1 + \xi_2 < y \right\} =
- \lambda e^{- \lambda y} \min \left( x, y \right) - e^{- \lambda \min \left( x, y \right) } + 1.
\end{split}
\end{equation*}
\end{enumerate}

\addcontentsline{toc}{section}{Домашнее задание}
\section*{Домашнее задание}

\subsubsection*{11.13}

\textit{Задание.} Пусть $ \xi $ и $ \eta $ --- независимые одинаково распределённые случайные величины с плотностью распределения
$$p \left( x \right) =
\frac{e^{- \left| x \right| }}{2}.$$
Найдите плотность распределения суммы $ \xi + \eta $.

\textit{Решение.} Записываем по определению
$$p_{ \xi + \eta } \left( x \right) =
\int \limits_{ \mathbb{R}} p \left( x-y \right) p \left( y \right) dy =
\int \limits_{ \mathbb{R}} \frac{e^{- \left| x-y \right| }}{2} \cdot \frac{e^{- \left| y \right| }}{2} dy.$$
Выносим константы за знак интеграла
$$p_{ \xi + \eta } \left( x \right) =
\frac{1}{4} \int \limits_{ \mathbb{R}} e^{- \left| x-y \right| - \left| y \right| } dy =
\frac{1}{4} \cdot I.$$

Рассматриваем три случая: $x < 0, x = 0$ и $x > 0$.
В первом случае
$$I =
\int \limits_{- \infty }^x e^{-x+y+y} dy + \int \limits_x^0 e^{x-y+y} dy + \int_0^{+ \infty } e^{x-y-y} dy.$$
Во втором слагаемом $y$ уничтожается
$$I =
\int \limits_{- \infty }^x e^{2y-x} dy + \int \limits_x^0 e^x dy + \int \limits_0^{+ \infty } e^{-2y+x} dy.$$
Вычисляем интегралы
$$I =
\left. \frac{1}{2} \cdot e^{2y-x} \right|_{- \infty }^x + \left. e^x \cdot y \right|_x^0 - \left. \frac{1}{2} \cdot e^{-2y+x} \right|_0^{+ \infty }.$$
Подставляем пределы интегрирования
$$I =
\frac{1}{2} \cdot e^{2 \cdot x - x} - \frac{1}{2} \cdot e^{- \infty } + e^x \cdot 0 - e^x \cdot x - \frac{1}{2} \cdot e^{- \infty } + \frac{1}{2} \cdot e^x.$$
Экспонента на $- \infty $ равна нулю
$$I =
\frac{1}{2} \cdot e^x - x \cdot e^x + \frac{1}{2} \cdot e^x =
e^x - x \cdot e^x =
e^x \left( 1 - x \right).$$

При $x = 0$ получаем
$$I =
\int_{- \infty }^{+ \infty } e^{- \left| -y \right| - \left| y \right| } dy =
\int \limits_{- \infty }^0 e^{- \left( -y \right) + y} dy + \int \limits_0^{+ \infty } e^{-y-y} dy.$$
Упрощаем степени в экспонентах
$$I =
\int \limits_{- \infty }^0 e^{2y} dy + \int \limits_0^{+ \infty } e^{-2y} dy =
\left. \frac{1}{2} \cdot e^{2y} \right|_{- \infty }^0 - \left. \frac{1}{2} \cdot e^{-2y} \right|_0^{+ \infty }.$$
Подставляем пределы интегрирования
$$I =
\frac{1}{2} \cdot e^0 - \frac{1}{2} \cdot e^{- \infty } - \frac{1}{2} \cdot e^{- \infty } + \frac{1}{2} \cdot e^0.$$
Экспонента на $- \infty $ равна нулю
$$I =
\frac{1}{2} + \frac{1}{2} =
1.$$

В третьем случае
$$I =
\int \limits_{- \infty }^0 e^{-x+y+y} dy + \int \limits_0^x e^{-x+y-y} dy + \int \limits_x^{+ \infty } e^{x-y-y} dy.$$
Во втором слагаемом $y$ уничтожается
$$I =
\int \limits_{- \infty }^0 e^{2y-x} dy + \int \limits_0^x e^{-x} dy + \int \limits_x^{+ \infty } e^{-2y+x} dy.$$
Вычисляем интегралы
$$I =
\frac{1}{2} \cdot e^{-x} - \frac{1}{2} \cdot e^{- \infty } + e^{-x} \cdot x - e^{-x} \cdot 0 - \frac{1}{2} \cdot e^{- \infty } + \frac{1}{2} \cdot e^{-2x+x}.$$
Экспонента на $- \infty $ равна нулю
$$I =
\frac{1}{2} \cdot e^{-x} + x \cdot e^{-x} + \frac{1}{2} \cdot e^{-x} =
e^{-x} + x \cdot e^{-x} =
e^{-x} \left( 1+x \right).$$

В итоге получаем
$$p_{ \xi + \eta } \left( x \right) =
\frac{1}{4} \cdot e^{ \left| x \right| } \left( 1 + \left| x \right| \right).$$

\subsubsection*{11.14}

\textit{Задание.}
Пусть $ \xi $ и $ \eta $ --- независимые случайные величины,
причём $ \xi $ имеет равномерное распределение на отрезке $ \left[ 0, 1 \right] $,
а $ \eta $ принимает значения $0, \pm 1, \pm 2, \dotsc $ с вероятностями $p_0, p_1, p_{-1}, \dotsc $ соответственно:
$$p_0 + p_1 + p_{-1} + \dotsc =
1.$$
Найдите плотность распределения суммы $ \xi + \eta $.

\textit{Решение.} Найдём функцию распределения суммы случайных величин по определению и применим формулу полной вероятности
$$F_{ \xi + \eta } \left( x \right) =
P \left( \xi + \eta \leq x \right) =
\sum \limits_{k = - \infty }^{+ \infty } P \left( \left. \xi + \eta \leq x \right| \eta = k \right) \cdot P \left( \eta = k \right).$$
Подставим $ \eta = k$ в условной вероятности и значение вероятности из условия, так как случайные величины независимы
Также заменим вероятность на функцию распределения
$$F_{ \xi + \eta } \left( x \right) =
\sum \limits_{k = - \infty }^{+ \infty } P \left( \eta = k \right) \cdot P \left( \xi + k \leq x \right).$$
Случайная величина $ \xi $ меняется в пределах $0 \leq \xi \leq x - k, k \leq \left[ x \right] $, поэтому
$$F_{ \xi + \eta } \left( x \right) =
\sum \limits_{k \leq \left[ x \right] } \min \left( 1, x-k \right) p_k.$$

Найдём плотность распределения как производную от функции распределения и запишем плотность распределения из условия.

Функция не дифференцируема, поэтому плотности нет.

\subsubsection*{11.15}

\textit{Задание.} Пусть $ \xi $ --- случайная величина с функцией распределения $F$.
Найдите функцию распределения случайного вектора $ \left( \xi, \xi \right) $.

\textit{Решение.} Есть случайный вектор, который имеет две компоненты.
Поэтому фукция распределения будет иметь 2 аргумента.

По определению $F_{ \left( \xi, \xi \right) } \left( x, y \right) = P \left( \xi \leq x, \xi \leq y \right) $.
Всё касается одной и той же случайной величины.
Изображаем все случаи на рисунке $ \ref{fig:1115}$.

\begin{figure}[h!]
  \centering
  \includegraphics[width=.8\textwidth]{./pictures/11_15.png}
  \caption{Возможные случаи}
  \label{fig:1115}
\end{figure}

Получаем
$$F_{ \left( \xi, \xi \right) } \left( x, y \right) =
\begin{cases}
P \left( \xi \leq x \right), \qquad x \leq y, \\
P \left( \xi \leq y \right), \qquad x \geq y.
\end{cases}$$

Выражаем это через функцию распределения
$$F_{ \left( \xi, \xi \right) } \left( x, y \right) =
\begin{cases}
F \left( x \right), \qquad x \leq y, \\
F \left( y \right), \qquad x \geq y.
\end{cases} =
F \left( \min \left( x, y \right) \right).$$

\subsubsection*{11.16}

\textit{Задание.} Случайный вектор $ \left( \xi_1, \xi_2 \right) $ имеет плотность распределения
$$p \left( x, y \right) =
\frac{c}{1+x^2 + x^2 y^2 + y^2}.$$
Найдите:
\begin{enumerate}[label=\alph*)]
\item параметр $c$;
\item плотности распределения $p_{ \xi_1} \left( x \right), p_{ \xi_2} \left( x \right) $;
\item $P \left( \left| \xi_1 \right| \leq 1, \left| \xi_2 \right| \leq 1 \right).$
\end{enumerate}
Являются ли $ \xi_1, \xi_2$ независимыми?

\textit{Решение.}
\begin{enumerate}[label=\alph*)]
\item Находим неизвестный параметр, который входит в плотность распределения из условия нормировки, которое выполняется для всякой плотности
$$1 =
\int \limits_{- \infty }^{+ \infty } \int \limits_{- \infty }^{+ \infty } p \left( x, y \right) dxdy =
\int \limits_{- \infty }^{+ \infty } \int \limits_{- \infty }^{+ \infty } \frac{c}{1 + x^2 + x^2 y^2 + y^2} dxdy.$$
Выпишем дробь и разложим знаменатель на множители
$$ \frac{c}{1 + x^2 + x^2 y^2 + y^2} =
\frac{c}{x^2 \left( 1 + y^2 \right) + \left( 1 + y^2 \right) } =
\frac{c}{ \left( 1 + y^2 \right) \left( 1 + x^2 \right) }.$$
Подставим в интеграл
$$1 =
\int \limits_{- \infty }^{+ \infty } \int \limits_{- \infty }^{+ \infty } \frac{c}{ \left( 1 + y^2 \right) \left( 1 + x^2 \right) } dxdy =
\int \limits_{- \infty }^{+ \infty } \frac{c}{1 + y^2} \int \limits_{- \infty }^{+ \infty } \frac{1}{1 + x^2} dxdy.$$
Вычислим интеграл по $dx$.
Получим
$$1 =
\int \limits_{- \infty }^{+ \infty } \left. \frac{c}{1 + y^2} \cdot arctg x \right|_{- \infty }^{+ \infty } dy.$$
Подставим пределы интегрирования
$$1 =
\int \limits_{- \infty }^{+ \infty } \frac{c}{1 + y^2} \left[ arctg \left( + \infty \right) - arctg \left( - \infty \right) \right] dy.$$
На $- \infty $ арктангенс равен $ - \pi/2$, а на $+ \infty $ --- $ \pi/2$.
Поэтому
$$1 =
\int \limits_{- \infty }^{+ \infty } \frac{c}{1 + y^2} \left[ \frac{ \pi }{2} - \left( - \frac{ \pi }{2} \right) \right] dy =
\int \limits_{- \infty }^{+ \infty } \frac{c}{1 + y^2} \left( \frac{ \pi }{2} + \frac{ \pi }{2} \right) dy.$$
Приведём подобные и вынесем константу за знак интеграла
$$1 =
\pi c \int \limits_{- \infty }^{+ \infty } \frac{dy}{1 + y^2} =
\left. \pi c \cdot arctg y \right|_{- \infty }^{+ \infty } =
\pi c \cdot \pi =
\pi^2 c.$$
Отсюда следует, что
$$c =
\frac{1}{ \pi^2};$$
\item находим плотности каждой из случайных величин как плотности компонент вектора
$$p_{ \xi_1} \left( x \right) =
\int \limits_{- \infty }^{+ \infty } p \left( x, y \right) dy =
\int \limits_{- \infty }^{+ \infty } \frac{c}{1 + x^2 + x^2 y^2 + y^2} dy.$$
Разложим знаменатель дроби на множители и подставим найденное значение константы
$$p_{ \xi_1} \left( x \right) =
\int \limits_{- \infty }^{+ \infty } \frac{1}{ \pi^2 \left( 1 + y^2 \right) \left( 1 + x^2 \right) } dy =
\frac{1}{ \pi^2 \left( 1 + x^2 \right) } \int \limits_{- \infty }^{+ \infty } \frac{1}{1 + y^2} dy.$$
Вычислим интеграл
$$p_{ \xi_1} \left( x \right) =
\frac{1}{ \pi^2 \left( 1 + x^2 \right) } \cdot \pi.$$
Одно $ \pi $ сокращается
$$p_{ \xi_1} \left( x \right) =
\frac{1}{ \pi \left( 1 + x^2 \right) }.$$

Теперь ищем плотность $ \xi_2$ аналогичным образом
$$p_{ \xi_2} \left( y \right) =
\int \limits_{- \infty }^{+ \infty } p \left( x, y \right) dx =
\int \limits_{- \infty }^{+ \infty } \frac{c}{1 + x^2 + x^2 y^2 + y^2 } dx.$$
Разложим знаменатель дроби на множители и подставим найденное значение константы
$$p_{ \xi_2} \left( y \right) =
\int \limits_{- \infty }^{+ \infty } \frac{1}{ \pi^2 \left( 1 + y^2 \right) \left( 1 + x^2 \right) } dx =
\frac{1}{ \pi^2 \left( 1 + y^2 \right) } \int \limits_{- \infty }^{+ \infty } \frac{1}{1 + x^2} dx.$$
Вычислим интеграл
$$p_{ \xi_2} \left( y \right) =
\frac{1}{ \pi^2 \left( 1 + y^2 \right) } \cdot \pi.$$
Одно $ \pi $ сокращается
$$p_{ \xi_2} \left( y \right) =
\frac{1}{ \pi \left( 1 + y^2 \right) };$$
\item ищем вероятность
$$P \left( \left| \xi_1 \right| \leq 1, \left| \xi_2 \right| \leq 1 \right) =
P \left( -1 \leq \xi_1 \leq 1, -1 \leq \xi_2 \leq 1 \right).$$
Вероятность представляем как математическое ожидание индикатора события
$$P \left( \left| \xi_1 \right| \leq 1, \left| \xi_2 \right| \leq 1 \right) =
M \mathbbm{1} \left( -1 \leq \xi_1 \leq 1, -1 \leq \xi_2 \leq 1 \right).$$
Вычисляем математическое ожидание
$$P \left( \left| \xi_1 \right| \leq 1, \left| \xi_2 \right| \leq 1 \right) =
\int \limits_{- \infty }^{+ \infty } \int \limits_{- \infty }^{+ \infty } \mathbbm{1} \left( -1 \leq \xi_1 \leq 1, -1 \leq \xi_2 \leq 1 \right) p \left( x, y \right) dxdy.$$
Меняем пределы интегрирования за счёт индикатора
$$P \left( \left| \xi_1 \right| \leq 1, \left| \xi_2 \right| \leq 1 \right) =
\int \limits_{-1}^1 \int \limits_{-1}^1 p \left( x, y \right) dxdy.$$
Подставляем вид плотности распределения
$$P \left( \left| \xi_1 \right| \leq 1, \left| \xi_2 \right| \leq 1 \right) =
\int \limits_{-1}^1 \int \limits_{-1}^1 \frac{1}{ \pi^2 \left( 1 + y^2 \right) \left( 1 + x^2 \right) } dxdy.$$
Вычисляем внутренний интеграл
$$P \left( \left| \xi_1 \right| \leq 1, \left| \xi_2 \right| \leq 1 \right) =
\frac{1}{ \pi^2} \int \limits_{-1}^1 \left. \frac{1}{1 + y^2} \cdot arctg x \right|_{-1}^1 dy.$$
Подставляем пределы интегрирования
$$P \left( \left| \xi_1 \right| \leq 1, \left| \xi_2 \right| \leq 1 \right) =
\frac{1}{ \pi^2} \int \limits_{-1}^1 \frac{ \pi }{2} \cdot \frac{1}{1 + y^2} dy =
\frac{1}{ \pi^2} \cdot \frac{ \pi }{2} \cdot \frac{ \pi }{2}.$$
Сокращаем $ \pi^2$ и в итоге получаем
$$P \left( \left| \xi_1 \right| \leq 1, \left| \xi_2 \right| \leq 1 \right) =
\frac{1}{4}.$$
\end{enumerate}

Проверим, являются ли случайные величины независимыми.
Для этого сравним произведение плотностей компонент вектора с плотностью вектора
$$p_{ \xi_1} \left( x \right) \cdot p_{ \xi_2} \left( y \right) =
\frac{1}{ \pi \left( 1 + x^2 \right) } \cdot \frac{1}{ \pi \left( 1 + y^2 \right) } =
\frac{1}{ \pi^2 \left( 1 + x^2 \right) \left( 1 + y^2 \right) } =
p_{ \left( \xi_1, \xi_2 \right) } \left( x, y \right).$$
Отсюда следует, что $ \xi_1, \xi_2$ --- независимы.

\subsubsection*{11.17}

\textit{Задание.} Пусть $ \xi $ и $ \eta $ --- независимые случайные величины, каждая из которых распределена по показательному распределению с параметром
$$ \alpha > 0.$$
Докажите, что $ \xi/ \left( \xi + \eta \right) $ имеет равномерное распределение на отрезке $ \left[ 0, 1 \right] $.

\textit{Решение.} Найдём функцию распределения по определению
$$F_{ \frac{ \xi }{ \xi + \eta }} \left( x \right) =
P \left( \frac{ \xi }{ \xi + \eta } \leq x \right) =
M \mathbbm{1} \left\{ \frac{ \xi }{ \xi + \eta } \leq x \right\}.$$

Так как случайные величины $ \xi $ и $ \eta $ --- независимы, то
$p_{ \left( \xi, \eta \right) } \left( x, y \right) = \\
= p_{ \xi } \left( x \right) p_{ \eta } \left( y \right) =
\alpha e^{- \alpha x} \cdot \mathbbm{1} \left( x \geq 0 \right) \cdot \alpha e^{- \alpha y} \cdot \mathbbm{1} \left( y \geq 0 \right) =
\alpha^2 e^{- \alpha x} e^{- \alpha y} \mathbbm{1} \left( x \geq 0, y \geq 0 \right) $.

Из этого следует, что
$$F_{ \frac{ \xi }{ \xi + \eta }} \left( x \right) =
\iint \limits_{ \mathbb{R}^2} \mathbbm{1} \left( \frac{z}{z+y} \leq x \right) \cdot
\mathbbm{1} \left( x \geq 0, y \geq 0 \right) \cdot \alpha^2 e^{- \alpha z} e^{- \alpha y} dzdy.$$
Изменим пределы интегрирования за счёт индикатора
\begin{equation*}
\begin{split}
F_{ \frac{ \xi }{ \xi + \eta }} \left( x \right) =
\int \limits_0^{+ \infty } \int \limits_0^{+ \infty } \mathbbm{1} \left( z \leq xz + xy \right) \alpha^2 e^{- \alpha z} e^{- \alpha y} dzdy = \\
= \alpha^2 \int \limits_0^{+ \infty } \int \limits_0^{+ \infty } \mathbbm{1} \left( z - xz \leq xy \right) e^{- \alpha z} e^{- \alpha y} dzdy = \\
= \alpha^2 \int \limits_0^{+ \infty } \int \limits_0^{+ \infty } \mathbbm{1} \left( z \left( 1 - x\right) \leq xy \right) e^{- \alpha z} e^{- \alpha y} dzdy = \\
= \alpha^2 \int \limits_0^{+ \infty } \int \limits_0^{+ \infty } \mathbbm{1} \left( z \leq \frac{xy}{1-x} \right) e^{- \alpha z} e^{- \alpha y} dzdy =
\alpha^2 \int \limits_0^{+ \infty } \int \limits_0^{ \frac{xy}{1-x}} e^{- \alpha z} e^{- \alpha y} dzdy = \\
= \alpha^2 \int \limits_0^{+ \infty } \left. e^{- \alpha y} \frac{1}{- \alpha } \cdot e^{- \alpha z} \right|_0^{ \frac{xy}{1-x}} dy =
- \alpha \int \limits_0^{+ \infty } e^{- \alpha y} \left( e^{- \alpha \cdot \frac{xy}{1-x}} - 1 \right) dy = \\
= - \alpha \int \limits_0^{+ \infty } \left( e^{- \alpha y - \alpha \cdot \frac{xy}{1-x}} - e^{- \alpha y} \right) dy =
- \alpha \int \limits_0^{+ \infty } e^{- \alpha y \left( 1 + \frac{x}{1-x} \right) } dy + \alpha \int \limits_0^{+ \infty } e^{- \alpha y} dy = \\
= - \alpha \int \limits_0^{+ \infty } e^{- \alpha y \cdot \frac{1-x+x}{1-x}} dy - \left. \alpha \cdot \frac{1}{- \alpha } \cdot e^{- \alpha y} \right|_0^{+ \infty } =
\left. - \alpha \cdot \frac{1}{- \alpha } \cdot \left( 1-x \right) e^{- \alpha y} \right|_0^{+ \infty } - \\
- \left. e^{- \alpha y} \right|_0^{+ \infty } =
\left. \left( 1-x \right) e^{- \alpha y} \right|_0^{+ \infty } - \left. e^{- \alpha y} \right|_0^{+ \infty } =
\left. e^{- \alpha y} \right|_0^{+ \infty } \left( 1-x-1 \right) = \\
= -x \left( e^{- \infty } -1 \right) =
x.
\end{split}
\end{equation*}

Запишем результат в виде системы
$$F_{ \frac{ \xi }{ \xi + \eta }} \left( x \right) =
\begin{cases}
0, \qquad x < 0, \\
x, \qquad x \in \left[ 0, 1 \right], \\
1, \qquad x > 1.
\end{cases}$$

Найдём плотность распределения как производную от функции распределения
$$p_{ \frac{ \xi }{ \xi + \eta }} \left( x \right) =
F'_{ \frac{ \xi }{ \xi + \eta }} \left( x \right) =
\begin{cases}
0, \qquad x < 0, \\
1, \qquad x \in \left[ 0, 1 \right], \\
0, \qquad x > 1
\end{cases} =
\mathbbm{1} \left( x \in \left[ 0, 1 \right] \right).$$

\subsubsection*{11.18}

\textit{Задание.} Пусть $ \xi_1, \dotsc, \xi_n$ --- независимые одинаково распределённые случайные величины с непрерывной функцией распределения $F$.
Упорядочим их по величине $ \xi_{ \left( 1 \right) } \leq \xi_{ \left( 2 \right) } \leq \dotsc \leq \xi_{ \left( n \right) }$.
Найдите функцию распределения случайных величин:
\begin{enumerate}[label=\alph*)]
\item $ \xi_{ \left( 1 \right) } = \min \left( \xi_1, \dotsc, \xi_n \right) $;
\item $ \xi_{ \left( n \right) } = \max \left( \xi_1, \dotsc, \xi_n \right) $;
\item $ \xi_{ \left( m \right) }$.
\end{enumerate}

\textit{Решение.}
\begin{enumerate}[label=\alph*)]
\item По определению
$F_{ \xi_{ \left( 1 \right) }} \left( x \right) =
P \left\{ \min \left( \xi_1, \dotsc, \xi_n \right) \leq x \right\}$.
Перейдём к противоположному событию
$F_{ \xi_{ \left( 1 \right) } } \left( x \right) =
1 - P \left\{ \min \left( \xi_1, \dotsc, \xi_n \right) > x \right\}$.
Это значит, что каждая из случайных величин больше $x$.
Получаем
$$F_{ \xi_{ \left( 1 \right) }} \left( x \right) =
1 - P \left( \xi_1 > x, \dotsc, \xi_n > x \right) =
1 - P \left( \xi_1 > x \right) \cdot \dotsc \cdot P \left( \xi_n > x \right).$$
Перейдём к противоположным событиям
$$F_{ \xi_{ \left( 1 \right) }} \left( x \right) =
1 - \left[ 1 - P \left( \xi_1 \leq x \right) \right] \cdot \dotsc \cdot \left[ 1 - P \left( \xi_n \leq x \right) \right].$$
Запишем через функцию распределения
$$F_{ \xi_{ \left( 1 \right) }} \left( x \right) =
1 - \left[ 1 - P \left( \xi_1 \leq x \right) \right]^n = 1 - \left[ 1 - F \left( x \right) \right]^n;$$
\item аналогично
$F_{ \xi_{ \left( n \right) }} \left( x \right) =
P \left( \xi_{ \left( n \right) } \leq x \right) =
P \left\{ \max \left( \xi_1, \dotsc, \xi_n \right) \leq x \right\}$.
Это значит, что каждая из $n$ случайных величин не превосходит $x$.
Получаем
$F_{ \xi_{ \left( n \right) }} \left( x \right) =
P \left( \xi_1 \leq x, \dotsc, \xi_n \leq x \right) =
P \left( \xi_1 \leq x \right) \cdot \dotsc \cdot P \left( \xi_n \leq x \right)$.
Так как случайные величины одинаково распределены, то
$$F_{ \xi_{ \left( n \right) }} \left( x \right) =
\left[ P \left( \xi_1 \leq x \right) \right]^n =
\left[ F \left( x \right) \right]^n;$$
\item  найдём функцию распределения $F_{ \xi_{\left( m \right) }} \left( x \right) = P \left( \xi_{ \left( m \right) } \leq x \right)$.
Из условия знаем, что $ \xi_{ \left( 1 \right) } \leq \xi_{ \left( 2 \right) } \leq \dotsc \leq \xi_{ \left( m \right) } \leq \dotsc \leq \xi_{ \left( n \right) }$.
Если $ \xi_{ \left( m \right) } \leq x$, то как минимум $ \xi_{ \left( 1 \right) }, \dotsc, \xi_{ \left( m \right) } \leq x$.
Так как $x$ может быть равен бесконечности, то хотя бы $m$ из $n$ случайных величин не должно превосходить значение $x$.
Получаем $F_{ \xi_{\left( m \right) }} \left( x \right) = P$(хотя бы $m$ из $n$ случайных величин не превышают $x$)$=$
\begin{equation*}
\begin{split}
= \sum \limits_{i=m}^n C_n^i \left[ P \left( \xi \leq x \right) \right]^i \left[ P \left( \xi > x \right) \right]^{n-i} = \\
= \sum \limits_{i=m}^n C_n^i \left[ F \left( x \right) \right]^i \cdot \left[ 1 - P \left( \xi \leq x \right) \right]^{n-i} =
\sum \limits_{i=m}^n C_n^i \left[ F \left( x \right) \right]^i \cdot \left[ 1 - F \left( x \right) \right]^{n-i}.
\end{split}
\end{equation*}
\end{enumerate}

\subsubsection*{11.19}

\textit{Задание.} Пусть $ \xi_1, \xi_2$ --- независимые случайные величины со стандартным нормальным распределением.
Найдите плотность распределения случайной величины $ \xi_1^2 + \xi_2^2$.

\textit{Решение.} Выпишем плотности $ \xi_1, \xi_2$ каждую из условия
$$p_{ \xi_1} \left( x \right) =
p_{ \xi_2} \left( x \right) =
\frac{1}{ \sqrt{2 \pi} \sigma} \cdot e^{- \frac{ \left( x - a \right)^2}{2 \sigma^2}}.$$
Стандартное нормальное распределение означает, что $a = 0, \sigma = 1$, то есть
$$p_{ \xi_1} \left( x \right) =
p_{ \xi_2} \left( x \right) =
\frac{1}{ \sqrt{2 \pi}} \cdot e^{- \frac{x^2}{2}}.$$

Тогда совместная плотность вектора из независимости случайных величин
$$p_{ \left( \xi_1, \xi_2 \right) } \left( x, y \right) =
p_{ \xi_1} \left( x \right) \cdot p_{ \xi_2} \left( y \right) =
\frac{1}{ \sqrt{2 \pi }} \cdot e^{- \frac{x^2}{2}} \cdot \frac{1}{ \sqrt{2 \pi }} \cdot e^{- \frac{y^2}{2}} =
\frac{1}{2 \pi } \cdot e^{- \frac{x^2 + y^2}{2}}.$$

Функция распределения по определению равна
$$F_{ \xi_1^2 + \xi_2^2} \left( t \right) =
P \left( \xi_1^2 + \xi_2^2 \leq t \right).$$
Это есть смысл продолжать при $t \geq 0$.
Записываем вероятность события как индикатор этого события $F_{ \xi_1^2 + \xi_2^2} \left( t \right) = M \mathbbm{1} \left( \xi_1^2 + \xi_2^2 \leq t \right) $.
Есть две случайные величины.
Отсюда следует, что будет двойной интеграл
$$F_{ \xi_1^2 + \xi_2^2} \left( t \right) =
\iint \limits_{ \mathbb{R}^2} \mathbbm{1} \left( x^2 + y^2 \leq t \right) \cdot p_{ \left( \xi_1, \xi_2 \right) } \left( x, y \right) dxdy.$$
Перепишем плотность
$$F_{ \xi_1^2 + \xi_2^2} \left( t \right) =
\iint \limits_{ \mathbb{R}^2} \mathbbm{1} \left( x^2 + y^2 \leq t \right) \cdot \frac{1}{2 \pi } \cdot e^{- \frac{x^2 + y^2}{2}} dxdy.$$
Перейдём в полярную систему координат
$x = \rho \cos \phi,
y = \rho \cos \phi,
dxdy = \\
= \rho d \rho d \phi,
\rho: 0 \rightarrow \sqrt{t},
\phi: 0 \rightarrow 2 \pi,
x^2 + y^2 = \rho^2 \cos^2 \phi + \rho^2 \sin \phi = \\
= \rho^2 \left( \sin^2 \phi + \cos^2 \phi \right) = \rho^2,
\mathbbm{1} \left( x^2 + y^2 \leq t \right) = \mathbbm{1} \left( \rho^2 \leq t \right) $.
Получаем
$$F_{ \xi_1^2 + \xi_2^2} \left( t \right) =
\int \limits_0^{2 \pi } \int \limits_0^{ \sqrt{t}} \frac{1}{2 \pi } \cdot e^{- \frac{ \rho^2}{2}} \rho d \rho d \phi.$$
По $d \phi $ сразу проинтегрируем
$$F_{ \xi_1^2 + \xi_2^2} \left( t \right) =
\frac{1}{2 \pi } \int \limits_0^{ \sqrt{t}} \left. e^{- \frac{ \rho^2}{2}} \cdot \phi \right|_0^{2 \pi } \rho d \rho =
\frac{1}{2 \pi } \int \limits_0^{ \sqrt{t}} \rho e^{- \rho^2} \cdot 2 \pi d \rho =
\int \limits_0^{ \sqrt{t}} \rho e^{- \frac{ \rho^2}{2}} d \rho.$$
Заметим, что
$$d \left( e^{- \frac{ \rho^2}{2}} \right) =
- \frac{2 \rho }{2} \cdot e^{- \frac{ \rho^2}{2}} d \rho =
- \rho e^{- \frac{ \rho^2}{2}} d \rho.$$
Отсюда выражаем подынтегральную функцию
$$ \rho e^{- \frac{ \rho^2}{2}} d \rho =
-d \left( e^{- \frac{ \rho^2}{2}} \right) .$$
Подставляем полученный результат
$$F_{ \xi_1^2 + \xi_2^2} \left( t \right) =
- \int \limits_0^{ \sqrt{t}} d \left( e^{- \frac{ \rho^2}{2}} \right) =
\left. - e^{- \frac{ \rho^2}{2}} \right|_0^{ \sqrt{t}} =
-e^{- \frac{t}{2}} + 1 =
1 - e^{- \frac{t}{2}}$$
--- это при $t \geq 0$.

Нужно продифференцировать по $t$.
Получим искомую плотность распределения
$$p_{ \xi_1^2 + \xi_2^2} \left( t \right) =
\frac{1}{2} \cdot e^{- \frac{t}{2}} \cdot \mathbbm{1} \left( t \geq 0 \right).$$
