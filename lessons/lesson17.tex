\addcontentsline{toc}{chapter}{Занятие 17. Центральная предельная теорема}
\chapter*{Занятие 17. Центральная предельная теорема}

\addcontentsline{toc}{section}{Контрольные вопросы и задания}
\section*{Контрольные вопросы и задания}

\subsubsection*{Сформулируйте центральную предельную теорему,
                запишите центральную предельную теорему для последовательности испытаний Бернулли.}

Центральная предельная теорема.
$ \xi_1, \dotsc, \xi_n$ -- независимые одинаково распределённые случайные величины,
$M \xi_1 = a, \,
  D \xi_1 = \sigma^2 < \infty $.
Тогда
$$ \frac{S_n - MS_n}{ \sqrt{D \xi_1}} \overset{d}{ \rightarrow } N \left( 0, 1 \right), \,
  n \to \infty,$$
где
$$S_n =
  \sum \limits_{i = 1}^n \xi_i.$$

Подставляя значения математического ожидания и дисперсии получим
$$ \frac{S_n - na}{ \sqrt{n}} \overset{d}{ \rightarrow } N \left( 0, \sigma^2 \right), \
  n \to \infty.$$

Пусть $ \left\{ \varepsilon_k \right\} $ ---
последовательность независимых одинаково распределённых случайных величин, таких, что
$$ \varepsilon_k =
  \begin{cases}
    1, \qquad p, \\
    0, \qquad 1 - p.
  \end{cases}$$
Тогда
$$ \frac{S_n - np}{ \sqrt{n}} \overset{d}{ \rightarrow }
  N \left( 0, p \left( 1 - p \right) \right).$$

Перенесём $p \left( 1 - p \right)$ влево
$$ \frac{S_n - np}{ \sqrt{n} \sqrt{p \left( 1 - p \right) }} \overset{d}{ \rightarrow }
  N \left( 0, 1 \right).$$

\addcontentsline{toc}{section}{Аудиторные задачи}
\section*{Аудиторные задачи}

\addcontentsline{toc}{section}{Домашнее задание}
\section*{Домашнее задание}
