\addcontentsline{toc}{chapter}{Занятие 16. Слабая сходимость случайных величин. Заон больших чисел}
\chapter*{Занятие 16. Слабая сходимость случайных величин. Заон больших чисел}

\addcontentsline{toc}{section}{Контрольные вопросы и задания}
\section*{Контрольные вопросы и задания}

\subsubsection*{Приведите определение видов сходимости случайных величин; какая связь между ними?}

Последовательность $ \left\{ \xi_n: \, n \geq 1 \right\} $ сходится к случайной величине $ \xi $ почти наверное (с вероятностью 1),
если $ \exists \Omega_0 \subset \Omega, \, \Omega_0 $ ---
случайное событие $ \left( \Omega_0 \in \mathcal{F} \right): \, P \left( \Omega_0 \right) = 1$ и
$ \forall \omega \in \Omega_0:
\xi_n \left( \omega \right) \rightarrow \xi \left( \omega \right),
n \rightarrow \infty $
(поточечная сходимость на множестве полной вероятности), т.е. $P \left\{ \lim \limits_{n \to \infty} \xi_n \left( \omega \right) = \xi \left( \omega \right) \right\} = 1$.

Последовательность случайных величин $ \left\{ \xi_n \right\} $ сходится по вероятности к случайной величине $ \xi $,
если $ \forall \epsilon > 0 \, P \left\{ \left| \xi_n - \xi \right| > \epsilon \right\} \rightarrow \infty, \, n \rightarrow \infty $.

Лемма.
Пусть $ \xi_n \overset{almost sure (a.s.)}{ \rightarrow } \xi, \, n \rightarrow \infty $, тогда $ \xi_n \overset{P}{ \rightarrow } \xi, \, n \rightarrow \infty $,
т.е. их сходимости почти наверное следует сходимость по вероятности.

Лемма Рисса.
Пусть $ \xi_n \overset{P}{ \rightarrow } \xi, \, n \rightarrow \infty $.
Тогда существует подпоследовательность: $ \left\{ \xi_{n_k}: \, k \geq 1 \right\} $ такая, что $ \xi_{n_k} \overset{a.s.}{ \rightarrow } \xi, \, k \rightarrow \infty $.

Лемма (характеризация сходимости по вероятности).
Если последовательность $ \left\{ \xi_n \right\} $ и случайная величина $ \xi $ таковы,
что их любой подпоследовательности $ \left\{ \xi_{n_k}: \, k \geq 1 \right\} $ можно выбрать подподпоследовательность
$$ \left\{ \xi_{n_{k_j}}: \, j \geq 1 \right\} $$
такую,
что $ \xi_{n_{k_j}} \overset{a.s.}{ \rightarrow } \xi, \, j \rightarrow \infty $, то сама $ \xi_n \overset{P}{ \rightarrow } \xi, \, n \rightarrow \infty $.

Случайные величины $ \xi_n$ слабо сходятся (сходятся по распределению) к случайной величине $ \xi $,
если для всякой непрерывной и ограниченной функции
$f: \,
\mathbb{R} \rightarrow \mathbb{R} \,
Mf \left( \xi_n \right) \rightarrow Mf \left( \xi \right) $ при $n \rightarrow \infty $.

\subsubsection*{Приведите критерии слабой сходимости (в терминах сходимости функций распределения и в терминах сходимости характеристических функций.}

Последовательность функций распределения $ \left\{ F_n \right\} $ слабо сходится к функции распределения $F$,
если для всякой непрерывной и ограниченной функции
$f: \, \mathbb{R} \rightarrow \mathbb{R} \,
\int \limits_{- \infty }^{+ \infty } f \left( x \right) dF_n \left( x \right) \rightarrow \int \limits_{- \infty }^{+ \infty } f \left( x \right) dF \left( x \right) $
при $n \rightarrow \infty $.

Если $ \xi_n \Rightarrow \xi $ при $n \rightarrow \infty $,
то $ \varphi_{ \xi_n} \left( t \right) = Me^{it \xi_n} \rightarrow Me^{it \xi } = \varphi_{ \xi } \left( t \right), \, n \rightarrow \infty $.

\subsubsection*{Сформулируйте закон больших чисел.}

Пусть $ \xi_1, \dotsc, \xi_n$ --- последовательность независимых одинаково распределённых случайных величины (функции распределения $ \xi_n: n \geq 1$ совпадают).
Организуем нарастающие суммы
$$S_n =
\sum \limits_{k = 1}^n \xi_k.$$
Берём среднее (ограниченные нормированные суммы):
$$ \frac{1}{n} \cdot S_n.$$
Если $ \exists M \xi_1 = a$, то
$$ \frac{1}{n} \cdot S_n \overset{P}{ \rightarrow } a, \,
n \rightarrow \infty.$$

\addcontentsline{toc}{section}{Аудиторные задачи}
\section*{Аудиторные задачи}

\subsubsection*{16.2}

\textit{Задание.} Пусть $ \xi_{ \lambda }$ --- случайная величина, распределённая по закону Пуассона с параметром $ \lambda $.
Докажите, что при $ \lambda \rightarrow \infty $ распределение случайной величины
$$ \frac{ \xi_{ \lambda } - \lambda }{ \sqrt{ \lambda }}$$
слабо сходится к распределению случайной величины, которая имеет нормлаьное стандартное распределение.

\textit{Решение.} $ \xi_{ \lambda } \sim Pois \left( \lambda \right), \, \lambda > 0$.

Нужно проверить, что
$$ \eta_{ \lambda } = \frac{ \xi_{ \lambda } - \lambda }{ \sqrt{ \lambda }} \overset{d}{ \rightarrow } \eta, \,
\lambda \rightarrow + \infty, \,
\eta \sim N \left( 0, 1 \right).$$

Будем проверять сходимость характеристических функций.

Теорема.
Пусть $ \left\{ \xi_n \right\}_{n \geq 1}$ --- это последовательность случайных величин,
тогда $ \xi_n \overset{d}{ \rightarrow } \xi, \, n \rightarrow \infty $ тогда и только тогда,
когда $ \forall t \in \mathbb{R} \, \varphi_{ \xi_n} \left( t \right) \rightarrow \varphi_{ \xi } \left( t \right) $.

В данной задаче $ \eta_{ \lambda }$ --- не последовательность, а семейство случайных величин.

Посчитаем характеристическую функцию случайной величины $ \eta_{ \lambda }$.
По определению $ \varphi_{ \eta_{ \lambda }} \left( t \right) = Me^{it \cdot \frac{ \xi_{ \lambda } - \lambda }{ \sqrt{ \lambda }}}$.
Характеристическая функция пуассоновской случайной величины $ \xi_{ \lambda }$ с параметром $ \lambda $ имеет вид
$$ \varphi_{ \xi_{ \lambda }} \left( t \right) =
e^{ \lambda \left( e^{it} - 1 \right) } =
Me^{it \xi_{ \lambda }}.$$
Тогда
$ \varphi_{ \eta_{ \lambda }} \left( t \right) =
Me^{it \cdot \frac{ \xi_{ \lambda }}{ \sqrt{ \lambda }}} \cdot e^{-it \sqrt{ \lambda }} =
e^{-it \sqrt{ \lambda }} e^{ \lambda \left( e^{i \cdot \frac{t}{ \sqrt{ \lambda }}} - 1 \right) }$.
Запишем под одну экспоненту $ \varphi_{ \eta_{ \lambda }} \left( t \right) = e^{ \lambda  \left( e^{i \cdot \frac{t}{ \sqrt{ \lambda }}} - 1 \right) - it \sqrt{ \lambda }}$.

Нужно проверить, что $ \varphi_{ \eta_{ \lambda }} \left( t \right) \rightarrow \varphi_{ \eta } \left( t \right) = e^{- \frac{t^2}{2}}, \, \eta \sim N \left( 0, 1 \right) $.

Нужно показать, что показатель в экспоненте сходится к
$$ - \frac{t^2}{2}, \, \lambda \rightarrow \infty .$$

Воспользуемся формулой Эйлера
$$ \lim \limits_{ \lambda \to + \infty } \left[ \lambda \left( e^{i \cdot \frac{t}{ \sqrt{ \lambda }} - 1} \right) - it \sqrt{ \lambda } \right] =
\lim \limits_{ \lambda \to + \infty }
\left( - \sqrt{ \lambda } it + \lambda \left( \cos \frac{t}{ \sqrt{ \lambda }} + i \sin \frac{t}{ \sqrt{ \lambda }} - 1 \right) \right).$$
Отделим действительную и мнимую части
\begin{equation*}
\begin{split}
\lim \limits_{ \lambda \to + \infty } \left[ \lambda \left( e^{i \cdot \frac{t}{ \sqrt{ \lambda }} - 1} \right) - it \sqrt{ \lambda } \right] = \\
= \lim \limits_{ \lambda \to + \infty }
\left( \lambda \left( \cos \frac{t}{ \sqrt{ \lambda }} - 1 \right) + i \left( \lambda \sin \frac{t}{ \sqrt{ \lambda }} - \sqrt{ \lambda } t \right) \right) = \\
= \lim \limits_{ \lambda \to + \infty }
\left( \lambda \left( 1 - \frac{t^2}{2 \lambda } + o \left( \frac{t^3}{ \sqrt{ \lambda }} \right) - 1 \right) +
i \left( \lambda \sin \frac{t}{ \sqrt{ \lambda }} - \sqrt{ \lambda } t \right) \right).
\end{split}
\end{equation*}
Действительная часть стремится к $- t^2 / 2, \, \lambda \rightarrow + \infty $.

Оценим мнимую часть
$$ \lambda \sin \frac{t}{ \sqrt{ \lambda }} - \sqrt{ \lambda } t =
\lambda \left( \frac{t}{ \sqrt{ \lambda }} - \frac{t^3}{6 \lambda^{ \frac{3}{2}}} + o \left( \frac{t^4}{ \lambda^2} \right) \right) - \sqrt{ \lambda } t.$$
Первое слагаемое в скобках и слагаемое после скобок уничтожаются
$$ \lambda \sin \frac{t}{ \sqrt{ \lambda }} - \sqrt{ \lambda } t =
- \frac{t^3}{6 \sqrt{ \lambda }} + o \left( 1 \right) \rightarrow 0 $$
при $ \lambda \rightarrow + \infty $, так как каждое из слагаемых стремится к нулю.

Значит, показатель экспоненты стремится к $- t^2 / 2$.

\subsubsection*{16.3}

\textit{Задание.}
Пусть последовательность случайных величин $ \left\{ \xi_n \right\}_{n \geq 1}$ слабо сходится к случайной величине $ \xi $,
и пусть $g: \, \mathbb{R} \rightarrow \mathbbm{R}$ --- непрерывная функция.
Докажите, что последовательность $ \left\{ g \left( \xi_n \right) \right\}_{n \geq 1}$ слабо сходится к $g \left( \xi \right) $.

\textit{Решение.} $ \xi_n \overset{d}{ \rightarrow } \xi, \, n \rightarrow \infty, \, g: \, \mathbb{R} \rightarrow \mathbb{R}$.

Доказать, что $g \left( \xi_n \right) \overset{d}{ \rightarrow } g \left( \xi \right) $.
Это означает, что $Mf \left( g \left( \xi_n \right) \right) \rightarrow Mf \left( g \left( \xi \right) \right) $.

Обозначим $h \left( x \right) = f \left( g \left( x \right) \right) $.
Тогда $Mh \left( \xi_n \right) \rightarrow Mh \left( \xi \right) $.

Функция $h$ --- непрерывная, так как это композиция двух непрерывных функций $f$ и $g$; функция $f$ --- ограничена.
Из этого следует, что $h$ --- ограничена.
Тогда $h \in C_b \left( \mathbb{R} \right) $, следовательно, $g \left( \xi_n \right) \overset{d}{ \rightarrow } g \left( \xi \right) $.

\addcontentsline{toc}{section}{Домашнее задание}
\section*{Домашнее задание}
