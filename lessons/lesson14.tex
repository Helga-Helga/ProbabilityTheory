\addcontentsline{toc}{chapter}{Занятие 14. Сходимость последовательностей случайных величин}
\chapter*{Занятие 14. Сходимость последовательностей случайных величин}

\addcontentsline{toc}{section}{Контрольные вопросы и задания}
\section*{Контрольные вопросы и задания}

\addcontentsline{toc}{section}{Аудиторные задачи}
\section*{Аудиторные задачи}

\subsubsection*{14.3}

\textit{Задание.} Пусть $ \left\{ \xi_n \right\}_{n \geq 1}$ --- последовательность нормально распределённых случайных величин,
$$ \xi_n \sim N \left( a, \frac{1}{n} \right).$$
Докажите, что:
\begin{enumerate}[label=\alph*)]
\item $ \xi_n \overset{P}{ \rightarrow } a$;
\item $ \xi_n \overset{L_2}{ \rightarrow } a$;
\item $ \xi_n \overset{L_p}{ \rightarrow } a, p \geq 1$.
\end{enumerate}

\textit{Решение.}
\begin{enumerate}[label=\alph*)]
\item Запишем определение сходимости по вероятности
$$ \forall \epsilon > 0,
\qquad P \left\{ \left| \xi_n - a \right| > \epsilon \right\} \rightarrow 0$$
при $n \rightarrow \infty $.

Неравенство Чебышева
$$P \left\{ \left| \xi_n - a \right| > \epsilon \right\} =
P \left\{ \left| \xi_n - a \right|^2 > \epsilon^2 \right\} \leq
\frac{M \left| \xi_n - a \right|^2}{ \epsilon^2} =
\frac{1}{n \epsilon^2} \rightarrow
0$$
при $n \to \infty $.
Сходимость по вероятности есть;
\item $ \xi_n \overset{L_2}{ \rightarrow } a$.
Это означает, что $M \left| \xi_n - a \right| \rightarrow 0$.
Имеем
$$M \left| \xi_n - a \right| =
\frac{1}{n} \rightarrow
0;$$
\item неравенство Ляпунова
$ \left( M \left| \zeta \right|^{r_1} \right)^{ \frac{1}{r_1}} \leq \left( M \left| \zeta \right|^{r_2} \right)^{ \frac{1}{r_2}},$ где $ \zeta = \xi_n - a$.

В этом случае
$$M \left| \xi_n - a \right|^{2k} =
\frac{ \left( 2k-1 \right)!!}{n^{2k}} \rightarrow
0.$$

Случайная величина имеет распределение $ \xi \sim N \left( 0, \sigma^2 \right) $.
Из этого следует, что $M \xi^{2k} = \left( 2k-1 \right)!! \cdot \sigma^{2k}$,
где двойной факториал означает произведение всех нечётных целых положительных числе до $ \left( 2k-1 \right) $.
Получаем, что в любом $L_{2k}$ есть сходимость.
\end{enumerate}

\addcontentsline{toc}{section}{Домашнее задание}
\section*{Домашнее задание}

\subsubsection*{14.18}

\textit{Задание.} Пусть $ \xi_n \overset{P}{ \rightarrow } \xi, \eta_n \overset{P}{ \rightarrow } \eta$.
Докажите, что:
\begin{enumerate}[label=\alph*)]
\item $ \xi_n + \eta_n \overset{P}{ \rightarrow } \xi + \eta$;
\item $ \left| \xi_n \right| \overset{P}{ \rightarrow } \left| \xi \right| $;
\item $ \xi_n \eta_n \overset{P}{ \rightarrow } \xi \eta $.
\end{enumerate}

\textit{Решение.}
\begin{enumerate}[label=\alph*)]
\item Из неравенства треугольника
$ \left| \xi_n - \xi + \eta_n - \eta \right| \leq \left| \xi_n - \xi \right| + \left| \eta_n - \eta \right| $.
Отсюда следует, что
$$ \left\{ \left| \xi_n - \xi + \eta_n - \eta \right| > \epsilon \right\} \subseteq
\left\{ \left| \xi_n - \xi \right| + \left| \eta_n - \eta \right| > \epsilon \right\}.$$
Разобъём на объединение двух множеств
$$ \left\{ \left| \xi_n - \xi + \eta_n - \eta \right| > \epsilon \right\} \subseteq
\left\{ \left| \xi_n - \xi \right| > \frac{ \epsilon }{2} \right\} \cup \left\{ \left| \eta_n - \eta \right| > \frac{ \epsilon }{2} \right\}.$$

Тогда вероятность, которая фигурирует в определении сходимости по вероятности
\begin{equation*}
\begin{split}
P \left\{ \left| \xi_n - \xi + \eta_n - \eta \right| > \epsilon \right\} \leq
P \left\{ \left| \xi_n - \xi \right| + \left| \eta_n - \eta \right| > \epsilon \right\} \leq \\
\leq P \left\{ \left| \xi_n - \xi \right| > \frac{ \epsilon }{2} \right\} + P \left\{ \left| \eta_n - \eta \right| > \frac{ \epsilon }{2} \right\}.
\end{split}
\end{equation*}
По условию каждая из этих вероятностей стремится к нулю при
$$n \rightarrow \infty,$$
поэтому и всё выражение стремится к нулю при $n \rightarrow \infty $;
\item оценим вероятность
$P \left( \left| \left| \xi_n \right| - \left| \xi \right| \right| < \epsilon \right) \geq
P \left( \left| \xi_n - \xi \right| < \epsilon \right) \rightarrow
1$
по условию;
\item выпишем нужную вероятность
$$P \left( \left| \xi_n \eta_n - \xi \eta \right| > \epsilon \right) =
P \left( \left| \xi_n \eta_n - \xi \eta_n + \xi \eta_n - \xi \eta \right| > \epsilon \right).$$
Сгруппируем слагаемые
\begin{equation*}
\begin{split}
P \left( \left| \xi_n \eta_n - \xi \eta \right| > \epsilon \right) =
P \left( \left| \eta_n \left( \xi_n - \xi \right) + \xi \left( \eta_n - \eta \right) \right| > \epsilon \right) \leq \\
\leq P \left( \left| \eta \right| \cdot \left| \xi_n - \xi \right| > \frac{ \epsilon }{2} \right) +
P \left( \left| \xi_n \right| \cdot \left| \eta_n - \eta \right| > \frac{ \epsilon }{2} \right) \rightarrow
0,
\end{split}
\end{equation*}
так как по условию каждая из этих вероятностей стремится к нулю.
\end{enumerate}
