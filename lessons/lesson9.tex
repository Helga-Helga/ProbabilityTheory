\addcontentsline{toc}{chapter}{Занятие 9. Распределение случайных величин}
\chapter*{Занятие 9. Распределение случайных величин}

\addcontentsline{toc}{section}{Контрольные вопросы и задания}
\section*{Контрольные вопросы и задания}

\subsubsection*{Приведите определение случайной величины, $ \sigma $-алгебры, порождённой случайной величиной.}

Функция $ \xi: \, \Omega \rightarrow \mathbb{R}$ называется случайной величиной,
если
$ \forall c \in \mathbb{R}: \,
\left\{ \omega \; \middle| \; \xi \left( \omega \right) \leq c \right\} =
\xi^{-1} \left( \left( - \infty, c \right] \right) \in
\mathcal{F} $.

Функция $ \xi: \, \Omega \rightarrow \mathbb{R}$ называется случайной величиной,
если выполнено следующее требование
$ \forall \Delta \in \mathcal{B} \left( \mathbb{R} \right): \,
\xi^{-1} \left( \Delta \right) =
\left\{ \omega \; \middle| \; \xi \left( \omega \right) \in \Delta \right\} \in
\mathcal{F}$
(является случайным событием).

$ \sigma $-алгебра,
порождённая случайной величиной $ \xi $ ---
это совокупность всех случайных событий вида $ \xi^{-1} \left( \Delta \right) $, где $ \Delta \in \mathcal{B} \left( \mathbb{R} \right) $.

\subsubsection*{Приведите определение функции распределения случайной величины, перечислите свойства функции распределения.}

Для случайной величины $ \xi $ функция распределения $F_{ \xi } \left( x \right) = P \left\{ \xi \leq x \right\} $.
Эта функция задана на $x \in \mathbb{R}$.

Свойства:
\begin{enumerate}
\item $F_{ \xi }: \, \mathbb{R} \rightarrow \left[ 0, 1 \right] $;
\item $x_1 \leq x_2: \, F_{ \xi } \left( x_1 \right) \leq F_{ \xi } \left( x_2 \right) $;
\item $F_{ \xi } \left( - \infty \right) = \lim \limits_{n \to - \infty } F_{ \xi } \left( x \right) = 0, \,
F_{ \xi } \left( + \infty \right) = \lim \limits_{n \to + \infty } F_{ \xi } \left( x \right) = 1$;
\item для каждого $x_0 \in \mathbb{R}: \, F_{ \xi } \left( x_0 + \right) = F_{ \xi } \left( x_0 \right) $.
\end{enumerate}

\subsubsection*{Приведите определение плотности распределения случайной величины.}

Случайная величина $ \xi $ имеет плотность распределения $p_{ \xi }$, если
$$ \forall a \leq b, \qquad
F_{ \xi } \left( b \right) - F_{ \xi } \left( a \right) =
\int \limits_a^b p_{ \xi } \left( x \right) dx.$$

\subsubsection*{Какая связь между функцией распределения и плотностью распределения случайной величины?}

$F_{ \xi } \left( b \right) - F_{ \xi } \left( a \right) $ ---
это вероятность попадания $ \xi $ в интервал $ \left( a, b \right] $, то есть $P \left( \xi \in \left( a, b \right] \right) $.

\subsubsection*{Сформулируйте свойства плотности распределения случайной величины.}

Свойства:
\begin{enumerate}
\item $p_{ \xi } \geq 0$;
\item $ \int_{- \infty }^{+ \infty }p_{ \xi } \left( x \right) dx = 1$.
\end{enumerate}

\addcontentsline{toc}{section}{Дополнительные задачи}
\section*{Дополнительные задачи}

\addcontentsline{toc}{section}{Домашнее задание}
\section*{Домашнее задание}
