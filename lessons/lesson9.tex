\addcontentsline{toc}{chapter}{Занятие 9. Распределение случайных величин}
\chapter*{Занятие 9. Распределение случайных величин}

\addcontentsline{toc}{section}{Контрольные вопросы и задания}
\section*{Контрольные вопросы и задания}

\subsubsection*{Приведите определение случайной величины, $ \sigma $-алгебры, порождённой случайной величиной.}

Функция $ \xi: \, \Omega \rightarrow \mathbb{R}$ называется случайной величиной,
если
$ \forall c \in \mathbb{R}: \,
\left\{ \omega \; \middle| \; \xi \left( \omega \right) \leq c \right\} =
\xi^{-1} \left( \left( - \infty, c \right] \right) \in
\mathcal{F} $.

Функция $ \xi: \, \Omega \rightarrow \mathbb{R}$ называется случайной величиной,
если выполнено следующее требование
$ \forall \Delta \in \mathcal{B} \left( \mathbb{R} \right): \,
\xi^{-1} \left( \Delta \right) =
\left\{ \omega \; \middle| \; \xi \left( \omega \right) \in \Delta \right\} \in
\mathcal{F}$
(является случайным событием).

$ \sigma $-алгебра,
порождённая случайной величиной $ \xi $ ---
это совокупность всех случайных событий вида $ \xi^{-1} \left( \Delta \right) $, где $ \Delta \in \mathcal{B} \left( \mathbb{R} \right) $.

\addcontentsline{toc}{section}{Дополнительные задачи}
\section*{Дополнительные задачи}

\addcontentsline{toc}{section}{Домашнее задание}
\section*{Домашнее задание}
