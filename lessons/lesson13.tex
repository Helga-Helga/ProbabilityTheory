\addcontentsline{toc}{chapter}{Занятие 13. Последовательности случайных величин. Лемма Бореля-Кантелли}
\chapter*{Занятие 13. Последовательности случайных величин. Лемма Бореля-Кантелли}

\addcontentsline{toc}{section}{Контрольные вопросы и задания}
\section*{Контрольные вопросы и задания}

\subsubsection*{Сформулируйте лемму Бореля-Кантелли, закон <<нуля или единицы>> Колмогорова; запишите теоретико-множественное изображение событий $ \varlimsup \limits_{n \to \infty } A_n, \, \varliminf \limits_{n \to \infty } A_n$.}

Лемма Бореля-Кантелли.
\begin{enumerate}
\item Если
$$ \sum \limits_{n=1}^{ \infty } P \left( A_n \right) < + \infty,$$
то $P \left( \varlimsup \limits_{n \to \infty } A_n \right) = 0$;
\item если $ \left\{ A_n \right\} $ --- независимы и
$$ \sum \limits_{n=1}^{ \infty } P \left( A_n \right) =
+ \infty $$
(ряд, составленный из вероятностей, расходится), то $P \left( \varlimsup \limits_{n \to \infty } A_n \right) = 1$.
\end{enumerate}

Закон <<нуля или единицы>> Колмогорова.
Пусть дано вероятностное пространство
$ \left( \Omega, \mathcal{F}, \mathbb{P} \right) $
и определённая на нём последовательность независимыых случайных величин
$ \left\{ X_n \right)_{n=1}^{ \infty } $ (не обязятельно одинаково распределённых).
Пусть $ \mathcal{F}_{ \infty }$ --- её остаточная $ \sigma $-алгебра, то есть
$$ \mathcal{F}_{ \infty } =
\sigma \left( \bigcap \limits_{m=1}^{ \infty } \bigcup \limits_{n \geq m} \mathcal{F}_n \right),$$
где $ \mathcal{F}_n $ есть $ \sigma $-алгебра, порождённая случайной величиной $X_n$.

Тогда если $A \in \mathcal{F}_{ \infty }$, то $ \mathbb{P} \left( A \right) = 0$ или $ \mathbb{P} \left( A \right) = 1$.

Другими словами, $A$ --- остаточное событие, если оно измеримо относительно $ \sigma $-алгебры,
порождённой случайными величинами $ \left\{ X_n \right\}_{n=1}^{ \infty }$, но независимо от любого конечного подмножества этих величин.
Согласно теореме, такое событие имеет вероятность ноль или единица.

Пусть $ \left\{ A_n \right\}_{n \geq 1}$ --- последовательность случайных событий.

$$ \varlimsup \limits_{n \to \infty } A_n =
\bigcap \limits_{n=1}^{ \infty } \bigcup \limits_{m=n}^{ \infty } A_m.$$

Это означает, что $ \forall n \, \exists m, \qquad A_m$ произошло.
В последовательности
$$ \left\{ A_n \right\}_{n \geq 1}$$
происходит бесконечно много событий $A_n$.

$$ \varliminf \limits_{n \to \infty } \bigcup \limits_{n=1}^{ \infty } \bigcap \limits_{m=n}^{ \infty } A_m =$$
= \{начиная с некоторого номера в $ \left\{ A_n \right\}_{n \geq 1}$ происходят все события $A_n$\}.

$ \exists n \, \forall m, \qquad A_m$ произошло.

\addcontentsline{toc}{section}{Аудиторные задачи}
\section*{Аудиторные задачи}

\subsubsection*{13.3}

\textit{Задание.} Найдите вероятность того, что при последовательных подбрасываниях монеты серия из пяти последовательных гербов выпадет бесконечно много раз.

\textit{Решение.}
Введём события $A_1 =$ \{герб выпадет при подбрасываниях 1 --- 5\}, $A_2 =$ \{герб выпадет при подбрасываниях 6 --- 10\},
$\dotsc, \, A_n =$ \{герб выпадет при подбрасываниях $\left( 5 \left( n-1 \right) + 1 \right) - 5n$\}.

Хотим доказать, что ряд из вероятностей таких событий расходится.
Для этого нужно посчитать вероятность такого события.

Подбрасывания независимы
$$P \left( A_k \right) =
\left( \frac{1}{2} \right)^5.$$

Имеем ряд
$$ \sum \limits_{k=1}^{ \infty } P \left( A_k \right) =
+ \infty.$$

По второй части леммы Бореля-Кантелли $P \left( \varlimsup \limits_{n \to \infty } A_n \right) = 1$.

\addcontentsline{toc}{section}{Дополнительные задачи}
\section*{Дополнительные задачи}

\addcontentsline{toc}{section}{Домашнее задание}
\section*{Домашнее задание}
