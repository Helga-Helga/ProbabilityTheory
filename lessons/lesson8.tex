\addcontentsline{toc}{chapter}{Занятие 8. Случайные величины. Измеримость}
\chapter*{Занятие 8. Случайные величины. Измеримость}

\addcontentsline{toc}{section}{Контрольные вопросы и задания}
\section*{Контрольные вопросы и задания}

\subsubsection*{Приведите определение $ \sigma $-алгебры борелевых множеств, измеримой функции, болелевской функции, случайной величины, $ \sigma $-алгебры, порождённой случайной величиной.}

Борелевская $ \sigma $-алгебра --- наименьшая $ \sigma $-алгебра, порождённая открытыми множествами.

Пусть $ \left( X, \mathcal{F} \right) $ и $ \left( Y, \mathcal{G} \right) $ --- два множества с выделенными алгебрами подмножеств.
Тогда функция $f: X \rightarrow Y$ называется $ \mathcal{F} / \mathcal{G} $-измеримой, или просто измеримой,
если полный прообраз любого множества из $ \mathcal{G} $ принадлежит $ \mathcal{F} $, то есть
$$ \forall B \in \mathcal{G}, \,
f^{-1} \left( B \right) \in \mathcal{F},$$
где $f^{-1} \left( B \right) $ означает полный прообраз множества $B$.

Борелева (борелевская) функция:
$ \xi: \left( \mathbb{R}, \mathcal{B} \left( \mathbb{R} \right) \right) \rightarrow
\left( \mathbb{R}, \mathcal{B} \left( \mathbb{R} \right) \right) $.

Есть $ \left( \Omega, \mathcal{F} \right) $ --- пространство элементарных событий $ \Omega $ с $ \sigma $-алгеброй событий $ \mathcal{F} $ на нём.
Есть пара $ \left( \mathbb{R}, \mathcal{B} \left( \mathbb{R} \right) \right) $,
где $ \mathcal{B} \left( \mathbb{R} \right) - \sigma $-алгебра борелевых подмножеств на $ \mathbb{R} $.
Отображение $ \xi: \Omega \rightarrow \mathbb{R} $ называется случайной величиной, если выполняется $ \forall B \subset \mathcal{B} \left( \mathbb{R} \right): \, \left\{ \omega: \xi \left( \omega \right) \in B \right\} \in \mathcal{F} $, другими словами, $ \xi^{-1} \left( B \right) \in \mathcal{F} $ (прообраз каждого множества $B$ должен быть измеримым).

$ \sigma $-алгебра, порождённая случайной величиной $ \xi: X \rightarrow \mathbb{R} $, определяется следующим образом:
$$ \sigma \left( \xi \right) =
\left\{ \left. \xi^{-1} \left( B \right) \right| B \in \mathcal{B} \left( \mathbb{R} \right) \right\},$$
где $ \mathcal{B} \left( \mathbb{R} \right) $ --- борелевская сигма-алгебра на вещественной прямой.

\subsubsection*{Сформулируйте меру про сходимость по вероятности множеств из минимальной $ \sigma $-алгебры.}

Последовательность случайных величин $ \left\{ \xi_n: \, n \geq 1 \right\} $ сходится по вероятности к случайное величине $ \xi $, если:
$$ \forall \epsilon > 0, \qquad
P \left\{ \left| \xi_n - \xi \right| > \epsilon \right\} \rightarrow 0, \qquad
n \rightarrow \infty.$$

\subsubsection*{Какие $ \sigma $-алгебры называются независимыми?}

Пусть $ \mathcal{A}_1, \mathcal{A}_2 \subset \mathcal{F} $ две сигма-алгебры на одном и том же вероятностном пространстве.
Они называются независимыми, если любые их представители независимы между собой, то есть:
$$ \mathbb{P} \left( A_1 \cap A_2 \right) =
\mathbb{P} \left( A_1 \right) \cdot \mathbb{P} \left( A_2 \right), \qquad
\forall A_1 \in \mathcal{A}_1, \,
A_2 \in \mathcal{A}_2.$$

\subsubsection*{Приведите определение независимых случайных величин.}

Пусть дано семейство случайных величин $ \left( X_i \right)_{i \in I} $,
так что
$$X_i: \,
\Omega \rightarrow \mathbb{R}, \qquad
\forall i \in I.$$
Тогда эти случайные величины попарно независимы,
если попарно независимы порождённые ими сигма-алгебры $ \left\{ \sigma \left( X_i \right) \right\}_{i \in I} $.
Случайные величины независимы в совокупности, если таковы порождённые ими сигма-алгебры.

\addcontentsline{toc}{section}{Аудиторные задачи}
\section*{Аудиторные задачи}

\subsubsection*{8.3}

\textit{Задание.} Пусть $ \xi, \eta $ --- случайные величины, которые определены на вероятностном пространстве $ \left( \Omega, \mathcal{F}, \mathbb{P} \right) $.
Докажите,
что множества
$A = \\
= \left\{ \omega \in \Omega: \,
\xi \left( \omega \right) <
\eta \left( \omega \right) \right\}, \,
B =
\left\{ \omega \in \Omega: \,
\xi \left( \omega \right) =
\eta \left( \omega \right) \right\}, \,
C = \\
= \left\{ \omega \in \Omega: \,
\xi \left( \omega \right) \leq
\eta \left( \omega \right) \right\} $
являются событиями.

\textit{Решение.} Есть $ \xi, \eta $ --- случайные величины на вероятностном пространстве $ \left( \Omega, \mathcal{F}, \mathbb{P} \right) $.
Нужно доказать, что множества являются событиями, то есть они принадлежат $ \sigma $-алгебре $ \mathcal{F} $.

Нужно проверить, что
$$A =
\left\{ \omega \in \Omega: \,
\xi \left( \omega \right) <
\eta \left( \omega \right) \right\} \in
\mathcal{F}.$$

Пусть
$$ \forall c \in \mathbb{R}: \qquad \left\{ \omega: \xi \left( \omega \right) \leq c \right\} \in \mathcal{F}, \\
\forall b \in \mathbb{R}: \qquad \left\{ \omega: \xi \left( \omega \right) \leq b \right\} \in \mathcal{F}.$$

Множество вещественных чисел $ \mathbb{R} $ является несчётным.
Нужно счётное количество операций.
Может разделить две случайные величины рациональной точкой
\begin{equation*}
\begin{split}
A =
\bigcup \limits_{c_i \in \mathbb{Q} } \left\{ \omega \in \Omega: \, \xi \left( \omega \right) < c_i < \eta \left( \omega \right) \right\} = \\
= \bigcup \limits_{c_i \in \mathbb{Q} }
\left( \left\{ \omega \in \Omega: \, \xi \left( \omega \right) < c_i \right\} \cap
\left\{ \omega: \, \eta \left( \omega \right) > c_i \right\} \right).
\end{split}
\end{equation*}

С каждым множеством $ \sigma $-алгебра содержит дополнение.
Тогда
$ \\
\left\{ \omega: \, \eta \left( \omega \right) > b \right\} \in
\mathcal{F} $.
По условию
$$B =
\left\{ \omega \in \Omega: \,
\xi \left( \omega \right) =
\eta \left( \omega \right) \right\}.$$
Дополнение к нему равно
$$ \overline{B} =
\left\{ \omega \in \Omega: \,
\xi \left( \omega \right) >
\eta \left( \omega \right) \right\} \cup
\left\{ \omega \in \Omega: \,
\xi \left( \omega \right) <
\eta \left( \omega \right) \right\}.$$
По предыдущим пунктам каждое из этих множеств принадлежит $ \mathcal{F} $.
Из этого следует, чо объединение принадлежит $ \mathcal{F} $.

$C$ можно представить как
$$C =
A \cup B \in \mathcal{F},$$
так как $A \in \mathcal{F} $ и $B \in \mathcal{F} $.

\addcontentsline{toc}{section}{Дополнительные задачи}
\section*{Дополнительные задачи}

\addcontentsline{toc}{section}{Домашнее задание}
\section*{Домашнее задание}
