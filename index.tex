\documentclass{book}
\usepackage[utf8]{inputenc}
\usepackage[russian]{babel}
\usepackage{color}
\usepackage{hyperref}
\hypersetup{
	colorlinks=true,
	linkcolor=blue,
   	linktoc=all
}

\begin{document}
\tableofcontents

\addcontentsline{toc}{chapter}{Занятие 1. Элементы комбинаторики}
\chapter*{Занятие 1. Элементы комбинаторики}

\addcontentsline{toc}{section}{Контрольные вопросы и задания}
\section*{Контрольные вопросы и задания}

\subsubsection*{Сформулируйте основной принцип комбинаторики (правило умножения).}
Если множества $A_1, ..., A_m$ содержат соответственно $n_1, ..., n_m$ элементов, то количество m-мерных векторов, которые получают выбором по одному элементу из каждого множества равно $n_1\cdot n_2\cdot...\cdot n_m$.

\subsubsection*{Что называется сочетанием из n элементов по k?} 
В комбинаторике сочетанием из n по k называется набор k элементов, выбранных из данного множества, содержащего n различных элементов.

\subsubsection*{Чему равно число сочетаний из n элементов по k?}
Если некоторое множество содержит n элементов, то количество её k-элементных подмножеств равно $C_n^k=C_n^{n-k}=\frac{n!}{k!\left(n-k\right)!=\frac{n\cdot\left(n-1\right)\cdot...\cdot\left(n-k+1\right)}{k!}}$, где $n!=1\cdot 2\cdot...\cdot n$. Считается, что $0!=1$.

\subsubsection*{Что называется сочетанием с повторениями из n элементов по k?}
Сочетанием (комбинацией) с повторениями называется набор из n элементов, каждый из которых может быть одного из k типов.

\subsubsection*{Чему равно число сочетаний с повторениями из n элементов по k?}
Количество разных комбинаций из элементов n типов по k с повторениями равно $C_n^k=C_{n+m-1}^{m-1}=C_{n+m-1}^n$.

\subsubsection*{Что называется размещением из n элементов по k?}
Размещения из n элементов по k --- это упорядоченные k-элементные подмножества множества, которое состоит из n элементов.

\subsubsection*{Чему равно число размещений из n элементов по k?}
Количество размещений из n элементов по k равно $A_n^k=k!C_n^k=n\cdot(n-1)\cdot...\cdot(n-k+1)$.

\subsubsection*{Чему равно число перестановок множества из n элементов?}
Количество перестановок равно $n!$.

\subsubsection*{Чему равно число способов разбиения множества из n элементов на m непересекающихся неупорядоченных подмножеств, которые содержат соответственно $k_1, ..., k_m$ элементов?}
Первое подмножество содержит $k_1$ элемент. Количество комбинаций, которыми можно выбрать эти элементы, равно $C_n^{k_1}$. Второе подмножество содержит $k_2$ элементов, тогда его элементы можно выбрать $C_{n-k_1}^{k_2}$ способами. Подмножество под номером m содержит $k_m$ элементов. Эти элементы можно выбрать из оставшихся $\left(n-k_1-k_2-...-k_{m-1}\right)$ $C_{n-k_1-k_2-...-k_{m-1}}^{k_m}$ способами. Число способов разбиения множества из n элементов на m непересекающихся неупорядоченных подмножеств равно $C_n^{k_1}\cdot C_{n-k_1}^{k_2}\cdot...\cdot C_{n-k_1-k_2-...-k_{m-1}}^{k_m}=\prod\limits_{i=1}^mC_{n-\sum\limits_{j=0}^{i-1}k_j}^{k_i}$. Так как $n=k_1+k_2+...+k_m=\sum\limits_{i=1}^m k_i=K_i$, то произведение можно записать в виде $\prod\limits_{i=1}^mC_{K_i}^{k_i}$.

\addcontentsline{toc}{section}{Домашние задачи}
\section*{Домашние задачи}

\subsubsection*{1.16. Подсчитать, сколько трёхзначных чисел можно записать с помощью: а) цифр 0, 1, 2, 3, 4, 5; б) цифр 0, 1, 2, 3, 4, 5, если каждую из цифр использовать не больше одного раза}

\textit{Решение.} Трёхзначное число можно рассматривать как трёхмерный вектор. Первой компонентой этого вектора может быть любая цифра из множества $A_1={1, 2, 3, 4, 5}$ (запись числа не может начинаться с 0).

а) На остальных позициях может стоять любая цифра, то есть $A_i={0, 1, 2, 3, 4, 5}, i=2, 3$. Отсюда имеем, что из указанных цифр можно составить $5\cdot 6\cdot 6=180$ трёхзначных чисел.

б) На остальных позициях могут стоять любые цифры (кроме тех, что стояли на предыдущих позициях). Отсюда имеем, что из указанных цифр можно составить $5\cdot 5\cdot 4=100$ трёхзначных чисел.

\subsubsection*{1.17. Подсчитать количество пятизначных чисел, которые делятся на 5}

\textit{Решение.} Пятизначное число можно рассматривать как пятимерный вектор. Первой компонентой этого вектора может быть любая цифра из множества $A_1={1, 2, 3, 4, 5, 6, 7, 8, 9}$ (запись числа не может начинаться с 0), а на остальных позициях (кроме последней) может стоять любая цифра, то есть $A_i={0, 1, 2, 3, 4, 5, 6, 7, 8, 9}, i=2, 3, 4$. На последней позиции может стоять цифра из множества $A_5={0, 5}$ (чтобы число делилось на 5, оно должно оканчиваться на 0 или 5). Отсюда имеем, что можно составить $9\cdot 10\cdot 10\cdot 10\cdot 2=18000$ пятизначных чисел, которые делятся на 5.

\subsubsection*{1.18. Замок компьютерного центра состоит из пяти кнопок, пронумерованных от 1 до 5. Чтобы открыть замок, необходимо первые две определённые кнопки нажать одновременно, а потом одну за другой нажать другие три кнопки в определённой последовательности. Подсчитать количество способов закодировать вход в компьютерный центр}

\textit{Решение.} Рассмотрим 3 случая: 

а) сначала необходимо нажать две разные кнопки, далее их отпускают, и все остальные кнопки могут быть любыми; 

б) первые две кнопки держатся нажатыми, следующие кнопки не могут быть такими, как первые две; 

в) нельзя нажать одну и ту же кнопку больше одного раза (кнопки остаются нажатыми).

Количество способов нажать первые две кнопки равна количеству двухэлементных подмножеств в множестве из пяти элементов, то есть $C_5^2=\frac{5!}{2!\left(5-2\right)!}=10$.

В случае а) количество способов закодировать вход в компьютерный центр равно $C_5^2\cdot 5^3=1250$. Общая формула: $C_N^n\cdot N^m$, где N --- количество кнопок, n кнопок нажимаются вместе, а затем m кнопок --- по очереди.

В случае б) после нажатия двух кнопок, остаётся только 3 кнопки, которые необходимо нажать в правильном порядке, поэтому количество способов по предыдущей формуле равно $C_5^2\cdot 3^3=270$.

В случае в) все кнопки должны быть нажаты один раз, поэтому количество способов закодировать вход равно $C_5^2\cdot 3\cdot 2\cdot 1=60$.

\end{document}
