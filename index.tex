\documentclass{book}
\usepackage[utf8]{inputenc}
\usepackage[russian]{babel}
\usepackage{color}
\usepackage{hyperref}
\usepackage{amsfonts}
\usepackage{graphicx}
\usepackage{amssymb}
\usepackage{indentfirst}
\usepackage{amsmath}
\usepackage{bbm}
\hypersetup{
	colorlinks=true,
	linkcolor=blue,
   	linktoc=all
}

\begin{document}
\tableofcontents

\pagestyle{plain}

\addcontentsline{toc}{chapter}{Занятие 1. Элементы комбинаторики}
\chapter*{Занятие 1. Элементы комбинаторики}

\addcontentsline{toc}{section}{Контрольные вопросы и задания}
\section*{Контрольные вопросы и задания}

\subsubsection*{Сформулируйте основной принцип комбинаторики (правило умножения).}

Если множества $A_1, ..., A_m$ содержат соответственно $n_1, ..., n_m$ элементов, то количество m-мерных векторов, которые получают выбором по одному элементу из каждого множества равно $n_1\cdot n_2\cdot...\cdot n_m$.

\subsubsection*{Что называется сочетанием из n элементов по k?}

В комбинаторике сочетанием из n по k называется набор k элементов, выбранных из данного множества, содержащего n различных элементов.

\subsubsection*{Чему равно число сочетаний из n элементов по k?}

Если некоторое множество содержит n элементов, то количество её k-элементных подмножеств равно $$C_n^k=C_n^{n-k}=\frac{n!}{k!\left(n-k\right)!=\frac{n\cdot\left(n-1\right)\cdot...\cdot\left(n-k+1\right)}{k!}},$$ где $n!=1\cdot 2\cdot...\cdot n$. Считается, что $0!=1$.

\subsubsection*{Что называется сочетанием с повторениями из n элементов по k?}

Сочетанием (комбинацией) с повторениями называется набор из n элементов, каждый из которых может быть одного из k типов.

\subsubsection*{Чему равно число сочетаний с повторениями из n элементов по k?}

Количество разных комбинаций из элементов n типов по k с повторениями равно $$C_n^k=C_{n+m-1}^{m-1}=C_{n+m-1}^n.$$

\subsubsection*{Что называется размещением из n элементов по k?}

Размещения из n элементов по k --- это упорядоченные k-элементные подмножества множества, которое состоит из n элементов.

\subsubsection*{Чему равно число размещений из n элементов по k?}

Количество размещений из n элементов по k равно $$A_n^k=k!C_n^k=n\cdot(n-1)\cdot...\cdot(n-k+1).$$

\subsubsection*{Чему равно число перестановок множества из n элементов?}

Количество перестановок равно $n!$.

\subsubsection*{Чему равно число способов разбиения множества из n элементов на m непересекающихся неупорядоченных подмножеств, которые содержат соответственно $k_1, ..., k_m$ элементов?}

Первое подмножество содержит $k_1$ элемент. Количество комбинаций, которыми можно выбрать эти элементы, равно $C_n^{k_1}$. Второе подмножество содержит $k_2$ элементов, тогда его элементы можно выбрать $C_{n-k_1}^{k_2}$ способами. Подмножество под номером m содержит $k_m$ элементов. Эти элементы можно выбрать из оставшихся $\left(n-k_1-k_2-...-k_{m-1}\right)$ $C_{n-k_1-k_2-...-k_{m-1}}^{k_m}$ способами. Число способов разбиения множества из n элементов на m непересекающихся неупорядоченных подмножеств равно $$C_n^{k_1}\cdot C_{n-k_1}^{k_2}\cdot...\cdot C_{n-k_1-k_2-...-k_{m-1}}^{k_m}=\prod\limits_{i=1}^mC_{n-\sum\limits_{j=0}^{i-1}k_j}^{k_i}.$$ Так как $$n=k_1+k_2+...+k_m=\sum\limits_{i=1}^m k_i=K_i,$$ то произведение можно записать в виде $$\prod\limits_{i=1}^mC_{K_i}^{k_i}.$$

\addcontentsline{toc}{section}{Аудиторные задачи}
\section*{Аудиторные задачи}

\subsubsection*{1.3}

\textit{Задание.} Посетителям кафе в качестве первого блюда предлагают борщ или суп, в качестве основного блюда --- мясо, рыбу или вегетарианский салат, в качестве десерта --- мороженое или пирожное. Подсчитать, сколько разных обедов из трёх блюд можно заказать в этом кафе.

\textit{Решение.} Есть два варианта выбора первого блюда: борщ и суп. Число способов выбрать первое блюдо равно $C_2^1=2$. Основное блюдо можно выбрать тремя, т.е. $C_3^1$ способами, так как есть 3 разных блюда на выбор. На десерт предлагают два блюда на выбор, поэтому есть $C_2^1=2$ способа его заказать. Отсюда имеем, что из указанных блюд можно заказать $$C_2^1\cdot C_3^1\cdot C_2^1=2\cdot 3\cdot 2=12$$ обедов.

\subsubsection*{1.4}

\textit{Задание.} Репертуар оркестра состоит из тридцати симфоний Гайдна, девяти симфоний Бетховена и четырёх концертов Моцарта. На концерте оркестр должен исполнить три произведения. Подсчитать количество способов составить концертную программу, если:

а) сначала необходимо исполнить произведение Гайдна, потом --- Бетховена и к завершению --- Моцарта;

б) в программу должно войти по одному произведению Гайдна, Бетховена и Моцарта, исполнять которые можно в произвольном порядке;

в) в программу должны войти три произвольные произведения из репертуара оркестра и выполнять их можно в произвольном порядке.

\textit{Решение.} 

а) Первым произведением должно быть произведение Гайдна. Таких в репертуаре оркестра есть тридцать. Одно произведение из тридцати можно выбрать $C_{30}^1=30$ способами. Вторым произведением должна быть симфония Бетховена. Количество способов выбрать одну симфонию из десяти равно $C_9^1=9$. Из четырёх концертов Моцарта один можно выбрать $C_4^1=4$ способами. Итого $$C_{30}^1\cdot C_9^1\cdot C_4^1=30\cdot 9\cdot 4$$ способа составить концертную программу.

б) Так как порядок исполнения произведений не важен, то на на любом из трёх мест может быть произведение любого из указанных авторов. Отсюда имеем, что результат, полученный в пункте а) нужно умножить на $3!$. Тогда количество способов составить концертную программу, где будет звучать по одному произведению каждого исполнителя, равно $$3!\cdot C_{30}^1\cdot C_9^1\cdot C_4^1=3!\cdot 30\cdot 9\cdot 4.$$

в) На первом месте в концертной программе может стоять любое произведение из $30+9+4=43$ возможных. Количество способов его выбрать равно $C_{43}^1=43$. На втором месте --- любое из 42 оставшихся. Это $C_{42}^1=42$. На третьем --- любое из оставшихся 41 произведения. Есть $C_{41}^1=41$ способ, чтобы его выбрать. Отсюда имеем, что составить концертную программу можно $$C_{43}^1\cdot C_{42}^1\cdot C_{41}^1=43\cdot 42\cdot 41$$ способом.

\subsubsection*{1.5}

\textit{Задание.} В цифровом компьютере один бит --- это одно из чисел $\{0, 1\}$, а слово --- это произвольная строка из 32-х бит. Подсчитать количество разных слов.

\textit{Решение.} В данном случае имеем множество, которое содержит 32 бита двух типов. В качестве первого символа 32-битного слова можно выбрать одно из двух чисел. Количество способов это сделать равно $C_{2}^1=2$. Аналогично на всех остальных позициях может стоять либо 0, либо 1. Отсюда имеем, что всего можно составить $2^{32}$ слова.

\subsubsection*{1.6}
\textit{Задание.} Сколькими способами можно упорядочить множество $\{1, 2, ..., 2n\}$ так, чтобы каждое чётное число имело чётный номер?

\textit{Решение.} Каждое чётное число должно стоять на чётной позиции, каждое нечётное число --- на нечётной. В множестве есть n чётных и n нечётных чисел. Количество способов разместить n чётных чисел на n чётных позициях равно $n!$. Аналогично можно разместить n нечётных чисел на n нечётных позициях $n!$ способами. Так как независимо заполняем n и n ячеек, то множество можно упорядочить $n!\cdot n!$ способами.

\subsubsection*{1.7}

\textit{Задание.} Среди 11-ти преподавателей кафедры есть 6 мужчин и 5 женщин. Сколькими способами из них можно выбрать комиссию из 3-х человек так, чтобы в ней было 2 мужчины и одна женщина?

\textit{Решение.} Двух мужчин из шести можно выбрать $C_6^2$ способами. Одну женщину из пяти --- $C_5^1$ способами. Тогда комиссию можно выбрать $C_6^2\cdot C_5^1$ способами.

\subsubsection*{1.8}

\textit{Задание.} Сколькими способами можно разместить 15 томов на книжной полке так, чтобы том I и II не стояли рядом?

\textit{Решение.} Определим общее число перестановок из 15 элементов по формуле $P_{15}=15!$.

Чтобы вычислить число <<лишних>> перестановок, сначала определим, сколько вариантов, в которых 2-й том находится рядом с 1-ым справа от него. В таких перестановках 1-ый том может занимать места с первого по 14-е, а второй со второго по 15-е --- всего 14 мест для этой пары книг. И при каждом таком положении первых двух томов остальные 13 книг могут занимать остальные 13 мест в произвольном порядке. Вариантов перестановки 13 книг $P_{13}=13!$. Всего лишних вариантов 2-го тома справа от 1-го получится $14\cdot 13!=14!$.

Аналогично рассмотрим случай, когда второй том расположен рядом с 1-ым, но слева от него. Получается такое же число вариантов $14\cdot 13!=14!$.

Значит всего <<лишних>> перестановок $2\cdot 14!$, а нужных способов расстановки $15!-2\cdot 14!$.

\subsubsection*{1.9}

\textit{Задание.} Сколькими способами можно распределить 12 разных предметов между тремя людьми так, чтобы каждый получил 4 предмета?

\textit{Решение.} Первый человек может получить любые 4 предмета из 12. Способов это сделать $C_{12}^4$. Второй человек может выбрать любые 4 предмета из 8 оставшихся --- это $C_8^4$. Третьему человеку число способов дать 4 предмета равно $C_4^4$. Отсюда имеем, что распределить предметы можно $C_{12}^4\cdot C_8^4\cdot C_4^4$ способами.

\subsubsection*{1.10}

\textit{Задание.} Сколькими способами можно распределить 3 ириски, 5 карамелек и 12 шоколадных конфет среди 20-ти детей так, чтобы:

а) каждый ребёнок получил по конфете?

б) каждый ребёнок мог получить произвольное количество конфет?

\textit{Решение.}

а) В данном случае имеем множество, которое содержит 20 конфет 3 типов, причём ириски встречаются 3 раза, карамельки --- 5 раз, а шоколадные конфеты --- 12 раз.

Допустим, нужно распределить среди детей $3+5+12$ разных конфет. Есть $\left(3+5+12\right)!$ способов это сделать. А теперь нужно учесть, что из них 3 одинаковые, и не брать во внимание перестановки между ними --- $$\frac{\left(3+5+12\right)!}{3!}.$$ Но 5 и 12 из них тоже одинаковые. Тогда количество разных перестановок равно $$C_{20}\left(3, 5, 12\right)=\frac{20!}{3!5!12!}.$$

б) Имеется $n_1=3$ конфеты одного сорта, $n_2=5$ другого, $n_3=12$ --- 3-го сорта. Используем метод перегородок. 

Сначала нужно распределить 3 одинаковые ириски на 20 групп. Вводим 19 перегородок, которые определяют, сколько именно конфет достанется каждому ребёнку. Обозначаем конфету символом <<0>>, а перегородку символом <<1>>. Тогда распределение ирисок однозначно характеризуется последовательностью из 3 нулей и 19 единиц. Количество способов, которыми можно распределить 3 ириски, совпадает с количеством разных последовательностей указанного вида. Последовательность определяется однозначно, если выбрать 19 мест из 22, где будут размещены единицы. Количество таких комбинаций равно $C_{22}^{19}$.

Аналогично распределяем 5 карамелек среди 20 детей. Вводим 19 перегородок, которые определяют, сколько именно конфет достанется каждому из детей. В результате чего получаем ответ $C_{24}^{19}$.

Остаётся распределить 12 одинаковых шоколадных конфет на 20 групп. Используя метод перегородок, получаем ответ $C_{31}^{19}$.

По комбинаторному правилу умножения всего способов $$C_{22}^{19}\cdot C_{24}^{19}\cdot C_{31}^{19}.$$

\subsubsection*{1.11}

\textit{Задание.} Вычислить количество неубывающих путей на двумерной целочисленной решётке $$\mathbb{Z}_+^2=\{\left(i, j\right): i, j=0, 1, 2, ...\},$$ которые начинаются в точке $\left(0, 0\right)$ и приводят в точку $\left(m, n\right)$. (Путь считается неубывающим, если на каждом шаге изменяется только одна координата, увеличиваясь на единицу.)

\textit{Решение.} Имеем пути, состоящие из $m+n$ ходов, среди которых ровно m <<горизонтальных>> и n <<вертикальных>> ходов. Если выберем, на каких шагах увеличиваем одну координату (например, вертикальную), будем однозначно знать, где увеличиваем вторую координату (идём по горизонтали). Значит, достаточно расставить m одинаковых шагов вверх на $n+m$ разных ячеек. Это будет $$C_{n+m}^m=\frac{\left(n+m\right)!}{n!m!}.$$

\subsubsection*{1.12}

\textit{Задание.} Вычислить количество разных частных производных порядка r бесконечно дифференцируемой функции n переменных $f\left(x_1, ..., x_n\right)$.

\textit{Решение.} Функция бесконечно дифференцируема, значит, производная порядка r есть. Рассмотрим функцию $f(x, y)$. Производные первого порядка: $f'_x, f'_y$. Первая производная --- одномерная матрица (вектор). Производные второго порядка для $f(x, y)$: $f''_{xx}, f''_{xy}, f''_{yx}, f''_{yy}$. $f''_{xy}$ и $f''_{yx}$ одинаковые. Представим вторую производную в виде матрицы:
$$
\begin{pmatrix}
  xx & yx \\
  xy & yy \\ 
\end{pmatrix}
$$

Если три переменные:
$$
\begin{pmatrix}
  xx & yx & zx \\
  xy & yy & zy \\
  xz & yz & zz \\ 
\end{pmatrix}
$$

Строим матрицы по определённому алгоритму. Столбцы --- это первое измерение, а строки --- второе. Столбцы соответствуют x, y и z, и в ячейках столбцов на первой позиции стоит соответствующий дифференциал. В ячейках строк указано, какой дифференциал будет вторым.

Рассмотрим третью производную для функции трёх переменных --- это трёхмерная матрица. Третье измерение показывает, какой дифференциал будет третьим. Значит, срез --- это то, где третьи дифференциалы фиксированы. Получится три обычные матрицы: в первой на третьей позиции будут x, во второй --- y, а в третьей --- z. Имеем: 
$$
\begin{pmatrix}
xxx & yxx & zxx \\ 
xyx & yyx & zyx \\
xzx & yzx & zzx \\
\end{pmatrix}
,
\begin{pmatrix}
xxy & yxy & zxy \\
xyy & yyy & zyy \\
xzy & yzy & zzy \\
\end{pmatrix}
,
\begin{pmatrix}
xxz & yxz & zxz \\
xyz & yyz & zyz \\
xzz & yzz & zzz \\
\end{pmatrix}
.$$

Матрицы записаны с учётом повторения элементов, значит нужно взять только те элементы, которые лежат на диагонали и с одной стороны от неё (треугольник). Задача сведена к подсчёту количества элементов в r-мерной матрице $n\times n$.

Производный первого порядка есть столько же, сколько переменных у функции, т.е. n.

Найдём количество элементов двумерной матрицы $n\times n$, её верхнего треугольника. Всего элементов в матрице $n\cdot n=n^2$. Возьмём половину: $$\frac{n\cdot n}{2}$$ и прибавим ещё половину диагонали, чтобы получить треугольник, получаем $$\frac{n\cdot n}{2}+\frac{n}{2}=\frac{n(n+1)}{2}.$$

Для трёхмерной матрицы $n\times n$ проделываем аналогичные шаги. Всего есть $n\cdot n\cdot n=n^3$ элементов, на диагонали --- $n\cdot n=n^2$ элементов. Имеем: $$\frac{n\cdot n\cdot n}{2}+\frac{n\cdot n}{2}=\frac{n^2(n+1)}{2}.$$

Обобщим для r-мерной матрицы. Всего элементов $n^r$, элементов на диагонали $n^{r-1}$. Отсюда имеем, что элементов на диагонали и выше её $$\frac{n^r}{2}+\frac{n^{r-1}}{2}=\frac{n^{r-1}(n+1)}{2}.$$

Это и есть число разных частных производных функции n переменных порядка r.

\subsubsection*{1.13}

\textit{Задание.} Пусть $\omega_m$ --- количество таких перестановок элементов множества $\{1, 2, ..., n\}$, что ни одно из чисел не остаётся на своём месте. Доказать, что величина $p_n=\frac{\omega_n}{n!}$ равна $$p_n=\frac{1}{2!}-\frac{1}{3!}+...+\frac{\left(-1\right)^n}{n!}.$$

\textit{Решение.} Найдём значение величины $\omega_n$.

Любая перестановка $k_1k_2...k_n$ чисел $1, 2, ..., n$ означает вариант перестановки чисел, при котором i-е число стоит на $k_i$-м месте. Например, в случае четырёх чисел перестановка 3241 означает, что на первом месте стоит тройка $\left(k_1=3\right)$, на втором --- двойка (на своём месте, $k_2=2$), на третьем --- четвёртая $\left(k_3=4\right)$ и на последнем --- первая $\left(k_4=1\right)$. Наоборот, каждый вариант перестановки чисел обозначается единственной перестановкой чисел $1, 2, ..., n$.

Будем говорить, что в перестановки $k_1k_2...k_n$ чисел $1, 2, ..., n$ число i стоит на своём месте, если $k_i=i$ (например, в перестановке 3241 двойка стоит на своём месте). Нас интересует количество беспорядков, то есть таких перестановок, в которых ни одно из чисел не стоит на своём месте. Число беспорядков можно найти, вычитая из общего количества перестановок, равного $n!$, количество тех перестановок, в которых хотя бы одно из чисел стоит на своём месте.

Пусть $A_i$ --- множество перестановок, в которых число i стоит на своём месте $\left(i=1, 2, ..., n\right)$. Искомое число $\omega_n$ беспорядков, таким образом, равно $$\omega_n=n!-|A_1\cup A_2\cup...\cup A_n|=n!-\sum\limits_i|A_i|+\sum\limits_{i<j}|A_i\cap A_j|-$$$$-\sum\limits_{i<j<k}|A_i\cap A_j\cap A_k|+...+\left(-1\right)^n|A_1\cap A_2\cap...\cap A_n|.$$

$|A_i|=\left(n-1\right)!$, поэтому $$\sum\limits_i|A_i|=n\cdot\left(n-1\right)!=n!.$$

Так же $|A_i\cap A_j|=\left(n-2\right)!$ и $$\sum\limits_{i<j}|A_i\cap A_j|=C_n^2\cdot\left(n-2\right)!=\frac{n\left(n-1\right)}{2!}\cdot\left(n-2\right)!=\frac{n!}{2!}.$$

Аналогично $|A_i\cap A_j\cap A_k|=\left(n-3\right)!$ и $$\sum\limits_{i<j<k}|A_i\cap A_j\cap A_k|=C_n^3\cdot\left(n-3\right)!=\frac{n\left(n-1\right)\left(n-3\right)}{3!}\cdot\left(n-3\right)!=\frac{n!}{3!}.$$

Теперь приходим к нужной формуле: $$\omega_n=n!-n!+\frac{n!}{2!}=\frac{n!}{3!}+...+\left(-1\right)^n=n!\cdot\sum\limits_{k=0}^n\frac{\left(-1\right)^k}{k!}.$$

Так как числа переставляются случайным образом, то вероятность беспорядка равна $$p_n=\frac{\omega_n}{n!}=\frac{1}{n!}\left(\frac{n!}{2!}-\frac{n!}{3!}+...+\left(-1\right)^n\right)=\frac{1}{2!}-\frac{1}{3!}+...+\frac{\left(-1\right)^n}{n!}.$$

\subsubsection*{1.14}

\textit{Задание.} В классе учится 35 учеников. Из них 20 занимаются в математическом кружке, 11 --- в физическом, а 10 учеников не посещают ни одного кружка. Сколько учеников посещают математический и физический кружок. Сколько учеников посещают только математический кружок?

\textit{Решение.} Изобразим диаграмму Эйлера-Венна для данной задачи на рисунке \ref{fig:114}. 

\begin{figure}[h!]
  \centering
  \includegraphics[width=.4\textwidth]{./pictures/1_14.png}
  \caption{Диаграмма Эйлера-Венна для задачи 1.14}
  \label{fig:114}
\end{figure}

Найдём, сколько учеников занимается хотя бы в одном кружке. Это будет $35-10=25$ учеников. Это есть количество элементов объединения двух множеств: $$|M\cup F|=|M|+|F|-|M\cap F|=25.$$ Отсюда можем найти, сколько учеников посещают и математический, и физический кружки: $$|M\cap F|=|M|+|F|-|M\cup F|=20+11-25=6.$$ Только математический кружок посещает $|M|-|M\cap F|=20-6=14$ учеников.

\addcontentsline{toc}{section}{Дополнительные задачи}
\section*{Дополнительные задачи}

\subsubsection*{1.15}

\textit{Задание.} Пусть множество X содержит n элементов, а множество Y --- m элементов. Вычислить:

а) количество функций из X в Y;

б) количество инъекций из X в Y $\left(n\leq m\right)$;

в) количество биекций из X в Y $\left(n=m\right)$.

\textit{Решение.}

а) Можно считать, что $X=\{1, ..., n\}, Y=\{1, ..., m\}$. Каждую функцию можно отождествлять с последовательностью $<f(1), ..., fn)>=<y_1, ..., y_n>$. Каждый член $y_i$ последовательности можно выбрать m способами, что даёт $m^n$ возможностей выбора $<y_1, ..., y_n>$.

б) Будем определять число инъективных (то есть имеющих все различные члены) последовательностей $<y_1, ..., y_n>$. Элемент $y_1$ может быть выбран m способами, элемент $y_2$ можно выбрать $m-1$ способом из оставшихся элементов. Если уже выбраны элементы $y_1, ..., y_{i-1}$, то в качестве $y_i$ может быть выбран любой из $m-i+1$ элементов множества $Y\setminus\{y_1, ..., y_{i-1}\}$. Принимаем, что $m\geq n$. если $n>m$, то искомое число функций равно 0. Это даёт $$m\left(m-1\right)...\left(m-n+1\right)=\frac{m!}{\left(m-n\right)!}$$ возможность выбора инъективных последовательностей $<y_1, ..., y_n>$.

в) Если $n=m$, то любое инъективное отображение будет биективным. Число всех биективных отражений X в Y равно $n!$ при $m=n$ и 0 при $m\neq n$.

\addcontentsline{toc}{section}{Домашнее задание}
\section*{Домашнее задание}

\subsubsection*{1.16}

\textit{Задание.} Подсчитать, сколько трёхзначных чисел можно записать с помощью: а) цифр 0, 1, 2, 3, 4, 5; б) цифр 0, 1, 2, 3, 4, 5, если каждую из цифр использовать не больше одного раза.

\textit{Решение.} Трёхзначное число можно рассматривать как трёхмерный вектор. Первой компонентой этого вектора может быть любая цифра из множества $$A_1=\{1, 2, 3, 4, 5\}$$ (запись числа не может начинаться с 0).

а) На остальных позициях может стоять любая цифра, то есть $$A_i=\{0, 1, 2, 3, 4, 5\}, i=2, 3.$$ Отсюда имеем, что из указанных цифр можно составить $5\cdot 6\cdot 6=180$ трёхзначных чисел.

б) На остальных позициях могут стоять любые цифры (кроме тех, что стояли на предыдущих позициях). Отсюда имеем, что из указанных цифр можно составить $5\cdot 5\cdot 4=100$ трёхзначных чисел.

\subsubsection*{1.17}

\textit{Задание.} Подсчитать количество пятизначных чисел, которые делятся на 5.

\textit{Решение.} Пятизначное число можно рассматривать как пятимерный вектор. Первой компонентой этого вектора может быть любая цифра из множества $$A_1=\{1, 2, 3, 4, 5, 6, 7, 8, 9\}$$ (запись числа не может начинаться с 0), а на остальных позициях (кроме последней) может стоять любая цифра, то есть $$A_i=\{0, 1, 2, 3, 4, 5, 6, 7, 8, 9\}, i=2, 3, 4.$$ На последней позиции может стоять цифра из множества $A_5=\{0, 5\}$ (чтобы число делилось на 5, оно должно оканчиваться на 0 или 5). Отсюда имеем, что можно составить $9\cdot 10\cdot 10\cdot 10\cdot 2=18000$ пятизначных чисел, которые делятся на 5.

\subsubsection*{1.18}

\textit{Задание.} Замок компьютерного центра состоит из пяти кнопок, пронумерованных от 1 до 5. Чтобы открыть замок, необходимо первые две определённые кнопки нажать одновременно, а потом одну за другой нажать другие три кнопки в определённой последовательности. Подсчитать количество способов закодировать вход в компьютерный центр.

\textit{Решение.} Рассмотрим 3 случая: 

а) сначала необходимо нажать две разные кнопки, далее их отпускают, и все остальные кнопки могут быть любыми; 

б) первые две кнопки держатся нажатыми, следующие кнопки не могут быть такими, как первые две; 

в) нельзя нажать одну и ту же кнопку больше одного раза (кнопки остаются нажатыми).

Количество способов нажать первые две кнопки равна количеству двухэлементных подмножеств в множестве из пяти элементов, то есть $$C_5^2=\frac{5!}{2!\left(5-2\right)!}=10.$$

В случае а) количество способов закодировать вход в компьютерный центр равно $$C_5^2\cdot 5^3=1250.$$ Общая формула: $C_N^n\cdot N^m$, где N --- количество кнопок, n кнопок нажимаются вместе, а затем m кнопок --- по очереди.

В случае б) после нажатия двух кнопок, остаётся только 3 кнопки, которые необходимо нажать в правильном порядке, поэтому количество способов по предыдущей формуле равно $$C_5^2\cdot 3^3=270.$$

В случае в) все кнопки должны быть нажаты один раз, поэтому количество способов закодировать вход равно $$C_5^2\cdot 3\cdot 2\cdot 1=60.$$

\subsubsection*{1.19}

\textit{Задание.} Колоду игральных карт (52 карты, 4 масти по 13 карт в каждой) тщательно перетасовали. Подсчитать количество способов выбрать из неё 6 карт без возвращения так, чтобы среди них: а) был пиковый король; б) были представители всех мастей; в) было ровно 5 карт одной масти.

\textit{Решение.}

а) Выбрать пикового короля есть только один способ. Остальные 6 карт могут быть любыми из оставшихся 51 карты. Выбрать эти 5 карт можно $C_{51}^5$ способами. Отсюда имеем, что из колоды можно выбрать 6 карт, среди которых был бы пиковый король, $1\cdot C_{51}^5$ числом способов.

б) Сначала выберем по одной карте каждой масти. Количество способов выбрать одну карту определённой масти равно $C_{13}^1$, так как имеется 13 карт каждой масти. Так как всего есть 4 масти, то нужно выбрать 4 карты (по одной карте каждой масти). Тогда количество способов выбрать 4 карты разных мастей равно $$C_{13}^1\cdot C_{13}^1\cdot C_{13}^1\cdot C_{13}^1=\left(C_{13}^1\right)^4.$$ Остаётся выбрать две произвольные карты из оставшихся. После выбора четырёх карт разных мастей в колоде осталось $52-4=48$ карт. Число способов выбрать из них две карты равно $C_{48}^2$. Отсюда имеем, что количество способов выбрать из данной колоды 6 карт без возвращения таким образом, чтобы среди них были представители всех мастей, равно $$\left(C_{13}^1\right)\cdot C_{48}^2.$$

в) Сначала нужно выбрать 5 карт одной масти. В одной масти 13 карт, следовательно число способов выбрать 5 карт одной масти равно $C_{13}^5$. Так как всего мастей 4, и нам не важно, какой именно масти будут 5 вытянутых карт (главное, чтобы одной), то число способов будет равно $$4\cdot C_{13}^5.$$ Шестая карта должна быть любой, но отличатся с предыдущими пятью мастью, то есть её можно выбрать из $52-13=39$ карт. Число способов это сделать равно $C_{39}^1$. Отсюда имеем, что число способов выбрать из колоды карт 6 карт так, чтобы среди них было ровно 5 карт одной масти, равно $$4\cdot C_{13}^5\cdot C_{39}^1.$$

\subsubsection*{1.20}

\textit{Задание.} Сколькими способами можно разместить 10 одинаковых открыток в 4 почтовых ящиках так, чтобы: а) не было пустых ящиков; б) во втором ящике было 3 открытки.

\textit{Решение.}

а) Рассмотрим случай, когда в каждый ящик должна быть помещена хотя бы одна открытка. Используем метод перегородок. Выложим открытки в ряд. Для определения расклада открыток по четырём почтовым ящикам разделим ряд тремя перегородками на 4 группы: первая группа для первого ящика, вторая --- для второго и так далее. Таким образом, число вариантов раскладки открыток по ящикам равно числу способов разложения трёх перегородок. Перегородки могут стоять на любом из 9 мест (между 10 открытками --- 9 промежутков). Поэтому число возможных расположений равно $C_9^3$.

б) Рассмотрим случай, когда во второй ящик должны быть помещены 3 открытки. Поскольку открытки одинаковые, то во второй ящик можно сразу положить 3 открытки. В этом случае нужно распределить $10-3=7$ одинаковых открыток между тремя почтовыми ящиками.

Используем метод перегородок. Рассмотрим ряд из 9 предметов: 7 одинаковых открыток и 2 одинаковые перегородки, расположенных в произвольном порядке. Каждый такой ряд однозначно соответствует некоторому способу раскладки открыток по ящикам: в первый ящик попадают открытки, расположенные левее первой перегородки, во второй --- расположенные между первой и второй перегородками и т.д. (между какими-то перегородками открыток может и не быть). Поэтому число способов раскладки открыток по ящикам равно числу различных рядов из 7 открыток и 2 перегородок, т.е равно $C_9^2$ (ряд определяется теми двумя местами из 9, на которых стоят перегородки).

\subsubsection*{1.21}

\textit{Задание.} Сколькими способами можно распределить 10 путёвок среди 10 студентов (по одной каждому), если: а) все путёвки разные; б) есть 4 путёвки одного типа и 6 --- другого?

\textit{Решение.}

а) Количество возможных перестановок 10 разных путёвок равно $10!$.

б) Если будем считать все 10 элементов перестановки с повторениями различными, то всего различных вариантов перестановок 10 путёвок --- $$(4+6)!=10!.$$ Однако среди этих перестановок не все различны. Все путёвки одного типа можно переставлять местами друг с другом, и от этого перестановка не изменится. Точно так же, можем переставлять путёвки другого типа. Таким образом, перестановка может быть записана $4!6!$ способами. Следовательно, число различных перестановок с повторениями равно $$\frac{(4+6)!}{4!6!}=\frac{10!}{4!6!}.$$

\subsubsection*{1.22}

\textit{Задание.} Доказать, что количество неубывающих путей на r-мерной целочисленной решётке $$\mathbb{Z}_+^r=\{\left(i_1, ..., i_r\right): i_1, ..., i_r=0, 1, 2, ...\},$$ которые начинаются в точке $\left(0, ..., 0\right)$ и приводят в точку $\left(n_1, ..., n_r\right)$, равно $$C_N\left(n_1, ..., n_r\right)=\frac{N!}{n_1!...n_r!},$$ где $N=\sum\limits_{i=1}^rn_i$. (Путь считается неубывающим, если на каждом шаге изменяется только одна координата, увеличиваясь на единицу.)

\textit{Решение.} Имеем пути, состоящие из $n_1+...+n_r$ ходов, среди которых ровно $n_1$ ходов в направлении $r_1$, ..., и $n_r$ ходов в направлении $r_n$. Если выберем, на каких шагах увеличиваем первую координату, будем знать, где увеличиваем остальные координаты. Далее выберем, на каких шагах увеличиваем вторую координату. И так необходимо определить, где увеличиваем $n_1+...n_{r-1}$ координат. Тогда будем однозначно знать, где увеличиваем последнюю координату. Это будет $$\frac{\left(n_1+...+n_r\right)}{n_1!...n_r!}.$$

\subsubsection*{1.23}

\textit{Задание.} Из 100 студентов английский язык знают 28, немецкий --- 30, французский --- 42, английский и немецкий --- 8, английский и французский --- 10, немецкий и французский --- 5, а все три языка знают 3 студента. Сколько студентов не знают ни одного языка?

\textit{Решение.} Условие задачи представлено на рисунке \ref{fig:123} в виде диаграммы Эйлера-Венна. 

\begin{figure}[h!]
  \centering
  \includegraphics[width=.4\textwidth]{./pictures/1_23.png}
  \caption{Диаграмма Эйлера-Венна для задачи 1.23}
  \label{fig:123}
\end{figure}

Сначала найдём, сколько студентов знает хотя бы один язык. Это есть мощность (количество элементов) объединения трёх множеств: Английский, Немецкий и Французский, которые обозначим первыми буквами. Мощность объединения трёх множеств можно найти как сумму мощностей этих множеств, но, так как они пересекаются, необходимо отнять мощности их пересечений и добавить пересечение всех трёх множеств, потому что его отняли дважды. Имеем $$|H\cup A\cup F|=|H|+|A|+|F|-|H\cap A|-|A\cap F|-|H\cap F|+|H\cap A\cap F|=$$$$=30+42+28-5-8-10+3=80.$$

Все остальные студенты из ста не знают ни одного языка. Их $100-80=20$.

\addcontentsline{toc}{chapter}{Занятие 2. События и операции над ними. Пространство элементарных событий}
\chapter*{Занятие 2. События и операции над ними. Пространство элементарных событий}

\addcontentsline{toc}{section}{Контрольные вопросы и задания}
\section*{Контрольные вопросы и задания}

\subsubsection*{Приведите определение вероятностного эксперимента, вероятностного пространства, случайного события.}

Вероятностным экспериментом называется явление, исход которого для нас не определён, и которое можно повторить любое число раз независимым образом.

Вероятностное пространство --- совокупность всех исходов вероятностного эксперимента, $\Omega$.

Случайное событие --- подмножество всех исходов вероятностного эксперимента, $A\subset\Omega$.

\subsubsection*{Запишите основные операции над случайными событиями и дайте их теоретико-множественную интерпретацию.}

Операции будем иллюстрировать на диаграммах Эйлера-Венна. На рис.\ref{fig:2} заштрихованы области, которые соответствуют событиям, являющимся результатами таких операций.

Пересечением (произведением) двух событий А и В называют событие С, происходящее тогда и только тогда, когда одновременно происходят оба события А и В, т.е событие, состоящее из тех и только тех элементарных исходов, которые принадлежат и события А, и событию В (рис.\ref{fig:2}, а).

Пересечение событий А и В записывают следующим образом: $C=A\cap B$, или $C=AB$.

События А и В называют несовместимыми, или непересекающимися, если их пересечение является невозможным событием, т.е если $A\cap B=\varnothing$ (рис.\ref{fig:2}, б).

В противном случае события называют совместимыми, или пересекающимися.

\begin{figure}[h!]
  \centering
  \includegraphics[width=.7\textwidth]{./pictures/2.png}
  \caption{Диаграммы Эйлера-Венна для операций над событиями}
  \label{fig:2}
\end{figure}

Объединением (суммой) двух событий А и В называют событие С, происходящее тогда и только тогда, когда происходит хотя бы одно из событий А или В, т.е. событие С, состоящее из тех элементарных исходов, которые принадлежат хотя бы одному из подмножеств А или В (рис.\ref{fig:2}, в).

Объединение событий А и В записывают в виде $C=A\cup B$.

Разностью двух событий А и В называют событие С, происходящее тогда и только тогда, когда происходит событие А, но не происходит событие В, т.е. событие С, состоящее из тех элементарных исходов, которые принадлежат А, но не принадлежат В (рис.\ref{fig:2}, г).

Разность событий А и В записывают в виде: $C=A\setminus B$.

Дополнением события А (обычно обозначают $\overline{A}$) называют событие, происходящее тогда и только тогда, когда не происходит событие А (рис.\ref{fig:2}, д). Другими словами, $\overline{A}=\Omega\setminus A$.

Событие А включено в событие В, что записывается $A\subset B$, если появление события А обязательно влечёт за собой наступление события В (рис.\ref{fig:2}, е), или каждый элементарный исход $\omega$, принадлежащий А, обязательно принадлежит и событию В.

Верхний предел последовательности $\{A_n: n\geq 1\}$ --- это случайное событие, состоящее в том, что произошло бесконечно много событий из исходной последовательности: $$\varlimsup\limits_{n\to\infty}A_n=\bigcap\limits_{n\geq 1}\bigcup\limits_{m\geq n}A_m.$$

$\forall n \exists m\geq n: A_m$ произошло.

Нижний предел последовательности: $\{A_n: n\geq 1\}$ --- это случайное событие, состоящее в том, что произошли все события, начиная с некоторого из исходной последовательности: $$\varlimsup\limits_{n\to\infty}A_n=\bigcup\limits_{n=1}^\infty\bigcap\limits_{m=n}^\infty A_m.$$

$\exists n \forall m\geq n$ происходит событие $A_m$.

\subsubsection*{Сформулируйте законы де Моргана.}

Первый закон де Моргана гласит: <<Если неверно, что есть и первое, и второе, то неверно либо одно из, либо оба>>, что выражается следующей формулой: $\overline{AB}=\overline{A}\cup\overline{B}$.

Второй закон де Моргана гласит: <<Если неверно, что есть первое, или неверно, что есть второе, то неверно, что есть первое и второе>>, что выражается следующей формулой: $\overline{A\cup B}=\overline{A}$ $\overline{B}$.

Законы де Моргана верны для любого конечного числа событий: $$\overline{A_1\cup A_2\cup...\cup A_n}=\overline{A_1}\,\overline{A_2}\,...\overline{A_n},$$ $$\overline{A_1A_2...A_n}=\overline{A_1}\cup\overline{A_2}\cup...\cup\overline{A_n}.$$

\addcontentsline{toc}{section}{Аудиторные задачи}
\section*{Аудиторные задачи}

\subsubsection*{2.3}

\textit{Задание.} Рассмотрим эксперимент, который состоит в подбрасывании трёх монет. Постройте множество $\Omega$ элементарных событий этого эксперимента. Опишите событие A, которое состоит в том, что выпало не меньше двух гербов. Вычислите вероятность события A.

\textit{Решение.} Пусть эксперимент состоит в подбрасывании одной монеты. При математическом описании этого опыта естественно отвлечься от несущественных возможностей (например, монета встанет на ребро) и ограничиться только двумя элементарными исходами: выпадение <<герба>> (обозначим этот исход $\omega_1$) и выпадением <<цифры>> (обозначим этот исход $\omega_2$). Таким образом, $\Omega_1=\{\omega_1, \omega_2\}$.

При подбрасывании двух монет пространство элементарных исходов будет содержать три элемента, т.е. $\Omega_2=\{\omega_{11}, \omega_{12}, \omega_{22}\}$, где, например, $\omega_{11}$ --- появление <<герба>> и на первой, и на второй монете.

При подбрасывании трёх монет пространство элементарных исходов будет содержать элементов, т.е. $\Omega_3=\Omega=\{\omega_{111}, \omega_{112}, \omega_{122}, \omega_{222}\}$, где, например, $\omega_{111}$ --- появление <<герба>> и на первой, и на второй, и на третьей монете.

Событие A состоит в том, что выпало не меньше двух гербов, т.е. два или три герба. Выберем из $\Omega$ такие исходы: $A=\{\omega_{122}, \omega_{222}\}$.

Поскольку $|A|=2, |\Omega|=4$, то $$P(A)=\frac{|A|}{|\Omega|}=\frac{2}{4}=\frac{1}{2}.$$

\subsubsection*{2.4}

\textit{Задание.} Пусть A, B, C --- произвольные события. Найдите выражения для событий, который состоят в том, что из событий A, B и C:

а) произошло только  A;

б) произошли A и B, но не произошло C;

в) произошли все три события;

г) произошло хотя бы одно из этих событий;

д) произошло хотя бы два события;

е) произошло одно и только одно событие;

ё) произошло два и только два события;

ж) ни одно из событий не произошло;

з) произошло не больше двух событий.

\textit{Решение.}

а) $A\cap\overline{B}\,\overline{C}$;

б) $A\cap B\cap\overline{C}$;

в) $A\cap B\cap C$;

г) $A\cup B\cup C$;

д) $\left(A\cap B\right)\cup\left(A\cap C\right)\cup\left(B\cap C\right)$;

е) $\left(A\cap\overline{B}\cap\overline{C}\right)\cup\left(B\cap\overline{A}\cap\overline{C}\right)\cup\left(C\cap\overline{A}\cap\overline{B}\right)$;

ё) $\left(A\cap B\cap\overline{C}\right)\cup\left(A\cap C\cap\overline{B}\right)\cup\left(B\cap C\cap\overline{A}\right)$;

ж) $\overline{A}\cap\overline{B}\cap\overline{C}$;

з) $\left(\overline{A}\cap\overline{B}\cap\overline{C}\right)\cup\left(A\cap\overline{B}\cap\overline{C}\right)\cup\left(B\cap\overline{A}\cap\overline{C}\right)\cup\left(C\cap\overline{A}\cap\overline{B}\right)\cup\left(A\cap B\cap\overline{C}\right)\cup\left(A\cap C\cap\overline{B}\right)\cup\left(B\cap C\cap\overline{A}\right)$.

\subsubsection*{2.5}

\textit{Задание.} Пусть A и B --- некоторые события. Упростите выражение $$C=\overline{\overline{A\cap\overline{B}}\cup\overline{A\cup B}\cup\left(B\cap A\right)}.$$

\textit{Решение.} $C=\overline{\overline{A\cap\overline{B}}}\cup\overline{\overline{A\cup B}}\cup\overline{B\cap A}=\left(A\cap\overline{B}\right)\cup A\cup B\cup\overline{B}\cup\overline{A}=\left(A\cap\overline{B}\right)\cup A\cup\overline{A}\cup B\cup\overline{B}=\left(A\cap\overline{B}\right)\cup\Omega\cup\Omega=\left(A\cap\overline{B}\right)\cup\Omega=\Omega$.

\subsubsection*{2.6}

\textit{Задание.} Пусть $\{A_n, n\in\mathbb{N}\}, \{B_n, n\in\mathbb{N}\}$ --- некоторые последовательности событий. Объясните, что значат события $\varlimsup\limits_{n\to\infty}A_n, \varliminf\limits_{n\to\infty}A_n$. Докажите, что:

а) $\overline{\varlimsup\limits_{n\to\infty}A_n}=\varliminf\limits_{n\to\infty}\overline{A_n}$;

б) $\varlimsup\limits_{n\to\infty}\left(A_n\cup B_n\right)=\varlimsup\limits_{n\to\infty}A_n\cup\varlimsup\limits_{n\to\infty}B_n$.

\textit{Решение.} Верхний предел последовательности $\{A_n: n\geq 1\}$ --- это случайное событие, состоящее в том, что произошло бесконечно много событий из исходной последовательности: $$\varlimsup\limits_{n\to\infty}A_n=\bigcap\limits_{n\geq 1}\bigcup\limits_{m\geq n}A_m.$$

$\forall n \exists m\geq n: A_m$ произошло.

Нижний предел последовательности: $\{A_n: n\geq 1\}$ --- это случайное событие, состоящее в том, что произошли все события, начиная с некоторого из исходной последовательности: $$\varlimsup\limits_{n\to\infty}A_n=\bigcup\limits_{n=1}^\infty\bigcap\limits_{m=n}^\infty A_m.$$

$\exists n \forall m\geq n$ происходит событие $A_m$.

а)

б)

\subsubsection*{2.7}

\textit{Задание.} Пусть $\{A_n, n\in\mathbb{N}\}$ --- некоторая последовательность событий, $\mathbbm{1}(A)$ --- индикатор события A. Докажите, что $\mathbbm{1}\left(\varlimsup\limits_{n\to\infty}A_n\right)=\varlimsup\limits_{n\to\infty}\mathbbm{1}\left(A_n\right)$.

\textit{Решение.}

\subsubsection*{2.8}

\textit{Задание.} Пусть B и C --- два события. Положим $A_n=B$, если n чётное и $A_n=C$, если n нечётное. Найдите событие, которое состоит в том, что:

а) произошло бесконечно много событий из последовательности $\{A_n\}_{n=1}^\infty$;

б) произошло только конечное количество событий из последовательности $\{A_n\}_{n=1}^\infty$.

\textit{Решение.}

\subsubsection*{2.9}

\textit{Задание.} Рассмотрим эксперимент, который состоит в выборе наугад точки в квадрате с вершинами в точках $A(0, 0), B(0, 1), C(1, 1)$ и $D(1, 0)$. Опишите и изобразите вероятностное пространство этого эксперимента и следующие события:

а) A = {данная точка оказалась на расстоянии не меньшем чем 1/4 от сторон квадрата};

б) B = {данная точка оказалась внутри круга с центром в начале координат и радиусом 1/2};

в) $\overline{A}, \overline{B}, A\cap B$.

\textit{Решение.} Пространство элементарных событий опишем как множество упорядоченных пар $\Omega=\{(x, y), 0\leq x\geq 1, 0\leq y\geq 1\}$, где x --- первая координата точки, y --- вторая координата точки.

а) $A=\{(x, y), \frac{3}{4}\leq x\geq\frac{1}{4}, \frac{3}{4}\leq y\geq\frac{1}{4}, \}$;

б) $B=\{(x, y), x\geq 0, y\geq 0, x^2+y^2\leq\frac{1}{2}\}$;

в)событие $\overline{A}$ означает, что событие A не произошло, то есть расстояние от точки до сторон квадрата оказалось больше 1/4: $$\overline{A}=\{(x, y), \frac{3}{4}>x>\frac{1}{4}, \frac{3}{4}>y>\frac{1}{4}\}.$$

Событие $\overline{B}$ означает, что данная точка не оказалась внутри круга с центром в начале координат и радиусом 1/2: $$\overline{B}=\{(x, y), \frac{1}{2}>x^2+y^2\leq 1, x\leq 1, y\leq 1.\}$$

Событие $A\cap B$ означает, что произошло и событие A, и событие B. Отсюда имеем, что данная точка оказалась на расстоянии не меньшем чем 1/4 от сторон квадрата, а так же внутри круга с центром в начале координат и радиусом 1/2, т.е. $$A\cap B=\{(x, y), x\geq\frac{1}{4}, y\geq\frac{1}{4}, x^2+y^2\leq\frac{1}{2}\}.$$

\addcontentsline{toc}{section}{Дополнительные задачи}
\section*{Дополнительные задачи}

\subsubsection*{2.10}

\textit{Задание.} Пусть A --- множество из n элементов, $A_1, ..., A_k$ --- подмножества A такие, что ни одно из подмножеств не является частью другого, а $i_1, ..., i_k$ --- количество элементов подмножеств $A_1, ..., A_k$ соответственно. Докажите, что $$\sum\limits_{r=1}^k\frac{1}{C_n^{i_r}}\leq 1.$$

\textit{Решение.}

\addcontentsline{toc}{section}{Домашнее задание}
\section*{Домашнее задание}

\subsubsection*{2.11}

\textit{Задание.} Рассмотрим эксперимент, который состоит в подбрасывании трёх игральных кубиков. Опишите множество $\Omega$ элементарных событий этого эксперимента; из скольки элементарных событий оно состоит? Опишите событие С, которое состоит в том, что на всех кубиках выпало одинаковое количество очков. Вычислите вероятность события С.

\textit{Решение.} Пространство элементарных событий опишем как множество упорядоченных троек $\Omega=\{\left(i, j, k\right), i=\overline{1, 6}, j=\overline{1, 6}, k=\overline{1, 6}\}$, где i --- количество очков, которые выпали на первом кубике, j --- количество очков, которые выпали на втором кубике, k --- количество очков, которые выпали на третьем кубике.

Можем рассматривать тройку чисел как вектор длины 3. Первой компонентой вектора может быть любое значение из $\{1, 2, 3, 4, 5, 6\}$. Его можно выбрать $C_6^1=6$ способами. Таким же образом находим, что есть 6 способов выбрать вторую компоненту вектора и 6 --- третью. По правилу умножения имеем $6\cdot 6\cdot 6=6^3=108$ разных векторов указанного вида, или элементарных событий.

$C=\{(1, 1, 1), (2, 2, 2), (3, 3, 3), (4, 4, 4), (5, 5, 5), (6, 6, 6)\}$. Поскольку $|C|=6$, $|\Omega|=108$, то $$P(C)=\frac{|C|}{|\Omega|}=\frac{6}{108}=\frac{1}{36}.$$

\subsubsection*{2.12}

\textit{Задание.} Монету подбрасывают до тех пор, пока она не выпадет 2 раза подряд одной и той же стороной, но не больше четырёх раз. Опишите множество $\Omega$ элементарных событий. Опишите следующие события и вычислите их вероятности:

а) А = { эксперимент закончился на втором подбрасывании};

б) В = { эксперимент закончился на третьем подбрасывании};

в) C = { эксперимент закончился на четвёртом подбрасывании}.

\textit{Решение.}  Пусть опыт состоит в однократном подбрасывании монеты. При математическом описании этого опыта естественно отвлечься от несущественных возможностей (например, монета встанет на ребро) и ограничиться только двумя элементарными исходами: выпадение <<герба>> (обозначим этот исход $\omega_1$) и выпадением <<цифры>> (обозначим этот исход $\omega_2$). Таким образом, $\Omega_1=\{\omega_1, \omega_2\}$.

При двукратном подбрасывании монеты пространство элементарных исходов будет содержать четыре элемента, т.е. $$\Omega_2=\{\omega_{11}, \omega_{12}, \omega_{21}, \omega_{22}\},$$ где, например, $\omega_{11}$ --- появление <<герба>> и при первом, и при втором подбрасываниях. В данном случае эксперимент может завершиться, если при двукратном подбрасывании монеты она выпала два раза одной и той же стороной. Поэтому $A=\{\omega_{11}, \omega_{22}\}$.

При трёхкратном подбрасывании монеты пространство элементарных исходов будет содержать 8 элементов, т.е. $$\Omega_3=\{\omega_{111}, \omega_{112}, \omega_{122}, \omega_{121}, \omega_{211}, \omega_{221}, \omega_{212}, \omega_{222}\},$$ где, например, $\omega_{111}$ --- появление <<герба>> и при первом, и при втором, и при третьем подбрысываниях. Исходом эксперимента могут быть такие 3 подбрасывания, при которых в первый раз выпала одна сторона монеты, а в следующие 2 --- другая, т.е. $B=\{\omega_{122}, \omega_{211}\}$.

При четырёхкратном подбрасывании монеты пространство исходов будет содержать 16 элементов, т.е $$\Omega_4=\{\omega_{1111}, \omega_{1112}, \omega_{1121}, \omega_{1211}, \omega_{2111}, \omega_{1122}, \omega_{1212}, \omega_{2112}, \omega_{2211}, \omega_{2121}, \omega_{1221},$$$$\omega_{1222}, \omega_{2122}, \omega_{2212}, \omega_{2221}, \omega_{2222}\},$$ где, например, $\omega_{1111}$ --- появление <<герба>> при всех четырёх подбрысываниях. Чтобы эксперимент не закончился раньше четвёртого подбрасывания, уберём из множества $\Omega_4$ такие его элементы, которые обеспечивают конец эксперимента при втором и третьем подбрасывании. Получим $C=\{\omega_{1211}, \omega_{1212}, \omega_{2121}, \omega_{2122}\}$.

Пространство элементарных исходов состоит из всех элементов множеств A, B и C, т.е. $$\Omega=\{\omega_{11}, \omega_{22}, \omega_{122}, \omega_{211}, \omega_{1211}, \omega_{1212}, \omega_{2121}, \omega_{2122}\}.$$

Поскольку $|A|=2$, $|\Omega|=8$, то $$P(A)=\frac{|A|}{|\Omega|}=\frac{2}{8}=\frac{1}{4}.$$

Так же и $|B|=2$, $|\Omega|=8$, поэтому $$P(B)=\frac{|B|}{|\Omega|}=\frac{2}{8}=\frac{1}{4}.$$

Поскольку $|C|=4$, $|\Omega|=8$, то $$P(C)=\frac{|C|}{|\Omega|}=\frac{4}{8}=\frac{1}{2}.$$

\subsubsection*{2.13}

\textit{Задание.} Рабочий произвёл n деталей. Пусть событие $A_i$ состоит в том, что i-я деталь имеет дефект. Запишите событие, которое состоит в том, что:

а) ни одна из деталей не имеет дефектов;

б) хотя бы одна из деталей имеет дефект;

в) ровно одна деталь имеет дефект;

г) ровно две детали имеют дефект;

д) хотя бы две детали не имеют дефектов;

е) не больше двух деталей имеют дефект.

\textit{Решение.}

а) $\overline{A_1}\cap\overline{A_2}\cap...\cap\overline{A_n}=\bigcap\limits_{i=1}^n\overline{A_i}$;

б) $A_1\cup A_2\cup...\cup A_n=\bigcup\limits_{i=1}^nA_i$;

в) $\bigcup\limits_{i=1}^n\left(A_i\bigcap\limits_{j\neq i}\overline{A_j}\right)$;

г) $\bigcup\limits_{i=1}^n\bigcup\limits_{j=1}^n\left(\bigcap\limits_{k\neq i, j}\overline{A_k}\right)$;

д) $\bigcup\limits_{i=1}^n\bigcup\limits_{j=1}^n\overline{A_i}\,\overline{A_j}$;

е) $\bigcup\limits_{i=1}^n\bigcup\limits_{j=1}^n\left(A_iA_j\bigcap\limits_{k\neq i, j}\overline{A_k}\right)$.

\subsubsection*{2.14}

\textit{Задание.} Пусть А, В, С --- некоторые события. Что означают равенства:

а) $A\cap B\cap C=A$?

б) $A\cup B\cup C=A$?

\textit{Решение.}

а) Событие А содержится и в событии В, и в событии С;

б) событие А содержит и событие В, и событие С.

\subsubsection*{2.15}

\textit{Задание.} Упростите выражение:

а) $\left(A\cup B\right)\cup\left(A\cup\overline{B}\right)$;

б) $\left(A\cup B\right)\cap\left(\overline{A}\cup B\right)\cap\left(A\cup\overline{B}\right)$.

\textit{Решение.} Используя свойства операций над событиями, получаем:

а) $\left(A\cup B\right)\cup\left(A\cup\overline{B}\right)=A\cup B\cup A\cup\overline{B}=A\cup A\cup B\cup\overline{B}=\left(A\cup A\right)\cup\left(B\cup\overline{B}\right)=A\cup\Omega=\Omega$.

б) $\left(A\cup B\right)\cap\left(\overline{A}\cup B\right)\cap\left(A\cup\overline{B}\right)=\left(A\cup B\right)\cap\left(A\cup\overline{B}\right)\cap\left(\overline{A}\cup B\right)=\left(A\cup\left(B\cap\overline{B}\right)\right)\cap\left(\overline{A}\cup B\right)=\left(A\cup\varnothing\right)\cap\left(\overline{A}\cup B\right)=A\cap\left(\overline{A}\cup B\right)=\left(A\cap\overline{A}\right)\cup\left(A\cap B\right)=\varnothing\cup\left(A\cap B\right)=A\cap B$.

\subsubsection*{2.16}

\textit{Задание.} Пусть $\{A_n, n\in\mathbb{N}\}, \{B_n, n\in\mathbb{N}\}$ --- некоторые последовательности событий. Докажите, что:

а) $\overline{\varliminf\limits_{n\to\infty}A_n}=\varlimsup\limits_{n\to\infty}\overline{A_n}$;

б) $\varlimsup\limits_{n\to\infty}A_n\cap\varliminf\limits_{n\to\infty}B_n\subseteq\varlimsup\limits_{n\to\infty}\left(A_n\cap B_n\right)\subseteq\varlimsup\limits_{n\to\infty}A_n\cap\varlimsup\limits_{n\to\infty}B_n$.

\textit{Решение.}

\subsubsection*{2.17}

\textit{Задание.} Пусть $\{A_n, n\in\mathbb{N}\}$ --- некоторая последовательность событий, $\mathbbm{1}(A)$ --- индикатор события А. Докажите, что: $\mathbbm{1}\left(\varliminf\limits_{n\to\infty}A_n\right)=\varliminf\limits_{n\to\infty}\mathbbm{1}\left(A_n\right)$.

\textit{Решение.}

\subsubsection*{2.18}

\textit{Задание.} Рассмотрим эксперимент, который состоит в выборе наугад точки в круге единичного радиуса с центром в начале координат. Опишите и изобразите пространство элементарных событий этого эксперимента, а также следующие события:

а) A = {произведение координат точки не превышает 1/8};

б) B = {данная точка оказалась внутри круга с центром в начале координат и радиусом 1/4};

в) $\overline{A}, \overline{B}, A\cap B$.

\textit{Решение.} Пространство элементарных событий опишем как множество упорядоченных пар $\Omega=\{(x, y), x^2+y^2\leq 1\}$, где x --- первая координата точки, y --- вторая координата точки.

а) $A=\{(x, y), x^2+y^2\leq 1, xy\leq\frac{1}{8}\}$;

б) $B=\{(x, y), x^2+y^2<\frac{1}{4}\}$;

в) cобытие $\overline{A}$ означает, что событие А не произошло, то есть произведение координат точки превышает 1/8: $$\overline{A}=\{(x, y), xy>1/8, x^2+y^2\leq 1\}.$$

Событие $\overline{B}$ означает, что точка не оказалась внутри круга радиуса 1/4: $$\overline{B}=\{(x, y), 1\geq x^2+y^2>\frac{1}{4}\}.$$

Событие $A\cap B$ означает, что произошло и событие A, и событие B. Отсюда имеем, что произведение координат точки не должно превышать 1/8, а сумма квадратов координат должна быть меньше 1/4, т.е. $$A\cap B=\{(x, y), x^2+y^2<\frac{1}{4}, xy\leq\frac{1}{8}\}.$$

\addcontentsline{toc}{chapter}{Занятие 3. Классическое определение вероятности}
\chapter*{Занятие 3. Классическое определение вероятности}

\addcontentsline{toc}{section}{Контрольные вопросы и задания}
\section*{Контрольные вопросы и задания}

\subsubsection*{Приведите определение вероятностного эксперимента, вероятностного пространства, случайного события.}

Вероятностным экспериментом называется явление, исход которого для нас не определён, и который можно повторить любое число раз независимым образом.

Вероятностное пространство --- совокупность всех исходов вероятностного эксперимента, $\Omega$.

Случайное событие --- подмножество всех исходов вероятностного эксперимента.

\subsubsection*{Как вычислить вероятность события в случае, когда вероятностное пространство состоит из конечного количества равновозможных элементарных событий?}

В этом случае вероятность любого события A вычисляется по формуле $$P(A)=\frac{|A|}{|\Omega|},$$ называемой классическим определением вероятности.

\subsubsection*{Запишите формулу включений и исключений.}

Если $B_1, B_2, ..., B_m$ --- некоторые события, то имеет место равенство $P\{\bigcup\limits_{i=1}^mB_i\}=\sum\limits_{i=1}^mP\{B_i\}-\sum\limits_{1\leq i_1<i_2\leq m}\{B_{i_1}\cap B_{i_2}\}+...+(-1)^{k-1}\sum\limits_{1\leq i_1<i_2<...<i_k\leq m}P\{B_{i_1}\cap B_{i_2}\cap...\cap B_{i_k}\}+...+(-1)^{m-1}P\{B_1\cap B_2\cap...\cap B_m\}.$

\addcontentsline{toc}{section}{Домашнее задание}
\section*{Домашнее задание}

\subsubsection*{3.15}

\textit{Задание.} Докажите, что для произвольных событий A и B $P(AB)=1-P\left(\overline{A}\right)-P\left(\overline{B}\right)+P\left(\overline{A}\,\overline{B}\right)$.

\textit{Решение.} $P(AB)=P(A)+P(B)-P\left(A\cup B\right)=1-P\left(\overline{A}\right)+1-P\left(\overline{B}\right)-\left(1-P\left(\overline{A\cup B}\right)\right)=1-P\left(\overline{A}\right)+1-P\left(\overline{B}\right)-1+P\left(\overline{A}\,\overline{B}\right)=1-P\left(\overline{A}\right)-P\left(\overline{B}\right)+P\left(\overline{A}\,\overline{B}\right)$.

\subsubsection*{3.16}

\textit{Задание.} Подбрасывают 4 игральных кубика. Найдите вероятность того, что на них выпадет одинаковое количество очков.

\textit{Решение.} На выпавшей грани <<первого>> игрального кубика может появиться одно очко, два очка, ..., шесть очков. Аналогичные шесть элементарных исходов возможны при бросании остальных кубиков. Таким образом, общее число возможных элементарных исходов испытания равно $6\cdot 6\cdot 6\cdot 6=1296$. Эти исходы в силу симметрии кубиков равновозможны.

Благоприятствующие интересующему нас событию (на всех гранях появится одинаковое количество очков) являются следующие шесть исходов (первым записано число очков, выпавших на <<первом>> кубике, вторым -- число очков, выпавших на <<втором>> кубике и т.д.): 1) 1, 1, 1, 1, 2) 2, 2, 2, 2, 3) 3, 3, 3, 3, 4) 4, 4, 4, 4, 5) 5, 5, 5, 5, 6) 6, 6, 6, 6.

Искомая вероятность равна отношению числа исходов, благоприятствующих событию, к числу всех возможных элементарных исходов: $$P=\frac{6}{6^4}=\frac{1}{6^3}.$$

\subsubsection*{3.17}

\textit{Задание.} Группа состоит из r студентов. Найдите вероятность того, что по крайней мере 2 студента родились в одном и том же месяце (считайте, что все месяцы года являются равновозможными для рождения).

\textit{Решение.} Год имеет 12 месяцев. День рождения каждого из студентов может приходиться на любой месяц года (12 вариантов).

Если студентов больше чем месяцев ($r>12$), то не существует исходов, при которых в один месяц попадает один человек. $P=0$.

Рассмотрим случай, когда студентов в группе не больше чем месяцев в году ($r\leq 12$). По правилу умножения существует всего $n=12^r$ вариантов размещения дней рождения студентов. Найдём количество вариантов, когда никакие два студента не имеют день рождения в тот же месяц. Для этого нужно вычислить количество способов, которыми из 12 месяцев можно выбрать упорядоченное множество из r месяцев. Используя формулу для размещения из 12 элементов по r, имеем $A_{12}^r$. Поэтому $$P=1-\frac{A_{12}^r}{12^r}.$$

\end{document}
