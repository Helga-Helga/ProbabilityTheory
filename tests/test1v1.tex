\addcontentsline{toc}{chapter}{Контрольная работа 1}
\chapter*{Контрольная работа 1}

\addcontentsline{toc}{section}{Вариант 1}
\section*{Вариант 1}

\subsubsection*{1}

\textit{Задание.} В сундуке лежат 10 красных, 6 синих и 4 зелёных пуговицы.
Найдите вероятность того, что две наугад вынутые пуговицы будут разного цвета.

\textit{Решение.}
Опишем пространство элементарных исходов: $ \Omega = \\ = \left\{ \left( x, y \right), \, x, y \in \right.$ красный, синий, зелёный\}\}.
Нужно выбрать две пуговицы из двадцати, поэтому $ \left| \Omega \right| = C_{20}^2$.

Опишем событие $A =$ \{две наугад вынутые пуговицы будут разного цвета\} $= \left\{ \left( x, y \right) in \Omega: \, x \neq y \right\} $.
Выберем сначала одну красную пуговицу --- $C_{10}^1$, а затем --- одну синюю --- $C_6^1$.
По правилу умножения есть $C_{10}^1 \cdot C_6^1$ способов вынуть одну красную и одну синюю пуговицу.
Второй случай: одна из пуговиц красная, другая --- зелёная --- $C_{10}^1 \cdot C_4^1$.
Третий случай: одна из пуговиц синяя, другая --- зелёная --- $C_6^1 \cdot C_4^1$.
По правилу суммы имеем, что $ \left| A \right| = C_{10}^1 \cdot C_6^1 + C_{10}^1 \cdot C_4^1 + C_6^1 \cdot C_4^1$.

Тогда вероятность равна
$$P \left( A \right) =
\frac{ \left| A \right| }{ \left| \Omega \right| } =
\frac{C_{10}^1 \cdot C_6^1 + C_{10}^1 \cdot C_4^1 + C_6^1 \cdot C_4^1}{C_{20}^2}.$$
