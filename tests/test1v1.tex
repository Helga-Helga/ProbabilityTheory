\addcontentsline{toc}{chapter}{Контрольная работа 1}
\chapter*{Контрольная работа 1}

\addcontentsline{toc}{section}{Вариант 1}
\section*{Вариант 1}

\subsubsection*{1}

\textit{Задание.} В сундуке лежат 10 красных, 6 синих и 4 зелёных пуговицы.
Найдите вероятность того, что две наугад вынутые пуговицы будут разного цвета.

\textit{Решение.}
Опишем пространство элементарных исходов: $ \Omega = \\ = \left\{ \left( x, y \right), \, x, y \in \right.$ красный, синий, зелёный\}\}.
Нужно выбрать две пуговицы из двадцати, поэтому $ \left| \Omega \right| = C_{20}^2$.

Опишем событие $A =$ \{две наугад вынутые пуговицы будут разного цвета\} $= \left\{ \left( x, y \right) in \Omega: \, x \neq y \right\} $.
Выберем сначала одну красную пуговицу --- $C_{10}^1$, а затем --- одну синюю --- $C_6^1$.
По правилу умножения есть $C_{10}^1 \cdot C_6^1$ способов вынуть одну красную и одну синюю пуговицу.
Второй случай: одна из пуговиц красная, другая --- зелёная --- $C_{10}^1 \cdot C_4^1$.
Третий случай: одна из пуговиц синяя, другая --- зелёная --- $C_6^1 \cdot C_4^1$.
По правилу суммы имеем, что $ \left| A \right| = C_{10}^1 \cdot C_6^1 + C_{10}^1 \cdot C_4^1 + C_6^1 \cdot C_4^1$.

Тогда вероятность равна
$$P \left( A \right) =
\frac{ \left| A \right| }{ \left| \Omega \right| } =
\frac{C_{10}^1 \cdot C_6^1 + C_{10}^1 \cdot C_4^1 + C_6^1 \cdot C_4^1}{C_{20}^2}.$$

\subsubsection*{2}

\textit{Задание.} Подбросили 10 игральных кубиков.
Найдите вероятность того, что единица выпала хотя бы на двух кубиках, если известно, что она выпала хотя бы на одном кубике.

\textit{Решение.}
Опишем пространство элементарных событий:
$ \Omega = \\
= \left\{ \left( x_1, x_2, \dotsc, x_{10} \right): \,
x_i \in \left\{1, 2, 3, 4, 5, 6 \right\}, \,
i =
\overline{1, 6} \right\} $.
На каждом кубике может выпасть число от одного до шести (6 вариантов), поэтому $ \left| \Omega \right| = 6^{10}$.

Рассмотрим событие $A =$ \{единица не выпала ни на одном кубике\}.
На каждом кубике может выпасть число от одного до пяти.
Таких вариантов есть $5^{10}$.
Тогда вероятность события $A$ равна
$$P \left( A \right) =
\frac{ \left| A \right| }{ \left| \Omega \right| } =
\frac{5^{10}}{6^{10}}.$$

Тогда событие $ \overline{A} =$ \{единица выпала хотя бы на одном кубике\} имеем вероятность
$$P \left( \overline{A} \right) =
1 - P \left( A \right) =
1 - \frac{5^{10}}{6^{10}}.$$

Рассмотрим событие $B =$ \{единица выпала хотя бы на двух кубиках\}.
Рассмотрим пересечение событий $ \overline{A} \cap B =$ \{единица выпала хотя бы на двух кубиках\} $= B$.
Рассмотрим противоположное событие: $ \overline{B} =$ \{единица выпала ровно на одном кубике\} $
\cup $ \{единица не выпала ни на одном кубике\} $= B_1 \cup A$.
Найдём количество способов выпадения цифр на кубиках, которые ему удовлетворяют.

$B_1 =$ \{единица выпала ровно на одном кубике\}.
Нужно выбрать 1 кубик из десяти, на котором выпала единица --- это $C_{10}^1$.
На остальных десяти кубиках может быть любое число от двух до шести --- $5^9$.
По правилу умножения $ \left| B_1 \right| = C_{10}^1 \cdot 5^9$.
Вероятность этого события равна
$$P \left( B_1 \right) =
\frac{ \left| B_1 \right) }{ \left| \Omega \right| } =
\frac{C_{10}^1 \cdot 5^9}{6^{10}}.$$

Тогда вероятность события $B$ равна
$$P \left( B \right) =
P \left( A \right) + P \left( B_1 \right) =
\frac{5^{10}}{6^{10}} + \frac{C_{10}^1 \cdot 5^9}{6^{10}} =
\frac{5^{10} + C_{10}^1 \cdot 5^9}{6^{10}}.$$

Условная вероятность события $B$ при условии, что событие $ \overline{A}$ произошло, равна
$$P \left( \left. B \right| \overline{A} \right) =
\frac{P \left( \overline{A} \cap B \right)}{P \left( \overline{A} \right) } =
\frac{P \left( B \right)}{P \left( \overline{A} \right) } =
\frac{ \frac{5^{10} + C_{10}^1 \cdot 5^9}{6^{10}}}{1 - \frac{5^{10}}{6^{10}} }.$$
